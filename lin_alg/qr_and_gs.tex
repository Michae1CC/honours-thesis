\subsection{Gram-Schmidt Process and QR factorisations}\label{Section4.2}

Many areas of linear algebra involving studying the column space of matrices. The $QR$ factorisation both provides a powerful tool to better understand the column space of a matrix and serves as an important factorisation mechanism for many numerical methods. Suppose that a matrix $\bm{A} = \left[ \bm{a}_1 , \bm{a}_2 , \ldots , \bm{a}_n \right] \in \RR^{n \times n}$ has full rank. The idea of a $QR$ factorisation is to find an alternative orthornormal basis for $\left\{ \bm{a}_i \right\}_{i=1}^{n}$, say $\left\{ \bm{q}_i \right\}_{i=1}^{n}$, and to somehow relate the original matrix $\bm{A}$ to a new matrix whose columns are $\left\{ \bm{q}_i \right\}_{i=1}^{n}$. Consider the following procedure that allows us to find an orthornormal basis $\left\{ \bm{q}_i \right\}_{i=1}^{n}$ for which $\operatorname{l.s} \left\{ \bm{a}_i \right\}_{i=1}^{n} = \operatorname{l.s} \left\{ \bm{q}_i \right\}_{i=1}^{n} $. First set $\bm{q}_1 = \frac{\bm{a}_1}{\| \bm{a}_i \|}$, clearly $\operatorname{l.s} \left\{ \bm{a}_1 \right\} = \operatorname{l.s} \left\{ \bm{q}_1 \right\}$. Next, construct a vector $\bm{q}_2' = \bm{a}_2 - r_{1,2} \cdot \bm{q}_1$ so that $\bm{q}_2' \perp \bm{q}_1$. This means
\begin{align*}
    0       & = \langle \bm{q}_1, \bm{q}_2' \rangle                                                   \\
    0       & = \langle \bm{q}_1, \bm{a}_2 - r_{1,2} \cdot \bm{q}_1 \rangle                           \\
    0       & = \langle \bm{q}_1, \bm{a}_2 \rangle - r_{1,2} \cdot \langle \bm{q}_1, \bm{q}_1 \rangle \\
    r_{1,2} & = \langle \bm{q}_1, \bm{a}_2 \rangle
\end{align*}
Since $\bm{q}_2'$ may not be a unit vector we set $\bm{q}_2 = \frac{\bm{q}_2'}{\| \bm{q}_2' \|}$ where $\operatorname{l.s} \left\{ \bm{a}_1, \bm{a}_2 \right\} = \operatorname{l.s} \left\{ \bm{q}_1, \bm{q}_2 \right\}$. Continuing in this fashion, the vector $\bm{q}_3'$ is constructed so that
\[
    \bm{q}_3' = \bm{a}_3 - \bm{r}_{1,3} \bm{q}_1 - \bm{r}_{2,3} \bm{q}_2
\]
is orthogonal to both $\bm{q}_2$ and $\bm{q}_1$. This amounts to setting $r_{1,3} = \langle \bm{q}_1, \bm{a}_3 \rangle$ and $r_{2,3} = \langle \bm{q}_2, \bm{a}_{3} \rangle$. Similarly, $\bm{q}_3'$ is normalized so that $\bm{q}_3 = \frac{\bm{q}_3'}{\| \bm{q}_3' \|}$ and $\operatorname{l.s} \left\{ \bm{a}_1, \bm{a}_2, \bm{a}_3 \right\} = \operatorname{l.s} \left\{ \bm{q}_1, \bm{q}_2, \bm{q}_3 \right\}$. Continuing in this fashion the $k^{th}$ vector in our orthornormal basis is computed as
\begin{equation}\label{eq: comp_orth_basis}
    \bm{q}_k = \frac{\bm{a}_k - \sum_{i=1}^{k-1} r_{i,k} \cdot \bm{q}_i}{r_{k,k}}
\end{equation}
where $r_{i,k} = \langle \bm{q}_i, \bm{a}_k \rangle$, $r_{k,k} = \| \bm{a}_k - \sum_{i=1}^{k-1} r_{i,k} \cdot \bm{q}_i \|$ and $\operatorname{l.s} \left\{ \bm{a}_1, \bm{a}_2, \ldots , \bm{a}_k \right\} = \operatorname{l.s} \left\{ \bm{q}_1, \bm{q}_2, \ldots , \bm{q}_k \right\}$. This procedure is known as the Gram-Schmidt process \cites{BerezanskyMakarovich1996FaV1,TrefethenLloydN.LloydNicholas1997Nla/,DemmelJamesW1997Anla} and is summarized in the following algorithm.

    % https://tex.stackexchange.com/questions/463359/algorithm-inside-a-tcolorbox-how-to-put-a-label-to-the-algorithm-but-the-captio
    % http://cfrgtkky.blogspot.com/2018/12/algorithm-inside-tcolorbox-how-to-put.html
    % {\centering
    %     \begin{minipage}{.85\linewidth}
    %         \begin{tcolorbox}[colback=white!100,colframe=black!100]
    %             \begin{algorithm}[H]
    %                 \caption{Classical Gram-Schmidt}
    % \label{alg: Classical_Gram-Schmidt}
    % \SetAlgoLined
    % \DontPrintSemicolon
    % \SetKwInOut{Input}{input}\SetKwInOut{Output}{output}

    % \Input{A basis $\left\{ \bm{a}_i \right\}_{i=1}^{n}$.}
    % \Output{An orthornormal basis $\left\{ \bm{q}_i \right\}_{i=1}^{n}$ such that $\operatorname{l.s} \left\{ \left\{ \bm{a}_i \right\}_{i=1}^{n} \right\} = \operatorname{l.s} \left\{ \left\{ \bm{q}_i \right\}_{i=1}^{n} \right\}$}
    % \BlankLine
    % \For{$k = 1$ \KwTo $n$}{
    %     $\bm{q}_k' = \bm{a}_k$\;
    %     \For{$i = 1$ \KwTo $k-1$}{
    %         $r_{i,k} = \langle \bm{q}_i, \bm{a}_k \rangle$\;
    %         $\bm{q}_k' = \bm{q}_k' - r_{i,k} \bm{q}_i$\;
    %     }
    %     $r_{k,k} = \| \bm{q}_k' \|$\;
    %     $\bm{q}_k = \bm{q}_k' / r_{k,k}$\;
    % }
    % \Return{$\left\{ \bm{q}_i \right\}_{i=1}^{n}$}
    % \BlankLine
    %             \end{algorithm}
    %         \end{tcolorbox}
    %     \end{minipage}
    %     \par
    % }


    {\centering
        \begin{minipage}{.85\linewidth}
            \begin{algorithm}[H]
                \caption{Classical Gram-Schmidt}
                \label{alg: Classical_Gram-Schmidt}
                \SetAlgoLined
                \DontPrintSemicolon
                \SetKwInOut{Input}{input}\SetKwInOut{Output}{output}

                \Input{A basis $\left\{ \bm{a}_i \right\}_{i=1}^{n}$.}
                \Output{An orthornormal basis $\left\{ \bm{q}_i \right\}_{i=1}^{n}$ such that $\operatorname{l.s} \left\{ \left\{ \bm{a}_i \right\}_{i=1}^{n} \right\} = \operatorname{l.s} \left\{ \left\{ \bm{q}_i \right\}_{i=1}^{n} \right\}$}
                \BlankLine
                \For{$k = 1$ \KwTo $n$}{
                    $\bm{q}_k' = \bm{a}_k$\;
                    \For{$i = 1$ \KwTo $k-1$}{
                        $r_{i,k} = \langle \bm{q}_i, \bm{a}_k \rangle$\;
                        $\bm{q}_k' = \bm{q}_k' - r_{i,k} \bm{q}_i$\;
                    }
                    $r_{k,k} = \| \bm{q}_k' \|$\;
                    $\bm{q}_k = \bm{q}_k' / r_{k,k}$\;
                }
                \Return{$\left\{ \bm{q}_i \right\}_{i=1}^{n}$}
                \BlankLine
            \end{algorithm}
        \end{minipage}
        \par
    }

Relating the column space of $\bm{A}$ to the orthornormal basis $\left\{ \bm{q}_{i} \right\}_{i=1}^{n}$ can be done in a matrix form as
\[
    \left[ \bm{a}_1 , \bm{a}_2 , \ldots \bm{a}_n \right] =
    \left[ \bm{q}_1 , \bm{q}_2 , \ldots \bm{q}_n \right]
    \begin{bmatrix}
        r_{1,1} & r_{1,2} & \cdots & r_{1,n} \\
                & r_{2,2} &        & \vdots  \\
                &         & \ddots & \vdots  \\
                &         &        & r_{n,n}
    \end{bmatrix}
\]
or more written succinctly as
\begin{equation}\label{eq: QR_factorisation}
    \bm{A} = \bm{Q} \bm{R}
\end{equation}
where $\bm{Q} = \left[ \bm{q}_1 , \bm{q}_2 , \ldots \bm{q}_n \right]$ and $\bm{R}_{i,j} = r_{i,j}$ for $i \leq j$ and $\bm{R}_{i,j} = 0$ for $i > j$. This is exactly the $QR$ factorisation for a full rank matrix. Note that $\operatorname{Range} \left( \bm{A} \right) = \operatorname{Range} \left( \bm{Q} \right)$. In general, any square matrix  $\bm{A} \in \KK^{m \times n}$ may be decomposed as $\bm{A} = \bm{Q} \bm{R}$ where $\bm{Q} \in \KK^{m \times m}$ is an orthogonal matrix and $\bm{R} \in \KK^{m \times n}$ is an upper triangular matrix. This is known as a full $QR$ factorisation. Since the bottom $(m-n)$ rows of this $\bm{R}$ consists entirely of zeros, it is often useful to partition the full $QR$ factorisation in the following manner to shed vacuous entries
\[
    \bm{A} = \bm{Q} \bm{R} = \bm{Q}
    \begin{bmatrix}
        \hat{\bm{R}} \\
        \bm{0}_{(m-n) \times n}
    \end{bmatrix}
    =
    \begin{bmatrix}
        \hat{\bm{Q}} & \bm{Q}'
    \end{bmatrix}
    \begin{bmatrix}
        \hat{\bm{R}} \\
        \bm{0}_{(m-n) \times n}
    \end{bmatrix}
    = \hat{\bm{Q}} \hat{\bm{R}}.
\]
This alternate decomposition is called the reduced (or sometimes the thin) QR-factorization. The following two theorems on the $QR$ factorization are stated without proof.

\begin{thm} \label{theorem: QR_general_existence}
    Every $\bm{A} \in \KK^{m \times n}, \; (m \geq n)$ has a full $QR$ factorisation, hence also a reduced $QR$ factorisation.
    \cite{TrefethenLloydN.LloydNicholas1997Nla/}
\end{thm}

\begin{thm} \label{theorem: QR_full_rank_unique}
    Each $\bm{A} \in \KK^{m \times n}, \; (m \geq n)$ of full rank has a unique reduced $QR$ factorisation $\bm{A} = \hat{\bm{Q}} \hat{\bm{R}}$ with $r_{k,k} > 0$.
    \cite{TrefethenLloydN.LloydNicholas1997Nla/}
\end{thm}

In practice the classical Gram-Schmidt process described in \Cref{alg: Classical_Gram-Schmidt} is rarely used as the procedure becomes numerically unstable if $\left\{ \bm{a}_i \right\}_{i=1}^{n}$ are almost linearly dependent. Before looking at ways to resolve these numerical instabilities, a quick recap of projectors is useful. A square matrix $\bm{P}_{G}$ acting on a Hilbert space $H$ that sends $\bm{x} \in H$ to its projection onto a subspace $G$ is called the projector onto $G$. If $\left\{ \bm{q}_k \right\}_{k=1}^{m}$ is an orthornormal basis in $G$ then
\[
    \bm{P}_{G} = \bm{Q} \bm{Q}^{\ast}
\]
where $\bm{Q} = \left[ \bm{q}_1 , \bm{q}_2 , \ldots \bm{q}_m, \bm{0} , \ldots , \bm{0} \right] \in \RR^{n \times n}$. A special class of projectors which isolates the components of a given vector onto a one dimensional subspace spanned by a single unit vector $\bm{q}$ called a rank one orthogonal projector, denoted as $\bm{P}_{q}$. Each $k$ in the classical Gram-Schmidt process creates $\bm{q}_k'$ using the following orthogonal projection
\begin{equation}\label{eq: classical_GS_proj}
    \bm{q}_k' = \bm{P}_{A_{k}^{\perp}} \bm{a}_k
\end{equation}
where $A_k = \operatorname{l.s} \left\{ \bm{a}_i \right\}_{i=1}^{k}$ and $\bm{P}_{A_{1}^{\perp}} = \Id_{n \times n}$ for convenience. A modified version of the Gram-Schmidt process performs the same orthogonal projection broken up as $k-1$ orthogonal projections of rank $n-1$ as so
\begin{align*}
    \bm{q}_k' & = \bm{P}_{A_{k}^{\perp}} \bm{a}_k                                                                                                                                                                          \\
              & = \left( \Id_{n \times n} - \bm{Q}_{k} \bm{Q}_{k}^{\ast} \right) \bm{a}_k                                                                                                                                  \\
              & = \left( \prod_{i=1}^{k-1} \left( \Id_{n \times n} - \bm{q}_i \bm{q}_i^{\ast} \right) \right)\bm{a}_k                                                                                                      \\
              & = \left( \Id_{n \times n} - \bm{q}_1 \bm{q}_1^{\ast} \right) \left( \Id_{n \times n} - \bm{q}_1 \bm{q}_1^{\ast} \right) \cdots \left( \Id_{n \times n} - \bm{q}_{k-1} \bm{q}_{k-1}^{\ast} \right) \bm{a}_k \\
              & = \bm{P}_{\bm{q}_{k}^{\perp}} \cdots \bm{P}_{\bm{q}_{1}^{\perp}} \bm{a}_k
\end{align*}

where $\bm{Q} = \left[ \bm{q}_1 , \bm{q}_2 , \ldots \bm{q}_k, 0 , \ldots , 0 \right]$. While its clear that $\bm{P}_{A_{k}^{\perp}} \bm{a} $ and $\bm{P}_{\bm{q}_{k}^{\perp}} \cdots \bm{P}_{\bm{q}_{1}^{\perp}} \bm{a}_k$ used for computing $\bm{q}_k'$ are algebraically equivalent, they differ arithmetically as the latter expression evaluates $\bm{q}_k'$ using the follow procedure

\begin{align*}
    \bm{q}_k^{(1)}             & = \bm{a}_k                                        \\
    \bm{q}_k^{(2)}             & = \bm{P}_{\bm{q}_{1}^{\perp}} \bm{q}_k^{(1)}      \\
    \bm{q}_k^{(3)}             & = \bm{P}_{\bm{q}_{2}^{\perp}} \bm{q}_k^{(2)}      \\
                               & \vdots                                            \\
    \bm{q}_k' = \bm{q}_k^{(k)} & = \bm{P}_{\bm{q}_{k-1}^{\perp}} \bm{q}_k^{(k-1)}.
\end{align*}

Applying projections sequentially in this manner produces smaller numerical errors. The modified Gram-Schmidt process \cite{TrefethenLloydN.LloydNicholas1997Nla/,DemmelJamesW1997Anla} is summarized in the following algorithm.

    {\centering
        \begin{minipage}{.85\linewidth}
            \begin{algorithm}[H]
                \caption{Modified Gram-Schmidt}
                \label{alg: Modified_Gram-Schmidt}
                \SetAlgoLined
                \DontPrintSemicolon
                \SetKwInOut{Input}{input}\SetKwInOut{Output}{output}

                \Input{A basis $\left\{ \bm{a}_i \right\}_{i=1}^{n}$.}
                \Output{An orthornormal basis $\left\{ \bm{q}_i \right\}_{i=1}^{n}$ such that $\operatorname{l.s} \left\{ \bm{a}_i \right\}_{i=1}^{n} = \operatorname{l.s} \left\{  \bm{q}_i \right\}_{i=1}^{n}$}
                \BlankLine
                \For{$k = 1$ \KwTo $n$}{
                    $\bm{q}_k' = \bm{a}_k$\;
                }
                \For{$k = 1$ \KwTo $n$}{
                    $r_{k,k} = \| \bm{q}_k' \|$\;
                    $\bm{q}_k = \bm{q}_k' / r_{k,k}$\;
                    \For{$i = k+1$ \KwTo $n$}{
                        $r_{i,k} = \langle \bm{q}_k, \bm{q}_i' \rangle$\;
                        $\bm{q}_i = \bm{q}_i - r_{i,k} \bm{q}_i$\;
                    }
                }
                \Return{$\left\{ \bm{q}_i \right\}_{i=1}^{n}$}
                \BlankLine
            \end{algorithm}
        \end{minipage}
        \par
    }