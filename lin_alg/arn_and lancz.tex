\subsection{Arnoldi and Lanczos Algorithm}\label{Section4.3}

As a quick reminder, we are in search of an iterative process to solve the linear system $\bm{A} \bm{x}^{\star} = \bm{b}$ where no explicit form of $\bm{A}$ is available and we may only rely on a routine that computes $\bm{A} \bm{v}$ for any $\bm{v}$ to extract information on $\bm{A}$. In \Cref{Section4.1} it was shown that $\bm{x}^{\star} \in \calK_{t_{\bm{r}_0}, \bm{A}} \left( \bm{A}, \bm{r}_0 \right)$. With many iterative methods, computing an exact value for $\bm{x}^{\star}$ is out the question with the view that $t_{\bm{r}_0, \bm{A}}$ is impractically large. We must instead resort to approximating $\bm{x}^{\star}$ by $\bm{x}_k$ for which $\bm{x}^{k} \in \calK_{k} \left( \bm{A}, \bm{r}_0 \right)$ where $k \ll t_{\bm{r}_0}$. To find an appropriate value for $\bm{x}_k$, a good start would be to find a basis $\calK_{k} \left( \bm{A}, \bm{r}_0 \right)$. \Cref{defe: krylov_subspace} indicated that $\left\{ \bm{A}^{i-1} \bm{r}_0 \right\}_{i=1}^{k}$ serves as a basis for $\calK_{k} \left( \bm{A}, \bm{r}_0 \right)$. However, for numerical reasons this is a poor choice of basis since each consecutive term becomes closer and closer to being linearly dependent. From now on, for more convenient notation we shall set $n = t_{\bm{r}_0, \bm{A}}$ so that $\bm{x}^{\star} \in \calK_{n} \left( \bm{A}, \bm{r}_0 \right)$. To search for a more appropriate basis let $\bm{K} \in \KK^{n \times n}$ be the invertible matrix
\[
    \bm{K} = \left[ \bm{r}_0 , \bm{A} \bm{r}_0, \ldots , \bm{A}^{n-1} \bm{r}_0 \right].
\]
Since $\bm{K}$ is invertible we can compute $\bm{c} = - \bm{K}^{-1} \bm{A}^{n} \bm{r}_0$ so that
\begin{align*}
    \bm{A} \bm{K} & = \left[ \bm{A} \bm{r}_0, \bm{A}^{2} \bm{r}_0, \ldots , \bm{A}^{n} \bm{r}_0 \right]                     \\
    \bm{A} \bm{K} & = \bm{K} \cdot \left[ \bm{e}_2, \bm{e}_3, \ldots , \bm{e}_n, - \bm{c}  \right] \triangleq \bm{K} \bm{C}
\end{align*}
or, in other terms
\[
    \bm{K}^{-1} \bm{A} \bm{K} = \bm{C} =
    \begin{bmatrix}
        0      & 0      & \cdots & 0      & -c_1   \\
        1      & 0      & \cdots & 0      & -c_2   \\
        0      & 1      & \cdots & 0      & \vdots \\
        \vdots & \vdots & \cdots & \vdots & \vdots \\
        0      & 0      & \cdots & 1      & -c_n
    \end{bmatrix}.
\]
Note here that $\bm{C}$ is upper Hessenberg. While this form is simple, it is of little practical use since the matrix $\bm{K}$ is very likely to be ill-conditioned. To remedy this we can replace $\bm{K}$ with an orthogonal matrix which spans the same space. These are exactly the properties that the $\bm{V}$ matrix offers in the $QR$-factorisation of $\bm{K}$. With this in mind let $\bm{K} = \bm{V} \bm{R}$ be the full $QR$-factorisation of $\bm{K}$. Then
\begin{align*}
    \bm{A} \bm{V} \bm{R} & = \bm{A} \bm{K}                    \\
    \bm{A} \bm{V}        & = \bm{A} \bm{K} \bm{R}^{-1}        \\
    \bm{A} \bm{V}        & = \bm{K} \bm{C} \bm{R}^{-1}        \\
    \bm{A} \bm{V}        & = \bm{V} \bm{R} \bm{C} \bm{R}^{-1} \\
    \bm{A} \bm{V}        & \triangleq \bm{V} \bm{H}.
\end{align*}
Since $\bm{R}$ and $\bm{R}^{-1}$ and both upper triangular and $\bm{C}$ is upper Hessenberg, $\bm{H}$ is also upper Hessenberg. This form provides us with a $\bm{V}$ such that the range of $\bm{V}$ is $\calK_{n} \left( \bm{A}, \bm{r}_0 \right)$ and
\begin{equation}\label{eq: QTAQ_eq_H}
    \bm{V}^{\intercal} \bm{A} \bm{V} = \bm{H}.
\end{equation}
Again, in practice, it may be very difficult to compute this entire expression forcing us to search for approximative alternatives. Consider \Cref{eq: QTAQ_eq_H} for which the only first $k$ columns of $\bm{V}$ have been computed. Let $\bm{V}_k = \left[ \bm{v}_1 , \bm{v}_2 , \ldots , \bm{v}_k \right]$ and $\bm{V}_u = \left[ \bm{v}_{k+1} , \bm{v}_{k+2} , \ldots , \bm{v}_{n} \right]$. Then
\begin{align*}
    \bm{V}^{\intercal} \bm{A} \bm{V}                                                         & = \bm{H} \\
    \left[ \bm{V}_k , \bm{V}_u \right]^{\intercal} \bm{A} \left[ \bm{V}_k , \bm{V}_u \right] & =
    \begin{bmatrix}
        \bm{H}_k     & \bm{H}_{u,k} \\
        \bm{H}_{k,u} & \bm{H}_{u}
    \end{bmatrix}                                                                           \\
    \begin{bmatrix}
        \bm{V}_{k}^{\intercal} \bm{A} \bm{V}_{k} & \bm{V}_{k}^{\intercal} \bm{A} \bm{V}_{u} \\
        \bm{V}_{u}^{\intercal} \bm{A} \bm{V}_{k} & \bm{V}_{u}^{\intercal} \bm{A} \bm{V}_{u}
    \end{bmatrix}
                                                                                             & =
    \begin{bmatrix}
        \bm{H}_k     & \bm{H}_{u,k} \\
        \bm{H}_{k,u} & \bm{H}_{u}
    \end{bmatrix}
\end{align*}
where $\bm{H}_k , \bm{H}_{u,k}, \bm{H}_{k,u}$ and $\bm{H}_u$ are the relevant sub matrices. This provides us with the equality
\begin{equation}\label{eq: QTkAQk_eq_Hk}
    \bm{V}_{k}^{\intercal} \bm{A} \bm{V}_{k} = \bm{H}_k
\end{equation}
noting that $\bm{H}_{k}$ is upper Hessenberg for the same reason that $\bm{H}$ is. We know that when $n = t_{\bm{r}_0, \bm{A}}$ we can find a $\bm{V} \in \KK^{n \times n}$ and $\bm{H} \in \KK^{n \times n}$ that satisfies $\bm{A} \bm{V} = \bm{V} \bm{H}$. However, in general, we may not be so fortunate in finding a $\bm{V}_{k} \in \KK^{n \times k}$ and $\bm{H}_{k} \in \KK^{n \times k}$ to satisfy $\bm{A} \bm{V}_{k} = \bm{V}_{k} \bm{H}_k$ for any $k < n$. Instead we can adjust this equality by adding an error $\bm{E}_k \in \KK^{n \times k}$ to force an equality. Our expression now becomes
\begin{equation}\label{eq: QTkAQk_eq_HkEk}
    \bm{V}_{k}^{\intercal} \bm{A} \bm{V}_{k} = \bm{H}_k + \bm{E}_k.
\end{equation}
A judicious choice of $\bm{E}_k$ must be made to also retain equality in \Cref{eq: QTkAQk_eq_Hk}, meaning $\bm{V}_{k}^{\intercal} \bm{E}_k = \bm{0}$. Since $\left\{ \bm{v}_i \right\}_{i=1}^{k}$ forms an orthornormal basis for $\calK_{n} \left( \bm{A}, \bm{r}_0 \right)$, consider the following choice of $\bm{E}_k$,
\[
    \bm{E}_k = \bm{v}_{k+1} \bm{h}_{k}^{\intercal}
\]
where $\bm{h}_k$ is any vector in $\KK^{k}$. Notice that
\[
    \bm{V}_{k}^{\intercal} \bm{E} = \bm{V}^{\intercal} \left( \bm{v}_{k+1} \bm{h}_k \right) = \left( \bm{V}^{\intercal} \bm{v}_{k+1} \right) \bm{h}_{k}^{\intercal} = \bm{0}.
\]
Since this holds for any $\bm{h}_k \in \KK^{k}$, to preserve sparsity and to keep this form as simple as possible we can set $\bm{h}_k = \left[ 0,0, \ldots , h_{k+1,k} \right]^{\intercal}$. This means $\bm{A} \bm{V}_k$ can be written as
\begin{equation}\label{eq: QTkAQk_eq_Hk_p_qkhk}
    \bm{A} \bm{V}_k =  \bm{V}_k \bm{H}_k + \bm{v}_{k+1} \bm{h}_{k}^{\intercal}
\end{equation}
where
\[
    \bm{V}_k \bm{H}_k =
    \left[ \bm{v}_1 , \bm{v}_2 , \ldots , \bm{v}_k \right]
    \begin{bmatrix}
        h_{1,1} & \cdots & \cdots & \cdots    & h_{1,k} \\
        h_{2,1} & \cdots & \cdots & \cdots    & \vdots  \\
        0       & \ddots & \ddots & \ddots    & \vdots  \\
        \vdots  & \ddots & \ddots & \ddots    & \vdots  \\
        0       & \cdots & 0      & h_{k,k-1} & h_{k,k} \\
        0       & \cdots & 0      & 0         & 0
    \end{bmatrix}.
\]
Equating the $j^{th}$ columns of \Cref{eq: QTkAQk_eq_Hk_p_qkhk} yields
\[
    \bm{A} \bm{v}_j = \sum_{i=1}^{j+1} h_{i,j} \bm{v}_{i}.
\]
Again since $\left\{ \bm{v}_i \right\}_{i=1}^{n}$ form an orthornormal basis, multiplying both sides by $\bm{v}_m$ for $1 \leq m \leq j$ gives
\[
    \bm{v}_m^{\intercal} \bm{A} \bm{v}_j = \sum_{i=1}^{j+1} h_{i,j} \bm{v}_m^{\intercal} \bm{v}_{i} = h_{m,j}
\]
and so
\begin{equation}\label{eq: arn_eq_1}
    h_{j+1,j} \bm{v}_{j+1} = \bm{A} \bm{v}_j - \sum_{i=1}^{j} h_{i,j} \bm{v}_{i}.
\end{equation}
From \Cref{eq: arn_eq_1} we find that $\bm{v}_{j+1}$ can be computed using a recurrance involving its previous Krylov factors. This bears a striking resemblance to \Cref{eq: comp_orth_basis} having a virtually identical setup to computing an orthornormal basis using the modified Gram-Schmidt process (\Cref{alg: Modified_Gram-Schmidt}). As such, values for $\bm{v}_{j+1}$ and $h_{j+1,j}$ can be evaluted using a procedure very similar to the modified Gram-Schmidt process better known as the Arnoldi Algorithm \cite{TrefethenLloydN.LloydNicholas1997Nla/,DemmelJamesW1997Anla}, presented in \Cref{alg: Arnoldi_Algorithm}.

{\centering
\begin{minipage}{.85\linewidth}
    \begin{algorithm}[H]
        \caption{Arnoldi Algorithm}
        \label{alg: Arnoldi_Algorithm}
        \SetAlgoLined
        \DontPrintSemicolon
        \SetKwInOut{Input}{input}\SetKwInOut{Output}{output}

        \Input{$\bm{A}, \bm{r}_0$ and $k$, the number of columns of $\bm{V}$ to compute.}
        \Output{$\bm{V}_k , \bm{H}_k$.}
        \BlankLine
        $\bm{v}_1 = \bm{r}_0 / \| \bm{r}_0 \|$\;
        \For{$j = 1$ \KwTo $k$}{
            $\bm{z} = \bm{A} \bm{v}_j$\;
            \For{$i = 1$ \KwTo $j$}{
                $h_{i,j} = \langle \bm{v}_{i} , \bm{z} \rangle$\;
                $\bm{z} = \bm{z} - h_{i,j} \bm{v}_{i}$\;
            }
            $h_{j+1,j} = \| \bm{z} \|$\;
            \If{$h_{j+1,j} = 0$}{
                \Return{$\bm{V}_k , \bm{H}_k$}
            }
            $\bm{v}_{j+1} = \bm{z} / h_{j+1,j}$\;
        }
        \Return{$\bm{V}_k , \bm{H}_k$}
        \BlankLine
    \end{algorithm}
\end{minipage}
\par
}

When $\bm{A}$ is symmertic then $\bm{H} = \bm{T}$ becomes a tridiagonal matrix, simplifying a large amount of the Arnoldi algorithm since the matrix elements from $\bm{T}$ can be written as
\[
    \bm{T} =
    \begin{bmatrix}
        \alpha_1 & \beta_1 &        &             &             \\
        \beta_1  & \ddots  & \ddots &             &             \\
                 & \ddots  & \ddots & \ddots      &             \\
                 &         & \ddots & \ddots      & \beta_{n-1} \\
                 &         &        & \beta_{n-1} & \alpha_{n}
    \end{bmatrix}.
\]
As before, equating the $j^{th}$ columns of $\bm{A} \bm{V} = \bm{V} \bm{T}$ yields
\begin{equation}\label{eq: lancz_orth_basis}
    \bm{A} \bm{v}_{j} = \beta_{j-1} \bm{v}_{j-1} + \alpha_{j} \bm{v}_j + \beta_j \bm{v}_{j+1}.
\end{equation}
Again since $\left\{ \bm{v}_{i} \right\}_{i=1}^{n}$ form an orthornormal basis, multiplying both sides of \Cref{eq: lancz_orth_basis} by $\bm{v}_j$ gives $\bm{v}_j \bm{A} \bm{v}_j = \alpha_j$. A simplified version of the Arnoldi algorithm can be devised to compute $\left\{ \bm{v}_{i} \right\}_{i=1}^{n}$ and $\bm{T}$ for symmetric matrices known as the Lanczos algorithm \cite{DemmelJamesW1997Anla}. The Lanczos algorithm is presented in \Cref{alg: Lanczos_Algorithm}.

{\centering
\begin{minipage}{.85\linewidth}
    \begin{algorithm}[H]
        \caption{Lanczos Algorithm}
        \label{alg: Lanczos_Algorithm}
        \SetAlgoLined
        \DontPrintSemicolon
        \SetKwInOut{Input}{input}\SetKwInOut{Output}{output}

        \Input{$\bm{A}, \bm{r}_0$ and $k$, the number of columns of $\bm{V}$ to compute.}
        \Output{$\bm{V}_k , \bm{T}_k$.}
        \BlankLine
        $\bm{v}_1 = \bm{r}_0 / \| \bm{r}_0 \|$, $\beta_0 = 0$, $\bm{v}_0 = 0$\;
        \For{$j = 1$ \KwTo $k$}{
            $\bm{z} = \bm{A} \bm{v}_j$\;
            $\alpha_j = \langle \bm{v}_{j}, \bm{z} \rangle$\;
            $\bm{z} = \bm{z} - \alpha_j \bm{v}_{j} - \beta_{j-1} \bm{v}_{j-1}$\;
            $\beta_j = \| z \|$\;
            \If{$\beta_{j} = 0$}{
                \Return{$\bm{V}_k , \bm{T}_k$}
            }
            $\bm{v}_{j+1} = \bm{z} / \beta_{j}$\;
        }
        \Return{$\bm{V}_k , \bm{T}_k$}
        \BlankLine
    \end{algorithm}
\end{minipage}
\par
}

For the Lanczos algorithm, \Cref{eq: QTkAQk_eq_Hk_p_qkhk} can be re-written in the a more compact form as
\begin{equation}\label{eq: AVk_eq_VkTk1k}
    \bm{A} \bm{V}_{k} \triangleq \bm{V}_{k} \bm{T}_{k+1,k}
\end{equation}
where $\bm{T}_{k+1,k} = \bm{T}_{k} + \bm{v}_{k+1} \bm{t}_{k}^{\intercal}$.