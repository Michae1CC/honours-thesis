\section{Krylov Subspace Methods}\label{Chapter4}
In this section we will focus on how iterative methods, in particular a class of iterative methods called Krylov Subspace methods, may be used to solve a linear system $\bm{A} \bm{x} = \bm{b}$. While non-iterative methods exist to solve such systems virtually all of them carry an unwieldy runtime of $\mathcal{O} \left( n^3 \right)$ for a system of $n$ parameters. Even for current computer systems, this renders many common matrix problems untractable. Consequently the focus of solving linear systems has shifted towards iterative methods. While iterative methods typically demand certain structural properties of the matrices, such as symmetry and positive definiteness, this generally is not a problem since the majority of large matrix problems that, by mature, endow these systems with the desired properties. For example, in the context of this paper the Gram matrices used to solve linear systems in Gaussian Processes possess both symmetry and positive definiteness. There are also a number of other properties of iterative methods which make them rather attractive to users. To start, iterative Krylov subspace methods are guranteed to converge to an exact solution within a finite number of iterations and even if the method is prematurely stopped before reaching an exact solution, the approximation obtained on the final iteration will in some sense be a good enough estimate of our exact solution. Furthermore, unlike most non-iterative methods, Krylov subspace methods do not require an explicit form of the matrix $\bm{A}$ and instead only requires some routine or process for computing $\bm{A} \bm{x}$.


\subsection{Krylov Subspaces}\label{Section4.1}

We will motivate the Krylov subspaces by observing their usefullness in solving linear systems. To this end, consider the problem of solving the linear system
\begin{equation}\label{eq: lin_sys_1}
    \bm{A} \bm{x^{\star}} = \bm{b}
\end{equation}
where no explicit form of $\bm{A}$ is available and instead one must draw information from $\bm{A}$ solely through a routine that can evaluate $\bm{A} \bm{v}$ for any $\bm{v}$. How could this routine be utilized in such a manner to provide with a solution to equation \ref{eq: lin_sys_1}? Before answering this, consider the following theorem

\begin{thm} \label{theorem: invert_mat_norm}
    For $\bm{A} \in \CC^{n \times n}$ if $\| \bm{A} \| = q < 1$ then $\Id - \bm{A}$ is invertible and its inverse admits the following representation
    \begin{equation*} \label{eq: convg-of-part}
        \left( \Id - \bm{A} \right)^{-1} = \sum_{k=0}^{\infty} \bm{A}^k
    \end{equation*}
    \cite{BerezanskyMakarovich1996FaV1}*{page 287}.
\end{thm}

\begin{proof}
    Let us show that the sequence $\bm{S}_{n} = \sum_{k=0}^{n} \bm{A}^k$ (where $n$ is some natural number) of partial sums of the series on the right-hand side of \Cref{eq: convg-of-part} is fundamental in $\CC^{n \times n}$. Indeed, by using the fact that $\| \bm{A B} \| \leq \| \bm{A} \| \| \bm{B} \|$ for any $\bm{A} , \bm{B} \in \CC^{n \times n}$, we conclude that
    \begin{equation*}
        \| \bm{S}_{n+p} - \bm{S}_{n} \| \leq \| \bm{A}^{n+1} \| + \cdots + \| \bm{A}^{n+p} \| \leq {q}^{n+1} + \cdots + {q}^{n+p}
    \end{equation*}
    for all $n,p \in \NN$. Since $q < 1$, this yields the desired result. Since $\CC^{n \times n}$ is a Banach space, the sequence $\left\{ \bm{S}_{n} \right\}_{n}^{\infty}$ uniformly converges to an operator $\bm{S} \in \CC^{n \times n}$. Let us show that $\bm{S} \left( \Id_{n \times n} - \bm{A} \right) = \left( \Id_{n \times n} - \bm{A} \right) \bm{S} = \Id_{n \times n}$. To this end, we note that
    \[
        \| \left( \Id_{n \times n} - \bm{A} \right) \bm{S}_n - \left( \Id_{n \times n} - \bm{A} \right) \bm{S} \| \leq \|  \Id_{n \times n} - \bm{A} \| \cdot \| \bm{S} - \bm{S}_{n} \| \to 0
    \]
    as $n \to \infty$. Therefore, it suffices to show that the sequence $\left\{ \left( \Id_{n \times n} - \bm{A} \right) \bm{S}_n \right\}_{n=1}^{\infty}$ uniformly converges to the identity operator. We have
    \[
        \bm{S}_{n} \left( \Id_{n \times n} - \bm{A} \right) = \left( \Id_{n \times n} - \bm{A} \right) \bm{S}_{n} = \sum_{k=0}^{n} \bm{A}^k - \sum_{k=1}^{n+1} \bm{A}^k = \Id_{n \times n} - \bm{A}^{n+1},
    \]
    that is, $\| \left( \Id_{n \times n} - \bm{A} \right) \bm{S}_{n} - \Id_{n \times n} \| = \| \bm{A}^{n+1} \| \leq q^{n+1}$, which vanishes as $n \to \infty$. Thus $\bm{S} = \left( \Id_{n \times n} - \bm{A} \right)^{-1}$.
\end{proof}

Consider a matrix for which $\| \bm{A} \| < 2$, it follows that $\| \Id_{n \times n} - \bm{A} \| < 1$ meaning $\Id_{n \times n} - \left( \Id_{n \times n} - \bm{A} \right)$ is invertible and $\bm{A}^{-1} = \left( \Id_{n \times n} - \left( \Id_{n \times n} - \bm{A} \right) \right)^{-1} = \sum_{k=0}^{\infty} \left( \Id_{n \times n} - \bm{A} \right)^{k}$. Thinking back to equation \ref{eq: lin_sys_1} for any $\bm{x}_0 \in \RR^{n}$ we have
\begin{align*}
    \bm{x^{\star}} & = \bm{A}^{-1} \bm{b} = \bm{A}^{-1} \left( \bm{A} \bm{x^{\star}} - \bm{A} \bm{x_0} + \bm{A} \bm{x_0} \right) \\
                   & = \bm{x_0} + \bm{A}^{-1} \bm{r_0}                                                                           \\
                   & = \bm{x_0} + \sum_{k=0}^{\infty} \left( \Id_{n \times n} - \bm{A} \right)^k \bm{r_0}
\end{align*}
where $\bm{r_0} = \bm{A} \bm{x^{\star}} - \bm{A} \bm{x_0}$. A natural question that arises is that can we find a closed form solution of the above equation? To answer this question we need to enlist the help of the Cayley-Hamilton theorem.
\begin{thm}[Cayley-Hamilton] \label{theorem: cayley_amilton}
    Let $p_n \left( \lambda \right) = \sum_{i=0}^{n} c_i \lambda^{i}$ be the characteristic polynomial of the matrix $\bm{A} \in \CC^{n \times n}$, then $p_n \left( \bm{A} \right) = \bm{0}$. {\color{red} \textbf{THIS NEEDS A CITATION}}
\end{thm}
The Cayley-Hamilton theorem implies that
\begin{align*}
    \bm{0}      & = c_0 + c_1 \bm{A} + \ldots + c_{n-1} \bm{A}^{n-1} + c_{n} \bm{A}^{n}                      \\
    \bm{0}      & = \bm{A}^{-1} c_0 + c_1 + \ldots + c_{n-1} \bm{A}^{n-2} + c_{n} \bm{A}^{n-1}               \\
    \bm{A}^{-1} & = \alpha_1 \Id_{n \times n} + \ldots + \alpha_{n-1} \bm{A}^{n-2} + \alpha_{n} \bm{A}^{n-1}
\end{align*}
where $\alpha_i = -c_i / c_0$. The above demonstrates that $\bm{A}^{-1}$ can be represented as a matrix polynomial of degree $n-1$. This means that $\sum_{k=0}^{\infty} \left( \Id - \bm{A} \right)^k$ indeed possess a closed form solution namely
\begin{equation} \label{eq: x_ast_via_cayley}
    \bm{x^{\star}} = \bm{x}_0 + \bm{A}^{-1} \bm{r}_0 = \bm{x}_0 + \alpha_1 \bm{r}_0 + \ldots + \alpha_{n-1} \bm{A}^{n-2} \bm{r}_0 + \alpha_{n} \bm{A}^{n-1} \bm{r}_0.
\end{equation}
This also shows that $\bm{x^{\star}} \in \bm{x}_0 + \operatorname{l.s} \left\{ \bm{r}_0, \bm{A} \bm{r}_0, \bm{A}^2 \bm{r}_0, \ldots , \bm{A}^{n-1} \bm{r}_0 \right\}$. One idea for finding a solution to equation \ref{eq: lin_sys_1} is to use our routine for evaluting $\bm{A} \bm{v}$ to iteratively compute new basis elements for the space generated by $\left\{ \bm{r}_0, \bm{A} \bm{r}_0, \bm{A}^2 \bm{r}_0, \ldots , \bm{A}^{n-1} \bm{r}_0 \right\}$ and at each step carefully choosing a $\bm{x_k}$ such that $\bm{x_k}$ approaches $\bm{x^{\star}}$, in some form. The subspace constructed using this technique is so important that is has its own name.
\begin{defe}[Krylov Subspace] \label{defe: krylov_subspace}
    The Krylov Subspace of order $k$ generated by the matrix $\bm{A} \in \CC^{n \times n}$ and the vector $\bm{v} \in \CC^{n}$ is defined as
    \[
        \calK_{k} \left( \bm{A},\bm{v} \right) = \operatorname{l.s} \left\{ \bm{r}_0, \bm{A} \bm{r}_0, \bm{A}^2 \bm{r}_0, \ldots , \bm{A}^{k-1} \bm{r}_0 \right\}
    \]
    for $k \geq 1$ and $\calK_{0} \left( \bm{A},\bm{v} \right) = \left\{ \bm{0} \right\}$ \cite{TrefethenLloydN.LloydNicholas1997Nla/}*{page 245}.
\end{defe}
For the purposes of solving equation \ref{eq: lin_sys_1} it is of much interest to understand how $\calK_{k} \left( \bm{A},\bm{v} \right)$ grows for larger and larger $k$ since a solution for equation \ref{eq: lin_sys_1} will be present in a Krylov Subspace that cannot be grown any larger, according to equation \ref{eq: x_ast_via_cayley}. In other words, an exact solution can be constructed once we have extracted all the information from $\bm{A}$ through multiplication of $\bm{r_0}$. The following theorem provides information on how exactly the Krylov Subspace grows as $k$ increases.
\begin{thm} \label{theorem: grade_of_v}
    There is a positive called the grade of $\bm{v}$ with respect to $\bm{A}$, denoted $t_{\bm{v}, \bm{A}}$, where
    \[
        \operatorname{dim} \left( \calK_{k} \left( \bm{A} , \bm{v} \right) \right) = \left\{
        \begin{matrix}
            k,                  & k \leq t_{\bm{v}, \bm{A}} \\
            t_{\bm{v}, \bm{A}}, & k \geq t_{\bm{v}, \bm{A}}
        \end{matrix}
        \right.
    \]
\end{thm}
Theorem \ref{theorem: grade_of_v} essentially tells us for $k \leq t_{\bm{v}, \bm{A}}$ that $\bm{A}^k \bm{v}$ is linearly independent to $\bm{A}^i \bm{v}$ for $0 \leq i \leq k-1$ meaning $\left\{ \bm{v}, \bm{A} \bm{v}, \bm{A}^2 \bm{v}, \ldots , \bm{A}^{k-1} \bm{v} \right\}$ serves as a basis for $\calK_{k} \left( \bm{A},\bm{v} \right)$ and $\calK_{k-1} \left( \bm{A},\bm{v} \right) \subsetneq \calK_{k} \left( \bm{A},\bm{v} \right)$. Conversely, any new vectors formed beyond $t_{\bm{v}, \bm{A}}$ will be linearly independent meaning $\calK_{k} \left( \bm{A},\bm{v} \right) \subsetneq \calK_{k+1} \left( \bm{A},\bm{v} \right)$ for $k \geq t_{\bm{v}, \bm{A}}$. While $t_{\bm{v}, \bm{A}}$ obviously plays a central role in determining a suitable basis whose span contains $\bm{A}^{-1} \bm{b}$, its importance is made abundantly clear in the following corollary.
\begin{cor} \label{theorem: grade_as_min}
    \[
        t_{\bm{v}, \bm{A}} = \min \left\{k \mid \bm{A}^{-1} \bm{v} \in \calK_{k} \left( \bm{A},\bm{v} \right) \right\}
    \]
\end{cor}
\begin{proof}
    Recall from Cayley-Hamilton (theorem \ref{theorem: cayley_amilton}) that
    \[
        \bm{A}^{-1} \bm{v} = \sum_{i=0}^{n-1} \alpha_{i} \bm{A}^{i} \bm{v}
    \]
    But since $\calK_{k} \left( \bm{A},\bm{v} \right) = \calK_{k+1} \left( \bm{A},\bm{v} \right)$ for $k \geq t_{\bm{v}, \bm{A}}$
    \[
        \bm{A}^{-1} \bm{v} = \sum_{i=0}^{t-1} \beta_{i} \bm{A}^{i} \bm{v}
    \]
    meaing $\bm{A}^{-1} \bm{v} \in \calK_{k} \left( \bm{A},\bm{v} \right)$ for $k \geq t_{\bm{v}, \bm{A}}$. Suppose for the sake of contradiction that this also holds for $k = t_{\bm{v}, \bm{A}} - 1$, that is, $\bm{A}^{-1} \bm{v} = \sum_{i=0}^{t-2} \gamma_{i} \bm{A}^{i} \bm{v}$. However, this gives
    \[
        \bm{v} = \sum_{i=0}^{t-2} \gamma_{i} \bm{A}^{i+1} \bm{v} = \sum_{i=0}^{t-1} \gamma_{i-1} \bm{A}^{i} \bm{v}
    \]
    implying $\left\{ \bm{v}, \bm{A} \bm{v}, \bm{A}^2 \bm{v}, \ldots , \bm{A}^{t-1} \bm{v} \right\}$ are linearly dependent which means that $\operatorname{dim} \left( \calK_{k} \left( \bm{A} , \bm{v} \right) \right) < t$, which provides us with our contradiction.
\end{proof}
This allows us to make a much stronger statement on the where abouts of $\bm{x^{\star}}$ in relation to the Krylov Subspaces.
\begin{cor} \label{theorem: sol_in_krylov}
    For any $\bm{x_0}$ we have
    \[
        \bm{x^{\star}} \in \bm{x_0} + \calK_{t_{\bm{r_0}, \bm{A}}} \left( \bm{A},\bm{r_0} \right)
    \]
    where $\bm{r_0} = \bm{b} - \bm{A} \bm{x_0}$.
\end{cor}

\subsection{Gram-Schmidt Process and QR factorisations}\label{Section4.2}

Many areas of linear algebra involving studing the column space of matrices. The $QR$ factorisation provides us with a powerful tool to better understand the column space of a matrix as well as serving as an important factorisation mechanism for many numerical methods. Suppose that a matrix $\bm{A} = \left[ \bm{a}_1 , \bm{a}_2 , \ldots , \bm{a}_n \right] \in \KK^{n \times n}$ has full rank. The idea of a $QR$ factorisation is to find an alternative orthornormal basis for $\left( \bm{a}_i \right)_{i=1}^{n}$, say $\left( \bm{q}_i \right)_{i=1}^{n}$, and to somehow relate the original matrix $\bm{A}$ to a new matrix whose columns are $\left( \bm{q}_i \right)_{i=1}^{n}$. Consider the following procedure that allows us to find an orthornormal basis $\left( \bm{q}_i \right)_{i=1}^{n}$ for which $\operatorname{l.s} \left\{ \left( \bm{a}_i \right)_{i=1}^{n} \right\} = \operatorname{l.s} \left\{ \left( \bm{q}_i \right)_{i=1}^{n} \right\}$. First set $\bm{q}_1 = \frac{\bm{a}_1}{\| \bm{a}_i \|}$, clearly $\operatorname{l.s} \left\{ \bm{a}_1 \right\} = \operatorname{l.s} \left\{ \bm{q}_1 \right\}$. Next, construct a vector $\bm{q}_2' = \bm{a}_2 - r_{1,2} \cdot \bm{q}_1$ so that $\bm{q}_2' \perp \bm{q}_1$. This means
\begin{align*}
    0       & = \langle \bm{q}_1, \bm{q}_2' \rangle                                                   \\
    0       & = \langle \bm{q}_1, \bm{a}_2 - r_{1,2} \cdot \bm{q}_1 \rangle                           \\
    0       & = \langle \bm{q}_1, \bm{a}_2 \rangle - r_{1,2} \cdot \langle \bm{q}_1, \bm{q}_1 \rangle \\
    r_{1,2} & = \langle \bm{q}_1, \bm{a}_2 \rangle
\end{align*}
Since $\bm{q}_2'$ may not be a unit vector we set $\bm{q}_2 = \frac{\bm{q}_2'}{\| \bm{q}_2' \|}$ where $\operatorname{l.s} \left( \left\{ \bm{a}_1, \bm{a}_2 \right\} \right) = \operatorname{l.s} \left( \left\{ \bm{q}_1, \bm{q}_2 \right\} \right)$. Continuing the vector $\bm{q}_3'$ is constructed so that
\[
    \bm{q}_3' = \bm{a}_3 - \bm{r}_{1,3} \bm{q}_1 - \bm{r}_{2,3} \bm{q}_2
\]
are chosen so that $\bm{q}_3'$ is orthogonal to both $\bm{q}_2$ and $\bm{q}_1$. This amounts to setting $r_{1,3} = \langle \bm{q}_1, \bm{a}_3 \rangle$ and $r_{2,3} = \langle \bm{q}_2, \bm{a}_{3} \rangle$. Similarly, $\bm{q}_3'$ is normalized so that $\bm{q}_3 = \frac{\bm{q}_3'}{\| \bm{q}_3' \|}$ and $\operatorname{l.s} \left( \left\{ \bm{a}_1, \bm{a}_2, \bm{a}_3 \right\} \right) = \operatorname{l.s} \left( \left\{ \bm{q}_1, \bm{q}_2, \bm{q}_3 \right\} \right)$. Continuing in this fashion the $k^{th}$ vector in our orthornormal basis is computed as
\begin{equation}\label{eq: comp_orth_basis}
    \bm{q}_k = \frac{\bm{a}_k - \sum_{i=1}^{k-1} r_{i,k} \cdot \bm{q}_i}{r_{k,k}}
\end{equation}
where $r_{i,k} = \langle \bm{q}_i, \bm{a}_k \rangle$, $r_{k,k} = \| \bm{a}_k - \sum_{i=1}^{k-1} r_{i,k} \cdot \bm{q}_i \|$ and $\operatorname{l.s} \left( \left\{ \bm{a}_1, \bm{a}_2, \ldots , \bm{a}_k \right\} \right) = \operatorname{l.s} \left( \left\{ \bm{q}_1, \bm{q}_2, \ldots , \bm{q}_k \right\} \right)$. This procedure is famiously known as the Gram-Schmidt process \cite{BerezanskyMakarovich1996FaV1,TrefethenLloydN.LloydNicholas1997Nla/,DemmelJamesW1997Anla} and is summarized in the following algorithm.

    % https://tex.stackexchange.com/questions/463359/algorithm-inside-a-tcolorbox-how-to-put-a-label-to-the-algorithm-but-the-captio
    % http://cfrgtkky.blogspot.com/2018/12/algorithm-inside-tcolorbox-how-to-put.html
    % {\centering
    %     \begin{minipage}{.85\linewidth}
    %         \begin{tcolorbox}[colback=white!100,colframe=black!100]
    %             \begin{algorithm}[H]
    %                 \caption{Classical Gram-Schmidt}
    % \label{alg: Classical_Gram-Schmidt}
    % \SetAlgoLined
    % \DontPrintSemicolon
    % \SetKwInOut{Input}{input}\SetKwInOut{Output}{output}

    % \Input{A basis $\left( \bm{a}_i \right)_{i=1}^{n}$.}
    % \Output{An orthornormal basis $\left( \bm{q}_i \right)_{i=1}^{n}$ such that $\operatorname{l.s} \left\{ \left( \bm{a}_i \right)_{i=1}^{n} \right\} = \operatorname{l.s} \left\{ \left( \bm{q}_i \right)_{i=1}^{n} \right\}$}
    % \BlankLine
    % \For{$k = 1$ \KwTo $n$}{
    %     $\bm{q}_k' = \bm{a}_k$\;
    %     \For{$i = 1$ \KwTo $k-1$}{
    %         $r_{i,k} = \langle \bm{q}_i, \bm{a}_k \rangle$\;
    %         $\bm{q}_k' = \bm{q}_k' - r_{i,k} \bm{q}_i$\;
    %     }
    %     $r_{k,k} = \| \bm{q}_k' \|$\;
    %     $\bm{q}_k = \bm{q}_k' / r_{k,k}$\;
    % }
    % \Return{$\left( \bm{q}_i \right)_{i=1}^{n}$}
    % \BlankLine
    %             \end{algorithm}
    %         \end{tcolorbox}
    %     \end{minipage}
    %     \par
    % }


    {\centering
        \begin{minipage}{.85\linewidth}
            \begin{algorithm}[H]
                \caption{Classical Gram-Schmidt}
                \label{alg: Classical_Gram-Schmidt}
                \SetAlgoLined
                \DontPrintSemicolon
                \SetKwInOut{Input}{input}\SetKwInOut{Output}{output}

                \Input{A basis $\left( \bm{a}_i \right)_{i=1}^{n}$.}
                \Output{An orthornormal basis $\left( \bm{q}_i \right)_{i=1}^{n}$ such that $\operatorname{l.s} \left\{ \left( \bm{a}_i \right)_{i=1}^{n} \right\} = \operatorname{l.s} \left\{ \left( \bm{q}_i \right)_{i=1}^{n} \right\}$}
                \BlankLine
                \For{$k = 1$ \KwTo $n$}{
                    $\bm{q}_k' = \bm{a}_k$\;
                    \For{$i = 1$ \KwTo $k-1$}{
                        $r_{i,k} = \langle \bm{q}_i, \bm{a}_k \rangle$\;
                        $\bm{q}_k' = \bm{q}_k' - r_{i,k} \bm{q}_i$\;
                    }
                    $r_{k,k} = \| \bm{q}_k' \|$\;
                    $\bm{q}_k = \bm{q}_k' / r_{k,k}$\;
                }
                \Return{$\left( \bm{q}_i \right)_{i=1}^{n}$}
                \BlankLine
            \end{algorithm}
        \end{minipage}
        \par
    }

Relating the column space of $\bm{A}$ to the orthornormal basis $\left( \bm{q}_{i} \right)_{i=1}^{n}$ in a matrix form
\[
    \left[ \bm{a}_1 , \bm{a}_2 , \ldots \bm{a}_n \right] =
    \left[ \bm{q}_1 , \bm{q}_2 , \ldots \bm{q}_n \right]
    \begin{bmatrix}
        r_{1,1} & r_{1,2} & \cdots & r_{1,n} \\
                & r_{2,2} &        & \vdots  \\
                &         & \ddots & \vdots  \\
                &         &        & r_{n,n}
    \end{bmatrix}
\]
or more succinctly
\begin{equation}\label{eq: QR_factorisation}
    \bm{A} = \bm{Q} \bm{R}
\end{equation}
where $\bm{Q} = \left[ \bm{q}_1 , \bm{q}_2 , \ldots \bm{q}_n \right]$ and $\left( \bm{R} \right)_{i,j} = r_{i,j}$ for $i \leq j$ and $\left( \bm{R} \right)_{i,j} = 0$ for $i > j$. This is exactly the $QR$ factorisation for a full rank matrix. Note that $\operatorname{Range} \left( \bm{A} \right) = \operatorname{Range} \left( \bm{Q} \right)$. In general, any square matrix  $\bm{A} \in \KK^{m \times n}$ may be decomposed as $\bm{A} = \bm{Q} \bm{R}$ where $\bm{Q} \in \KK^{m \times m}$ is an orthogonal matrix and $\bm{R} \in \KK^{m \times n}$ is an upper triangular matrix. This is known as a full $QR$ factorisation. Since bottom $(m-n)$ rows of this $\bm{R}$ consists entirely of zeros, it is often useful to partition the full $QR$ factorisation in the following manner to shed vacuous entries
\[
    \bm{A} = \bm{Q} \bm{R} = \bm{Q}
    \begin{bmatrix}
        \hat{\bm{R}} \\
        \bm{0}_{(m-n) \times n}
    \end{bmatrix}
    =
    \begin{bmatrix}
        \hat{\bm{Q}} & \bm{Q}'
    \end{bmatrix}
    \begin{bmatrix}
        \hat{\bm{R}} \\
        \bm{0}_{(m-n) \times n}
    \end{bmatrix}
    = \hat{\bm{Q}} \hat{\bm{R}}.
\]
This alternate decomposition is called the reduced (or somtimes the thin) $QR$ factorization. We shall state the following two theorems on the $QR$ factorization are stated without proof.

\begin{thm} \label{theorem: QR_general_existence}
    Every $\bm{A} \in \KK^{m \times n}, \; (m \geq n)$ has a full $QR$ factorisation, hence also a reduced $QR$ factorisation.
    \cite{TrefethenLloydN.LloydNicholas1997Nla/}
\end{thm}

\begin{thm} \label{theorem: QR_full_rank_unique}
    Each $\bm{A} \in \KK^{m \times n}, \; (m \geq n)$ of full rank has a unique reduced $QR$ factorisation $\bm{A} = \hat{\bm{Q}} \hat{\bm{R}}$ with $r_{k,k} > 0$.
    \cite{TrefethenLloydN.LloydNicholas1997Nla/}
\end{thm}

In practice the classical Gram-Schmidt process described in algorithm \ref{alg: Classical_Gram-Schmidt} is rarely used as the procedure becomes numerically unstable if $\left( \bm{a}_i \right)_{i=1}^{n}$ are almost linearly dependent. Before looking for ways to resolve these numerical instabilities a quick recap of projectors has been devised. A square matrix $\bm{P}_{G}$ acting on a Hilbert space $H$ that sends $\bm{x} \in H$ to its projection onto a subspace $G$ is called the projector onto $G$. If $\left( \bm{q}_k \right)_{k=1}^{m}$ is an orthornormal basis in $G$ then
\[
    \bm{P}_{G} = \bm{Q} \bm{Q}^{\ast}
\]
where $\bm{Q} = \left[ \bm{q}_1 , \bm{q}_2 , \ldots \bm{q}_m, 0 , \ldots , 0 \right] \in \KK^{n \times n}$. A special class of projectors which isolates the components of a given vector onto a one dimensional subspace spanned by a single unit vector $\bm{q}$ called a rank one orthogonal projector, denoted as $\bm{P}_{q}$. Each $k$ in the classical Gram-Schmidt process $\bm{q}_k'$ using the following orthogonal projection
\begin{equation}\label{eq: classical_GS_proj}
    \bm{q}_k' = \bm{P}_{A_{k}^{\perp}} \bm{a}_k
\end{equation}
where $A_k = \operatorname{l.s} \left\{ \bm{a}_i \right\}_{i=1}^{k}$ and $\bm{P}_{A_{1}^{\perp}} = \Id$ for convenience. A modified version of the Gram-Schmidt process performs the same orthogonal projection broken up as $k-1$ orthogonal projections of rank $n-1$ as so
\begin{align*}
    \bm{q}_k' & = \bm{P}_{A_{k}^{\perp}} \bm{a}_k                                                                                                                                   \\
              & = \left( \Id - \bm{Q}_{k} \bm{Q}_{k}^{\ast} \right) \bm{a}_k                                                                                                        \\
              & = \left( \prod_{i=1}^{k-1} \left( \Id - \bm{q}_i \bm{q}_i^{\ast} \right) \right)\bm{a}_k                                                                            \\
              & = \left( \Id - \bm{q}_1 \bm{q}_1^{\ast} \right) \left( \Id - \bm{q}_1 \bm{q}_1^{\ast} \right) \cdots \left( \Id - \bm{q}_{k-1} \bm{q}_{k-1}^{\ast} \right) \bm{a}_k \\
              & = \bm{P}_{\bm{q}_{k}^{\perp}} \cdots \bm{P}_{\bm{q}_{1}^{\perp}} \bm{a}_k
\end{align*}

While its clear that $\bm{P}_{A_{k}^{\perp}} \bm{a} $ and $\bm{P}_{\bm{q}_{k}^{\perp}} \cdots \bm{P}_{\bm{q}_{1}^{\perp}} \bm{a}_k$ used for computing $\bm{q}_k'$ are algebraically, they differ arithmetically as the latter expression evaluates $\bm{q}_k'$ using the follow procedure

\begin{align*}
    \bm{q}_k^{(1)}             & = \bm{a}_k                                       \\
    \bm{q}_k^{(2)}             & = \bm{P}_{\bm{q}_{1}^{\perp}} \bm{q}_k^{(1)}     \\
    \bm{q}_k^{(3)}             & = \bm{P}_{\bm{q}_{2}^{\perp}} \bm{q}_k^{(2)}     \\
                               & \vdots                                           \\
    \bm{q}_k' = \bm{q}_k^{(k)} & = \bm{P}_{\bm{q}_{k-1}^{\perp}} \bm{q}_k^{(k-1)}
\end{align*}

Applying projections sequentially in this manner produces smaller numerical errors. The modified Gram-Schmidt process \cite{TrefethenLloydN.LloydNicholas1997Nla/,DemmelJamesW1997Anla} is summarized in the following algorithm.

    {\centering
        \begin{minipage}{.85\linewidth}
            \begin{algorithm}[H]
                \caption{Modified Gram-Schmidt}
                \label{alg: Modified_Gram-Schmidt}
                \SetAlgoLined
                \DontPrintSemicolon
                \SetKwInOut{Input}{input}\SetKwInOut{Output}{output}

                \Input{A basis $\left\{ \bm{a}_i \right\}_{i=1}^{n}$.}
                \Output{An orthornormal basis $\left\{ \bm{q}_i \right\}_{i=1}^{n}$ such that $\operatorname{l.s} \left\{ \bm{a}_i \right\}_{i=1}^{n} = \operatorname{l.s} \left\{  \bm{q}_i \right\}_{i=1}^{n}$}
                \BlankLine
                \For{$k = 1$ \KwTo $n$}{
                    $\bm{q}_k' = \bm{a}_k$\;
                }
                \For{$k = 1$ \KwTo $n$}{
                    $r_{k,k} = \| \bm{q}_k' \|$\;
                    $\bm{q}_k = \bm{q}_k' / r_{k,k}$\;
                    \For{$i = k+1$ \KwTo $n$}{
                        $r_{i,k} = \langle \bm{q}_k, \bm{q}_i' \rangle$\;
                        $\bm{q}_i = \bm{q}_i - r_{i,k} \bm{q}_i$\;
                    }
                }
                \Return{$\left\{ \bm{q}_i \right\}_{i=1}^{n}$}
                \BlankLine
            \end{algorithm}
        \end{minipage}
        \par
    }

\subsection{Arnoldi and Lanczos Algorithm}\label{Section4.3}

As a quick reminder, we are in search of an iterative process to solve the linear system $\bm{A} \bm{x}^{\star} = \bm{b}$ where no explicit form of $\bm{A}$ is available and we may only rely on a routine that computes $\bm{A} \bm{v}$ for any $\bm{v}$ to extract information on $\bm{A}$. In \Cref{Section4.1} it was shown that $\bm{x}^{\star} \in \calK_{t_{\bm{r}_0}, \bm{A}} \left( \bm{A}, \bm{r}_0 \right)$. With many iterative methods, computing an exact value for $\bm{x}^{\star}$ is out the question with the view that $t_{\bm{r}_0, \bm{A}}$ is impractically large. We must instead resort to approximating $\bm{x}^{\star}$ by $\bm{x}_k$ for which $\bm{x}^{k} \in \calK_{k} \left( \bm{A}, \bm{r}_0 \right)$ where $k \ll t_{\bm{r}_0}$. To find an appropriate value for $\bm{x}_k$, a good start would be to find a basis $\calK_{k} \left( \bm{A}, \bm{r}_0 \right)$. \Cref{defe: krylov_subspace} indicated that $\left\{ \bm{A}^{i-1} \bm{r}_0 \right\}_{i=1}^{k}$ serves as a basis for $\calK_{k} \left( \bm{A}, \bm{r}_0 \right)$. However, for numerical reasons this is a poor choice of basis since each consecutive term becomes closer and closer to being linearly dependent. From now on, for more convenient notation we shall set $n = t_{\bm{r}_0, \bm{A}}$ so that $\bm{x}^{\star} \in \calK_{n} \left( \bm{A}, \bm{r}_0 \right)$. To search for a more appropriate basis let $\bm{K} \in \KK^{n \times n}$ be the invertible matrix
\[
    \bm{K} = \left[ \bm{r}_0 , \bm{A} \bm{r}_0, \ldots , \bm{A}^{n-1} \bm{r}_0 \right].
\]
Since $\bm{K}$ is invertible we can compute $\bm{c} = - \bm{K}^{-1} \bm{A}^{n} \bm{r}_0$ so that
\begin{align*}
    \bm{A} \bm{K} & = \left[ \bm{A} \bm{r}_0, \bm{A}^{2} \bm{r}_0, \ldots , \bm{A}^{n} \bm{r}_0 \right]                     \\
    \bm{A} \bm{K} & = \bm{K} \cdot \left[ \bm{e}_2, \bm{e}_3, \ldots , \bm{e}_n, - \bm{c}  \right] \triangleq \bm{K} \bm{C}
\end{align*}
or, in other terms
\[
    \bm{K}^{-1} \bm{A} \bm{K} = \bm{C} =
    \begin{bmatrix}
        0      & 0      & \cdots & 0      & -c_1   \\
        1      & 0      & \cdots & 0      & -c_2   \\
        0      & 1      & \cdots & 0      & \vdots \\
        \vdots & \vdots & \cdots & \vdots & \vdots \\
        0      & 0      & \cdots & 1      & -c_n
    \end{bmatrix}.
\]
Note here that $\bm{C}$ is upper Hessenberg. While this form is simple, it is of little practical use since the matrix $\bm{K}$ is very likely to be ill-conditioned. To remedy this we can replace $\bm{K}$ with an orthogonal matrix which spans the same space. These are exactly the properties that the $\bm{V}$ matrix offers in the $QR$ factorisation of $\bm{K}$. With this in mind let $\bm{K} = \bm{V} \bm{R}$ be the full $QR$ factorisation of $\bm{K}$. Then
\begin{align*}
    \bm{A} \bm{V} \bm{R} & = \bm{A} \bm{K}                    \\
    \bm{A} \bm{V}        & = \bm{A} \bm{K} \bm{R}^{-1}        \\
    \bm{A} \bm{V}        & = \bm{K} \bm{C} \bm{R}^{-1}        \\
    \bm{A} \bm{V}        & = \bm{V} \bm{R} \bm{C} \bm{R}^{-1} \\
    \bm{A} \bm{V}        & \triangleq \bm{V} \bm{H}.
\end{align*}
Since $\bm{R}$ and $\bm{R}^{-1}$ and both upper triangular and $\bm{C}$ is upper Hessenberg, $\bm{H}$ is also upper Hessenberg. This form provides us with a $\bm{V}$ such that the range of $\bm{V}$ is $\calK_{n} \left( \bm{A}, \bm{r}_0 \right)$ and
\begin{equation}\label{eq: QTAQ_eq_H}
    \bm{V}^{\intercal} \bm{A} \bm{V} = \bm{H}.
\end{equation}
Again, in practice, it may be very difficult to compute this entire expression forcing us to search for approximative alternatives. Consider \Cref{eq: QTAQ_eq_H} for which the only first $k$ columns of $\bm{V}$ have been computed. Let $\bm{V}_k = \left[ \bm{v}_1 , \bm{v}_2 , \ldots , \bm{v}_k \right]$ and $\bm{V}_u = \left[ \bm{v}_{k+1} , \bm{v}_{k+2} , \ldots , \bm{v}_{n} \right]$. Then
\begin{align*}
    \bm{V}^{\intercal} \bm{A} \bm{V}                                                         & = \bm{H} \\
    \left[ \bm{V}_k , \bm{V}_u \right]^{\intercal} \bm{A} \left[ \bm{V}_k , \bm{V}_u \right] & =
    \begin{bmatrix}
        \bm{H}_k     & \bm{H}_{u,k} \\
        \bm{H}_{k,u} & \bm{H}_{u}
    \end{bmatrix}                                                                           \\
    \begin{bmatrix}
        \bm{V}_{k}^{\intercal} \bm{A} \bm{V}_{k} & \bm{V}_{k}^{\intercal} \bm{A} \bm{V}_{u} \\
        \bm{V}_{u}^{\intercal} \bm{A} \bm{V}_{k} & \bm{V}_{u}^{\intercal} \bm{A} \bm{V}_{u}
    \end{bmatrix}
                                                                                             & =
    \begin{bmatrix}
        \bm{H}_k     & \bm{H}_{u,k} \\
        \bm{H}_{k,u} & \bm{H}_{u}
    \end{bmatrix}
\end{align*}
where $\bm{H}_k , \bm{H}_{u,k}, \bm{H}_{k,u}$ and $\bm{H}_u$ are the relevant sub matrices. This provides us with the equality
\begin{equation}\label{eq: QTkAQk_eq_Hk}
    \bm{V}_{k}^{\intercal} \bm{A} \bm{V}_{k} = \bm{H}_k
\end{equation}
noting that $\bm{H}_{k}$ is upper Hessenberg for the same reason that $\bm{H}$ is. We know that when $n = t_{\bm{r}_0, \bm{A}}$ we can find a $\bm{V} \in \KK^{n \times n}$ and $\bm{H} \in \KK^{n \times n}$ that satisfies $\bm{A} \bm{V} = \bm{V} \bm{H}$. However, in general, we may not be so fortunate in finding a $\bm{V}_{k} \in \KK^{n \times k}$ and $\bm{H}_{k} \in \KK^{n \times k}$ to satisfy $\bm{A} \bm{V}_{k} = \bm{V}_{k} \bm{H}_k$ for any $k < n$. Instead we can adjust this equality by adding an error $\bm{E}_k \in \KK^{n \times k}$ to force an equality. Our expression now becomes
\begin{equation}\label{eq: QTkAQk_eq_HkEk}
    \bm{V}_{k}^{\intercal} \bm{A} \bm{V}_{k} = \bm{H}_k + \bm{E}_k.
\end{equation}
A judicious choice of $\bm{E}_k$ must be made to also retain equality in \Cref{eq: QTkAQk_eq_Hk}, meaning $\bm{V}_{k}^{\intercal} \bm{E}_k = \bm{0}$. Since $\left\{ \bm{v}_i \right\}_{i=1}^{k}$ forms an orthornormal basis for $\calK_{n} \left( \bm{A}, \bm{r}_0 \right)$, consider the following choice of $\bm{E}_k$,
\[
    \bm{E}_k = \bm{v}_{k+1} \bm{h}_{k}^{\intercal}
\]
where $\bm{h}_k$ is any vector in $\KK^{k}$. Notice that
\[
    \bm{V}_{k}^{\intercal} \bm{E} = \bm{V}^{\intercal} \left( \bm{v}_{k+1} \bm{h}_k \right) = \left( \bm{V}^{\intercal} \bm{v}_{k+1} \right) \bm{h}_{k}^{\intercal} = \bm{0}.
\]
Since this holds for any $\bm{h}_k \in \KK^{k}$, to preserve sparsity and to keep this form as simple as possible we can set $\bm{h}_k = \left[ 0,0, \ldots , h_{k+1,k} \right]^{\intercal}$. This means $\bm{A} \bm{V}_k$ can be written as
\begin{equation}\label{eq: QTkAQk_eq_Hk_p_qkhk}
    \bm{A} \bm{V}_k =  \bm{V}_k \bm{H}_k + \bm{v}_{k+1} \bm{h}_{k}^{\intercal}
\end{equation}
where
\[
    \bm{V}_k \bm{H}_k =
    \left[ \bm{v}_1 , \bm{v}_2 , \ldots , \bm{v}_k \right]
    \begin{bmatrix}
        h_{1,1} & \cdots & \cdots & \cdots    & h_{1,k} \\
        h_{2,1} & \cdots & \cdots & \cdots    & \vdots  \\
        0       & \ddots & \ddots & \ddots    & \vdots  \\
        \vdots  & \ddots & \ddots & \ddots    & \vdots  \\
        0       & \cdots & 0      & h_{k,k-1} & h_{k,k} \\
        0       & \cdots & 0      & 0         & 0
    \end{bmatrix}.
\]
Equating the $j^{th}$ columns of \Cref{eq: QTkAQk_eq_Hk_p_qkhk} yields
\[
    \bm{A} \bm{v}_j = \sum_{i=1}^{j+1} h_{i,j} \bm{v}_{i}.
\]
Again since $\left\{ \bm{v}_i \right\}_{i=1}^{n}$ form an orthornormal basis, multiplying both sides by $\bm{v}_m$ for $1 \leq m \leq j$ gives
\[
    \bm{v}_m^{\intercal} \bm{A} \bm{v}_j = \sum_{i=1}^{j+1} h_{i,j} \bm{v}_m^{\intercal} \bm{v}_{i} = h_{m,j}
\]
and so
\begin{equation}\label{eq: arn_eq_1}
    h_{j+1,j} \bm{v}_{j+1} = \bm{A} \bm{v}_j - \sum_{i=1}^{j} h_{i,j} \bm{v}_{i}.
\end{equation}
From \Cref{eq: arn_eq_1} we find that $\bm{v}_{j+1}$ can be computed using a recurrance involving its previous Krylov factors. This bears a striking resemblance to \Cref{eq: comp_orth_basis} having a virtually identical setup to computing an orthornormal basis using the modified Gram-Schmidt process (\Cref{alg: Modified_Gram-Schmidt}). As such, values for $\bm{v}_{j+1}$ and $h_{j+1,j}$ can be evaluted using a procedure very similar to the modified Gram-Schmidt process better known as the Arnoldi Algorithm \cite{TrefethenLloydN.LloydNicholas1997Nla/,DemmelJamesW1997Anla}, presented in \Cref{alg: Arnoldi_Algorithm}.

{\centering
\begin{minipage}{.85\linewidth}
    \begin{algorithm}[H]
        \caption{Arnoldi Algorithm}
        \label{alg: Arnoldi_Algorithm}
        \SetAlgoLined
        \DontPrintSemicolon
        \SetKwInOut{Input}{input}\SetKwInOut{Output}{output}

        \Input{$\bm{A}, \bm{r}_0$ and $k$, the number of columns of $\bm{V}$ to compute.}
        \Output{$\bm{V}_k , \bm{H}_k$.}
        \BlankLine
        $\bm{v}_1 = \bm{r}_0 / \| \bm{r}_0 \|$\;
        \For{$j = 1$ \KwTo $k$}{
            $\bm{z} = \bm{A} \bm{v}_j$\;
            \For{$i = 1$ \KwTo $j$}{
                $h_{i,j} = \langle \bm{v}_{i} , \bm{z} \rangle$\;
                $\bm{z} = \bm{z} - h_{i,j} \bm{v}_{i}$\;
            }
            $h_{j+1,j} = \| \bm{z} \|$\;
            \If{$h_{j+1,j} = 0$}{
                \Return{$\bm{V}_k , \bm{H}_k$}
            }
            $\bm{v}_{j+1} = \bm{z} / h_{j+1,j}$\;
        }
        \Return{$\bm{V}_k , \bm{H}_k$}
        \BlankLine
    \end{algorithm}
\end{minipage}
\par
}

When $\bm{A}$ is symmertic then $\bm{H} = \bm{T}$ becomes a tridiagonal matrix, simplifying a large amount of the Arnoldi algorithm since the matrix elements from $\bm{T}$ can be written as
\[
    \bm{T} =
    \begin{bmatrix}
        \alpha_1 & \beta_1 &        &             &             \\
        \beta_1  & \ddots  & \ddots &             &             \\
                 & \ddots  & \ddots & \ddots      &             \\
                 &         & \ddots & \ddots      & \beta_{n-1} \\
                 &         &        & \beta_{n-1} & \alpha_{n}
    \end{bmatrix}.
\]
As before, equating the $j^{th}$ columns of $\bm{A} \bm{V} = \bm{V} \bm{T}$ yields
\begin{equation}\label{eq: lancz_orth_basis}
    \bm{A} \bm{v}_{j} = \beta_{j-1} \bm{v}_{j-1} + \alpha_{j} \bm{v}_j + \beta_j \bm{v}_{j+1}.
\end{equation}
Again since $\left\{ \bm{v}_{i} \right\}_{i=1}^{n}$ form an orthornormal basis, multiplying both sides of \Cref{eq: lancz_orth_basis} by $\bm{v}_j$ gives $\bm{v}_j \bm{A} \bm{v}_j = \alpha_j$. This simplified version of the Arnoldi algorithm used for computing $\left\{ \bm{v}_{i} \right\}_{i=1}^{n}$ and $\bm{T}$ for symmetric matrices is better known as the Lanczos algorithm \cite{DemmelJamesW1997Anla}, presented more explicitly in \Cref{alg: Lanczos_Algorithm}.

{\centering
\begin{minipage}{.85\linewidth}
    \begin{algorithm}[H]
        \caption{Lanczos Algorithm}
        \label{alg: Lanczos_Algorithm}
        \SetAlgoLined
        \DontPrintSemicolon
        \SetKwInOut{Input}{input}\SetKwInOut{Output}{output}

        \Input{$\bm{A}, \bm{r}_0$ and $k$, the number of columns of $\bm{V}$ to compute.}
        \Output{$\bm{V}_k , \bm{T}_k$.}
        \BlankLine
        $\bm{v}_1 = \bm{r}_0 / \| \bm{r}_0 \|$, $\beta_0 = 0$, $\bm{v}_0 = 0$\;
        \For{$j = 1$ \KwTo $k$}{
            $\bm{z} = \bm{A} \bm{v}_j$\;
            $\alpha_j = \langle \bm{v}_{j}, \bm{z} \rangle$\;
            $\bm{z} = \bm{z} - \alpha_j \bm{v}_{j} - \beta_{j-1} \bm{v}_{j-1}$\;
            $\beta_j = \| z \|$\;
            \If{$\beta_{j} = 0$}{
                \Return{$\bm{V}_k , \bm{T}_k$}
            }
            $\bm{v}_{j+1} = \bm{z} / \beta_{j}$\;
        }
        \Return{$\bm{V}_k , \bm{T}_k$}
        \BlankLine
    \end{algorithm}
\end{minipage}
\par
}

For the Lanczos algorithm, \Cref{eq: QTkAQk_eq_Hk_p_qkhk} can be re-written in the a more compact form as
\begin{equation}\label{eq: AVk_eq_VkTk1k}
    \bm{A} \bm{V}_{k} \triangleq \bm{V}_{k} \bm{T}_{k+1,k}
\end{equation}
where $\bm{T}_{k+1,k} = \bm{T}_{k} + \bm{v}_{k+1} \bm{t}_{k}^{\intercal}$.

\subsection{Optimality Conditions}\label{Section4.4}

So far we have shown that $\bm{x}^{\star} \in \calK_{t_{\bm{r}_0}, \bm{A}} \left( \bm{A}, \bm{r}_0 \right)$ where $n = t_{\bm{r}_0}$ is the grade of $\bm{r}_0$ with respect to $\bm{A}$. Moreover from section \ref{Section4.3} we found ways to construct a basis for $\calK_{t_{\bm{r}_0}, \bm{A}} \left( \bm{A}, \bm{r}_0 \right)$ allowing us to generate vectors with these affine spaces, namely the Arnoldi algorithm (algorithm \ref{alg: Arnoldi_Algorithm}) and Lanczos algorithm (algorithm \ref{alg: Lanczos_Algorithm}) for non-symmertic and symmertic systems respectively. From now on $\calK_{t_{\bm{r}_0}, \bm{A}} \left( \bm{A}, \bm{r}_0 \right)$ will be abbreviated to $\calK_{t_{\bm{r}_0}, \bm{A}}$ when the context is clear. The question still remains however, how should one choose an $\bm{x}_k$ that best approximates $\bm{x}^{\ast}$ satisfying equation \ref{eq: lin_sys_1}? Here are a few of the most well known methods for selecting a suitable $\bm{x}_k$.

\begin{enumerate}

    \item Select an $\bm{x}_k \in \bm{x}_0 + \calK_k$ which minimizes $\| \bm{x}_k - \bm{x}^{\ast} \|_2$. While this method seems like the most intuitive and natural way to select $\bm{x}_k$, it is unfortunately of no practical use since there is not enough information in the Krylov subspace to find an $\bm{x}_k$ which matches this profile.

    \item Select an $\bm{x}_k \in \bm{x}_0 + \calK_k$ which minimizes $\| \bm{r}_k \|_2$ (recall this is the residual of $\bm{x}_k$, that is, $\bm{r}_k = \bm{b} - \bm{A} \bm{x}_k$). This method is possible to implement. Two well known algorithms stem from this class of methods, namely MINRES (minimum residual) and GMRES (general minimum residual) which solve linear systems for symmetric and non-symmertic $\bm{A}$ respectively.

    \item When $\bm{A}$ is a positive definite matrix it defines a norm $\| \bm{r} \|_{\bm{A}} = \left( \bm{r}^{\intercal} \bm{A} \bm{r} \right)^{\frac{1}{2}}$, called the energy norm. Select an $\bm{x}_k \in \bm{x}_0 + \calK_k$ which minimizes $\| \bm{r} \|_{\bm{A}^{-1}}$ which is equivalent to minimizing $\| \bm{x}_k - \bm{x} \|_{A}$. This technique is known as the CG (conjugate gradient) algorithm.

    \item Select an $\bm{x}_k \in \bm{x}_0 + \calK_k$ for which $\bm{r}_{k} \perp \calW_k$ where $\calW_k$ is some $k$-dimensional subspace. Two well known algorithms that belong to this family of methods are SYMMLQ (Symmetric LQ Method) and a variant of GMRES used for solving symmetric and non-symmetric methods respectively.

\end{enumerate}

Interestingly, when $\bm{A}$ is symmetric positive definite and $\calW_k = \calK_k$ the last two selection methods are equivalent. This is stated more precisely in theorem \ref{theorem: 3_4_method_eq} without proof.

\begin{thm} \label{theorem: 3_4_method_eq}
    In the context of the above selection method, if $\bm{A} \succ \bm{0}$ and $\calW_k = \calK_k$ in method (4) then it produces the same $\bm{x}_k$ in method (3) \cite{DemmelJamesW1997Anla}.
\end{thm}

In fact the very last method can be used to bring together a number of different analytical aspects and unify them in a general framework known as projection methods. Selecting an $\bm{x}_k$ from our Krylov subspace allows $k$ degrees of freedom meaning $k$ constraints must be used to determine a unique $\bm{x}_k$ for selection. As seen in method (4) already, typically orthogonality constraints are imposed on the residual $\bm{r}_k$. Specifically we would like to find a $\bm{x}_k \in \bm{x}_0 + \calK_k$ where $\bm{r}_k \perp \calW_k$. This is sometimes referred to as the Petrov-Galerkin (or just Galerkin) conditions. Projection methods for which $\calW_k = \calK_k$ are known as orthogonal projections while methods for which $\calW_k = \bm{A} \calK_k$ are known as oblique projections. If we set $\bm{x}_k = \bm{x}_0 + \bm{z}_k$ for some $\bm{z}_k \in \calK_k$ then the Petrov-Galerkin conditions imply $\bm{r}_0 - \bm{A} \bm{z}_k \perp \calW_k$, or alternatively $\langle \bm{r}_0 - \bm{A} \bm{z}_k , \bm{w} \rangle = 0$ for every $\bm{w} \in \calW_k$. To impose these conditions it will help to have an appropriate basis for $\calK$ and $\calW$. Suppose we have access to such a basis where $\left\{ \bm{v}_i \right\}_{i=1}^{k}$ and $\left\{ \bm{w}_i \right\}_{i=1}^{k}$ are basis elements for $\calK$ and $\calW$ respectively. Let
\begin{align*}
    \bm{K}_k & \triangleq \left[ \bm{v}_1 , \bm{v}_2 , \ldots , \bm{v}_k \right] \in \KK^{n \times k} \\
    \bm{W}_k & \triangleq \left[ \bm{w}_1 , \bm{w}_2 , \ldots , \bm{w}_k \right] \in \KK^{n \times k}
\end{align*}
then the Petrov-Galerkin conditions can be imposed as follows
\begin{align*}
    \bm{K}_k \bm{y}_k                                                       & = \bm{z}_k , \quad \text{for some} \; \bm{y}_k \in \KK^k \\
    \bm{W}_k^{\intercal} \left( \bm{r}_0 - \bm{A} \bm{K}_k \bm{y}_k \right) & = \bm{0}.
\end{align*}
Moreover if $\bm{W}_k^{\intercal} \bm{A} \bm{K}_k$ is invertible then $\bm{x}_k$ can be expressed as
\begin{equation} \label{eq: expr_x_Petrov_Galerkin_1}
    \bm{x}_k = \bm{x}_0 + \bm{K}_k \left( \bm{W}_k^{\intercal} \bm{A} \bm{K}_k \right)^{-1} \bm{W}_k \bm{r}_0.
\end{equation}
This justifies a general form of the projection method algorithm presented in algorithm \ref{alg: General_Projection}.

{\centering
\begin{minipage}{.85\linewidth}
    \begin{algorithm}[H]
        \caption{General Projection Method}
        \label{alg: General_Projection}
        \SetAlgoLined
        \DontPrintSemicolon
        \SetKwInOut{Input}{input}\SetKwInOut{Output}{output}

        \Output{An approximation of $\bm{x}^{\ast}$, $\bm{x}_k$.}
        \BlankLine
        \For{$k = 1 , \ldots $ \Until convergence}{
        Select $\calK_k$ and $\calW_k$\;
        Form $\bm{K}_k$ and $\bm{W}_k$\;
        Solve $\left( \bm{W}_k^{\intercal} \bm{A} \bm{K}_k \right) \bm{y}_k = \bm{W}_k^{\intercal} \bm{r}_0$\;
        $\bm{x}_k = \bm{x}_0 + \bm{K}_k \bm{y}_k$\;
        }
        \Return{$\bm{x}_k$}
        \BlankLine
    \end{algorithm}
\end{minipage}
\par
}

\subsection{Conjugate Gradient Algorithm}\label{Section4.5}

From \Cref{Section4.4} that the Petrov-Galerkin conditions for the CG algorithm used an orthogonal projection and the matrix $\bm{A}$ was assumed to be positive definite. To derive the CG algorithm we can start be using some machinery that the Lanczos algorithm provides us with. Recall, the Lanczos algorithm produces the form $\bm{A}\bm{V}_{k} = \bm{V}_{k} \bm{T}_k + \bm{v}_{k+1} \bm{t}_{k}^{\intercal}$ where $\bm{t}_{k} \triangleq \left[ 0,0, \ldots , 0, \beta_k \right]^{\intercal} \in \KK^k$ and the columns of $\bm{V}_{k}$ span $\calK_k$. Recall that $\bm{x}_k$ can be expressed as $\bm{x}_k = \bm{x}_0 + \bm{K}_k \left( \bm{W}_k^{\intercal} \bm{A} \bm{K}_k \right)^{-1} \bm{W}_k \bm{r}_0$ (\Cref{eq: expr_x_Petrov_Galerkin_1}) when $\bm{W}_k^{\intercal} \bm{A} \bm{K}_k$ is invertible. For the CG algorithm $\calK = \calW$ and $\bm{A} \succ \bm{0}$. Under these conditions we can easily show that $\bm{W}_k^{\intercal} \bm{A} \bm{K}_k$ is indeed invertible. Thus the approximate vector can be expressed as $\bm{x}_k = \bm{x}_0 + \bm{z}_k$ where $\bm{z}_k \in \calK_k$. In terms of the Petrov-Galerkin conditions this means that $\bm{z}_k$ must satisfy $\bm{r}_0 - \bm{A} \bm{z}_k \perp \calW_k$. Furthermore since $\calK_k = \operatorname{Range} \left( \bm{V}_{k} \right)$ where $\bm{V}_{k}$ has full column rank then $\bm{z}_k$ can be represented as $\bm{z}_k = \bm{V}_{k} \bm{y}$ for a unique $\bm{y} \in \RR^k$ so that
\begin{equation} \label{eq: x_eq_Qky}
    \bm{x}_k = \bm{x}_0 + \bm{V}_{k} \bm{y}.
\end{equation}
Coupling this with the Petrov-Galerkin conditions ensures
\begin{align} \label{eq: Tky_eq_normr0e1}
    \bm{V}_{k}^{\intercal} \left( \bm{r}_0 - \bm{A} \bm{V}_{k} \bm{y} \right) & = \bm{0}                        \nonumber   \\
    \bm{V}_{k}^{\intercal} \bm{A} \bm{V}_{k} \bm{y}                           & = \bm{V}_{k}^{\intercal} \bm{r}_0 \nonumber \\
    \bm{T}_k \bm{y}                                                           & = \| \bm{r}_0 \| \bm{e}_1.
\end{align}
In the CG algorithm $\bm{x}_{k+1}$ is computed as the recurrance of the following three sets of vectors
\begin{enumerate}
    \item The approximate solutions $\bm{x}_{k}$
    \item The residual vectors $\bm{r}_{k}$
    \item The conjugate gradient vectors $\bm{p}_k$
\end{enumerate}
The conjugate gradient vectors include the word 'gradient' since the attempt to find the direction of steepest descent that minimizes $\| \bm{r}_{k} \|_{\bm{A}^{-1}}$. It also includes the word 'conjugate' since $\langle \bm{p}_k, \bm{A} \bm{p}_j \rangle = 0$ for $i \neq j$, that is, vectors $\bm{p}_i$ and $\bm{p}_j$ are mutally $A$-conjugate (shown in \Cref{lemma: Pk_cols_A_conj}).

Since $\bm{A}$ is symmetric positive definite then so is $\bm{T}_k  = \bm{V}_{k} \bm{A} \bm{V}_{k}$. We can take the Cholesky decomposition of $\bm{T}_k$ to get
\begin{equation} \label{eq: Tk_Cholesky}
    \bm{T}_k = \bm{L}_k \bm{D}_k \bm{L}_k^{\intercal}
\end{equation}
where $\bm{L}_k$ is a unit lower bidiagonal matrix and $\bm{D}_k$ is diagonal written as
\[
    \bm{L}_k =
    \begin{bmatrix}
        1   &        &         &   \\
        l_1 & \ddots &         &   \\
            & \ddots & \ddots  &   \\
            &        & l_{k-1} & 1
    \end{bmatrix}, \quad
    \bm{D}_k =
    \begin{bmatrix}
        d_1 &     &        &     \\
            & d_2 &        &     \\
            &     & \ddots &     \\
            &     &        & d_k
    \end{bmatrix}.
\]
Combining equations \Cref{eq: x_eq_Qky}, \Cref{eq: Tky_eq_normr0e1} and \Cref{eq: Tk_Cholesky}
\begin{align*}
    \bm{x}_k & = \bm{x}_0 + \bm{V}_{k} \bm{y}                                                                                                  \\
    \bm{x}_k & = \bm{x}_0 + \| \bm{r}_0 \| \bm{V}_{k} \bm{T}_k^{-1} \bm{e}_1                                                                   \\
    \bm{x}_k & = \bm{x}_0 + \| \bm{r}_0 \| \bm{V}_{k} \left( \bm{L}_k \bm{D}_k \bm{L}_k^{\intercal} \right)^{-1} \bm{e}_1                      \\
    \bm{x}_k & = \bm{x}_0 + \left( \bm{V}_{k} \bm{L}_k^{-\intercal} \right) \left( \| \bm{r}_0 \| \bm{D}_k^{-1} \bm{L}_k^{-1} \bm{e}_1 \right) \\
    \bm{x}_k & \triangleq \bm{x}_0 + \tilde{\bm{P}}_k \tilde{\bm{y}}_k
\end{align*}
where $\tilde{\bm{P}}_k = \bm{V}_{k} \bm{L}_k^{-\intercal}$ and $\tilde{\bm{y}}_k = \| \bm{r}_0 \| \bm{D}_k^{-1} \bm{L}_k^{-1} \bm{e}_1$. The matrix $\tilde{\bm{P}}_k$ can be written as
$\tilde{\bm{P}}_k = \left[ \tilde{\bm{p}}_1 , \tilde{\bm{p}}_2 , \ldots , \tilde{\bm{p}}_k \right]$. \Cref{lemma: Pk_cols_A_conj} shows that the columns of $\tilde{\bm{P}}_k$ are $A$-conjugate.

\begin{lem} \label{lemma: Pk_cols_A_conj}
    The columns of $\tilde{\bm{P}}_k$ are $A$-conjugate, in otherwords $\tilde{\bm{P}}_k^{\intercal} \bm{A} \tilde{\bm{P}}_k$ is diagonal.
\end{lem}

\begin{proof}
    We compute
    \begin{align*}
        \tilde{\bm{P}}_k^{\intercal} \bm{A} \tilde{\bm{P}}_k
         & = \left( \bm{V}_{k} \bm{L}_k^{-\intercal} \right)^{\intercal} \bm{A} \left( \bm{V}_{k} \bm{L}_k^{-\intercal} \right)      \\
         & = \bm{L}_k^{-1} \left( \bm{V}_{k}^{\intercal} \bm{A} \bm{V}_{k} \right) \bm{L}_k^{-\intercal}                             \\
         & = \bm{L}_k^{-1} \left( \bm{T}_k \right) \bm{L}_k^{-\intercal}                                                             \\
         & = \bm{L}_k^{-1} \left( \bm{L}_k \bm{D}_k \bm{L}_k^{\intercal} \right) \bm{L}_k^{-\intercal} \tag*{\Cref{eq: Tk_Cholesky}} \\
         & = \bm{D}_k
    \end{align*}
    as wanted.
\end{proof}

Since $\bm{L}_k$ is a lower bidiagonal, setting $\bm{a} \triangleq l_{k-1} \bm{e}_{k-1}$, it can be written in the form
\[
    \bm{L}_k =
    \begin{bmatrix}
        \bm{L}_{k-1}       & \bm{0} \\
        \bm{a}^{\intercal} & 1
    \end{bmatrix}
\]
meaning
\[
    \bm{L}_k^{-1} =
    \begin{bmatrix}
        \bm{L}_{k-1}^{-1} & \bm{0} \\
        \star             & 1
    \end{bmatrix}
\]

where the elements of $\star$ are of no importance. With this a recurrance for the columns of $\tilde{\bm{P}}_k$ can now be derived in terms of $\bm{y}_k$. To start we can show that the first $k-1$ entries of $\tilde{\bm{y}}_{k}$ shares the first $k-1$ entires with $\tilde{\bm{y}}_{k-1}$ and that $\tilde{\bm{P}}_k$ and $\tilde{\bm{P}}_{k-1}$ share the same first $k-1$ columns. We can begin by computing a recurrance for $\tilde{\bm{y}}_{k}$ as follows
\begin{align*}
    \tilde{\bm{y}}_{k} & = \| \bm{r}_0 \| \bm{D}_k^{-1} \bm{L}_k^{-1} \bm{e}_1^k \\
                       & = \| \bm{r}_0 \|
    \begin{bmatrix}
        \bm{D}_{k-1}^{-1} & \bm{0}   \\
        \bm{0}            & d_k^{-1}
    \end{bmatrix}
    \begin{bmatrix}
        \bm{L}_{k-1}^{-1} & \bm{0} \\
        \star             & 1
    \end{bmatrix}
    \bm{e}_1^k                                                                   \\
                       & = \| \bm{r}_0 \|
    \begin{bmatrix}
        \bm{D}_{k-1}^{-1} \bm{L}_{k-1}^{-1} & \bm{0}   \\
        \star                               & d_k^{-1}
    \end{bmatrix}
    \begin{bmatrix}
        \bm{e}_1^k \\
        0
    \end{bmatrix}                                                   \\
                       & =
    \begin{bmatrix}
        \tilde{\bm{y}}_{k-1} \\
        \eta_k
    \end{bmatrix}
\end{align*}

where $\bm{e}_i^k$ is the $i^{th}$ unit vector with $k$ dimensions. To get a recurrance for the columns of $\tilde{\bm{P}}_{k-1} = \left[ \tilde{\bm{p}}_1 , \tilde{\bm{p}}_2 , \ldots , \tilde{\bm{p}}_k \right]$ since $\bm{L}_{k-1}^{\intercal}$ is upper triangular then so is $\bm{L}_{k-1}^{-\intercal}$, thus forming the leading $(k-1) \times (k-1)$ submatrix of $\bm{L}_{k}^{-\intercal}$. In effect, $\tilde{\bm{P}}_{k-1}$ is identical to the leading $k-1$ columns of

\[
    \tilde{\bm{P}}_{k} = \bm{V}_{k} \bm{L}_k^{-\intercal} = \left[ \bm{V}_{k-1} , \bm{v}_k \right]
    \begin{bmatrix}
        \bm{L}_{k-1}^{-1} & \bm{0} \\
        \star             & 1
    \end{bmatrix}
    = \left[ \bm{V}_{k-1} \bm{L}_{k-1}^{-1} , \tilde{\bm{p}}_{k} \right]
    = \left[ \tilde{\bm{P}}_{k-1} , \tilde{\bm{p}}_{k} \right].
\]
Moreover rearranging $\tilde{\bm{P}}_{k} = \bm{V}_{k} \bm{L}_k^{-\intercal}$ we get $\tilde{\bm{P}}_{k} \bm{L}_k^{\intercal} = \bm{V}_{k}$. Equating the $k^{th}$ column yields
\begin{equation} \label{eq: pk_rec}
    \tilde{\bm{p}}_{k} = \bm{v}_k - l_{k-1} \tilde{\bm{p}}_{k-1}.
\end{equation}
Finally we can use
\begin{equation} \label{eq: xk_rec}
    \bm{x}_k = \bm{x}_0 + \tilde{\bm{P}}_{k} \tilde{\bm{y}}_{k}                                 \\
    = \bm{x}_0 + \left[ \tilde{\bm{P}}_{k-1} , \tilde{\bm{p}}_{k} \right]
    \begin{bmatrix}
        \tilde{\bm{y}}_{k-1} \\
        \eta_k
    \end{bmatrix}                                                                    \\
    = \bm{x}_0 + \tilde{\bm{P}}_{k-1} \tilde{\bm{y}}_{k-1} + \eta_k \tilde{\bm{p}}_{k} \\
    = \bm{x}_{k-1} + \eta_k \tilde{\bm{p}}_{k}
\end{equation}
as a recurrance for $\bm{x}_k$. A recurrance for $\bm{r}_k$ is easily computed as
\begin{equation} \label{eq: rk_rec}
    \bm{r}_{k} = b - \bm{A} \bm{x}_k = b - \bm{A} \left( \bm{x}_{k-1} + \eta_k \tilde{\bm{p}}_{k} \right) = \left( b - \bm{A} \bm{x}_{k-1} \right) - \eta_k \bm{A} \tilde{\bm{p}}_{k} = \bm{r}_{k-1} - \eta_k \bm{A} \tilde{\bm{p}}_{k}
\end{equation}
Altogether we are left with recurrences for $\bm{v}_k$ from Lanczos, $\tilde{\bm{p}}_{k}$ \Cref{eq: pk_rec}, the residual $\bm{r}_k$ \Cref{eq: pk_rec},  and for the approximate solution $\bm{x}_k$ \Cref{eq: xk_rec}. However, futher simplifications can be made to bring about a more efficient algorithm. Recall from \Cref{Section4.3} that $\bm{A} \bm{V}_{k} =  \bm{V}_{k} \bm{T}_k + \bm{v}_{k+1} \bm{t}_{k}^{\intercal}$ where $\bm{t}_k = \left[ 0,0, \ldots , 0, \beta_k \right]^{\intercal} \in \RR^k$ meaning
\[
    \bm{r}_k = \bm{r}_0 - \bm{A} \bm{V}_{k} \bm{y}_k = \bm{r}_0 - \bm{V}_{k} \bm{T}_k \bm{y}_k - \langle \bm{t}_k , \bm{y} \rangle \bm{v}_{k+1} = - \beta_k y_k \bm{v}_{k+1}.
\]
This tells us that $\bm{r}_k$ is parallel to $\bm{v}_{k+1}$ and orthogonal to all $\bm{v}_{i}, \; 1 \leq i \leq k$. This further implies that $\bm{r}_k$ is orthogonal to all $\bm{r}_i, \; 1 \leq i \leq k-1$ since they are just $\bm{v}_{i}$ scaled by some constant factor. So replacing $\bm{r}_{k-1}$ with $\bm{v}_k / \eta_k$ and defining $\bm{p}_k \triangleq \tilde{\bm{p}}_k / \gamma_k$ gives us a new set of recurrences
\begin{align*}
    \bm{x}_k & = \bm{x}_{k-1} + \alpha_k \bm{p}_k        \\
    \bm{r}_k & = \bm{r}_{k-1} - \alpha_k \bm{A} \bm{p}_k \\
    \bm{p}_k & = \bm{r}_{k-1} + \beta_k \bm{p}_{k-1}
\end{align*}
where $\alpha_k = \eta_k / \gamma_k$. From \Cref{lemma: Pk_cols_A_conj} we have shown that the columns of $\tilde{\bm{P}}_k$ are $A$-conjugate (that is $\langle \tilde{\bm{p}}_i , \bm{A} \tilde{\bm{p}}_j \rangle = 0, \; i \neq j$) and that $\tilde{\bm{P}}_k^{\intercal} \bm{A} \tilde{\bm{P}}_k = \bm{D}_k$. This also means that $\langle \bm{r}_i , \bm{r}_j \rangle = 0, \; i \neq j$. Note that from our recurrence for $\bm{p}_k = \bm{r}_{k-1} + \beta_k \bm{p}_{k-1}$ we have
\[
    \langle \bm{A} \bm{p}_k ,\bm{p}_k \rangle = \langle \bm{A} \bm{p}_k , \bm{r}_{k-1} + \beta_k \bm{p}_{k-1} \rangle = \langle \bm{A} \bm{p}_k , \bm{r}_{k-1} \rangle.
\]
We can now find an expression for $\alpha_k$ as
\begin{align*}
    \langle \bm{r}_{k-1} , \bm{r}_{k} \rangle & = \langle \bm{r}_{k-1} , \bm{r}_{k-1} - \alpha_k \bm{A} \bm{p}_k \rangle                           \\
    \langle \bm{r}_{k-1} -1 \rangle           & = \langle \bm{r}_{k-1} , \bm{r}_{k-1} \rangle - \alpha_k \langle \bm{p}_k, \bm{A} \bm{p}_k \rangle \\
    \alpha_k                                  & = \frac{\langle \bm{r}_{k-1} , \bm{r}_{k-1} \rangle}{\langle \bm{p}_k, \bm{A} \bm{p}_k \rangle}.
\end{align*}
Similarly, using the recurrence for $\bm{p}_k$, an expression for $\beta_k$ can be computed as
\begin{align*}
    \langle \bm{A} \bm{p}_{k-1} , \bm{p}_k \rangle & = \langle \bm{A} \bm{p}_{k-1}, \bm{r}_{k-1} + \beta_k \bm{p}_{k-1} \rangle                                       \\
    \langle \bm{A} \bm{p}_{k-1} , \bm{p}_k \rangle & = \langle \bm{A} \bm{p}_{k-1}, \bm{r}_{k-1} \rangle + \beta_k \langle \bm{A} \bm{p}_{k-1}, \bm{p}_{k-1} \rangle  \\
    \beta_k                                        & = - \frac{\langle \bm{A} \bm{p}_{k-1}, \bm{r}_{k-1} \rangle}{\langle \bm{A} \bm{p}_{k-1}, \bm{p}_{k-1} \rangle}.
\end{align*}
This formula requires an additional dot product which was not present before. Fortunately, this dot product can be eliminated using our recurrence for $\bm{r}_k$
\begin{align*}
    \langle \bm{r}_k , \bm{r}_k \rangle & = \langle \bm{r}_k , \bm{r}_{k-1} - \alpha_k \bm{A} \bm{p}_k \rangle                            \\
    \langle \bm{r}_k , \bm{r}_k \rangle & = \langle \bm{r}_k , \bm{r}_{k-1} \rangle - \alpha_k \langle \bm{r}_k , \bm{A} \bm{p}_k \rangle \\
    \alpha_k                            & = - \frac{\langle \bm{r}_k , \bm{r}_k \rangle}{\langle \bm{r}_k , \bm{A} \bm{p}_k \rangle}.
\end{align*}
Equating the two expressions for $\alpha_k$ affords
\begin{align*}
    - \frac{\langle \bm{r}_k , \bm{r}_k \rangle}{\langle \bm{r}_k , \bm{A} \bm{p}_k \rangle}  & = \frac{\langle \bm{r}_{k-1} , \bm{r}_{k-1} \rangle}{\langle \bm{p}_k, \bm{A} \bm{p}_k \rangle} \\
    - \frac{\langle \bm{r}_k , \bm{r}_k \rangle}{\langle \bm{r}_{k-1} , \bm{r}_{k-1} \rangle} & = \frac{\langle \bm{r}_k , \bm{A} \bm{p}_k \rangle}{\langle \bm{p}_k, \bm{A} \bm{p}_k \rangle}.
\end{align*}
This means that
\[
    \beta_k = \frac{\langle \bm{r}_{k-1} , \bm{r}_{k-1} \rangle}{\langle \bm{r}_{k-2} , \bm{r}_{k-2} \rangle}.
\]
These recurrences are computed iteratively to form the basis of the CG algorithm, seen in Algorithm \ref{alg: CG}.

{\centering
\begin{minipage}{.85\linewidth}
    \begin{algorithm}[H]
        \caption{CG Algorithm}
        \label{alg: CG}
        \SetAlgoLined
        \DontPrintSemicolon
        \SetKwInOut{Input}{input}\SetKwInOut{Output}{output}

        \Input{$\bm{A} \succ \bm{0}$, $\bm{b}$ and an initial guess $\bm{x}_0$.}
        \Output{An approximation of $\bm{x}^{\ast}$, $\bm{x}_k$.}
        \BlankLine
        $\bm{r}_0 = \bm{b} - \bm{A} \bm{x}_0$, $\bm{p}_1 = \bm{r}_0$\;
        \For{$k = 1 , \ldots $ \Until $\| r_{k-1} \| \leq \tau$}{
            $\alpha_k = \frac{\langle \bm{r}_{k-1} , \bm{r}_{k-1} \rangle}{\langle \bm{p}_k, \bm{A} \bm{p}_k \rangle}$ \;
            $\bm{x}_k = \bm{x}_{k-1} + \alpha_k \bm{p}_k$ \;
            $\bm{r}_k = \bm{r}_{k-1} - \alpha_k \bm{A} \bm{p}_k$ \;
            $\beta_{k+1} = \frac{\langle \bm{r}_{k} , \bm{r}_{k} \rangle}{\langle \bm{r}_{k-1} , \bm{r}_{k-1} \rangle}$ \;
            $\bm{p}_{k+1} = \bm{r}_{k} + \beta_{k+1} \bm{p}_k$ \;
        }
        \Return{$\bm{x}_k$}
        \BlankLine
    \end{algorithm}
\end{minipage}
\par
}

The next few theorems pertain to the rate of convergence of the CG algorithm.

\begin{thm} \label{theorem: equiv-CG-spaces}
    Let the CG iteration (\Cref{alg: CG}) be applied to a symmetric positive definite matrix problem $\bm{A} \bm{x} = \bm{b}$. As long as the iteration has not yet converged (that is, $\bm{r}_{n-1} \neq 0$), the algorithm proceeds without divisions by zero, and we have the following identities of subspaces:
    \begin{equation} \label{eq: equiv-CG-spaces}
        \calK_n                                                  = \operatorname{l.s} \left\{ \bm{x}_i \right\}_{i=1}^{n} = \operatorname{l.s} \left\{ \bm{p}_i \right\}_{i=0}^{n-1} = \operatorname{l.s} \left\{ \bm{r}_i \right\}_{i=0}^{n-1} = \operatorname{l.s} \left\{ \bm{A}^{i} \bm{b} \right\}_{i=0}^{n-1}.
    \end{equation}
    Moreover the residuals are orthogonal,
    \begin{equation*}
        \langle \bm{r}_{n} , \bm{r}_{j} \rangle = 0 \quad (j<n)
    \end{equation*}
    and the search directions are $\bm{A}-$conjugate \cite{TrefethenLloydN.LloydNicholas1997Nla/}*{page 295}.
\end{thm}

\begin{proof}
    The orthogonality of the residuals has already been shown within the derivation of the CG algorithm. Seeing that the search directions are $\bm{A}-$conjugate is a very straight forward application of \Cref{lemma: Pk_cols_A_conj}. To show \Cref{eq: equiv-CG-spaces} we can induct on $k$. From the initial guess $\bm{x}_0$ and the expression $\bm{x}_k = \bm{x}_{k-1} + \alpha_k \bm{p}_k$, using an induction argument it follows that $\bm{x}_k$ belongs to $\operatorname{l.s} \left\{ \bm{p}_i \right\}_{i=0}^{k-1}$. Similarly, from $\bm{p}_k = \bm{r}_{k-1} + \beta_k \bm{p}_{k-1}$ it follows that this belongs to $\operatorname{l.s} \left\{ \bm{r}_i \right\}_{i=0}^{n-1}$. Finally, from $\bm{r}_k = \bm{r}_{k-1} - \alpha_k \bm{A} \bm{p}_k$ that this belongs to $\operatorname{l.s} \left\{ \bm{A}^{i} \bm{b} \right\}_{i=0}^{n-1}$.
\end{proof}

It is fairly straight forward to confirm that CG minimizes error at each step.

\begin{thm} \label{theorem: CG-error-min}
    Let the CG iteration be applied to a symmertic positive definite matrix $\bm{A} \bm{x} = \bm{b}$. If the iteration has not already converged (that is, $\| \bm{r}_{k-1} \| > 0 $), then $\bm{x}_{k}$ is the unique point in $\calK_{k}$ that minimizes $\| \bm{x}^{\star} - \bm{x}_{k} \|_{\bm{A}}$. The convergence is the monotonic,
    \[
        \| \bm{x}^{\star} - \bm{x}_{k} \|_{\bm{A}} \leq \| \bm{x}^{\star} - \bm{x}_{k-1} \|_{\bm{A}}
    \]
    and $\| \bm{x}^{\star} - \bm{x}_{k} \|_{\bm{A}} = 0$ is achieved for some $k \leq n$ \cite{TrefethenLloydN.LloydNicholas1997Nla/}*{page 296}.
\end{thm}

\begin{proof}
    From \Cref{theorem: equiv-CG-spaces} we know that $\bm{x}_{k}$ belongs to $\calK_{k}$. To show that it is the unique point in $\calK_{k}$ that minimizes the error, consider an arbitrary point $\bm{z} = \bm{x}_k - \Delta \bm{z} \in \calK_{k}$, with error $\bm{e} = \bm{x}_{\star} - \bm{z} = \bm{e}_{k} + \Delta \bm{x}$. Note in this context $\bm{e}_{k} = \bm{x}_{\star} - \bm{x}_{k}$, and does {\it not} represent standard basis vectors. We calculate
    \begin{align*}
        \| \bm{e} \|_{\bm{A}}^2 \
         & = \left( \bm{e}_{k} + \Delta \bm{x} \right)^{\intercal} \bm{A} \left( \bm{e}_{k} + \Delta \bm{x} \right)                                                                                  \\
         & = \bm{e}_{k}^{\intercal} \bm{A} \bm{e}_{k} + \left( \Delta \bm{x} \right)^{\intercal} \bm{A} \left( \Delta \bm{x} \right) + 2 \bm{e}_{k}^{\intercal} \bm{A} \left( \Delta \bm{x} \right).
    \end{align*}
    The final term of this expression is $2 \bm{r}_{k}^{\intercal} \left( \Delta \bm{x} \right)$, an inner product of $\bm{r}_{k}$ with a vector in $\calK_k$, and by \Cref{theorem: CG-error-min}, any such inner product is zero. Effectively, this means
    \[
        \| \bm{e} \|_{\bm{A}}^{2} = \bm{e}_{k}^{\intercal} \bm{A} \bm{e}_{k} + \left( \Delta \bm{x} \right)^{\intercal} \bm{A} \left( \Delta \bm{x} \right).
    \]
    Only the second of these terms depends on $\Delta \bm{x}$, since $\bm{A}$ is positive definite, that term is greater than or equal to $0$, attaining the value of $0$ if and only if $\Delta \bm{x} = \bm{0}$, that is, $\bm{x}_k = \bm{x}$. Thus $\| \bm{e} \|_{\bm{A}}$ is minimal if and only if $\bm{x}_{k} = \bm{x}$, as claimed.
\end{proof}

As seen in \Cref{eq: x_ast_via_cayley} Krylov subspace methods, in some sense, aim to find a polynomial $p_n \in P_n$ which minimizes $\| p_n \left( \bm{A} \right) \| = \| \sum_{k=0}^{n-1} \alpha_{k + 1} \bm{A}^{k} \bm{r}_{0} \|$, where $P_n$ is the set of polynomials with degree less than or equal to $n$ with $p(\bm{0}) = \Id$. In the context of the CG algorithm, this is equivalent to finding a polynomial $p_n \in P_n$ which instead minimizes
\begin{equation} \label{eq: CG-poly-min}
    \| p_n \left( \bm{A} \right) \bm{e}_{0} \|_{\bm{A}}.
\end{equation}
From \Cref{theorem: equiv-CG-spaces}, we can derive the following convergence theorem.

\begin{thm} \label{theorem: CG-convg}
    If the CG iteration has not already converged before step $k$ (that is, $\| \bm{r}_{k-1} \| > 0 $), then \Cref{eq: CG-poly-min} has a unique solution $p_n \in P_n$, and the iterate $\bm{x}_{k}$ has error $\bm{e}_k = p_n \left( \bm{A} \right) \bm{e}_{0}$ for this same polynomial $p_n$. Consequently we have
    \begin{equation} \label{eq: CG-err-bounds}
        \frac{\| \bm{e}_k \|_{\bm{A}}}{\| \bm{e}_{0} \|_{\bm{A}}} = \inf_{p \in P_n} \frac{\| p \left( \bm{A} \right) \bm{e}_{0} \|_{\bm{A}}}{\| \bm{e}_{0} \|_{\bm{A}}} \leq \inf_{p \in P_n} \max_{\lambda \in \Lambda \left( \bm{A} \right)} \left| p (\lambda) \right|
    \end{equation}
    \cite{TrefethenLloydN.LloydNicholas1997Nla/}*{page 298}.
\end{thm}

\begin{proof}
    From \Cref{theorem: equiv-CG-spaces} it follows that $\bm{e}_{k} = p \left( \bm{A} \right) \bm{e}_{0}$ for some $p \in P_n$. The equality in \Cref{eq: CG-err-bounds} is a consequence of this and \Cref{theorem: CG-error-min}. As for the inequality in \Cref{eq: CG-err-bounds}, if $\bm{e}_{0} = \sum_{j=1}^{m} a_j \bm{v}_j$ is an expansion of $\bm{e}_{0}$ in orthonormal eigenvectors of $\bm{A}$, then we have $p \left( \bm{A} \right) \bm{e}_{0} = \sum_{j=1}^{m} a_j p \left( \lambda_j \right) \bm{v}_{j}$ and thus
    \[
        \| \bm{e}_{0} \|_{\bm{A}}^{2} = \sum_{j=1}^{m} a_{j}^{2} \lambda_j, \qquad \| p \left( \bm{A} \right) \bm{e}_0 \|_{\bm{A}}^{2} = \sum_{j=1}^{m} a_{j}^{2} \lambda_j \left( p \left( \lambda_j \right) \right)^{2}.
    \]
    These identities indicate $\| p \left( \bm{A} \right) \bm{e}_{0} \|_{\bm{A}}^{2} / \| \bm{e}_{0} \|_{\bm{A}}^{2} \leq \max_{\lambda \in \Lambda \left( \bm{A} \right)} \left| p (\lambda) \right|^{2}$, which implies the inequality in question.
\end{proof}