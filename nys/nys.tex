\section{The Nystrom Method}\label{Chapter2}
In chapter \ref{Chapter1} we saw that GP regression and classification relied on a Gram matrix (see definition \ref{defe: Gram_Matrix}) to produce predictions. Unfortunately, from a computational perspective, constructing the Gram matrix for a data set $\calD = \left\{ \left( \bm{x}_i , y_{i} \right) \right\}_{i=1}^{n}$ brings about a nasty bottle neck owed by the $\calO \left( n^2 \right)$ kernel evaluations. Even before the rise of ML, there has been a lot of research devoted to creating numerical methods that quickly construct a low rank approximation of large matrices, $\bm{A}$, which ordinarily are a computational burdened to build exactly. These methods are centered around the idea of capturing the columns space of the matrix that best describes the the action of $\bm{A}$ as an operator. For lack of a better explanation, Mahoney gives a fantastic summary as to why the column space is of paramount importance in these approximation techniques
\begin{center}
    \emph{"To understand why sampling columns (or rows) from a matrix is of interest,recall that matrices are “about” their columns and rows that is, linear combinations are taken with respect to them; one all but understands a given matrix if one understands its column space, row space, and null
        spaces; and understanding the subspace structure of a matrix sheds a great deal of light on the linear transformation that the matrix represents."} \cite{DBLP:journals/corr/abs-1104-5557}*{page 13}
\end{center}
Moreover, this class of algorithms lend very nice forms when $\bm{A}$ possess positive definite structure, which is exactly the case for our Gram matrix.

\subsection{The Nystrom Method}\label{Section2.1}

Attempting to compute an entire kernel matrix can prove to be quite a computational headache, prompting us to seek estimative alternatives. The approximation techniques studied in this chapter have been spurred on by the John-Lindenstrauss lemma stated in lemma \ref{lem: John-Lindenstrauss}.

\begin{lem}[John-Lindenstrauss] \label{lem: John-Lindenstrauss}
    Given $0 < \varepsilon < 0$, any set of n points, $X$, in a high dimensional Euclidean space can be embedded into a $\ell-$dimensional Euclidean space where $\ell = \calO \left( \ln (n) \right)$ via some linear map $\bm{\Omega} \in \RR^{n \times \ell}$ which satisfies
    \[
        (1 - \varepsilon) \norm{\bm{u} - \bm{v}}^2 \leq \norm{\bm{\Omega} \bm{u} - \bm{\Omega} \bm{v}}^2 \leq \varepsilon \norm{\bm{u} - \bm{v}}^2
    \]
    for any $\bm{u}, \bm{v} \in X$ \cite{DBLP:journals/corr/abs-1104-5557}*{page 15}.
\end{lem}

The John-Lindenstrauss lemma tells us that $\bm{Q} \bm{Q}^{\ast} \bm{A}$ will serve as a good approximation to some matrix $\bm{A}$ where $\bm{Q} \bm{Q}^{\ast}$, in some sense, projects onto some rank-$k$ subspace of $\bm{A}$'s column space. This is because if $\bm{Q} \bm{Q}^{\ast}$ closesly matches the behavior of $\bm{\Omega}$ from the lemma then the pair-wise distances between points before and after applying $\bm{Q} \bm{Q}^{\ast}$ should remain fairly similar. To state this a little more explicitly, for a matrix $\bm{A}$ and a positive error tolerance $\varepsilon$ we seek a matrix $\bm{Q} \in \RR^{n \times k_{\varepsilon}}$ with orthonormal columns such that
\begin{equation*}
    \norm{\bm{A} - \bm{Q} \bm{Q}^{\ast} \bm{A}}_{F} \leq \varepsilon
\end{equation*}
which can be expressed a more short hand notation as
\begin{equation} \label{eq: nys-Q-cond}
    \bm{A} \simeq \bm{Q} \bm{Q}^{\ast} \bm{A}.
\end{equation}
This is commonly called the {\it fixed precision approximation problem}. Although, to simplify algorithmic development, a value of $k$ is specified in advanced (instead of $\varepsilon$, thus removing $k$'s dependence on $\varepsilon$) which is instead given the name {\it fixed rank problem}. Within the fixed rank problem framework, when $\bm{A}$ is hermitian, the matrix $\bm{Q} \bm{Q}^{\ast}$ acts as a good projection for both the columns and row space of $\bm{A}$ so that we have both $\bm{A} \simeq \bm{Q} \bm{Q}^{\ast} \bm{A}$ and $\bm{A} \simeq \bm{A} \bm{Q} \bm{Q}^{\ast}$ so that
\begin{equation} \label{eq: hermitian-apprx}
    \bm{A} \simeq \bm{Q} \bm{Q}^{\ast} \left( \bm{A} \right) \simeq \bm{Q} \bm{Q}^{\ast} \bm{A} \bm{Q} \bm{Q}^{\ast}.
\end{equation}
Furthermore, if $\bm{A}$ is positive semi-definite we can improve the quality of our approximation of our approximation at almost no additional cost \cite{halko2011finding}*{page 32}. Using the approximation from \ref{eq: hermitian-apprx}
\begin{align} \label{eq: nys-apprx}
    \bm{A} & \simeq \bm{Q} \left( \bm{Q}^{\ast} \bm{A} \bm{Q} \right) \bm{Q}^{\ast} \nonumber                                                                                            \\
           & = \bm{Q} \left( \bm{Q}^{\ast} \bm{A} \bm{Q} \right) \left( \bm{Q}^{\ast} \bm{A} \bm{Q} \right)^{\dagger} \left( \bm{Q}^{\ast} \bm{A} \bm{Q} \right) \bm{Q}^{\ast} \nonumber \\
           & \simeq \left( \bm{A} \bm{Q} \right) \left( \bm{Q}^{\ast} \bm{A} \bm{Q} \right)^{\dagger} \left( \bm{Q}^{\ast} \bm{A} \right).
\end{align}
This is known as the Nystrom method. A general Nystrom framework is presented in Algorithm TODO.

\subsection{Column Probabilities}\label{Section2.2}

Recall that the Nystrom method from Algorithm \ref{alg: nys-col-samp} is largely dependent on the random matrix multiplication algorithm (Algorithm \ref{alg: rand-mat-mult}) to produce a suitable sketching matrix. Moreover, improvements in the sketching matrix produced by the random matrix multiplication algorithmare reflected as smaller errors in the Nystrom approximation. Now, consider using the random matrix multiplication algorithm to approximate $\bm{A} \bm{A}^{\intercal}$ by setting $\bm{B} = \bm{A}$. The output is an approximation of the form
\begin{equation*}
    \bm{A} \bm{A}^{\intercal} \simeq \bm{C} \bm{C}^{\intercal} = \bm{C} \bm{R}.
\end{equation*}
The probability distribution
\begin{equation*} \label{eq: col-probs}
    p_i = \frac{\norm{\bm{A}_{(:,i)}}_2^2}{\norm{\bm{A}}_F}.
\end{equation*}
aims to minimize the error between $\bm{A} \bm{A}^{\intercal}$ and the approximation $\bm{C} \bm{C}^{\intercal}$. As a result, we should expect that $\bm{C}$ becomes a better estimate for $\bm{A} \bm{S}$, implying that the sketching matrix, $\bm{S}$, is using a better sampling and landmark selection criteria. Drineas and Mahoney give a precise bound on this error presented in theorem \ref{thm: col-pro-bounds} \cite{JMLR:v6:drineas05a}*{page 2158}.

\begin{thm} \label{thm: col-pro-bounds}
    Given $\bm{A} \in \RR^{m \times n}$, $1 \leq c \leq n$ and the probability distribution $\left\{ p_i \right\}_{i=1}^{n}$ described in equation \ref{eq: col-probs}. Construct $\bm{C}$ using algorithm \ref{alg: rand-mat-mult}, then
    \[
        \EE \left[ \norm{\bm{A} \bm{A}^{\intercal} - \bm{C} \bm{C}^{\intercal}}_{F} \right] \leq \frac{1}{\sqrt{c}} \norm{\bm{A}}^2_{F}.
    \]
\end{thm}

To show theorem \ref{thm: col-pro-bounds}, we can actually show something a little more general.

\subsection{Leverage Scores}\label{Section2.3}

\subsubsection{Statistical Leverage Scores}
Our next distribution originates from the least-squares problem. Breifly, in an over constrained least-squares problem, where $\bm{A} \in \RR^{n \times m}$, $\bm{b} \in \RR^{n}$, for $m \ll n$ there usually is not any $\bm{x} \in \RR^{m}$ for which $\bm{A} \bm{x} = \bm{b}$. Instead, alternative criteria are used to seek a $\bm{x}$ which in some way comes closest to satisfying this equality. Perhaps one of the more popular criterion is to minimize the $\ell^2-$norm, that is
\[
    \bm{x}_{opt} = \argmin_{x} \norm{\bm{A} \bm{x} - \bm{b}}
\]
\cite{DBLP:journals/corr/abs-1104-5557}*{page 19-21}. This is what the least-squares problem is. The optimal value for $\bm{x}$ can be solved as $\bm{x}_{opt} = \left( \bm{A}^{\ast} \bm{A} \right)^{-1} \bm{A}^{\ast} \bm{b}$. The least-squares solution is commonly used to find the best weight vector (in this case $\bm{x}$) for a linear model, given a dataset. Fitted or predicted values are usually obtained from $\hat{\bm{b}} = \bm{H} \bm{b}$ where the projector onto the column space of $\bm{A}$
\[
    \bm{H} = \bm{A} \left( \bm{A}^{\intercal} \bm{A} \right)^{-1} \bm{A}^{\intercal}
\]
is sometimes referred to as the {\it hat matrix}. The element $\bm{H}_{ij}$ has the direct interpretation as the influence or statistical leverage exerted on $\hat{\bm{b}}_i$. Thus, examining the hat matrix can reveal to us columns of $\bm{A}$ which bear a significant impact on $\hat{\bm{b}}$ \cite{HoaglinDavidC1978THMi}*{page 17}. Relatedly, if the element $\bm{H}_{ii}$ is particularly large this is indicative of the $i^{th}$ column of $\bm{A}$ having great influence in determining values of $\hat{\bm{b}}$, justifying the interpretation of $\bm{H}_{ii}$ as statistical leverage scores.

The statistical leverage scores are maximised when $\bm{A}_{(:,i)}$ is linearly independent from $\bm{A}$'s other columns and decreases when it aligns with many other columns or when the value of $\norm{\bm{A}_{(:,i)}}$ is small \cite{DBLP:journals/corr/CohenMM15}*{page 5}. To compute the statistical leverage scores, if $\bm{A} = \bm{U} \bm{\Sigma} \bm{V}^{\intercal}$ is the SVD of $\bm{A}$, then
\begin{align*}
    \bm{H}_{ii} & = \left( \bm{A} \left( \bm{A}^{\intercal} \bm{A} \right)^{-1} \bm{A}^{\intercal} \right)_{ii} \\
                & = \left( \bm{U} \bm{\Sigma}^{2} \left( \bm{\Sigma}^{2} \right)^{-1} \bm{U} \right)_{ii}       \\
                & = \norm{\bm{U}_{(i,:)}}_2^2.
\end{align*}
Note that $\bm{H}_{ii}$ may not constitute as a probability distribution, as may the other leverage scores which we will soon discuss. This is easily enough fixed by normalisation. The idea behind using statistical leverage scores as a probability distribution in the Nystrom method is that statistical leverage scores help us priorities selecting columns that are more linearly independent from other columns so that the range of our approximate more closely aligns with the range of our original $\bm{A}$.

\subsubsection{Rank$-k$ Statistical Leverage Scores}

We can generalize this notion of statistical leverage scores to include lower rank approximations. As before let $\bm{A} = \bm{U} \bm{\Sigma} \bm{V}^{\intercal}$ be the SVD of $\bm{A}$. The SVD can be partitioned as
\begin{equation*}
    \bm{U} = \left[ \bm{U}_1 , \bm{U}_2 \right] \qquad \bm{\Sigma} =
    \begin{bmatrix}
        \bm{\Sigma}_1 &               \\
                      & \bm{\Sigma}_2
    \end{bmatrix}
    \qquad
    \bm{V} = \left[ \bm{V}_1 , \bm{V}_2 \right].
\end{equation*}
Here $\bm{U}_1$ contains the first $k$ columns of $\bm{U}$, $\bm{V}_1$ the first $k$ rows of $\bm{V}$ and $\bm{\Sigma}_1$ is a $k \times k$ matrix containing the top $k$ singular values across its diagonal. The matrix $\bm{A}_k = \bm{U}_1 \bm{\Sigma}_1 \bm{V}_1$ then forms the best rank$-k$ approximation to $\bm{A}$. The statistical leverage scores relative to the best rank$-k$ approximation are again $\bm{H}_{ii}$, but this time $\bm{H}$ is computed only using the best rank$-k$ approximation of $\bm{A}$, that is $\bm{A}_k$. These low rank scores can be evaluated as
\begin{equation*} \label{eq: lev-scrs-1}
    \ell_i^k \triangleq \left( \bm{A}_{k} \left( \bm{A}_{k}^{\intercal} \bm{A}_{k} \right)^{-1} \bm{A}_{k}^{\intercal} \right)_{ii} = \norm{\left( \bm{U}_1 \right)_{(i,:)}}_2^2.
\end{equation*}
What makes low-rank statistical leverage scores particularly appealing is that they can be approximated quickly with a truncated SVD \cite{DBLP:journals/corr/abs-1303-1849}*{pages 3-4}.

\subsubsection{Ridge Leverage Scores}

The low rank leverage scores we saw in equation \ref{eq: lev-scrs-1} will not always be unique and can be sensitive to perturbations \cite{DBLP:journals/corr/CohenMM15}*{page 6}. As you could guess, the prediction results can very drastically when $\bm{A}$ is modified slightly or when we only have access to partial information on the matrix. This largely undermines the the possibility of computing good quality low rank approximations of statistical leverage scores. This motivates the next class of leverage score, ridge leverage scores. Ridge leverage scores are similar to statistical leverage scores although a ridge regression term (hence the name) is within the hat matrix for a given regularization parameter $\lambda$. The $\lambda-$ridge leverage score is defined as
\begin{equation*}
    r_{i}^{\lambda} \triangleq \left( \bm{A} \left( \bm{A}^{\intercal} \bm{A} + \lambda \Id_{n \times n} \right)^{-1} \bm{A}^{\intercal} \right)_{ii}.
\end{equation*}
A regularization parameter of
\begin{equation*} \label{eq: rid-lev-reg-param}
    \lambda = \frac{\norm{\bm{A} - \bm{A}_k}_F^{2}}{k}
\end{equation*}
is typically used since this choice of $\lambda$ will guarantee that the sum of the ridge leverage scores (keep in mind that the raw ridge leverage do not necessarily form a probability distribution) is bounded by $2k$, stated more formally in lemma \ref{lem: rid-lev-reg-param-bound}.
\begin{lem} \label{lem: rid-lev-reg-param-bound}
    When using a regularization parameter of $\lambda = \frac{\norm{\bm{A} - \bm{A}_k}_F^{2}}{k}$ we have $\sum_{i=1}^{n} r_{i}^{\lambda} \leq 2k$ \cite{DBLP:journals/corr/CohenMM15}*{pages 6-7}.
\end{lem}
From now on (unless otherwise stated) the regularization parameter seen in \ref{eq: rid-lev-reg-param} will always be used for ridge leverage scores where the notation
\begin{equation*}
    r_{i}^{k} \triangleq \left( \bm{A} \left( \bm{A}^{\intercal} \bm{A} + \left( \frac{\norm{\bm{A} - \bm{A}_k}_F^{2}}{k} \right) \Id_{n \times n} \right)^{-1} \bm{A}^{\intercal} \right)_{ii}
\end{equation*}
will be used to show that the best rank$-k$ matrix is used in the regularization parameter. Adding regularization to the hat matrix offers a smoother alternative which 'washes out' small singular directions meaning they are sampled with proportionally lower probability \cite{DBLP:journals/corr/CohenMM15}*{page 6}.