\section{The Nystrom Method}\label{Chapter2}
In chapter \ref{Chapter1} we saw that GP regression and classification relied on a Gram matrix (see \Cref{defe: Gram_Matrix}) to produce predictions. Unfortunately, from a computational perspective, constructing the Gram matrix for a data set $\calD = \left\{ \left( \bm{x}_i , y_{i} \right) \right\}_{i=1}^{n}$ brings about a nasty bottle neck on account of the $\calO \left( n^2 \right)$ kernel evaluations. Even before the rise of ML, a lot of research devoted to creating numerical methods that quickly construct a low rank approximation of large matrices, $\bm{A}$, which ordinarily are a computational burden to build exactly. These methods are based on the idea of capturing the columns space of the matrix that best describes the the action of $\bm{A}$ as an operator. Mahoney provides an enlightened summary as to why the column space is of paramount importance in these approximation techniques
\begin{center}
    \emph{"To understand why sampling columns (or rows) from a matrix is of interest,recall that matrices are “about” their columns and rows that is, linear combinations are taken with respect to them; one all but understands a given matrix if one understands its column space, row space, and null
        spaces; and understanding the subspace structure of a matrix sheds a great deal of light on the linear transformation that the matrix represents."} \cite{DBLP:journals/corr/abs-1104-5557}*{page 13}
\end{center}
Moreover, this class of algorithms lend very nice forms when $\bm{A}$ possess positive definite structure, which is exactly the case for Gram matrices.

\subsection{The Nystrom Method}\label{Section2.1}

Attempting to compute an entire kernel matrix can be a computational headache prompting an investigation of estimative alternatives. The approximation techniques studied in this chapter have been spurred on by the John-Lindenstrauss lemma stated in lemma \ref{lem: John-Lindenstrauss}.

\begin{lem}[John-Lindenstrauss] \label{lem: John-Lindenstrauss}
    Given $0 < \varepsilon < 0$, any set of n points, $X$, in a high dimensional Euclidean space can be embedded into a $\ell-$dimensional Euclidean space where $\ell = \calO \left( \ln (n) \right)$ via some linear map $\bm{\Omega} \in \RR^{n \times \ell}$ which satisfies
    \[
        (1 - \varepsilon) \norm{\bm{u} - \bm{v}}^2 \leq \norm{\bm{\Omega} \bm{u} - \bm{\Omega} \bm{v}}^2 \leq \varepsilon \norm{\bm{u} - \bm{v}}^2
    \]
    for any $\bm{u}, \bm{v} \in X$ \cite{DBLP:journals/corr/abs-1104-5557}*{page 15}.
\end{lem}

The John-Lindenstrauss lemma tells us that $\bm{Q} \bm{Q}^{\ast} \bm{A}$ will serve as a good approximation to some matrix $\bm{A} \in \RR^{n \times m}$ where $\bm{Q} \bm{Q}^{\ast}$, in some sense, projects onto a rank-$k$ subspace of $\bm{A}$'s column space. This is because if $\bm{Q} \bm{Q}^{\ast}$ closesly matches the behavior of $\bm{\Omega}$ from the lemma then the pair-wise distances between points before and after applying $\bm{Q} \bm{Q}^{\ast}$ should remain fairly similar. To state this a little more explicitly, for a matrix $\bm{A}$ and a positive error tolerance $\varepsilon$ we seek a matrix $\bm{Q} \in \RR^{n \times k_{\varepsilon}}$ with orthonormal columns such that
\begin{equation*}
    \norm{\bm{A} - \bm{Q} \bm{Q}^{\ast} \bm{A}}_{F} \leq \varepsilon
\end{equation*}
which can be expressed in a more short hand notation as
\begin{equation} \label{eq: nys-Q-cond}
    \bm{A} \simeq \bm{Q} \bm{Q}^{\ast} \bm{A}.
\end{equation}
This is commonly called the {\it fixed precision approximation problem}. To simplify algorithmic development, a value of $k$ is specified in advance (instead of $\varepsilon$, thus removing $k$'s dependence on $\varepsilon$) which is instead given the name {\it fixed rank problem}. Within the fixed rank problem framework, when $\bm{A}$ is hermitian, the matrix $\bm{Q} \bm{Q}^{\ast}$ acts as a good projection for both the columns and row space of $\bm{A}$ so that we have both $\bm{A} \simeq \bm{Q} \bm{Q}^{\ast} \bm{A}$ and $\bm{A} \simeq \bm{A} \bm{Q} \bm{Q}^{\ast}$ meaning
\begin{equation} \label{eq: hermitian-apprx}
    \bm{A} \simeq \bm{Q} \bm{Q}^{\ast} \left( \bm{A} \right) \simeq \bm{Q} \bm{Q}^{\ast} \bm{A} \bm{Q} \bm{Q}^{\ast}.
\end{equation}
Furthermore, if $\bm{A}$ is positive semi-definite we can improve the quality of our approximation of our approximation at almost no additional cost \cite{halko2011finding}*{page 32}. Using the approximation from \ref{eq: hermitian-apprx}
\begin{align} \label{eq: nys-apprx}
    \bm{A} & \simeq \bm{Q} \left( \bm{Q}^{\ast} \bm{A} \bm{Q} \right) \bm{Q}^{\ast} \nonumber                                                                                            \\
           & = \bm{Q} \left( \bm{Q}^{\ast} \bm{A} \bm{Q} \right) \left( \bm{Q}^{\ast} \bm{A} \bm{Q} \right)^{\dagger} \left( \bm{Q}^{\ast} \bm{A} \bm{Q} \right) \bm{Q}^{\ast} \nonumber \\
           & \simeq \left( \bm{A} \bm{Q} \right) \left( \bm{Q}^{\ast} \bm{A} \bm{Q} \right)^{\dagger} \left( \bm{Q}^{\ast} \bm{A} \right).
\end{align}
This is known as the Nystrom method. Since any Gram matrix is positive semi-definite, we can always applied the Nystrom method to find an approximation to it. A general Nystrom framework is presented in Algorithm \ref{alg: nys-gen}.

{\centering
\begin{minipage}{.85\linewidth}
    \begin{algorithm}[H]
        \caption{General Nystrom Framework}
        \label{alg: nys-gen}
        \SetAlgoLined
        \DontPrintSemicolon
        \SetKwInOut{Input}{input}\SetKwInOut{Output}{output}

        \Input{A positive semi-definite matrix $\bm{A} \in \RR^{n \times m}$, a matrix $\bm{Q} \in {n \times k}$ that satisfies \ref{eq: nys-Q-cond}.}
        \Output{A rank $k$ approximation $\bm{\overline{A}} \simeq \bm{A}$.}
        \BlankLine
        $\bm{C} = \bm{A} \bm{Q}$\;
        $\bm{W} = \bm{Q}^{\ast} \bm{C}$\;
        \Return{$\bm{C} \bm{W}^{\dagger} \bm{C}^{\ast}$}
        \BlankLine
    \end{algorithm}
\end{minipage}
\par}

However, Algorithm \ref{alg: nys-gen} assumes that $\bm{Q}$ has already been computed. Naturally, the next question is then how to efficiently construct a suitable matrix $\bm{Q}$ that satisfies equation \ref{eq: nys-Q-cond}? We can do this through a very popular column sampling technique ubiquitous in numerical linear algebra literature. This technique has been driven by theorem \ref{lem: best-col-sel}.

\begin{thm} \label{lem: best-col-sel}
    Every $\bm{A} \in \RR^{n \times m}$ matrix contains a $k-$column submatrix $\bm{C}$ for which
    \[
        \norm{\bm{A} - \bm{C} \bm{C}^{\dagger} \bm{A}}_{F} \leq \sqrt{1 + k(n-k)} \cdot \norm{\bm{A} - \bm{A}_{k}}
    \]
    where $\bm{A}_k$ is the best rank$-k$ approximation of $\bm{A}$ \cite{halko2011finding}*{page 11}.
\end{thm}
Before we delve further into this column sampling Nystrom method, we must first cover the random matrix multiplication algorithm which serves as a backbone for this technique. Therefore, let $\bm{A} \in \RR^{n \times m}$ be a target matrix we would like to approximate and suppose that $\bm{A}$ can be represented as the sum of 'simpler' (for example, sparse or low-rank) matrices, $\bm{A}_i$, so that
\begin{equation} \label{eq: A-as-summands}
    \bm{A} = \sum_{i=1}^{I} \bm{A}_i .
\end{equation}
The basic idea is to consider a Monte-Carlo approximation of equation \ref{eq: A-as-summands} that randomly selects $\bm{A}_i$ according to the distribution $\left\{ p_i \right\}_{i=1}^{I}$ to give an estimate
\begin{equation}
    \bm{A} \simeq \frac{1}{c} \sum_{t=1}^{c} p_{t_i}^{-1} \bm{A}_{t_i}
\end{equation}
where $c$ is the number of samples and each summand is rescaled by a factor of $p_{t_i}^{-1}$ to ensure our estimate is unbiased \cite{martinsson2021randomized}*{pages 24-27}. The random matrix multiplication algorithm works by attempting to find a Monte-Carlo estimate for $\bm{A}\bm{B}$, where $\bm{A} \in \RR^{n \times I}$ and $\bm{B} \in \RR^{I \times m}$. Recall that any matrix multiplication can be written in its outter product form
\begin{equation*}
    \bm{A} \bm{B} = \sum_{i=1}^{I} \bm{A}_{(:,i)} \bm{B}_{(i,:)}
\end{equation*}
\cites{Roosta2020, doi:10.1137/S0097539704442684}. A straight forward way to approximate this using the Monte-Carlo estimate is to simply set each $\bm{A}_i$ in \ref{eq: A-as-summands} to the corresponding rank$-1$ outter product summand $\bm{A}_{(:,i)} \bm{B}_{(i,:)}$. This justifies the random matrix multiplication algorithm seen in Algorithm \ref{alg: rand-mat-mult} \cite{drineas2017lectures}*{page 16}.

{\centering
\begin{minipage}{.85\linewidth}
    \begin{algorithm}[H]
        \caption{Random Matrix Multiplication}
        \label{alg: rand-mat-mult}
        \SetAlgoLined
        \DontPrintSemicolon
        \SetKwInOut{Input}{input}\SetKwInOut{Output}{output}

        \Input{$\bm{A} \in \RR^{n \times I}$ and $\bm{B} \in \RR^{I \times m}$, the number of samples $1 \leq c \leq I$ and a probability distribution over $I$, $\left\{ p_i \right\}_{i=1}^{I}$ .}
        \Output{Matricies $\bm{C} \in \RR^{n \times c}$ and $\bm{R} \in \RR^{c \times m}$ such that $\bm{CR} \simeq \bm{AB}$.}
        \BlankLine
        \For{$t = 1 , \ldots , c$}{
            Pick $i_t \in \left\{ 1, \ldots , I \right\}$ with $\PP \left[ i_t = k \right] = p_k$, independently and with replacement.\;
            $\bm{C}_{(:,t)} = \frac{1}{\sqrt{cp_{i_t}}} \bm{A}_{(:,i_t)}$\;
            $\bm{R}_{(:,t)} = \frac{1}{\sqrt{cp_{i_t}}} \bm{B}_{(i_t,:)}$\;
        }
        \Return{$\bm{C} \bm{R} = \sum_{t=1}^{c} \frac{1}{cp_{i_t}} \bm{A}_{(:,i_t)} \bm{B}_{(i_t,:)}$}
        \BlankLine
    \end{algorithm}
\end{minipage}
\par}

This algorithm makes this idea a little more precise, taking in the two matrices to multiply together as well as a probability distribution over $I$ to provide an estimate for $\bm{A}\bm{B}$ of the form
\begin{equation*}
    \bm{A}\bm{B} \simeq \sum_{t=1}^{c} \frac{1}{cp_{i_t}} \bm{A}_{(:,i_t)} \bm{B}_{(i_t,:)}.
\end{equation*}
Equivalently, the above can be restated as the product of two matrices $\bm{C} \bm{R}$ formed by Algorithm \ref{alg: rand-mat-mult}, where $\bm{C}$ consists of $c$ randomly selected rescaled columns of $\bm{A}$ and $\bm{R}$ is $c$ randomly selected rescaled rows of $\bm{B}$. Notice that
\begin{equation*}
    \bm{C} \bm{R} = \sum_{t=1}^{c} \bm{C}_{(:,i_t)} \bm{R}_{(i_t,:)} = \sum_{t=1}^{c} \left( \frac{1}{\sqrt{cp_{i_t}}} \bm{A}_{(:,i_t)} \right) \left( \frac{1}{\sqrt{cp_{i_t}}} \bm{B}_{(i_t,:)} \right) = \frac{1}{c} \sum_{t=1}^{c} \frac{1}{p_{i_t}} \bm{A}_{(:,i_t)} \bm{B}_{(i_t,:)}.
\end{equation*}
To make development easier, let us define a sampling and rescaling matrix, usually referred to as a sketching matrix, $\bm{S} \in \RR^{n \times c}$ to be the the matrix with elements $\bm{S}_{i_t , t} = 1 \sqrt{c p_{i_t}}$ if the $i_t$ column of $\bm{A}$ is chosen during the $t^{th}$ trial and all other entries of $\bm{S}$ are set to $0$. Then we have
\begin{equation*}
    \bm{C} = \bm{A} \bm{S} \quad \text{and} \quad \bm{R} = \bm{S}^{\intercal} \bm{B}
\end{equation*}
so that
\begin{equation} \label{eq: nys-sketch-apprx}
    \bm{C} \bm{R} = \bm{A} \bm{S} \bm{S}^{\intercal} \bm{B} \simeq \bm{A} \bm{B}.
\end{equation}
Notice that $\bm{S}$ is generally a very sparse matrix and therefore is generally not constructed explicitly and instead the matrix products $\bm{A} \bm{S}$ and $\bm{S}^{\intercal} \bm{B}$ are done through row and column rescaling \cite{drineas2017lectures}*{page 17}. Lemma \ref{lem: rmm-exp-var-bds} provides some bounds on $\bm{C} \bm{R}$ as an estimate for $\bm{A}\bm{B}$.

\begin{lem} \label{lem: rmm-exp-var-bds}
    Let $\bm{C}$ and $\bm{R}$ be constructed as described in Algorithm \ref{alg: rand-mat-mult}, then
    \[
        \EE \left[ \left( \bm{C} \bm{R} \right)_{ij} \right] = \left( \bm{A}\bm{B} \right)_{ij}.
    \]
    That is, $\bm{C} \bm{R}$ is an unbiased estimate of $\bm{A}\bm{B}$. Furthermore
    \[
        \VV \left[ \left( \bm{C} \bm{R} \right)_{ij} \right] \leq \frac{1}{c} \sum_{k=1}^{n} \frac{\bm{A}_{ik}^2 \bm{B}_{kj}^2}{p_k}.
    \]
\end{lem}

\begin{proof}
    For some fixed pair $i,j$ for each $t = 1 , \ldots , c$ define $\bm{X}_t = \left( \frac{\bm{A}_{(:,i_t)} \bm{B}_{(i_t,:)}}{c p_{i_t}} \right)_{ij} = \frac{\bm{A}_{(i,i_t)} \bm{B}_{(i_t,j)}}{c p_{i_t}}$. Thus, for any $t$,
    \begin{align*}
        \EE \left[ \bm{X}_t \right] = \sum_{k=1}^{n} p_k \frac{\bm{A}_{ik} \bm{B}_{kj}}{cp_k} = \frac{1}{c} \sum_{k=1}^{n} \bm{A}_{ik} \bm{B}_{kj} = \frac{1}{c} \left( \bm{A} \bm{B} \right)_{ij}.
    \end{align*}
    Since we have $\left( \bm{C} \bm{R} \right)_{ij} = \sum_{t=1}^{c} \bm{X}_t$, it follows that
    \begin{align*}
        \EE \left[ \left( \bm{C} \bm{R} \right)_{ij} \right] = \EE \left[ \sum_{t=1}^{c} \bm{X}_t \right] = \sum_{t=1}^{c} \left[ \EE \bm{X}_t \right] = \left( \bm{A} \bm{B} \right)_{ij}.
    \end{align*}
    Hence, $\bm{C} \bm{R}$ is an unbiased estimator of $\bm{A}\bm{B}$, regardless of the choice of the sampling probabilities. Using the fact that $\left( \bm{C} \bm{R} \right)_{ij}$ is the sum of $c$ independent random variables, we get
    \begin{equation*}
        \VV \left[ \left( \bm{C} \bm{R} \right)_{ij} \right] = \VV \left[ \sum_{t=1}^{c} \bm{X}_t \right] = \sum_{t=1}^{c} \VV \left[ \bm{X}_t \right].
    \end{equation*}
    Using the fact $\VV \left[ \bm{X}_t \right] \leq \EE \left[ \bm{X}_t^2 \right] = \sum_{k=1}^n \frac{\bm{A}_{ik}^2 \bm{B}_{kj}^2}{c^2 p_k}$, we get
    \begin{equation*}
        \VV \left[ \left( \bm{C} \bm{R} \right)_{ij} \right] = \sum_{t=1}^{c} \VV \left[ \bm{X}_t \right] \leq c \sum_{k=1}^n \frac{\bm{A}_{ik}^2 \bm{B}_{kj}^2}{c^2 p_k} = \frac{1}{c} \frac{\bm{A}_{ik}^2 \bm{B}_{kj}^2}{p_k}.
    \end{equation*}
\end{proof}

So how does this help us with the Nystrom method? Consider using the random matrix multiplication algorithm to approximate the matrix multiplication of a Gram matrix $\bm{K} \in \RR^{n \times n}$ and $\Id_{n \times n}$. Equation \ref{eq: nys-sketch-apprx} gives

\begin{equation*}
    \bm{K} \bm{S} \bm{S}^{\intercal} \Id_{n \times n} = \bm{K} \bm{S} \bm{S}^{\intercal} \simeq \bm{K}.
\end{equation*}

We see now that the sketching matrix produced by Algorithm \ref{alg: rand-mat-mult} provides a sketching matrix $\bm{S}$ that satisfies the properties of $\bm{Q}$ from equation \ref{eq: nys-Q-cond} meaning that $\bm{S}$ can be used in place of $\bm{Q}$ within the Nystrom estimate from equation \ref{eq: nys-apprx}. These ideas are used together in Algorithm \ref{alg: nys-col-samp} that uses the column sampling technique from Algorithm \ref{alg: rand-mat-mult} together with the general Nystrom framework (Algorithm \ref{alg: nys-gen}) to provide a new column sampling Nystrom method to approximate a Gram matrix for a provided dataset and probability distribution \cites{JMLR:v6:drineas05a,DBLP:journals/corr/abs-1303-1849}.

{\centering
\begin{minipage}{.85\linewidth}
    \begin{algorithm}[H]
        \caption{Nystrom Method via Column Sampling}
        \label{alg: nys-col-samp}
        \SetAlgoLined
        \DontPrintSemicolon
        \SetKwInOut{Input}{input}\SetKwInOut{Output}{output}

        \Input{Data matrix $\bm{X} = \left[ \bm{x}_1 , \ldots , \bm{x}_n \right]^{\intercal} \in \RR^{n \times d}$, the number of samples $1 \leq c \leq n$ and a probability distribution over $n$, $\left\{ p_i \right\}_{i=1}^{n}$ .}
        \Output{An approximation of the Gram matrix corresponding to $\bm{X}$, that is $\overline{\bm{K}} \simeq \bm{K}$ where $\bm{K}_{ij} = k \left( \bm{x}_i , \bm{x}_j \right)$.}
        \BlankLine
        Initialize $\bm{C}$ as an empty $n \times c$ matrix.\;
        Pick $c$ columns with the probability of choosing the $k^{th}$ column ($1 \leq k \leq n$) as $\PP \left[ k = i \right] = p_i$, independently and with replacement and let $I$ a list of indices of the sampled columns.\;
        \For{$i \in I$}{
            $\bm{K}_{(:,i)} = \left[ k \left( \bm{x}_1 , \bm{x}_i \right) , \ldots , k \left( \bm{x}_n , \bm{x}_i \right) \right]^{\intercal}$\;
            $\bm{C}_{(:,i)} = \bm{K}_{(:,i)} / \sqrt{c p_{i}}$\;
        }
        $\bm{W} = \bm{K}_{(I,I)} \in \RR^{c \times c}$\;
        Rescale each entry of $\bm{W}$, $\bm{W}_{ij}$, by $1 / c \sqrt{p_i p_j}$.\;
        Compute $\bm{W}^{\dagger}$\;
        \Return{$\bm{C} \bm{W}^{\dagger} \bm{C}^{\ast}$}
        \BlankLine
    \end{algorithm}
\end{minipage}
\par}

As we can tell from the algorithms inputs, this requires some sort of probability distribution to select the columns. As seen in lemma \ref{lem: rmm-exp-var-bds} any probability distribution will provide an unbiased estimate. However, some probability distributions can be used to lower the variance faster than others. Naively, we could just employ uniform sampling where each column in selected with equal probability although it should be cautioned that this is seldom a good idea since uniform sampling tend to over sample landmarks from one large cluster while under sampling or even missing entire small but important clusters. As a result, the approximation for $\bm{K}$ will decline \cite{musco2017recursive}*{page 3}. This is demonstarted in graphic form in Figure \ref{fig: uni-samp-v-nonuni-samp}.

\begin{figure}[h]
    \centering
    \subfloat[]{
        \begin{adjustbox}{width=0.4\textwidth}
            \begin{tikzpicture}[>=latex]
                \begin{axis}[
                        xmin=5.0,xmax=18,
                        ymin=5.0,ymax=18,
                        tick style={draw=none},
                        yticklabels=\empty,
                        xticklabels=\empty,
                        axis y line =left,
                        axis x line =bottom,
                        axis line style = thick,
                    ]

                    \addplot [
                        only marks,
                        mark=*,
                        blue!70,
                        fill opacity=0.5,
                        draw opacity=0,
                        mark size = 3,
                    ] table {./data/nys_samp_fig.csv};

                    \addplot [
                        only marks,
                        mark=*,
                        red,
                        fill opacity=1,
                        draw opacity=0,
                        mark size = 3,
                    ] table {./data/nys_samp_fig_r1.csv};
                \end{axis}
            \end{tikzpicture}
        \end{adjustbox}
    } \qquad
    \subfloat[]{
        \begin{adjustbox}{width=0.4\textwidth}
            \begin{tikzpicture}[>=latex]
                \begin{axis}[
                        xmin=5.0,xmax=18,
                        ymin=5.0,ymax=18,
                        tick style={draw=none},
                        yticklabels=\empty,
                        xticklabels=\empty,
                        axis y line =left,
                        axis x line =bottom,
                        axis line style = thick,
                    ]

                    \addplot [
                        only marks,
                        mark=*,
                        blue!70,
                        fill opacity=0.5,
                        draw opacity=0,
                        mark size = 3,
                    ] table {./data/nys_samp_fig.csv};

                    \addplot [
                        only marks,
                        mark=*,
                        red,
                        fill opacity=1,
                        draw opacity=0,
                        mark size = 3,
                    ] table {./data/nys_samp_fig_r2.csv};
                \end{axis}
            \end{tikzpicture}
        \end{adjustbox}
    }
    \caption{Employing uniform sampling in the column sampling Nystrom estimate can lead to oversampling from denser parts of the data set (Panel (A)). Instead data dependent probability densities are commonly used to better cover the relevant data (Panel (B)). Example taken from \cite{musco2017recursive}*{page 4}.}
    \label{fig: uni-samp-v-nonuni-samp}
\end{figure}

To combat this issue, alternative probabilites density can be constructed to take into account a measure of importance in landmark selection. Indeed there has been a plethora of research that has shown the importance of using data-dependent non-uniform probability distributions to obtain proveably better error bounds within the Nystrom framework \cites{JMLR:v6:drineas05a,DBLP:journals/corr/abs-1303-1849,musco2017recursive,DBLP:journals/corr/abs-1109-3843,DBLP:journals/corr/CohenMM15,kumar2009sampling}. A few of the more common distributions will be discussed in the coming sections.

\subsection{Column Probabilities}\label{Section2.2}

Recall that the Nystrom method from Algorithm \ref{alg: nys-col-samp} is largely dependent on the random matrix multiplication algorithm (Algorithm \ref{alg: rand-mat-mult}) to produce a suitable sketching matrix. Moreover, improvements in the sketching matrix produced by the random matrix multiplication algorithmare reflected as smaller errors in the Nystrom approximation. Now, consider using the random matrix multiplication algorithm to approximate $\bm{A} \bm{A}^{\intercal}$ by setting $\bm{B} = \bm{A}$. The output is an approximation of the form
\begin{equation*}
    \bm{A} \bm{A}^{\intercal} \simeq \bm{C} \bm{C}^{\intercal} = \bm{C} \bm{R}.
\end{equation*}
The probability distribution
\begin{equation*} \label{eq: col-probs}
    p_i = \frac{\norm{\bm{A}_{(i,:)}}_2^2}{\norm{\bm{A}}_F}.
\end{equation*}
aims to minimize the error between $\bm{A} \bm{A}^{\intercal}$ and the approximation $\bm{C} \bm{C}^{\intercal}$. As a result, we should expect that $\bm{C}$ becomes a better estimate for $\bm{A} \bm{S}$, implying that the sketching matrix, $\bm{S}$, is using a better sampling and landmark selection criteria. Drineas and Mahoney give a precise bound on this error presented in theorem \ref{thm: col-pro-bounds} \cite{JMLR:v6:drineas05a}*{page 2158}.

\begin{thm} \label{thm: col-pro-bounds}
    Given $\bm{A} \in \RR^{m \times n}$, $1 \leq c \leq n$ and the probability distribution $\left\{ p_i \right\}_{i=1}^{n}$ described in equation \ref{eq: col-probs}. Construct $\bm{C}$ using algorithm \ref{alg: rand-mat-mult}, then
    \[
        \EE \left[ \norm{\bm{A} \bm{A}^{\intercal} - \bm{C} \bm{C}^{\intercal}}_{F} \right] \leq \frac{1}{\sqrt{c}} \norm{\bm{A}}^2_{F}
    \]
    \cite{JMLR:v6:drineas05a}*{page 2158}.
\end{thm}

To show theorem \ref{thm: col-pro-bounds}, we can actually prove something a little more general.

\begin{lem} \label{lem: col-pro-bounds-gen}
    Given $\bm{A} \in \RR^{m \times n}$, $\bm{B} \in \RR^{n \times p}$, $1 \leq c \leq n$ and the probability distribution $\left\{ p_i \right\}_{i=1}^{n}$ as follows
    \[
        p_i = \frac{\norm{\bm{A}_{(k,:)}}_2 \norm{\bm{B}_{(:,k)}}_2}{\sum_{j=1}^{n} \norm{\bm{A}_{(k,:)}}_2 \norm{\bm{B}_{(:,k)}}}.
    \]
    Construct $\bm{C}$ using algorithm \ref{alg: rand-mat-mult}, using the probability distribution described above, then
    \[
        \EE \left[ \norm{\bm{A} \bm{B} - \bm{C} \bm{R}}_{F} \right] \leq \frac{1}{\sqrt{c}} \norm{\bm{A}}^2_{F} \norm{\bm{B}}^2_{F}.
    \]
    This choice of probability distribution minimises $\EE \left[ \norm{\bm{A} \bm{B} - \bm{C} \bm{R}}_{F} \right]$ among all possible sampling probabilites \cite{doi:10.1137/S0097539704442684}*{pages 9-12}.
\end{lem}

\begin{proof}
    First note that
    \[
        \sum_{i=1}^{m} \sum_{j=1}^{p} \EE \left[ \left( \bm{A} \bm{B} - \bm{C} \bm{R}\right)_{ij}^{2} \right] = \sum_{i=1}^{m} \sum_{j=1}^{p} \VV \left[ \left( \bm{C} \bm{R} \right)_{ij} \right].
    \]
    Thus from lemma \ref{lem: rmm-exp-var-bds}, it follows that
    \begin{align*}
          & \EE \left[ \norm{\bm{A} \bm{B} - \bm{C} \bm{R}}^2_F \right]                                                                                                                     \\
        = & \frac{1}{c} \sum_{i=1}^{n} \frac{1}{p_k} \left( \sum_{i=1}^{m} \bm{A}_{ik}^{2} \right) \left( \sum_{j=1}^{p} \bm{B}_{kj}^{2} \right) - \frac{1}{c} \norm{\bm{A} \bm{B}}_{F}^{2} \\
        = & \frac{1}{c} \sum_{i=1}^{n} \frac{1}{p_k} \norm{\bm{A}_{(i,:)}}_{2}^{2} \cdot \norm{\bm{B}_{(:,i)}}_{2}^{2} - \frac{1}{c} \norm{\bm{A} \bm{B}}_{F}^{2}.
    \end{align*}
    Substituting in a probability of
    \[
        p_i = \frac{\norm{\bm{A}_{(i,:)}}_2 \norm{\bm{B}_{(:,i)}}_2}{\sum_{j=1}^{n} \norm{\bm{A}_{(j,:)}}_2 \norm{\bm{B}_{(:,j)}}}.
    \]
    yields
    \begin{align*}
        \EE \left[ \norm{\bm{A} \bm{B} - \bm{C} \bm{R}}^2_F \right] = & \frac{1}{c} \left( \sum_{i=1}^{n} \norm{\bm{A}_{(i,:)}}_{2} \norm{\bm{B}_{(:,k)}}_{2} \right)^{2} - \frac{1}{c} \norm{\bm{A} \bm{B}}_{F}^{2} \\
        \leq                                                          & \frac{1}{c} \norm{\bm{A}}^2_{F} \norm{\bm{B}}^2_{F}.
    \end{align*}
    To verify that this choice of probability distribution minimises $\EE \left[ \norm{\bm{A} \bm{B} - \bm{C} \bm{R}}_{F} \right]$ define the function
    \[
        f \left( p_1 , \ldots , p_n \right) = \sum_{i=1}^{n} \frac{1}{p_i} \norm{\bm{A}_{(i,:)}}_{2}^{2} \cdot \norm{\bm{B}_{(:,i)}}_{2}^{2}
    \]
    which characterises the dependence of $\EE \left[ \norm{\bm{A} \bm{B} - \bm{C} \bm{R}}_{F} \right]$ on the probability distribution. To minimise $f$ subject to $\sum_{i=1}^{n} p_i = 1$, we introduce the Lagrange multiplier $\lambda$ and define the function
    \[
        g \left( p_1 , \ldots , p_n \right) = f \left( p_i , \ldots , p_n \right) + \lambda \left( \sum_{i=1}^{n} p_i - 1 \right).
    \]
    The minimum is then
    \[
        0 = \frac{\partial g}{\partial p_i} = - \frac{1}{p_i^2} \norm{\bm{A}_{(k,:)}}_{2}^{2} \cdot \norm{\bm{B}_{(:,k)}}_{2}^{2} + \lambda.
    \]
    Thus
    \[
        p_i = \frac{\norm{\bm{A}_{(i,:)}}_{2} \cdot \norm{\bm{B}_{(:,i)}}_{2}}{\sqrt{\lambda}} = \frac{\norm{\bm{A}_{(i,:)}}_{2} \cdot \norm{\bm{B}_{(:,i)}}_{2}}{\sum_{j=1}^{n} \norm{\bm{A}_{(j,:)}}_{2} \norm{\bm{B}_{(:,j)}}_{2}}
    \]
    where the second equality comes from solving for $\sqrt{\lambda}$ in $\sum_{i=1}^{n-1} p_i = 1$. These probabilities are indeed minimizing since $\frac{\partial^2 g}{\partial p_i^2} > 0$ for every $i$ such that $\norm{\bm{A}_{(i,:)}}_{2}^{2} \cdot \norm{\bm{B}_{(:,i)}}_{2}^{2} > 0$.
\end{proof}

\subsection{Leverage Scores}\label{Section2.3}

\subsubsection{Statistical Leverage Scores}
Our next distribution originates from the least-squares problem. Breifly, in an over constrained least-squares problem, where $\bm{A} \in \RR^{n \times m}$, $\bm{b} \in \RR^{n}$, for $m \ll n$ there usually are not any $\bm{x} \in \RR^{m}$ for which $\bm{A} \bm{x} = \bm{b}$. Instead, alternative criteria must be employed to seek a $\bm{x}$ which in some way comes "closest" to satisfying this equality. Perhaps one of the more popular criterion is to minimize the $\ell^2-$norm, that is
\[
    \bm{x}_{opt} = \argmin_{x} \norm{\bm{A} \bm{x} - \bm{b}}
\]
\cite{DBLP:journals/corr/abs-1104-5557}*{page 19-21}. The optimal value for $\bm{x}$ can be solved as $\bm{x}_{opt} = \left( \bm{A}^{\intercal} \bm{A} \right)^{-1} \bm{A}^{\intercal} \bm{b}$. The least-squares solution is commonly used to find the best weight vector (in this case $\bm{x}$) for a linear model, given a dataset. Fitted or predicted values are usually obtained from $\hat{\bm{b}} = \bm{H} \bm{b}$ where the projector onto the column space of $\bm{A}$
\[
    \bm{H} = \bm{A} \left( \bm{A}^{\intercal} \bm{A} \right)^{-1} \bm{A}^{\intercal}
\]
is sometimes referred to as the {\it hat matrix}. The element $\bm{H}_{ij}$ has the direct interpretation as the influence or statistical leverage exerted on $\hat{\bm{b}}_i$. Thus, examining the hat matrix can reveal to us columns of $\bm{A}$ which bear a significant impact on $\hat{\bm{b}}$ \cite{HoaglinDavidC1978THMi}*{page 17}. Relatedly, if the element $\bm{H}_{ii}$ is particularly large this is indicative of the $i^{th}$ column of $\bm{A}$ having a strong influence over values of $\hat{\bm{b}}$, justifying the interpretation of $\bm{H}_{ii}$ as {\it statistical leverage scores}.

The statistical leverage scores are maximised when $\bm{A}_{(:,i)}$ is linearly independent from $\bm{A}$'s other columns and decreases when it aligns with many other columns or when the value of $\norm{\bm{A}_{(:,i)}}$ is small \cite{DBLP:journals/corr/CohenMM15}*{page 5}. To compute the statistical leverage scores, if $\bm{A} = \bm{U} \bm{\Sigma} \bm{V}^{\intercal}$ is the SVD of $\bm{A}$, then
\begin{align*}
    \bm{H}_{ii} & = \left( \bm{A} \left( \bm{A}^{\intercal} \bm{A} \right)^{-1} \bm{A}^{\intercal} \right)_{ii} \\
                & = \left( \bm{U} \bm{\Sigma}^{2} \left( \bm{\Sigma}^{2} \right)^{-1} \bm{U} \right)_{ii}       \\
                & = \norm{\bm{U}_{(i,:)}}_2^2.
\end{align*}
Note that $\bm{H}_{ii}$ may not constitute as a probability distribution, as may the other leverage scores which we will soon discuss. This is easily remedied through normalisation, in this case dividing each statistical leverage score by $\operatorname{tr} \left( \bm{H} \right)$. The idea behind using statistical leverage scores as a probability distribution in the Nystrom method is that they help prioritize selection of columns that are more linearly independent from other columns so that the range of our approximate better aligns with the range of our original $\bm{A}$.

\subsubsection{Rank$-k$ Statistical Leverage Scores}

We can generalize this notion of statistical leverage scores to include lower rank approximations. Let $\bm{A} = \bm{U} \bm{\Sigma} \bm{V}^{\intercal}$ be the compact SVD of a $\bm{A}$ real $n \times m$ matrix. Setting $r = \min \left\{ n,m \right\}$, the compact SVD can be partitioned as
\begin{equation*}
    \bm{U} = \left[ \bm{U}_1 , \bm{U}_2 \right] \in \RR^{n \times r}, \qquad \bm{\Sigma} =
    \begin{bmatrix}
        \bm{\Sigma}_1 &               \\
                      & \bm{\Sigma}_2
    \end{bmatrix}
    \in \RR^{r \times r},
    \qquad
    \bm{V} = \left[ \bm{V}_1 , \bm{V}_2 \right] \in \RR^{m \times r}.
\end{equation*}
Here $\bm{U}_1$ contains the first $k \leq r$ columns of $\bm{U}$, $\bm{V}_1$ the first $k$ rows of $\bm{V}$ and $\bm{\Sigma}_1$ is a $k \times k$ matrix containing the top $k$ singular values across its diagonal. The matrix $\bm{A}_k = \bm{U}_1 \bm{\Sigma}_1 \bm{V}_1$ serves as the best rank$-k$ approximation to $\bm{A}$. The statistical leverage scores relative to the best rank$-k$ approximation are again $\bm{H}_{ii}$, but this time $\bm{H}$ is computed only using the best rank$-k$ approximation of $\bm{A}$, that is $\bm{A}_k$. These low rank scores can be evaluated as
\begin{equation*} \label{eq: lev-scrs-1}
    \ell_i^k \triangleq \left( \bm{A}_{k} \left( \bm{A}_{k}^{\intercal} \bm{A}_{k} \right)^{-1} \bm{A}_{k}^{\intercal} \right)_{ii} = \norm{\left( \bm{U}_1 \right)_{(i,:)}}_2^2.
\end{equation*}
What makes low-rank statistical leverage scores particularly appealing is that they can be approximated quickly with a truncated SVD \cite{DBLP:journals/corr/abs-1303-1849}*{pages 3-4}.

\subsubsection{Ridge Leverage Scores}

The low rank leverage scores we saw in equation \ref{eq: lev-scrs-1} will not always be unique and can be sensitive to perturbations \cite{DBLP:journals/corr/CohenMM15}*{page 6}. Consequently these scores can vary drastically when $\bm{A}$ is modified slightly or when we only have access to partial information on the matrix. This undermines the the possibility of computing good quality low rank approximations from statistical leverage scores. This shortcoming is addressed in the next class of leverage score, that is, ridge leverage scores. Ridge leverage scores are similar to statistical leverage scores although a ridge regression term (hence the name) is added to the hat matrix with a regularization parameter $\lambda$. The $\lambda-$ridge leverage score is defined as
\begin{equation*}
    r_{i}^{\lambda} \triangleq \left( \bm{A} \left( \bm{A}^{\intercal} \bm{A} + \lambda \Id_{n \times n} \right)^{-1} \bm{A}^{\intercal} \right)_{ii}.
\end{equation*}
A regularization parameter of
\begin{equation*} \label{eq: rid-lev-reg-param}
    \lambda = \frac{\norm{\bm{A} - \bm{A}_k}_F^{2}}{k}
\end{equation*}
is typically used since this choice of $\lambda$ will guarantee that the sum of the ridge leverage scores (keep in mind that the raw ridge leverage scores do not necessarily form a probability distribution) is bounded by $2k$, stated more formally in lemma \ref{lem: rid-lev-reg-param-bound}.
\begin{lem} \label{lem: rid-lev-reg-param-bound}
    When using a regularization parameter of $\lambda = \frac{\norm{\bm{A} - \bm{A}_k}_F^{2}}{k}$ we have $\sum_{i=1}^{n} r_{i}^{\lambda} \leq 2k$ \cite{DBLP:journals/corr/CohenMM15}*{pages 6-7}.
\end{lem}
\begin{proof}
    Writing $r_{i}^{\lambda}$ using the SVD of $\bm{A}$ where $\lambda = \frac{\norm{\bm{A} - \bm{A}_k}_F^{2}}{k}$ gives
    \begin{align*}
        r_{i}^{\lambda} & = \bm{A}_{(i,:)} \left( \bm{U} \bm{\Sigma} \bm{U}^{\intercal} + \frac{\norm{\bm{A} - \bm{A}_k}_F^{2}}{k} \bm{U} \bm{U}^{\intercal} \right)^{-1} \bm{A}_{(i,:)}^{\intercal} \\
                        & = \bm{A}_{(i,:)} \left( \bm{U} \overline{\bm{\Sigma}}^2 \bm{U}^{\intercal} \right)^{-1} \bm{A}_{(i,:)}^{\intercal}                                                         \\
                        & = \bm{A}_{(i,:)} \left( \bm{U} \overline{\bm{\Sigma}}^{-2} \bm{U}^{\intercal} \right) \bm{A}_{(i,:)}^{\intercal}
    \end{align*}
    where $\overline{\bm{\Sigma}}^{2}_{ii} = \sigma_{i}^{2} \left( \bm{A} \right) + \frac{\norm{\bm{A} - \bm{A}_k}_F^{2}}{k}$. Then
    \begin{align*}
        \sum_{i=1}^{n} r_{i}^{\lambda} & = \operatorname{tr} \left( \bm{A}^{\intercal} \bm{U} \overline{\bm{\Sigma}}^{-2} \bm{U}^{\intercal} \bm{A} \right) \\
                                       & = \operatorname{tr} \left( \bm{V} \bm{\Sigma} \overline{\bm{\Sigma}}^{-2} \bm{\Sigma} \bm{V}^{\intercal} \right)   \\
                                       & = \operatorname{tr} \left( \bm{\Sigma}^{2} \overline{\bm{\Sigma}}^{-2} \right).
    \end{align*}
    Here we have
    \[
        \left( \bm{\Sigma}^{2} \overline{\bm{\Sigma}}^{-2} \right)_{ii} = \frac{\sigma_{i}^{2} \left( \bm{A} \right)}{\sigma_{i}^{2} \left( \bm{A} \right) + \frac{\norm{\bm{A} - \bm{A}_k}_F^{2}}{k}}.
    \]
    For $i \leq k$ we simply upper bound this by $1$, yielding
    \begin{equation*}
        \operatorname{tr} \left( \bm{\Sigma}^{2} \overline{\bm{\Sigma}}^{-2} \right) =
        k + \sum_{i=k+1}^{n} \frac{\sigma_{i}^{2} \left( \bm{A} \right)}{\sigma_{i}^{2} \left( \bm{A} \right) + \frac{\norm{\bm{A} - \bm{A}_k}_F^{2}}{k}}
        \leq k + \sum_{i=k+1}^{n} \frac{\sigma_{i}^{2} \left( \bm{A} \right)}{\frac{\norm{\bm{A} - \bm{A}_k}_F^{2}}{k}}
        = k + \frac{\sum_{i=k+1}^{n} \sigma_{i}^{2} \left( \bm{A} \right)}{\frac{\norm{\bm{A} - \bm{A}_k}_F^{2}}{k}}
        \leq k + k.
    \end{equation*}
\end{proof}
From now on (unless otherwise stated) the regularization parameter seen in \ref{eq: rid-lev-reg-param} will always be used for ridge leverage scores where the notation
\begin{equation*}
    r_{i}^{k} \triangleq \left( \bm{A} \left( \bm{A}^{\intercal} \bm{A} + \left( \frac{\norm{\bm{A} - \bm{A}_k}_F^{2}}{k} \right) \Id_{n \times n} \right)^{-1} \bm{A}^{\intercal} \right)_{ii}
\end{equation*}
is employed to show that the best rank$-k$ matrix is utilized in the regularization parameter. Adding regularization to the hat matrix offers a smoother alternative which 'washes out' small singular directions meaning they are sampled with proportionally lower probability \cite{DBLP:journals/corr/CohenMM15}*{page 6}. Alaoui and Mahoney \cite{NIPS2015_f3f27a32} prove that ridge leverage scores provide theoretically better bounds over uniform sampling techniques when the number of sampled columns is proportional to $\operatorname{tr} \left( \bm{H}_{\lambda} \right) \cdot \ln \left( n \right)$ where $\bm{H}_{\lambda}$ is the hat matrix with added regularization, that is $\bm{H}_{\lambda} = \bm{A} \left( \bm{A}^{\intercal} \bm{A} + \lambda \Id_{n \times n} \right)^{-1} \bm{A}^{\intercal}$. With the rising popularity of ridged leverage scores, a number of iterative methods have been devised (and continue to be developed) that take advantage of parallel computing to provide fast approximations \cite{martinsson2021randomized}*{page 90}.

\subsection{Data Probabilities}\label{Section2.3}

Here we present two novel Nystrom probability sampling distributions that make use of the data matrix, $\bm{X}$, directly without requiring any knowledge of the associated kernel matrix. The first of these two methods is the data-column sampling distribution defined simply as
\begin{equation*}
    p_{i} = \norm{\bm{X}_{(i,:)}}_2 / \norm{\bm{X}}_{F}^2.
\end{equation*}
This sampling distribution can be thought of as the data-column dual of the kernel column sampling probabilities seen in \Cref{Section2.2}. The second sampling method is the $Q$ matrix column sampling distribution defined as
\begin{equation*}
    p_{i} = \norm{\bm{Q}_{(i,:)}}_2 / \norm{\bm{Q}}_{F}^2.
\end{equation*}
where $\bm{X} = \bm{Q} \bm{R}$ is the $QR$ factorisation (see \Cref{theorem: QR_general_existence}) of $\bm{X}$. This sampling technique can be though of as the data matrix equivalent of leverage score sampling. There are numerous benefits computing the sampling probabilities using the data matrix directly without needing any knowledge about the corresponding matrix. To start, the obvious benefit is that no properties or elements (along with their corresponding calculations) of the kernel matrix are required saving large amounts of computation time. Second, if the parameters of the kernel function are changed, or an entirely different kernel function is used altogether, then the data-column sampling probabilities need not be re-evaluted.

The main grounds for the theoretical justification for using the data matrix directly to produce a sampling distribution comes from a result shown in \cite{KarouiNoureddineEl2010TSOK}, presented below.

\renewcommand{\labelenumi}{(\alph{enumi})}

\begin{thm}[Inner Product of Kernel Matrices] \label{thm: inner-prof-kern-mat}
    Let us assume that we observe n i.i.d. random vectors, $X_i \in \RR^{p}$. let us consider the kernel matrix $\bm{M}$ with entries
    \begin{equation*}
        \bm{M}_{ij} = f \left( \frac{\bm{X}_{i}^{\intercal} \bm{X}_{i}}{p} \right).
    \end{equation*}
    We assume that
    \begin{enumerate}
        \item an apple
        \item a banana
        \item a carrot
        \item a durian
    \end{enumerate}
    \cite{KarouiNoureddineEl2010TSOK}*{page 9}.
\end{thm}