\section{Twisted Hecke Algebras of Finite Groups}\label{Chapter2}
We have now completed our investigation of the Hecke algebra $\calH(G,K)$.
The aim of this section is to generalise the results of chapter \ref{Chapter1} to the case of a non-trivial character $\sigma\colon K\to\CC^\times$.
Here the Hecke algebra $\calH=\calH(G,K,\sigma)$ is the convolution algebra of $(K,\sigma)$-bi-invariant functions on $G$.
In \Cref{Section2.1}, we discuss the theory of the induced representation $\Ind_K^G \sigma$.
In \Cref{Section2.2}, we revisit the Hecke algebra, identify its identity and describe its basis.
Notice that the results of \Cref{Section1.4} and \Cref{Section1.5} were independent of the choice $\sigma=\1$, so they still apply now that we are considering a non-trivial character.

In \Cref{Section2.3}, we generalise Gelfand's Trick from \Cref{Section1.7} to the case of a non-trivial character $\sigma$.
Naturally, we will need to reconsider the conditions that the anti-automorphism $\varphi\colon G\to G$ must satisfy.
As in \Cref{Section1.7}, we will investigate these conditions and conclude with a natural statement and proof of Gelfand's Trick in the twisted case.
We conclude with \Cref{Section2.4}, in which we investigate the Gelfand--Graev representation and use the results of this chapter to prove that it is multiplicity-free.


\subsection{The induced representation $\Ind_K^G \sigma$}\label{Section2.1}
Suppose that $\sigma\colon K \to\CC^\times$ is a character, i.e.\ a group homomorphism.
Consider the space
\[
    W := \{f\colon G\to \CC\ |\ f(gk) = f(g)\sigma(k),\ \forall g\in G, \forall k \in K\} \subseteq \Fun(G).
\]
As in the previous section, $W$ is called the induced representation and denoted $\Ind_K^G \sigma$.
We state and prove a lemma analogous to \Cref{lemma: W_left_ideal}.
\begin{lem}\label{lemma: W_left_ideal_two}
    $W$ is a left ideal of $(\Fun(G),\star)$.
\end{lem}
\begin{proof}
    We verify that $f\star w\in W$ whenever $w\in W$ and $f\in\Fun(G)$.
    Let $g\in G$ and $k\in K$.
    Then
    \begin{multline*}
        (f\star w)(gk) = \sum_{xy=gk} f(x)w(y) = \sum_{x\in G} f(x)w(x^{-1}gk) = \sum_{x\in G} f(x)w(x^{-1}g)\sigma(k) \\
        = \bigg[\sum_{x\in G} f(x)w(x^{-1}g)\bigg]\sigma(k) = \bigg[\sum_{xy=g} f(x)w(y)\bigg]\sigma(k) = (f\star w)(g)\sigma(k).\qedhere
    \end{multline*}
\end{proof}

%%%%%%%%%%%%%%%%%%%%%%%%%%%%%%%%%%%%%%%%%%%%%%%%%%%%%%%%%%%%%%%%%%%%%%%%%%%%%%%%%%%%%%%%%%%%%%

\subsection{The twisted Hecke algebra of a finite group $\calH(G,K,\sigma)$}\label{Section2.2}
The Hecke algebra $\calH = \calH(G,K,\sigma)$ is the space
\[
    \calH := \{f\colon G\to \CC\ |\ f(k_1gk_2) = \sigma(k_1)f(g)\sigma(k_2),\ \forall g\in G,\ \forall k_1,k_2\in K\} \subseteq \Fun(G).
\]
The proof of \Cref{lemma: W_left_ideal_two} can be adapted to show that $\calH$ is a two-sided ideal in $(\Fun(G),\star)$.
As before, the identity of $(\Fun(G),\star)$ does not lie in $\calH$.
Nevertheless, $\calH$ does have an identity of its own.
It is easy to verify that the identity is $\iota_K^\sigma$, which we define below.
\[
    \iota_K^\sigma:G\to\CC,\quad \iota_K^\sigma(g) := \begin{cases}
        \frac{1}{|K|}\sigma(g),\  & \text{if}\ g\in K, \\
        0,\                       & \text{else}.
    \end{cases}
\]
Thus, $(\calH,\star)$ is a unital associative algebra in its own right.

We now construct a basis for $\calH$.
Recall that when $\sigma=\1$, the basis of $\calH$ was described by the characteristic functions of $K$-double cosets.
To treat the case when $\sigma\neq\1$, we need a lemma about group actions.

Consider the finite group $K$ acting on a set $X$.
For each $x\in X$, let $\calO_x := \{g\cdot x\ |\ g\in G\}$ be the orbit containing $x$ and let $K_x := \{k\in K\ |\ k\cdot x = x\}$ be the stabiliser subgroup of $x$ in $K$ (also denoted as $\stab_K(x)$).
Consider the vector space
\[
    V := \{f\colon X\to\CC\ |\ f(k\cdot x) = \sigma(k)f(x),\ \forall k\in K,\ \forall x\in X\}\subseteq \Fun(X).
\]
An orbit $\calO_x$ is called \emph{$(K,\sigma)$-relevant} if there exists $f\in V$ such that $f|_{\calO_x}$ is non-zero.
Otherwise, we say $\calO_x$ is \emph{$(K,\sigma)$-irrelevant}.
We omit mention of $(K,\sigma)$ if it is clear from the context.

\begin{lem}\label{lemma: relevant_orbit}
    An orbit $\calO_x$ is $(K,\sigma)$-relevant if and only if $\sigma(K_x) = \{1\}$.
\end{lem}
\begin{proof}
    Assume that $\sigma(K_x)\neq\{1\}$.
    Then there exists $k\in K_x$  such that $\sigma(k)\neq 1$.
    Now recall that for $f\in V$ we have $f(x) = f(k\cdot x) = \sigma(k)f(x)$.
    However $\sigma(k)\neq 1$, so $f(x)=0$.
    Then $f$ must be zero on $\calO_x$ and $\calO_x$ is irrelevant.

    Conversely, the fact that $\sigma$ is trivial on $K_x$ implies that it factors through a well-defined function $\sigma_x\colon \calO_x\simeq K/K_x\rightarrow \CC$ given by $\sigma_x(kK_x) := \sigma(k)$.
    To see that this function is well-defined, suppose that $k_1K_x=k_2K_x$.
    Then $k_1k_2^{-1}\in K_x$.
    Since $\sigma$ is trivial on $K_x$, we know $1=\sigma(k_1k_2^{-1})=\sigma(k_1)\sigma(k_2)^{-1}$.
    Then $\sigma(k_1)=\sigma(k_2)$ so $\sigma_x(k_1K_x) = \sigma_x(k_2K_x)$.
    It is easy to check that $\sigma_x\in V$.
    Thus, $\calO_x$ is relevant.
\end{proof}
Now suppose $K$ is a subgroup of $G$ acting on $X=G$ from the left and right by translation.
Then the orbit $\calO_x$ is nothing but the double coset $KxK$ and $V$ becomes the Hecke algebra $\calH$.
Explicitly, we have
\[
    \calH = \{f\colon X\to\CC \ |\ f(k_1\cdot x \cdot k_2) = \sigma(k_1)f(x)\sigma(k_2),\ \forall k_1,k_2\in K,\ \forall x\in X\}\subseteq \Fun(X).
\]
We can re-write this data by considering the left action of $K\times K^\op$ on $X$, where $K^\op$ is the group opposite to $K$.
Then
\[
    \calH = \{f\colon X\to\CC \ |\ f((k_1,k_2)\cdot x) = \sigma(k_1)\sigma(k_2)f(x),\ \forall k_1,k_2\in K,\ \forall x\in X\}\subseteq\Fun(X).
\]
A double coset $KxK$ is relevant if it supports a non-zero function from $\calH$.
Let $X_\rel$ be a family of relevant coset representatives.
Define the family of functions $\{\chi_x\}_{x\in X_\rel}$ by
\[
    \chi_x(y) := \begin{cases}
        \sigma(k)\sigma(k'),\  & \text{if}\ y\in KxK\ \text{with}\ y=kxk', \\
        0,\                    & \text{if}\ y\notin KxK.
    \end{cases}
\]
One easily checks that $\chi_x$ is well-defined.
We call $\chi_x$ a \emph{twisted characteristic function} associated to the relevant orbit $KxK$.
When $\sigma=\1$, every orbit is relevant and $\sigma(k)\sigma(k')=1$, so we obtain the original characteristic functions described in \Cref{Section1.3}.
Define the map
\[
    \sigma\boxtimes\sigma\colon K\times K\to\CC^\times\times\CC^\times,\quad (\sigma\boxtimes\sigma)(k_1,k_2) := (\sigma(k_1),\sigma(k_2)).
\]
As a result of \Cref{lemma: relevant_orbit}, we see that an orbit under the left action of $K\times K^\op$ is relevant if and only if $(\sigma\boxtimes\sigma)(\stab_{K\times K}(x)) = \{1\}$.
As in \Cref{Section1.3}, it is not difficult to see that the twisted characteristic functions of relevant orbits form a basis of $\calH(G,K,\sigma)$.

%%%%%%%%%%%%%%%%%%%%%%%%%%%%%%%%%%%%%%%%%%%%%%%%%%%%%%%%%%%%%%%%%%%%%%%%%%%%%%%%%%%%%%%%%%%%%%

\subsection{Twisted Gelfand's Trick}\label{Section2.3}
Our goal in this section is to prove the twisted analogue of Gelfand's Trick.
\begin{thm}[Twisted Gelfand's Trick]\label{twisted_trick}
    Suppose that $G$ is a finite group with $K\leq G$ as a subgroup and character $\sigma\colon K\to\CC^\times$.
    Let $\varphi\colon G\to G$ be an anti-automorphism such that
    \begin{enumerate}[\itshape(i)]
        \item $\varphi^2=1$,
        \item $\varphi(K)=K$,
        \item $\sigma(\varphi(k))=\sigma(k)$ for all $k\in K$, and
        \item $\varphi(x) = x$ for all $x\in X_\mathrm{rel}$, a family of representatives for the $(K,\sigma)$-relevant $K$-double cosets.
    \end{enumerate}
    Then $\calH(G,K,\sigma)$ is commutative.
\end{thm}
This is a true generalisation of \Cref{theorem: Gelfand's_Trick}.
Indeed, if we consider the trivial representation $\sigma=\1$, condition {\itshape(iii)} is trivially satisfied, condition {\itshape(iv)} corresponds to the requirement that $K\varphi(x)K=KxK$ in \Cref{theorem: Gelfand's_Trick}, and condition {\itshape(ii)} is contained in the requirement that $K\varphi(x)K=KxK$.

As in \Cref{Section2.1}, the proof of \Cref{twisted_trick} relies on the observation that an anti-homomorphism of an algebra that acts as the identity on basis elements of the subalgebra is sufficient to conclude that the subalgebra is commutative (c.f.\ \Cref{lemma: subalgebra_commutative} and Corollary \ref{cor: H_commutative}).
This leaves us with a question: can we rewrite the condition $\varphi^\ast \chi_x = \chi_x$?

Recall that $X_\mathrm{rel}$ denotes a family of representatives for the relevant double cosets.
Recall the twisted characteristic functions $\{\chi_x\}_{x\in X_\rel}$ defined in \Cref{Section2.2} given by
\[
    \chi_x(y) = \begin{cases}
        \sigma(k)\sigma(k'),\  & \text{if}\ y\in KxK\ \text{with}\ y=kxk', \\
        0,\                    & \text{if}\ y\notin KxK.
    \end{cases}
\]
Thus,
\[
    (\varphi^\ast\chi_x)(g) = \begin{cases}
        \sigma(k)\sigma(k'),\  & \text{if}\ \varphi(g) \in KxK\ \text{with}\ \varphi(g) = kxk', \\
        0,\                    & \text{else}.
    \end{cases}
\]
If $\varphi\colon G\to G$ is an involutive homomorphism, then $\varphi(g) = kxk'$ is equivalent to $g=\varphi(k')\varphi(x)\varphi(k)$.
If we further suppose that $\varphi(x)=x$ for all $x\in X_\rel$ and $\varphi(K)=K$, then $g=\varphi(k')\varphi(x)\varphi(k)$ is equivalent to $g=\varphi(k')x\varphi(k)$.
Thus,
\[
    (\varphi^\ast\chi_x)(g) = \begin{cases}
        \sigma(\varphi(k'))\sigma(\varphi(k)),\  & \text{if}\ g \in KxK\ \text{with}\ g = \varphi(k')x\varphi(k), \\
        0,\                                      & \text{else}.
    \end{cases}
\]
This tells us that $\varphi^\ast\chi_x$ is also supported (i.e.\ non-zero) on $KxK$.
Now let's also assume that $\sigma(\varphi(k))=\sigma(k)$ for all $k\in K$.
Then we can easily verify that $\varphi^\ast\chi_x \in \calH(G,K,\sigma)$.
So $\varphi^\ast\chi_x$ must be a multiple of $\chi_x$.
In fact, this multiple is $1$, since
\[
    (\varphi^\ast\chi_x)(x) = \chi_x(\varphi(x)) = \chi_x(x) = 1.
\]
We are now ready to prove \Cref{twisted_trick}.
\begin{proof}[Proof of \Cref{twisted_trick}]
    Suppose that $\varphi\colon G\to G$ is an anti-automorphism.
    Also suppose that $\varphi^2=1$, $\varphi(K)=K$, $\sigma(\varphi(k))=\sigma(k)$ for all $k\in K$, and $\varphi(x)=x$ for all $x\in X_\rel$.
    The above discussion tells us that $\varphi^\ast \chi_x = \chi_x$.
    These are the basis elements of $\calH(G,K,\sigma)$.
    We apply Corollary \ref{cor: comm} to conclude that $\calH(G,K,\sigma)$ is commutative.
\end{proof}

%%%%%%%%%%%%%%%%%%%%%%%%%%%%%%%%%%%%%%%%%%%%%%%%%%%%%%%%%%%%%%%%%%%%%%%%%%%%%%%%%%%%%%%%%%%%%%

\subsection{The Gelfand--Graev representation}\label{Section2.4}
We construct the \emph{Gelfand--Graev representation} of $G=\GL_n(\FF_q)$.
First, consider the \emph{unipotent radical} of $G$, given by
\[
    U(\FF_q) := \begin{bmatrix}
        1      & \FF_q  & \hdots & \FF_q  \\
        0      & 1      & \ddots & \vdots \\
        \vdots & \ddots & 1      & \FF_q  \\
        0      & \hdots & 0      & 1
    \end{bmatrix}.
\]
Next, fix a non-trivial additive character $\psi\colon \FF_q\to\CC^\times$ (i.e.\ $\psi(a+b)=\psi(a)\psi(b)$).
Then define a character $\pi\colon U(\FF_q)\to\CC^\times$ by
\[
    \pi(x) := \psi(x_{12}+x_{23}+\cdots+x_{n-1,n}).
\]
To see that $\pi$ is a character, observe
\begin{align*}
    \pi(xy) & = \psi((xy)_{12} + (xy)_{23} + \cdots + (xy)_{n-1,n})                                                                                          \\
            & = \psi\bigg(\sum_{k=1}^{n-1} x_{1k}y_{k2} + \sum_{k=1}^{n-1} x_{2k}y_{k3} + \cdots + \sum_{k=1}^{n-1} x_{n-1,k}y_{kn}\bigg)                    \\
            & = \psi\bigg(\sum_{k=1}^{n-1} x_{1k}y_{k2}\bigg)\psi\bigg(\sum_{k=1}^{n-1} x_{2k}y_{k3}\bigg)\cdots\bigg(\sum_{k=1}^{n-1} x_{n-1,k}y_{kn}\bigg) \\
            & = \psi(x_{12}+y_{12})\psi(x_{23}+y_{23})\cdots\psi(x_{n-1,n}+y_{n-1,n})                                                                        \\
            & = \psi(x_{12})\psi(y_{12})\psi(x_{23})\psi(y_{23})\cdots\psi(x_{n-1,n})\psi(y_{n-1,n})                                                         \\
            & = \psi(x_{12})\psi(x_{23})\cdots\psi(x_{n-1,n})\psi(y_{12})\psi(y_{23})\cdots\psi(y_{n-1,n})                                                   \\
            & = \psi(x_{12} + \cdots + x_{n-1,n})\psi(y_{12} + \cdots + y_{n-1,n})                                                                           \\
            & = \pi(x)\pi(y).
\end{align*}
The \emph{Gelfand--Graev representation} of $G$ is $\Ind_U^G \pi$.
In \cite{Bump13}, Bump explains, ``this Gelfand–Graev representation is important because it contains \emph{most} irreducible representations of the group; those it contains are therefore called \emph{generic}.'' Furthermore, we have the following theorem.
\begin{thm}\label{thm: gelf_graev}
    The Gelfand--Graev representation is multiplicity-free.
    That is, $(\GL_n(\FF_q),U(\FF_q),\pi)$ is a twisted Gelfand pair.
\end{thm}
This theorem will be proven in two parts.
We begin with a lemma.
\begin{lem}\label{lemma: bruhat}
    \begin{enumerate}[\itshape(i)]
        \item We have the Bruhat decomposition
              \[
                  \GL_n(\FF_q) = \bigsqcup_{w\in W} BwB,
              \]
              where $W$ is the group of all $n\times n$ permutation matrices and $B$ is the subgroup of all $n\times n$ upper-triangular matrices.
        \item We can modify the Bruhat decomposition and write
              \[
                  \GL_n(\FF_q) = \bigsqcup_{m\in M} UmU,
              \]
              where $M$ is the group of all $n\times n$ monomial matrices.
              A monomial matrix is a matrix with exactly one non-zero element in each row and column.
    \end{enumerate}
\end{lem}
Before we prove \Cref{lemma: bruhat}, we recall a simple fact about matrices.
Define $x_{ij}(t) := I_{n\times n} + tE_{ij}$, where $1\leq i\leq j\leq n$ and $E_{ij}$ is the matrix of $0$'s except for a $1$ in the $i^\text{th}$ row and $j^\text{th}$ column.
Notice that $x_{ij}(t)\in B$ since $i\leq j$.
We can achieve the usual row and column operations on a matrix $A$ by multiplying on the left or the right by some $x_{ij}(t)$.
The following makes this statement precise.

Right-multiplying $A$ by $x_{ij}(t)$ corresponds to the column operation of $C_j\mapsto C_j+tC_i$, where $C_k$ is column $k$ of $A$.
Similarly, left-multiplying $A$ by $x_{ij}(t)$ corresponds to the row operation of $R_i \mapsto R_i+tR_j$, where $R_k$ is row $k$ of $A$.
Right-multiplying $A$ by $x_{ii}(\lambda-1)$ corresponds to the column operation $C_i\mapsto \lambda C_i$, for some scalar $\lambda$.
Similarly, left-multiplying $A$ by $x_{ii}(\lambda-1)$ corresponds to the row operation $R_i\mapsto\lambda R_i$.
We see that we can perform the usual row and column operations by right- and left-multiplying by elements of $B$.

\begin{proof}[Proof of \Cref{lemma: bruhat}]
    We begin by proving that $\GL_n(\FF_q) = \bigcup_{w\in W} BwB$ and will prove disjointness of the union later.
    We proceed by induction.
    The $n=1$ case is clearly true since all matrices in $\GL_1(\FF_q)$ are upper-triangular.
    Now let $n>1$ and $g\in\GL_n(\FF_q)$.
    We wish to find a permutation matrix $w$ in $BgB$.
    We have two cases: $g_{n,1}\neq 0$ and $g_{n,1}=0$.

    In the first case, the previous discussion tells us that we can multiply $g$ on the left and the right by appropriate elements of $B$ so that the resulting matrix has zeros in the left column and bottom row, except for the bottom left entry, which is $g_{n,1}$.
    This is non-zero so we can normalise this resulting matrix by $g_{n,1}$ to yield $\left(\begin{smallmatrix} 0 & g' \\ 1 & 0 \end{smallmatrix}\right)$.
    Here $g'$ lies in $\GL_{n-1}(\FF_q)$.
    The inductive hypothesis that the $n-1$ is true tells us that $g'$ lies in a double coset $Bw'B$ for some $(n-1)\times(n-1)$ permutation matrix $w'$.
    Then the desired $w$ is obtained by setting $w = \left(\begin{smallmatrix} 0 & w' \\ 1 & 0 \end{smallmatrix}\right)$.

    In the second case, choose $g_{i1}\neq 0$ and $g_{nj}\neq 0$ so that $i$ is as large as possible and $j$ is as small as possible.
    This amounts to choosing the two non-zero entries in the left column and bottom row that are closest to the bottom left entry.
    Left- and right-multiplication by appropriate elements of $B$ yields a matrix whose first and $j$th columns and $i$th and last rows are empty, except the entries $g_{i1}$ and $g_{nj}$.
    Since these entries are non-zero, we can normalise these to $1$ as well.
    Now we apply the inductive hypothesis to the matrix obtained by removing these two rows and two columns.
    We are left with a permutation matrix and this completes the induction.

    We verify that the union is disjoint.
    Let $w_1,w_2\in W$ be representatives for the same double coset.
    Then $Bw_1B=Bw_2B$ and, given any $b\in B$, there exists $b'\in B$ with $w_1bw_2^{-1} = b'$.
    In particular, $w_1w_2^{-1}\in B\cap W = \{1\}$.
    Thus $w_1=w_2$.

    We now prove the modified decomposition.
    Consider the subgroup $T$ of diagonal matrices in $\GL_n(\FF_q)$.
    Notice that $B=TU=UT$ and $M=TW=WT$ so the result follows from the regular Bruhat decomposition.
    Disjointness is proven as before.
\end{proof}
\begin{proof}[Proof of \Cref{thm: gelf_graev}]
    Consider the involutive anti-automorphism $\varphi\colon G \to G$ defined by
    \[
        \varphi(g) := w_0 g^t w_0,\quad\text{where}\ w_0 =
        \begin{bmatrix}
              &                       & 1 \\
              & \reflectbox{$\ddots$} &   \\
            1 &                       &
        \end{bmatrix}.
    \]
    We verify that $U\varphi(g)U=UgU$ for all $g\in G$.
    For each double coset $UgU$, we will show that $UgU$ has a certain coset representative $g'$ with $\varphi(g')=g'$, or $f(g)=0$ for all $f\in\calH$.

    The modification of the Bruhat decomposition in \Cref{lemma: bruhat} tells us that $UgU=UmU$ for some monomial matrix $m$.
    Let $f\in\calH$ be non-vanishing on $UmU$.
    That is, $f(m)\neq 0$.
    We show that $m$ has the form
    \[
        m = \begin{bmatrix}
                &                       &     & D_1 \\
                &                       & D_2 &     \\
                & \reflectbox{$\ddots$} &     &     \\
            D_r &                       &     &
        \end{bmatrix},
    \]
    for some diagonal matrices $D_1$,\ldots,$D_r$.
    Equivalently, we show that if $m_{ij}$ and $m_{i+1,k}$ are non-zero, then we must have $k\leq j+1$.

    To see this, assume that $m_{ij},m_{i+1,k}\neq 0$ and $k>j+1$.
    Then define $x :=I_n + m_{ij}E_{i,i+1}\in U$ and $y:= I_n +m_{i+1,k}E_{jk}\in U$.
    Simple computations tell us that $xm = m+m_{ij}m_{i+1,k}e_{ik} = my$, $\pi(x)=\psi(m_{ij})\neq 1$ and $\pi(y)=\pi(0)=1$.
    Then, since $f\in\calH$, there holds $\pi(x)f(m)=f(xm)=f(my)=f(m)\pi(y)$.
    Thus $(\pi(x)-\pi(y))f(m)=0$, so $f(m)=0$ since $\pi(x)\neq \pi(y)$.

    Now we show that each diagonal matrix $D_i$ is actually a matrix of scalars.
    In particular, we show that if $m_{i,j}$ and $m_{i+1,j+1}$ are non-zero then they are equal.
    Consider $x$ and $y$ as given above, with $k=j+1$.
    Then $xm=my$, $\pi(x)=\psi(m_{ij})$, $\pi(y)=\psi(m_{i+1,j+1})$ and $(\pi(x)-\pi(y))f(m)=0$.
    Recall that $f$ doesn't vanish on $UmU$ so $f(m)\neq 0$.
    Thus $\pi(x)=\pi(y)$, which tells us that $\psi(m_{ij})=\psi(m_{i+1,j+1})$ and $m_{ij}=m_{i+1,j+1}$ by injectivity of $\psi$.

    Finally, notice $\varphi(m)=m$.
    This is easy to see, since $m^t$ is simply $m$ with the elements on the opposite diagonal reversed, and left- and right-multiplying by $w_0$ also reverses the opposite diagonal.
    This completes the proof.
\end{proof}

%%%%%%%%%%%%%%%%%%%%%%%%%%%%%%%%%%%%%%%%%%%%%%%%%%%%%%%%%%%%%%%%%%%%%%%%%%%%%%%%%%%%%%%%%%%%%%