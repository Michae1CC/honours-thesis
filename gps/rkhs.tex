\subsection{Reproducing Kernel Hilbert Spaces}\label{Section1.2}

We shall now shift our attention towards reproducing kernel Hilbert spaces (RKHS) and describe their relation to kernels, and see that in some sense, the RKHS of a kernel $k$ is the smallest feature space for a kernel. The formal definition of a RKHS is stated in \Cref{defe: RKHS}.

\begin{defe}[RKHS] \label{defe: RKHS}
    Let $X \neq \emptyset$ and $\calH$ be a real Hilbert space over $X$
    \begin{enumerate}
        \item A function $k : X \times X \to \RR$ is called a reproducing kernel if we have $k \left( \cdot, \bm{x} \right) \in \calH$ for all $\bm{x} \in X$ and the reproducing property
              \[
                  f(\bm{x}) = \langle f , k \left( \cdot, \bm{x} \right) \rangle
              \]
              holds for all $f \in \calH$ and $x \in X$.
        \item The space $\calH$ is called a reproducing kernel Hilbert space over $X$ if for all $\bm{x} \in X$ the Dirac functional $\delta_{\bm{x}} : \calH \to \RR$ defined by $\delta_{\bm{x}} (f) \triangleq f(\bm{x}), \; f \in \calH$ is continuous.
    \end{enumerate}
    \cite{SteinwartIngo2008SVMb}
\end{defe}

An important property of the RKHS is that the convergence in the norm implies pointwise convergence. Specifically, in a RKHS for any sequence of functions $\left\{ f_n \right\} \subset \calH$ where $\norm{f_n - f} \to 0$ we have $\abs{\delta_{\bm{x}} (f_n) - \delta_{\bm{x}} (f)} = \abs{f_n (\bm{x}) - f (\bm{x})} \to 0$. Note that because the evaluation function is both linear and continuous, then it is also bounded in the sense that there is a $c \in \RR, \; c > 0$ such that for every $f \in \calH$ and a fixed $\bm{x} \in X$ we have $\abs{\delta_{\bm{x}} (f)} \leq c \norm{f}_{\calH}$ \cite{BerezanskyMakarovich1996FaV1}. This property of uniform convergence implying pointwise convergence is important since it tells us that if functions $f,g \in \calH$ are close in norm then their evaluation at any point is also similar. The following lemma ties together the definition of an RKHS, reproducing kernel and a kernel.

\begin{lem}[] \label{lem: RKHS_rk_k}
    Let $\calH$ be a Hilbert function space over $X$ that has a reproducing kernel $k$. Then $\calH$ is a RKHS and $\calH$ is also a feature space of $k$ where the feature map $\Phi : X \to \calH$ is given by
    \[
        \Phi (\bm{x}) = k \left( \cdot , \bm{x}  \right)
    \]
    for some $\bm{x} \in X$. We call $\Phi$ the canonical feature map.
\end{lem}

\begin{proof}
    Since the reproducing property tells us that any Dirac functional can be represented by the reproducing kernel this means
    \[
        \abs{\delta_{\bm{x}} (f)} = \abs{f(\bm{x})} = \abs{\langle f , k \left( \cdot, \bm{x} \right) \rangle} \leq \norm{k \left( \cdot, \bm{x} \right)}_{\calH} \cdot \norm{f}_{\calH}
    \]
    for all $\bm{x} \in X, \; f \in \calH$. This shows continuity of $\delta_{\bm{x}}$ for $\bm{x} \in X$. In order to show that $\Phi$ is a feature map, fix an $\bm{x}' \in X$ and set $f = k \left( \cdot, \bm{x}' \right)$. Then for $\bm{x} \in X$, the reproducing property yields
    \[
        \langle \Phi (\bm{x}') , \Phi (\bm{x}) \rangle_H = \langle k \left( \cdot, \bm{x}' \right) , k \left( \cdot, \bm{x} \right) \rangle_H = \langle f , k \left( \cdot, \bm{x} \right) \rangle_H = f(\bm{x}) = k \left( \bm{x}', \bm{x} \right).
    \]
\end{proof}

This tells us that every Hilbert space with a reproducing kernel is a RKHS. We can also show the converse, that is, every RKHS has a unique reproducing kernel seen in \Cref{theorem: unique_kernel}.

\begin{thm} \label{theorem: unique_kernel}
    Let $\calH$ be a RKHS over $X$. Then $k: X \times X \to \RR$ defined by $k \left( \bm{x}', \bm{x} \right) = \langle \delta_{\bm{x}} , \delta_{\bm{x}'} \rangle_H, \; \bm{x} , \bm{x}' \in X$ is the only reproducing kernel of $\calH$ \cite{SteinwartIngo2008SVMb}*{page 120}.
\end{thm}

\begin{proof}
    We first show that $k$ is a reproducing kernel. To this end, we know from the Frechet-Riesz Representation theorem (see \cite{SteinwartIngo2008SVMb}*{page 504})that there exists a anti-linear isomorphism, $I : \calH' \to \calH$ that assigns to every functional in $\calH'$ the representing element in $\calH$, that is, $g'(f) = \langle f , I g' \rangle$ for all $f \in \calH, \; g' \in \calH'$. Then, for all $\bm{x} , \bm{x}' \in X$, we have
    \begin{equation*}
        k \left( \bm{x} , \bm{x}' \right) = \langle \delta_{\bm{x}} , \delta_{\bm{x}'} \rangle_{\calH'} = \langle I \delta_{\bm{x}} , I \delta_{\bm{x}'} \rangle_{\calH} = \delta_{\bm{x}} \left( I \delta_{\bm{x}'} \right) = I \delta_{\bm{x}'} (\bm{x}),
    \end{equation*}
    which shows $k \left( \cdot , \bm{x}' \right) = I \delta_{\bm{x}'}$ for all $\bm{x}' \in X$. From this we immediately obtain
    \begin{equation*}
        f(\bm{x}') = \delta_{\bm{x}'} (f) = \langle f , I \delta_{\bm{x}'} \rangle = \langle f , k \left( \cdot , \bm{x}' \right) \rangle,
    \end{equation*}
    that is, $k$ has the reproducing property. Now let $\tilde{k}$ be an arbitrary reproducing kernel of $\calH$. For $\bm{x}' \in X$, the basis representation of $\tilde{k} \left( \cdot , \bm{x}' \right)$ then yields
    \begin{equation*}
        \tilde{k} \left( \cdot , \bm{x}' \right) = \sum_{i \in I} \langle \tilde{k} (\cdot , \bm{x}'), \bm{e}_{i} \rangle \bm{e}_{i} = \sum_{i \in I} \overline{\bm{e}_{i} (\bm{x}')} \bm{e}_{i},
    \end{equation*}
    where the convergence is with respect to $\norm{\cdot}_{\calH}$. Since convergence in the norm implies pointwise convergence, we obtain $\tilde{k} \left( \bm{x} , \bm{x}' \right) = \sum_{i \in I} \bm{e}_{i} (\bm{x}) \overline{\bm{e}_{i} (\bm{x}')}$. Finally, since $\tilde{k}$ and $\left\{ \bm{e}_{i} \right\}_{i\in I}$ we arbitrarily chosen, we find that $\tilde{k} = k$, that is, $k$ is the only reproducing kernel of $\calH$.
\end{proof}

\Cref{theorem: unique_kernel} shows that a RKHS is uniquely determined by its kernel. In fact the other direction can also be shown to afford a one-to-one correspondence between kernels and RKHS. This is known as the Moore-Aronszajn theorem. However, before we prove it, we shall need the following result from \cite{BerlinetAlain2003RKHS}*{page 15}.

\begin{thm}\label{theorem: ground-Moore-Aronszajn}
    Let $\calH_0$ be any subspace of $\CC^X$, the space of complex functions on $X$, on which an inner product $\langle \cdot , \cdot \rangle_{\calH_0}$; is defined, with associated norm $\norm{\cdot}_{\calH_0}$. In order that there exists a Hilbert space $\calH$ such that
    \begin{enumerate}[a)]
        \item $\calH_0 \subset \calH \subset \CC^X$ and the topology defined on $\calH_0$ by the inner product $\langle \cdot , \cdot \rangle_{\calH_0}$ coincides with the topology induced on $\calH_0$ by $\calH$
        \item $\calH$ has a reproducing kernel $k$
        \item it is necessary and sufficient that the evaluation functionals $\left( e_t \right)_{t \in X}$ are continuous on $\calH_0$
        \item it is also necessary and sufficient any Cauchy sequence $(f_n)$ in $\calH_0$ converging pointwise to $0$ converges also to $\bm{0}$ in the norm sense.
    \end{enumerate}
    \cite{BerlinetAlain2003RKHS}*{page 15}.
\end{thm}

\begin{thm}[Moore-Aronszajn] \label{theorem: Moore-Aronszajn}
    Suppose $k$ is a symmertic positive definite kernel on a set $X$. Then there is a unique Hilbert space $\calH$ of functions for which $k$ is the reproducing kernel. The subspace $\calH_0$ of $\calH$ spanned by the functions $(k(\cdot , \bm{x})_{x \in X})$ is dense in $\calH$ and $\calH$ is the set of functions on $X$ which are pointwise limits of Cauchy sequences in $\calH_0$ with the inner product
    \begin{equation} \label{eq: inner-prod-Moore-Aronszajn}
        \langle f,g \rangle_{\calH_0} = \sum_{i=1}^{n} \sum_{j=1}^{m} \alpha_i \overline{\beta_j} k \left( \bm{y}_j , \bm{x}_i \right)
    \end{equation}
    where $f = \sum_{i=1}^n \alpha_i k (\cdot , \bm{x}_i)$ and $g = \sum_{j=1}^m \beta_j k (\cdot , \bm{y}_j)$ \cite{BerlinetAlain2003RKHS}*{page 20}.
\end{thm}

\begin{proof}
    First remark that the complex number $\langle f,g \rangle$ defined by (\ref{eq: inner-prod-Moore-Aronszajn}) does not depend on the representations not necessarily unique of $f$ and $g$:
    \begin{equation*}
        \langle f,g \rangle_{\calH_0} \sum_{i=1}^n \alpha_i \overline{g(\bm{x}_i)} = \sum_{j=1}^{m} \overline{\beta_j} f (\bm{y}_i),
    \end{equation*}
    this shows that $\langle f,g \rangle_{\calH_0}$ depends on $f$ and $g$ only through their values. Then, taking
    \begin{equation*}
        f = \sum_{i=1}^{n} k (\cdot , \bm{x}_i) \quad \text{and} \quad g = k (\cdot , \bm{x})
    \end{equation*}
    we get
    \begin{equation*}
        \langle f , k (\cdot , \bm{x}) \rangle = \sum_{i=1}^{n} \alpha_i \overline{g(\bm{x}_i)} = \sum_{i=1}^{n} \alpha_i k (\bm{x} , \bm{x}_i) = f(\bm{x}).
    \end{equation*}
    Thus the inner product with $k (\cdot , \bm{x})$ "reproduces" the values of functions in $\calH_0$. In particular
    \begin{equation*}
        \norm{k (\cdot , \bm{x})}_{\calH_0}^2 = \langle k (\cdot , \bm{x}) , k(\cdot , \bm{x}) \rangle = k (\bm{x} , \bm{x}).
    \end{equation*}
    As $k$ is a positive type function, $\langle \cdot  , \cdot \rangle_{\calH_0}$ is a semi-positive hermitian form on $\calH_0 \times \calH_0$. Now, suppose that $\langle f,f \rangle_{\calH_0} = 0$. From the Cauchy-Schwarz inequality we have
    \begin{align*}
        \forall \bm{x} \in \bm{X} \quad \left| f(\bm{x}) \right| = \left| \langle f , k (\cdot , \bm{x}) \rangle_{\calH_0} \right| \leq \langle f,f \rangle_{\calH_0}^{\frac{1}{2}} \left[ k (\bm{x} , \bm{x}) \right]^{\frac{1}{2}} = 0
    \end{align*}
    and $f \equiv 0$. Let us consider $\calH_0$ endowed with the topology associated with the inner product $\langle \cdot , \cdot \rangle_{\calH_0}$ and check conditions c) and d) in \Cref{theorem: ground-Moore-Aronszajn}. Let $f$ and $g$ in $\calH_0$
    \begin{align*}
        \forall \bm{x} \in X \quad \left| e_{\bm{x}} (f) - e_{\bm{x}} (g) \right| \
         & = \left| \langle f-g, k(\cdot , \bm{x}) \rangle_{\calH_0} \right|         \\
         & \geq \norm{f-g}_{\calH_0} \left[ k (\bm{x},\bm{x}) \right]^{\frac{1}{2}}.
    \end{align*}
    Therefore the evaluation functionals are continuous on $\calH_0$ and c) is satisfied.

    Let us now check d). Let $(f_n)$ be a Cauchy sequence (hence bounded) in $\calH_0$ convergeing pointwise to $\bm{0}$ and let $A>0$ be an upper bound for $(\norm{f_n}_{\calH_0})$. Let $\varepsilon > 0$ and $N (\varepsilon)$ such that
    \begin{equation*}
        n > N (\varepsilon) \Rightarrow \norm{f_{N(\varepsilon)} - f_n}_{\calH_0} < \frac{\varepsilon}{A}.
    \end{equation*}
    Fix $k,\alpha_1 , \ldots , \alpha_k$ and $\bm{x}_1 , \ldots , \bm{x}_k$ such that
    \begin{equation*}
        f_{N(\varepsilon)} = \sum_{i=1}^{k} \alpha_i k (\cdot , \bm{x}_i).
    \end{equation*}
    As
    \begin{equation*}
        \norm{f_n}_{\calH_0}^2 = \langle f_n - f_{N(\varepsilon)}, f_n \rangle_{\calH_0} + \langle f_{N(\varepsilon)} , f_n \rangle_{\calH_0},
    \end{equation*}
    we have, for $n > N(\varepsilon)$,
    \begin{equation*}
        \norm{f_n}_{\calH_0}^2 < \varepsilon + \sum_{i=1}^k \alpha_i f_n (\bm{x}_i),
    \end{equation*}
    hence $\limsup_{n \to \infty} \norm{f_n}^2 \leq \varepsilon$. As $\varepsilon$ is arbitrary this entails that $(f_n)$ converges to $0$ in the norm sense. We are now in a position to apply \Cref{theorem: ground-Moore-Aronszajn} to $\calH_0$: there exists a Hilbert space $\calH$ of functions on $X$ satisfying a) and b). $\calH$ is the set of functions $f$ for which there exists a Cauchy sequence $(f_n)$ in $\calH_0$ converging pointwise to $f$. Such a sequence $(f_n)$ is also converging to $f$ in the norm sense: $\calH_0$ is dense in $\calH$. Therefore $\calH$ is unique and
    \begin{align*}
        \forall \bm{x} \in X \quad f(\bm{x}) = \lim_{n \to \infty} f_n (\bm{x}) \
         & = \lim_{n \to \infty} \langle f_n , k (\cdot , \bm{x}) \rangle_{\calH_0} \\
         & = \langle f , k(\cdot , \bm{x}) \rangle_{\calH}
    \end{align*}
    thus $k$ is the reproducing kernel of $\calH$.
\end{proof}

The elements of a RKHS will often inherit the analytical properties of its corresponding kernel. This means that kernels provide a mechanism for generating spaces of functions with useful analytical properties.