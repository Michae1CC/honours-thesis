\section{Random Fourier Features}\label{Chapter3}
We saw in section \ref{Chapter1} that GPs relied heavily on the Gram matrix (see definition \ref{defe: Gram_Matrix}) to create predictions based on training data $\calD = \left( \bm{X} , \bm{y} \right)$ where $\bm{X} = \left[ \bm{x}_1 , \bm{x}_2 , \ldots , \bm{x}_n \right]^{\intercal} \in \RR^{n \times d}$ and $\bm{y} = \left[ y_1 , y_2 , \ldots , y_n \right]^{\intercal} \in \RR^{n}$. Unfortunately, the size of the Gram matrix scales quadratically with the number of samples making it difficult to train using data sets with more than $10^5$ samples. Instead the kernel function itself can be factorized allowing one to convert training and kernel evaluation into the corresponding operations of a linear machine by mapping data into a relatively low-dimensional randomized feature space. This idea was first introduced by Rahimi and Recht \cite{NIPS2007_013a006f} where they propose that instead of using a kernel function to implicitly lift data into a higher dimensional feature space, an explicit feature map $\varphi : \RR^d \to \RR^D$ can be used to approximate $k$ as $k \left( \bm{x} , \bm{y} \right) = \langle \Phi (\bm{x}) , \Phi (\bm{y}) \rangle_{\RR^N} \approx \langle \varphi (\bm{x}) , \varphi (\bm{y}) \rangle_{\RR^D}$ where $D$ is chosen so that $n \gg  D$. Thus once $\varphi (\bm{x}_i)$ has been computed for each $\bm{x}_i$, each entry of the Gram matrix can be swiftly approximated as
\[
    \bm{K}_{ij} = \bm{K}_{ji} \approx \langle \varphi (\bm{x}_i) , \varphi (\bm{y}_j) \rangle_{\RR^D}.
\]
Already there have been numerous applications of this technique in GPs that have seen improved time performance with little loss in prediction accuracy \cite{PotapczynskiAndres2021BSGP}.

\subsection{Theory and Computation}\label{Section3.1}
Contrary to the kernel trick the Random Fouier Features (RFF) technique approximates $\langle \Phi \left( \cdot \right) , \Phi \left( \cdot \right) \rangle_{\RR^N}$ through an explicit feature mapping $\varphi$. The RFF techniques hinges on Bochners theorem stated without proof in theorem \ref{theorem: bochner} which characterises positive definite functions.

\begin{thm}{Bochner's} \label{theorem: bochner}
    A continuous and shift-invariant function $k \left( \bm{x} , \bm{y} \right) = k \left( \bm{x} - \bm{y} \right) = k \left( \Delta \right)$ is positive definite (see definition \ref{defe: PD}) if and only if it can be represented as
    \[
        k \left( \bm{x} - \bm{y} \right) = \int_{\RR} \exp \left( i \langle \bm{\omega} , \bm{x} - \bm{y} \rangle \right) \mu_k \left( d \; \bm{\omega} \right)
    \]
    where $\mu_k$ is a positive finite measure on the frequencies of $\bm{\omega}$ \cite{HahnHans1933SBVü,LiuFanghui2021RFfK}.
\end{thm}

The spectral distribution $\mu_k$ can be represented as finite measure induced by the Fourier transformation. Choosing a kernel for which $k (\bm{0}) = 1$ normalizes $\mu_k$ to a probability distribution $p (\cdot)$. For instance, the spectral distribution of the Gaussian RBF kernel is
\[
    p \left( \bm{w}\right) = \frac{1}{\sqrt{\left( 2 \pi\right)^{D} \left| \frac{\sigma^2}{2} \Id_{D \times D} \right|}} \exp \left( -\frac{1}{2} \bm{w}^\intercal \left( \frac{\sigma^2}{2} \Id_{D \times D} \right)^{-1} \bm{w} \right)
\]
\cite{NIPS2007_013a006f}*{page 3}. One important detail required for Bochner's theorem to work is that it requires our kernel to be shift-invariant (sometimes also referred to as stationary) stated in definition \ref{defe: shift-invar}.

\begin{defe}[Shift-Invariant] \label{defe: shift-invar}
    A kernel $k : \RR^N \times \RR^N \to \CC$ is called shift-invariant if $k \left( \bm{x}, \bm{y} \right) = g \left( \bm{x} - \bm{y} \right)$ for some positive definite function $g : \RR^N \to \CC$ \cite{JMLR:v17:14-538}*{page 3}.
\end{defe}

Clearly the Gaussian RBF kernel is shift invariant since it only relies on the bounding radius of $\bm{x}$ and $\bm{y}$. Thus from Bochners theorem a positive definite sift-invariant kernel with $k(0) = 1$ can be computed as
\begin{equation} \label{eq: rff_boch_int}
    k \left( \bm{x} - \bm{y} \right) = \int_{\RR} \exp \left( i \langle \bm{\omega} , \bm{x} - \bm{y} \rangle \right) p (\bm{\omega}) d \; \bm{\omega} .
\end{equation}
The main of idea of RFF is to approximate the integral in \ref{eq: rff_boch_int} using the following Monte-Carlo estimate
\begin{align*}
    k \left( \bm{x} - \bm{y} \right)
     & = \int_{\RR} \exp \left( i \langle \bm{\omega} , \bm{x} - \bm{y} \rangle \right) p (\bm{\omega}) d \; \bm{\omega}                                                                                                              \\
     & = \EE_{\bm{\omega} \sim p (\cdot)} \left( \exp \left( i \langle \bm{\omega} , \bm{x} - \bm{y} \rangle \right) \right)                                                                                                          \\
     & \approx \frac{1}{D} \sum_{j=1}^{D} \exp \left( i \langle \bm{\omega}_{j} , \bm{x} - \bm{y} \rangle \right)                                                                                                                     \\
     & = \sum_{j=1}^{D} \left( \frac{1}{\sqrt{D}} \exp \left( i \langle \bm{\omega}_{j} , \bm{x} \rangle \right) \right) \overline{\left( \frac{1}{\sqrt{D}} \exp \left( i \langle \bm{\omega}_{j} , \bm{y} \rangle \right) \right) } \\
     & = \langle \varphi (\bm{x}) , \varphi (\bm{y}) \rangle_{\RR^D}
\end{align*}
where $\bm{\omega}_i \stackrel{\text{iid}}{\sim} p(\cdot)$ using the feature map
\begin{equation}
    \varphi (\bm{x}) = \frac{1}{\sqrt{D}} \left[ z \left( \bm{\omega}_1, \bm{x} \right), z \left( \bm{\omega}_2, \bm{x} \right), \ldots , z \left( \bm{\omega}_D, \bm{x} \right) \right]^{\intercal}
\end{equation}
where for convenience $z \left( \bm{\omega}, \bm{x} \right) = \exp \left( i \langle \bm{\omega} , \bm{x} \rangle \right)$. This allows us to estimate the Gram matrix as $\bm{K} \approx \bm{\widetilde{K}} = \bm{Z} \bm{Z}^{\intercal}$ where $\bm{Z} = \left[ \varphi (\bm{x}_1), \varphi (\bm{x}_2), \ldots \varphi (\bm{x}_D) \right] \in \RR^{n \times D}$ \cite{NIPS2007_013a006f,LiuFanghui2021RFfK,JMLR:v17:14-538}. To simplify computation, in most settings both $p(\cdot)$ and $k(\Delta)$ are real valued functions meaning $\exp \left( i \langle \bm{\omega} , \bm{x} - \bm{y} \rangle \right)$ can replaced with its real component $\cos \left( \langle \bm{\omega} , \bm{x} - \bm{y} \rangle \right)$. Rahimi and Recht offer two embeddings for $\cos \left( \langle \bm{\omega} , \bm{x} - \bm{y} \rangle \right)$ for which $z \left( \bm{\omega}, \bm{x} \right)$ satisfies equation \ref{eq: rff_boch_int}, used in the vast majority of literature. The first embedding takes the form
\begin{equation} \label{eq: rff_emb}
    z \left( \bm{\omega}, \bm{x} \right) = \left[ \cos \left( \langle \bm{\omega} , \bm{x} \rangle \right) , \sin \left( \langle \bm{\omega} , \bm{x} \rangle \right) \right]
\end{equation}
which satisfies \ref{eq: rff_boch_int} since
\begin{align*}
     & z \left( \bm{\omega}, \bm{x} \right)^{\intercal} z \left( \bm{\omega}, \bm{y} \right)                                                                                                                                                                                                                                                                                                                                                                         \\
     & =
    \begin{bmatrix}
        \cos \left( \langle \bm{\omega} , \bm{x} \rangle \right) \\ \sin \left( \langle \bm{\omega} , \bm{x} \rangle \right)
    \end{bmatrix}
    \left[ \cos \left( \langle \bm{\omega} , \bm{y} \rangle \right), \sin \left( \langle \bm{\omega} , \bm{y} \rangle \right)  \right]                                                                                                                                                                                                                                                                                                                               \\
     & = \cos \left( \langle \bm{\omega} , \bm{x} \rangle \right) \cos \left( \langle \bm{\omega} , \bm{y} \rangle \right) + \sin \left( \langle \bm{\omega} , \bm{x} \rangle \right) \sin \left( \langle \bm{\omega} , \bm{y} \rangle \right)                                                                                                                                                                                                                       \\
     & = \frac{1}{2} \left( \cos \left( \langle \bm{\omega} , \bm{x} \rangle + \langle \bm{\omega} , \bm{y} \rangle \right) + \cos \left( \langle \bm{\omega} , \bm{x} \rangle - \langle \bm{\omega} , \bm{y} \rangle \right) \right) + \frac{1}{2} \left( \cos \left( \langle \bm{\omega} , \bm{x} \rangle - \langle \bm{\omega} , \bm{y} \rangle \right) - \cos \left( \langle \bm{\omega} , \bm{x} \rangle + \langle \bm{\omega} , \bm{y} \rangle \right) \right) \\
     & = \cos \left( \langle \bm{\omega} , \bm{x} - \bm{y} \rangle \right).
\end{align*}
The other embedding Rahimi and Recht give is
\begin{equation} \label{eq: rff_emb_alt}
    z (\bm{x}) = \sqrt{2} \cos \left(  \langle \bm{\omega} , \bm{x} \rangle + b \right)
\end{equation}
where $b \sim U \left[ 0,2 \pi \right]$. Using a similar argument we can show that this embedding also satisfies \ref{eq: rff_boch_int}. However, Sutherland's and Schneider's paper \cite{sutherland2015error} they argue that the Gaussian RBF kernel is better suited for the embedding given in \ref{eq: rff_emb}. To summarise their argument we denote
\[
    z_1 (\bm{x}) = \sqrt{\frac{2}{D}}
    \begin{bmatrix}
        \cos \left( \langle \bm{\omega}_{1} , \bm{x} \rangle \right)   \\
        \cos \left( \langle \bm{\omega}_{2} , \bm{x} \rangle \right)   \\
        \vdots                                                         \\
        \cos \left( \langle \bm{\omega}_{D/2} , \bm{x} \rangle \right) \\
        \sin \left( \langle \bm{\omega}_{1} , \bm{x} \rangle \right)   \\
        \vdots                                                         \\
        \sin \left( \langle \bm{\omega}_{D/2} , \bm{x} \rangle \right)
    \end{bmatrix}
\]
to be the feature map corresponding to embedding in equation \ref{eq: rff_emb} and
\[
    z_2 (\bm{x}) = \sqrt{\frac{2}{D}}
    \begin{bmatrix}
        \cos \left( \langle \bm{\omega}_{1} , \bm{x} \rangle + b_1 \right) \\
        \vdots                                                             \\
        \cos \left( \langle \bm{\omega}_{D} , \bm{x} \rangle + b_D \right)
    \end{bmatrix}
\]
to be the feature map corresponding to equation \ref{eq: rff_emb}. Furthermore, $\varphi_1$ and $\varphi_2$ denote the feature maps corresponding to $z_1$ and $z_2$ respectively. They then show that
\begin{align*}
    \VV \left[ \varphi_1 (\Delta) \right] & = \frac{1}{D} \left( 1 + k (2 \Delta) - 2 {k( \Delta)}^2 \right)           \\
    \VV \left[ \varphi_2 (\Delta) \right] & = \frac{1}{D} \left( 1 + \frac{1}{2} k (2 \Delta) - {k( \Delta)}^2 \right)
\end{align*}
so that the variance of $\varphi_1$ is smaller whenever
\[
    \VV \left[ \cos \left( \langle \bm{\omega} , \Delta \rangle \right) \right] = \frac{1}{2} + \frac{1}{2} k (2 \Delta) - {k( \Delta)}^2 \leq \frac{1}{2} .
\]
When using the Gaussian kernel,
\[
    \VV \left[ \cos \left( \langle \bm{\omega} , \Delta \rangle \right) \right] = \frac{1}{2} \left( 1 - \exp \left( - \frac{2\norm{\Delta}^2_2}{\sigma^2} \right) \right)^2 \leq \frac{1}{2}
\]
meaning $\varphi_1 (\Delta) \leq \varphi_2 (\Delta)$ for any $\Delta \in \RR^d$. This result is indeed consistent with our preliminary experiments. Thus for experimentation an embedding of $\varphi_1$ is always used.

\subsection{Orthogonal Random Features}\label{Section3.2}
In the previous chapter algorithm \ref{alg: RFF-algorithm} assumed some sort of mechanism for producing the transformation matrix $\bm{W}$. The construction presented in \ref{Section3.1} involved sampling $\bm{\omega}_{i} \stackrel{\text{iid}}{\sim} p(\cdot)$. For the Gaussian RBF kernel this meant sampling from the multivariate Gaussian distribution $\calN \left( \bm{0} , \Id_{D \times D} \right)$. The transformation matrix constructed in this manner was denoted $\bm{W}_{\text{RFF}}$. Recently, there has been a buzz in the literature exploring alternative constructs for the transformation matrix described in section \ref{Section3.1} \cite{LiuFanghui2021RFfK}. We shall consider the two methods proposed by Yu {\it et al.} \cite{YuFelixX2016ORF}; the first method here and the second in the following section (\ref{Section3.3}). The first method from Yu {\it et al.} is the Orthogonal Random Features (ORF) method with imposes orthogonality on the transformation matrix. To do this a Gaussian matrix $\bm{G} \in \RR^{D \times d}$ is first produced, much like in $\bm{W}_{\text{RFF}}$. An orthogonal matrix $\bm{Q}$ is then created by taking the QR-factorization (see section \ref{Section4.2}) of $\bm{G}$. However, the random orthogonal matrix, $\bm{Q}$, will not give an unbiased estimate of the kernel matrix. To fix this, the following common probabilistic identity is employed
\[
    \norm{\bm{z}}^2_2 \sim \chi^2_k, \; \text{where} \; \bm{z} \sim \calN \left( \bm{0}, \Id_{k \times k} \right)
\]
where $\chi^2_k$ is the chi-squared distribution with $k$ degrees of freedom \cite{BrockwellPeterJ1991TSTa}*{page 41}. This identity is easily demonstrated by equating a shared moment generating function of $(1-2t)^{-\frac{k}{2}}$ for $t < \frac{1}{2}$. Taking the square root of both sides gives $\norm{\bm{z}}_2 \sim \chi_k$ where $\chi_k$ is the chi distribution with $k$ degrees of freedom. In the RFF method, each $\bm{\omega}_i \in \RR^D$ was independently taken from the multivariate normal Gaussian distribution meaning that using the identity provided above $\norm{\bm{\omega}_i}_2 \sim \chi_D$. The ORF method augments $\bm{Q}$ by scaling its rows by iid $\chi_D$ values which can be accomplished through right multiplication with $\bm{S} = \operatorname{diag} \left( \psi_1 , \psi_2 , \ldots , \psi_D \right)$ where $\psi_i \stackrel{\text{iid}}{\sim} \chi_D$. This means
\[
    \norm{\left(\bm{S} \bm{Q}\right)_{(i)}}_2 = \norm{\psi_i \bm{Q}_{(i)}}_2 = \psi_i \sim \chi_D
\]
so that the row norms of $\bm{G}$ and $\bm{S} \bm{Q}$ have the same distribution. Thus the transformation matrix for the ORF method is
\begin{equation}
    \bm{W}_{\text{ORF}} = \left( \frac{\sigma}{\sqrt{2}} \right)^{-1} \bm{S} \bm{Q}.
\end{equation}
The main downside the the ORF method is that the QR-factorization brings a computational cost of $\calO \left( Dd \right)$. Fortunately when using $\bm{W}_{\text{ORF}}$ as our transformation matrix in algorithm \ref{alg: RFF-algorithm} the approximate Gram matrix $\bm{\widetilde{K}}_{\text{RFF}}$ is an unbiased estimate of $\bm{K}$, stated more formally in theorem \ref{thm: orf-unbiased}.

\begin{thm} \label{thm: orf-unbiased}
    $\bm{\widetilde{K}}_{\text{ORF}}$ is an unbiased estimate of $\bm{K}$, that is
    \[
        \EE \left[ \left( \bm{\widetilde{K}}_{\text{ORF}} \right)_{ij} \right] = \exp \left( \frac{- \norm{\bm{x}_i - {\bm{x}_j}}_{2}^{2}}{\sigma^2} \right)
    \] \cite{YuFelixX2016ORF}*{page 3}.
\end{thm}

Furthermore, the variance of $\left( \bm{\widetilde{K}}_{\text{ORF}} \right)_{ij}$ is bounded by
\[
    \VV \left[ \left( \bm{\widetilde{K}}_{\text{ORF}} \right)_{ij} \right] - \VV \left[ \left( \bm{\widetilde{K}}_{\text{RFF}} \right)_{ij} \right] = \frac{1}{D} \left( \frac{g (\tau)}{d} - \frac{(d-1)e^{-\tau^2}\tau^4}{2d} \right)
\]
where $\tau = \norm{\bm{x}_i - \bm{x}_j}_2 / \frac{\sigma}{\sqrt{2}}$ and
\[
    g (\tau) = \frac{e^{\tau^2} \left( \tau^8 + 6 \tau^6 + 7 \tau^4 + \tau \right)}{4} + \frac{e^{\tau^2} \tau^4 \left( \tau^6 + 2 \tau^4 \right)}{2d}
\]
\cite{LiuFanghui2021RFfK}*{page 8}. This shows that there are scenarios for which $\VV \left[ \left( \bm{\widetilde{K}}_{\text{ORF}} \right)_{ij} \right] < \VV \left[ \left( \bm{\widetilde{K}}_{\text{RFF}} \right)_{ij} \right]$, namely when $d$ is large and $\tau$ is small. Also, the ratio in variance between $\bm{\widetilde{K}}_{\text{ORF}}$ and $\bm{\widetilde{K}}_{\text{RFF}}$ for large $d$ can be approximated as
\[
    \frac{\VV \left[ \left( \bm{\widetilde{K}}_{\text{ORF}} \right)_{ij} \right]}{\VV \left[ \left( \bm{\widetilde{K}}_{\text{RFF}} \right)_{ij} \right]} \simeq 1 - \frac{(s-1)e^{-\tau^2}\tau^4}{d(1-e^{-\tau^2})^2}
\]
\cite{LiuFanghui2021RFfK}*{page 8}.

\subsection{Random Ortho-Matrices and Structured Orthogonal Random Matrices}\label{Section3.3}

The second method we shall consider for producing a transformation matrix also originates from Yu's {\it et al.} paper, which Choromanski {\it et al.} \cite{ChoromanskiKrzysztof2017TUEo} generalize as Random Ortho-Matrices (ROM). This second class of methods is motivated by creating transformation matrices with the same variance reductions as ORF with the added benefit time and memory savings. The transformation matrices generated using ROM take the form
\begin{equation} \label{eq: ROM-general}
    \bm{W}_{\text{ROM}} = \sqrt{d} \prod_{i=1}^{k} \bm{S} \bm{D}_{i}
\end{equation}
where $\bm{S} \in \RR^{D \times D}$ has orthogonal rows and $\bm{D} = \operatorname{diag} \left( \delta_1 , \ldots , \delta_D \right) \in \RR^{D \times D}$ where $\delta_i \stackrel{\text{iid}}{\sim} U \left( \left\{ -1, 1 \right\} \right)$, sometimes called the Radamach distribution. This matrix can be forced into a $\RR^{D \times d}$ sized matrix by simply extracting the first $d$ columns of $\bm{D}_1$. The matrix to take the role of $\bm{S}$ in virtually every application of ROM is the Hadamard matrix, defined in \ref{defe: Hadamard-Matrix}, which admits a fast $m \log (n)$ matrix multiplication for a size $m \times n$ Hadamard matrix called the Fast Walsh-Hadamard transform (FWHT) \cite{Fino1976UMTo}.

\begin{defe}[Hadamard Matrix] \label{defe: Hadamard-Matrix}
    The Hadamard matrix $\bm{H}_i \in \RR^{\left( 2^{i-1} \times 2^{i-1} \right)}$ is defined recursively as
    \[
        \bm{H}_i =
        \left\{
        \begin{array}{cc}
            \left[ 1 \right] & , i=1 \\
            \frac{1}{\sqrt{2}}
            \begin{bmatrix}
                \bm{H}_{i-1} & \bm{H}_{i-1}   \\
                \bm{H}_{i-1} & - \bm{H}_{i-1} \\
            \end{bmatrix}
                             & , i>1
        \end{array}
        \right. .
    \]
\end{defe}

Note that while Hadamard matrices are only defined for dimensions of exact powers of 2, other sizes can be constructed by removing portions of the matrix given in definition \ref{defe: Hadamard-Matrix} or by padding by $0$. This gives a concrete means for which one can generate a transformation matrix
\begin{equation} \label{eq: ROM-Hadamard}
    \sqrt{d} \prod_{i=1}^{k} \bm{H} \bm{D}_{i}
\end{equation}
where $\bm{H}$ is an appropriately sized Hadamard matrix. It is easy to check that the matrix generated by \ref{eq: ROM-Hadamard} shares the same expected rows norm lengths as $\bm{W}_{\text{ORF}}$ and thus enjoys the same variance reduction benefits.

Despite the wide use of the ROM method in various machine learning tasks \cite{ChoromanskiKrzysztof2017TUEo,AndoniAlexandr2015PaOL,ChoromanskiKrzysztof2020RAwP} a number of high-interest theoretical properties remain unsolved problems leaving many aspects of this method shrouded in mystery. Instead, much of what we understand about ROM's estimate capabilities comes from empirical analysis, although we will still cover a smaller number of important results that have been established.

Choromanski {\it et al.} \cite{ChoromanskiKrzysztof2017TUEo} show that there is diminishing returns, estimation wise, for choosing larger values of $k$ in equation \ref{eq: ROM-Hadamard}. They also show that choosing odd values of $k$ in \ref{eq: ROM-Hadamard} provide better estimates then its even $k-1$ and $k+1$ counterparts. For this reason a $k$ value of $3$ is usually chosen which yields the transformation matrix estimate given in equation \ref{eq: SORF-tm}. The method for constructing transformation matrices in this manner is referred to as Structured Orthogonal Random Feature (SORF).
\begin{equation} \label{eq: SORF-tm}
    \bm{W}_{\text{SORF}} = \sqrt{d} \bm{H} \bm{D}_{3} \bm{H} \bm{D}_{2} \bm{H} \bm{D}_{1}
\end{equation}
This is the same transformation matrix estimate that Yu {\it et al.} provides. Unfortunately using the SORF method in algorithm \ref{alg: RFF-algorithm} does not produce an unbiased estimate of the Gram matrix. However SORF does satisfy an asymptotic unbiased property
\begin{equation*}
    \left| \EE \left[ \left( \bm{\widetilde{K}}_{\text{SORF}} \right)_{ij} \right] - \EE \left[ \left( \bm{\widetilde{K}}_{\text{RFF}} \right)_{ij} \right] \right| \leq \frac{6 \tau}{\sqrt{d}}
\end{equation*}
where $\tau$ is again $\norm{\bm{x}_i - \bm{x}_j}_2 / \frac{\sigma}{\sqrt{2}}$ \cite{LiuFanghui2021RFfK}*{page 8}.

\begin{tikzpicture}[tdplot_main_coords, scale = 2.5]

    % Create a point (P)
    \coordinate (P) at ({1/sqrt(3)},{1/sqrt(3)},{1/sqrt(3)});

    % Draw shaded circle
    \shade[ball color = SkyBlue,
        opacity = 0.5
    ] (0,0,0) circle (1cm);

    % draw arcs 
    \tdplotsetrotatedcoords{0}{0}{0};
    \draw[dashed,
        tdplot_rotated_coords,
        darkgray
    ] (0,0,0) circle (1);

    % Axes in 3 d coordinate system
    \draw[-stealth] (-1.80,0,0) -- (1.80,0,0)
    node[below left] {$x$};

    \draw[-stealth] (0,-1.30,0) -- (0,1.30,0)
    node[below right] {$y$};

    \draw[-stealth] (0,0,-1.30) -- (0,0,1.30)
    node[above] {$z$};

    % Vector v
    \draw[-stealth] (0,0,0) -- (-0.19588084,  0.97940421,  0.04897021)
    node[above right] {$\bm{v}$};

    % Vector H D_1 v
    \draw[-stealth] (0,0,0) -- (-0.37139068,  0.74278135,  0.55708601)
    node[above right] {$\bm{H} \bm{D}_1 \bm{v}$};

\end{tikzpicture}

\begin{tikzpicture}[tdplot_main_coords, scale = 2.5]

    % Create a point (P)
    \coordinate (P) at ({1/sqrt(3)},{1/sqrt(3)},{1/sqrt(3)});

    % Axes in 3 d coordinate system
    \draw[-stealth] (-1.80,0,0) -- (1.80,0,0)
    node[below left] {$x$};

    \draw[-stealth] (0,-1.30,0) -- (0,1.30,0)
    node[below right] {$y$};

    \draw[-stealth] (0,0,-1.30) -- (0,0,1.30)
    node[above] {$z$};

    % Vector v
    \draw[-stealth] (0,0,0) -- (-0.37139068,  0.74278135,  0.55708601)
    node[above right] {$\bm{v}$};

    % Vector w
    \draw[-stealth] (0,0,0) -- (-0.55708601,  0.37139068,  0.74278135)
    node[above left] {$\bm{w}$};

    % Vector u
    \draw[-stealth] (0,0,0) -- (-0.46423834,  0.55708601,  0.64993368)
    node[above right] {$\bm{u}$};

    % Vector H D_2 v
    \draw[-stealth] (0,0,0) -- (0.83078669, -0.47668089, -0.28734788)
    node[above left] {$\bm{H} \bm{D}_2 \bm{v}$};

    % Vector H D_2 w
    \draw[-stealth] (0,0,0) -- (-0.73,-0.57,-0.38)
    node[above left] {$\bm{H} \bm{D}_2 \bm{w}$};

    % Vector H D_2 u
    \draw[-stealth] (0,0,0) -- (0.22058188, -0.57508846,  0.78779242)
    node[above left] {$\bm{H} \bm{D}_2 \bm{u}$};

\end{tikzpicture}

\begin{tikzpicture}[tdplot_main_coords, scale = 2.5]

    % Create a point (P)
    \coordinate (P) at ({1/sqrt(3)},{1/sqrt(3)},{1/sqrt(3)});

    % Draw shaded circle
    \shade[ball color = SkyBlue,
        opacity = 0.5
    ] (0,0,0) circle (1cm);

    % draw arcs 
    \tdplotsetrotatedcoords{0}{0}{0};
    \draw[dashed,
        tdplot_rotated_coords,
        darkgray
    ] (0,0,0) circle (1);

    % Axes in 3 d coordinate system
    \draw[-stealth] (-1.80,0,0) -- (1.80,0,0)
    node[below left] {$x$};

    \draw[-stealth] (0,-1.30,0) -- (0,1.30,0)
    node[below right] {$y$};

    \draw[-stealth] (0,0,-1.30) -- (0,0,1.30)
    node[above] {$z$};

    % Vector r
    \draw[-stealth] (0,0,0) -- (0.44366934, -0.44366934,  0.77866233)
    node[above right] {$\bm{r}$};

    % Vector w
    \draw[-stealth] (0,0,0) -- (-0.69631062, -0.69631062,  0.17407766)
    node[above right] {$\bm{w}$};

    % Vector v
    \draw[-stealth] (0,0,0) -- (0.87287156, -0.43643578,  0.21821789)
    node[above right] {$\bm{v}$};

\end{tikzpicture}

Bojarski {\it et al.} \cite{BojarskiMariusz2016Saar}*{page 4} give an intuitive explanation for the roles of each of the different blocks $\bm{H} \bm{D}_1$, $\bm{H} \bm{D}_2$ and $\bm{H} \bm{D}_3$. Since the first block can be shown to satisfy
\[
    \PP \left[ \norm{\bm{H}\bm{D}_1 \bm{x}}_{\infty} > \frac{\log D}{\sqrt{D}} \right] \leq 2 d \exp \left( {-\frac{\log^2 D}{8}} \right), \quad \bm{x} \in \RR^D
\]
\cite{LiuFanghui2021RFfK}*{page 8} so that it can be thought as a "balancer" leaving no single dimension bearingtoo much of the $l^2$ norm. For the secnd block, the cost of using a structured matrix is the loss of independence. The role of the second block is to mitigate this effect by making similar input vectors near-orthogonal. Finally the third block controls the capacity of the entire structure by providing a vector of parameters. Near-independence is now implied by the near-orthogonality achieved by the proceeding block and the fact that the projections of the Gaussian vector or Radamacher vector onto "almost orthogonal directions" and "close to independent".