\documentclass[11pt]{amsart}
\usepackage[utf8]{inputenc}
\usepackage[T1]{fontenc}
\allowdisplaybreaks
\usepackage{hyperref}
\hypersetup{
  colorlinks   = true, % Colours links instead of ugly boxes
  urlcolor     = {blue!80!black}, % Colour for external hyperlinks
  linkcolor    = {red!80!black}, % Colour of internal links
  citecolor   = {blue!50!black} % Colour of citations	
}
\usepackage[left=0.8in,right=0.8in,top=1in,bottom=1in]{geometry}
\usepackage{amsmath, amsthm, amssymb, amsfonts, amsthm, bbold, bm, xcolor}
\usepackage{tikz, tikz-cd, float, paralist} 
\usepackage{xcolor}
\usepackage[nobysame]{amsrefs}
\usepackage{tgpagella}
\usepackage{pdiag}

\usepackage{tikz, tcolorbox}

% Algorithm package
% \usepackage[linesnumbered,lined,boxed,vlined,ruled]{algorithm2e}
\usepackage[lined,boxed,vlined,ruled]{algorithm2e}
\SetKw{Continue}{continue}
\SetKw{Or}{or}
\SetKw{Until}{until}
% Set the algorithm comments to blue
\newcommand\mycommfont[1]{\ttfamily\textcolor{blue}{#1}}
\SetCommentSty{mycommfont}
% \SetAlFnt{\large}
% \SetAlCapFnt{\large}
% \SetAlCapNameFnt{\large}
% \SetProcFnt{\Large}

\newcommand{\nc}{\newcommand}
\newcommand{\rc}{\renewcommand}
\nc{\on}{\operatorname}

% Commands 
\rc{\AA}{\mathbb{A}}	\nc{\calA}{\mathcal{A}}	\nc{\fraka}{\mathfrak{a}}
\nc{\BB}{\mathbb{B}}	\nc{\calB}{\mathcal{B}}	\nc{\frakb}{\mathfrak{b}}
\nc{\CC}{\mathbb{C}}	\nc{\calC}{\mathcal{C}}	\nc{\frakc}{\mathfrak{c}}
\nc{\DD}{\mathbb{D}}	\nc{\calD}{\mathcal{D}}	\nc{\frakd}{\mathfrak{d}}
\nc{\EE}{\mathbb{E}}	\nc{\calE}{\mathcal{E}}	\nc{\frake}{\mathfrak{e}}
\nc{\FF}{\mathbb{F}}	\nc{\calF}{\mathcal{F}}	\nc{\frakf}{\mathfrak{f}}
\nc{\GG}{\mathbb{G}}	\nc{\calG}{\mathcal{G}}	\nc{\frakg}{\mathfrak{g}}
\nc{\HH}{\mathbb{H}}	\nc{\calH}{\mathcal{H}}	\nc{\frakh}{\mathfrak{h}}
\nc{\II}{\mathbb{I}}	\nc{\calI}{\mathcal{I}}	\nc{\fraki}{\mathfrak{i}}
\nc{\JJ}{\mathbb{J}}	\nc{\calJ}{\mathcal{J}}	\nc{\frakj}{\mathfrak{j}}
\nc{\KK}{\mathbb{K}}	\nc{\calK}{\mathcal{K}}	\nc{\frakk}{\mathfrak{k}}
\nc{\LL}{\mathbb{L}}	\nc{\calL}{\mathcal{L}}	\nc{\frakl}{\mathfrak{l}}
\nc{\MM}{\mathbb{M}}	\nc{\calM}{\mathcal{M}}	\nc{\frakm}{\mathfrak{m}}
\nc{\NN}{\mathbb{N}}	\nc{\calN}{\mathcal{N}}	\nc{\frakn}{\mathfrak{n}}
\nc{\OO}{\mathbb{O}}	\nc{\calO}{\mathcal{O}}	\nc{\frako}{\mathfrak{o}}
\nc{\PP}{\mathbb{P}}	\nc{\calP}{\mathcal{P}}	\nc{\frakp}{\mathfrak{p}}
\nc{\QQ}{\mathbb{Q}}	\nc{\calQ}{\mathcal{Q}}	\nc{\frakq}{\mathfrak{q}}
\nc{\RR}{\mathbb{R}}	\nc{\calR}{\mathcal{R}}	\nc{\frakr}{\mathfrak{r}}
\rc{\SS}{\mathbb{S}}	\nc{\calS}{\mathcal{S}}	\nc{\fraks}{\mathfrak{s}}
\nc{\TT}{\mathbb{T}}	\nc{\calT}{\mathcal{T}}	\nc{\frakt}{\mathfrak{t}}
\nc{\UU}{\mathbb{U}}	\nc{\calU}{\mathcal{U}}	\nc{\fraku}{\mathfrak{u}}
\nc{\VV}{\mathbb{V}}	\nc{\calV}{\mathcal{V}}	\nc{\frakv}{\mathfrak{v}}
\nc{\WW}{\mathbb{W}}	\nc{\calW}{\mathcal{W}}	\nc{\frakw}{\mathfrak{w}}
\nc{\XX}{\mathbb{X}}	\nc{\calX}{\mathcal{X}}	\nc{\frakx}{\mathfrak{x}}
\nc{\YY}{\mathbb{Y}}	\nc{\calY}{\mathcal{Y}}	\nc{\fraky}{\mathfrak{y}}
\nc{\ZZ}{\mathbb{Z}}	\nc{\calZ}{\mathcal{Z}}	\nc{\frakz}{\mathfrak{z}}
\newcommand{\Id}{\mathbb{1}}

\nc{\Lie}{\on{Lie}}
\nc{\GL}{\on{GL}}
\nc{\SL}{\on{SL}}
\nc{\Sp}{\on{Sp}}
\nc{\gl}{\on{\mathfrak{gl}}}
\rc{\sl}{\on{\mathfrak{sl}}}
\nc{\Mat}{\on{Mat}}
\nc{\mO}{\on{O}}

\nc{\Fun}{\on{Fun}}
\nc{\Aut}{\on{Aut}}
\nc{\End}{\on{End}}
\nc{\Hom}{\on{Hom}}
\nc{\Span}{\on{Span}}
\nc{\Ind}{\on{Ind}}
\nc{\Res}{\on{Res}}
\nc{\stab}{\on{stab}}
\rc{\ker}{\on{ker}}
\nc{\im}{\on{im}}

\nc{\Sym}{\on{Sym}}
\nc{\tr}{\on{tr}}
\nc{\Rank}{\on{Rank}}
\nc{\Nullity}{\on{Nullity}}
\nc{\ch}{\on{char}}
\nc{\supp}{\on{supp}}
\nc{\Spec}{\on{Spec}}
\nc{\Jac}{\on{Jac}}

% \nc{\Id}{\on{Id}}
\nc{\op}{\mathrm{op}}
\nc{\triv}{\mathrm{triv}}
\nc{\alg}{\mathrm{alg}}
\nc{\rel}{\mathrm{rel}}
\nc{\1}{\mathbf{1}}
\nc{\diag}{\mathrm{diag}}

% Environments
\newtheorem{thm}{Theorem}
\newtheorem{lem}[thm]{Lemma}
\newtheorem{prop}[thm]{Proposition}
\newtheorem{conj}[thm]{Conjecture}
\newtheorem{cor}[thm]{Corollary}
\newtheorem{defe}[thm]{Definition}
\newtheorem{prob}[thm]{Problem} 
\theoremstyle{remark}
\newtheorem{rem}[thm]{Remark}
\newtheorem{exam}[thm]{Example}
\newtheorem{ntn}[thm]{Notation}
\newtheorem{claim}[thm]{Claim}

% TOC SC font
\makeatletter
%Table of Contents
\setcounter{tocdepth}{2}
% Add smallcaps to \section titles in ToC and remove . after numbers
\renewcommand{\tocsection}[3]{%
  \indentlabel{\@ifnotempty{#2}{\scshape\ignorespaces#1 #2.\quad}}\scshape#3\dotfill}
% Add smallcaps to \subsection titles in ToC and remove . after numbers
\renewcommand{\tocsubsection}[3]{%
  \indentlabel{\@ifnotempty{#2}{\scshape\ignorespaces#1 #2.\quad}}\scshape#3\dotfill}
%\let\tocsubsubsection\tocsubsection% Update for \subsubsection
%...
\def\l@subsection{\@tocline{2}{0pt}{2.5pc}{5pc}{}}
\makeatother

\makeatletter
\def\thm@space@setup{\thm@preskip=2mm
\thm@postskip=2mm}
\makeatother

\linespread{1.2}

\begin{document}
% Title page
\thispagestyle{empty}
\begin{center}
    \includegraphics{university.png} \\
    \vspace{3cm}
    {\LARGE\textsc{Optimizing performance \\ in Gaussian Processes}} \\
    \vspace{0.3cm}
    {\textsc{Michael Ciccotosto-Camp}} \\
    \vspace{1cm}
    {\textsc{Supervisor: \href{https://people.smp.uq.edu.au/FredRoosta/}{Fred (Farbod) Roosta}}} \\
    {\textsc{Co-Supervisors: \href{https://researchers.uq.edu.au/researcher/2466}{Andries Potgieter} \\ \href{https://researchers.uq.edu.au/researcher/14230}{Yan Zhao}}} \\
    \vspace{5cm}
    {\textsc{Bachelor of Mathematics (Honours)}} \\
    {\textsc{June 2022}} \\
    \vspace{1cm}
    {\textsc{\href{https://www.uq.edu.au/}{The University of Queensland}}} \\
    {\textsc{\href{https://smp.uq.edu.au/}{School of Mathematics and Physics}}}
\end{center}
\newpage

% blank page
%\thispagestyle{empty}
%\ 
%\newpage

% blank page
%\thispagestyle{empty}
%\hspace{0pt}\vfill\begin{center}\emph{For mum and dad}\end{center}\vspace{18cm}\hspace{0pt}
%\newpage

% blank page
\thispagestyle{empty}
\
\newpage

% frontmatter
\pagenumbering{roman}
\tableofcontents
\setlength{\parindent}{0pt} % Default is 15pt.
\setlength{\parskip}{2mm}
\newpage

% blank page
\
\newpage

% blank page
\section*{Acknowledgements}
I would like to deeply thank my supervisor Dr.\ Masoud Kamgarpour for his advice and all of his time spent with me. I consider myself lucky and am glad to have been his student for my honours year. I would also like to thank my co-supervisor Dr.\ Anna Puskás for the same reasons. A special thanks to Dr.\ Valentin Buciumas for his time spent teaching me while he was at The University of Queensland.
\newpage

% blank page
\newpage

% thesis start
\pagenumbering{arabic}

\section*{Introduction}
\addtocontents{toc}{\protect\setcounter{tocdepth}{0}}
The problem of time series prediction (and related regressional tasks) have been a long standing subject of high interest in many disciplines of science and mathematics. The problem itself is fairly straight forward, given a data set of $n$ observations $\calD = \left\{ \left( x_i , y_i \right) \right\}_{i=1}^{n}$, where each input $x_i \in \RR_{>0}$ is a time value and $y_i \in \RR$ is a that acts a function of time, the question then is how do we go about predicting a value $y_{\star}$ for the phenomena being modeled at time $x_{\star}$? With computing power becoming ever more affordable and advanced, many have taken to Machine Learning (ML) to develope sophisticated models to address the problem of developing accurate time series predictors. ML, broadly speaking, is any class of heuristic algorithm that attempts to refine and develope functionality by learning through some form of input. The idea of ML is founded on the idea that any form of task learning is done through sensory input from the world around them. More formally speaking, in ML we are trying to develope a function $f : X \to Y$ for some input set $X$ and observation set $Y$ were the outputs of $f$ closely align to actual observations. A model will attempt to make accurate prediction using some simplified formulation of the world required for the task. The distribution corresponding to the probability of a prediction within the context of the "state of the world" is referred to as the {\it likelihood}. The uncertainty within the likelihood stems from the predictive limits of the model. These limitation usually arise as a consequence of selecting a model too simple or complex for the task. The "state of the world" is sometimes internally captured by the model as a set of mutable parameters $\bm{\theta}$. The process of taking observations and using them to form predictions is called {\it inference} which in some sense is synonymous with learning.

ML can be applied the problem of time series prediction in a fairly straight forward manner, by simply teaching a ML algorithm $\calM$ the time series data set $\calD$ to hopefully produce a function $f$ that serves as a good approximant for event prediction. In this thesis we shall focus on particular class of ML algorithms called Baysian ML models which, unsurprisingly, makes use of Bayesian statistics to drive inference.

In Baysian models a {\it prior} distribution is used to represent the uncertainty of the current state of the model before any observations are made. The model can then be updated once data is observed by using the likelihood to update the state of the model to give a {\it posterior} distribution which represents the reduced uncertainty after "teaching" the model with these new observations. Methods of "teaching" a model how to behave using a new set of observations sometimes involves the use of a {\it loss function} $L$ is used to aid us in deciding what action $a$ should be taken in changing the state of the model to best minimize uncertainty. The best action, roughly speaking, can be evaluated as
\begin{equation*}
    a_{\star} =  \argmin_{a} \int L \left( y_{\star} , a \right) p \left( y_{\star} \mid \bm{x}_{\star}, \bm{X}, \bm{y} \right) \; d y_{\star}.
\end{equation*}
Interestingly, the best action does not rely so much on the model internalized parameters of the structure of the model itself but more so the predictive distribution $p \left( y_{\star} \mid \bm{x}_{\star}, \bm{X}, \bm{y} \right)$. This key insight has spawned a class of ML algorithms that focus on infering the function $f$ directly by computing $p \left( f \mid \calD \right)$ instead of finding optimal internal parameters using $p \left( \bm{\theta} \mid \calD \right)$. Models that perform inference in this manner as called {\it non-parameteric} models. Within the {\it non-parameteric} model paradigm, the predictive distribution can be expressed as
\begin{equation*}
    p \left( y_{\star} \mid x_{\star} , \bm{X} , \bm{y} \right) = \int p \left( y_{\star} \mid f , x_{\star} \right) p \left( f \mid \bm{X} , \bm{y} \right) \; df
\end{equation*}
and once new data is observed the posterior can be updated using Baye's rule
\begin{equation*}
    \text{posterior} = \frac{\text{likelihood} \times \text{prior}}{\text{marginal likelihood}}, \qquad p \left( \bm{f} , f_{\star} \mid \bm{y} \right) = \frac{p \left( \bm{y} \mid \bm{f} \right) p \left( \bm{f} , f_{\star} \right) }{p \left( \bm{y} \right)}.
\end{equation*}
This thesis will focus on a particular non-parameteric Bayesian ML model called Gaussian processes (GP). The over arching idea of GPs is to define a prior to every possible function mapping from $X$ to $Y$. On the surface, this does not seem computationally tractable as we would seemingly need to check through an uncountable infinite number of mappings. It turns out, computation can be completed given we are only seeking predictions at a finite number of points using a finite number of observations. GPs occupy a special place within the realm of ML since they account for uncertainty in a principled way, relatively simple to implement and are highly modular allowing them to easily be incorporated into a larger systems. It is no surprise that while other kernel methods are still overshadowed by their neural network cousins, GPs are making a quiet comeback in the ML community.


\newpage
\addtocontents{toc}{\protect\setcounter{tocdepth}{2}}
%%%%%%%%%%%%%%%%%%%%%%%%%%%%%%%%%%%%%%%%%%%%%%%%%%%%%%%%%%%%%%%%%%%%%%%%%%%%%%%
\newpage

\section{Gaussian Processes}\label{Chapter1}
The aim of this chapter is to build up the theory behind GPs from the ground up. First, we shall review some essential theory from functional analysis on kernels and reproducing kernel Hilbert spaces which are not used in GPs but are used in a vast array of machine learning models, aptly named kernel machines. Afterward, we shall go through the underlying statistics that drive GP prediction and use it to form algorithms for both regression and classification tasks. Note that most of the theory presented here is only for real-values data sets although most the time complex-valued generalizations do exist.

\subsection{Kernels}\label{Section1.1}

Often in machine learning we are often met with the challenge of how to best represent data instances as fixed size feature vectors $\bm{x}_i \in X$. For certain objects it might not be obvious at all how to represent the data as a fixed length vector. Good examples of variable length data include textual documents and genomic data. For these data types we can define a method of measuring similarity between objects which requires them to first be converted to a fixed length feature vector first \cite{MurphyKevinP2012Ml}. To do this we begin by mapping the feature vectors into a Hilbert space $H$ which enriches the vector space with an inner product $\langle \cdot , \cdot \rangle_H : H \times H \to \RR$ and a norm $\| \cdot \|_H : H \to \RR$. Input data is transformed into feature space vectors via a non-linear feature mapping $\Phi : X \to H$. The benefit of using feature maps in this way is that a non-linear descision boundary can be constructed using linear models. In some instances a similarity measure can be computed directly using a function $k : X \times X \to \RR$, instead of needing to construct a $\Phi$ and then computing the inner product of the transformed instances. Functions that act directly on our data instances are known as kernel functions and using them to avoid computation associated with the underlying feature space is known as the kernel trick \cite{SteinwartIngo2008SVMb}. These ideas are stated more formally in definition \ref{defe: kernel}.

\begin{defe}[Kernel] \label{defe: kernel}
    Let $X$ be a non-empty set. Then a function $k : X \times X \to \RR$ is called a kernel on $X$ if there exists a Hilbert space and a map $\Phi : X \to H$ such that for all $\bm{x} , \bm{x}' \in X$ we have $k \left( \bm{x} , \bm{x}' \right) = \langle \Phi \left( \bm{x} \right), \Phi \left( \bm{x}' \right) \rangle_H$. We call the $\Phi$ the feature map and $H$ the feature space of $k$.
\end{defe}

It is worth noting that almost no conditions are placed on the set $X$, allowing it to accommodate virtually any form of data. It is not surprising then that neither the feature map nor the feature space are uniquely determined by the kernel. As shown by the example from Steinwart and Christmann \cite{SteinwartIngo2008SVMb}, when $X = \RR$ and $k \left( x , x' \right) = x \cdot x'$ where $x , x' \in X$, we can see that $k$ is a kernel using the feature map $\Phi \left( x \right) = x$ and $H = \RR$. However, another suitable feature map for this particular kernel is $\Phi' \left( x \right) = \left( x / \sqrt{2} , x / \sqrt{2} \right)$ with a corresponding feature space of $H = \RR^2$ since
\[
    \langle \Phi' \left( x \right), \Phi' \left( x' \right) \rangle_{\RR^2} = \frac{x'}{\sqrt{2}} \cdot \frac{x}{\sqrt{2}} + \frac{x'}{\sqrt{2}} \cdot \frac{x}{\sqrt{2}} = x \cdot x'
\]
for $x,x' \in X$. While their might be numerous functions that provide some notion of similarity between data entries, these functions might not be valid kernels. Instead of needing to construct a feature map and feature space to verify that a chosen function is a valid kernel using definition \ref{defe: kernel}, we can make use of a much simpler set of criteria. Before embarking on this train of thought, we need to define the following.

\begin{defe}[Positive Definite and Positive Semidefinite] \label{defe: PD}
    A function $k : K \times K \to \RR$ is positive semidefinite if for all $n \in \NN$ and $\alpha_1 , \ldots , \alpha_n \in \RR$ and all $\bm{x}_1 ,\ldots , \bm{x}_n \in X$ we have
    \begin{equation}\label{eq: PSD}
        \sum_{i=1}^{n} \sum_{j=1}^{n} \alpha_i \alpha_j k \left( \bm{x}_j , \bm{x}_i \right) \geq 0.
    \end{equation}
    Furthermore, $k$ is said to be positive definite if for mutually distinct $\bm{x}_1 ,\ldots , \bm{x}_n \in X$ equality \ref{eq: PSD} only holds for $\alpha_1 = \ldots = \alpha_n = 0$ \cite{SteinwartIngo2008SVMb}.
\end{defe}

\begin{defe}[Symmetric] \label{defe: Symmetric_function}
    A function $k : K \times K \to \RR$ is called symmetric if $k \left( \bm{x} , \bm{x}' \right) = k \left( \bm{x}' , \bm{x} \right)$ for any inputs $\bm{x}' , \bm{x} \in X$ \cite{SteinwartIngo2008SVMb}.
\end{defe}

\begin{defe}[Gram Matrix] \label{defe: Gram_Matrix}
    For fixed $\bm{x}_1 ,\ldots , \bm{x}_n \in X$ the matrix $\bm{K} \in \RR^{n \times n}$ where $\bm{K}_{i,j} \triangleq k \left( \bm{x}_j , \bm{x}_i \right)$ is the Gram matrix \cite{SteinwartIngo2008SVMb}.
\end{defe}

Note that checking if a function is positive (semi) definite is equivalent to checking that any Gram matrix produced by a function is positive (semi) definite. If $k$ is a real valued kernel corresponding to the feature map $\Phi$, then $k$ is symmertic by virtue of the fact that the inner product of a real Hilbert space is symmetric. Moreover $k$ is positive definite since for $\alpha_1 , \ldots , \alpha_n \in \RR$ and $\bm{x}_1 ,\ldots , \bm{x}_n \in X$ we have
\begin{align*}
     & \sum_{i=1}^{n} \sum_{j=1}^{n} \alpha_i \alpha_j k \left( \bm{x}_j , \bm{x}_i \right)                                           \\
     & = \sum_{i=1}^{n} \sum_{j=1}^{n} \alpha_i \alpha_j \langle \Phi \left( \bm{x}_i \right), \Phi \left( \bm{x}_j \right) \rangle_H \\
     & = \norm{ \sum_i^n \alpha_i \Phi \left( \bm{x}_i \right) }_{H}^{2}                                                              \\
     & \geq 0.
\end{align*}

The following theorems tell us that it is not only necessary for a kernel to be positive semi definite but it is also a sufficient condition.

\begin{thm} \label{theorem: nec_and_suf_kernel_1}
    A function $k : K \times K \to \RR$ is a kernel if and only if it is symmertic and positive semidefinite \cite{SteinwartIngo2008SVMb}.
\end{thm}

\subsection{Reproducing Kernel Hilbert Spaces}\label{Section1.2}

We shall now shift our attention towards reproducing kernel Hilbert spaces (RKHS) and describe their relation to kernels, and see that in some sense, the RKHS of a kernel $k$ is the smallest feature space for a kernel. The formal definition of a RKHS is stated in definition \ref{defe: RKHS}.

\begin{defe}[RKHS] \label{defe: RKHS}
    Let $X \neq \emptyset$ and $H$ be a real Hilbert space over $X$
    \begin{enumerate}
        \item A function $k : X \times X \to \RR$ is called a reproducing kernel if we have $k \left( \cdot, \bm{x} \right) \in H$ for all $\bm{x} \in X$ and the reproducing property
              \[
                  f(\bm{x}) = \langle f , k \left( \cdot, \bm{x} \right) \rangle
              \]
              holds for all $f \in H$ and $x \in X$.
        \item The space $H$ is called a reproducing kernel Hilbert space over $X$ if for all $\bm{x} \in X$ the Dirac functional $\delta_{\bm{x}} : H \to \RR$ defined by $\delta_{\bm{x}} (f) \triangleq f(x), \; f \in H$ is continuous.
    \end{enumerate}
    \cite{SteinwartIngo2008SVMb}
\end{defe}

An important property of the RKHS is that the convergence in the norm implies pointwise convergence. Specifically, in a RKHS for any sequence of functions $\left\{ f_n \right\} \subset H$ where $\norm{f_n - f} \to 0$ we have $\abs{\delta_{\bm{x}} (f_n) - \delta_{\bm{x}} (f)} = \abs{f_n (x) - f (x)} \to 0$. Note that because the evaluation function is both linear and continuous, then it is also bounded in the sense that there is an $c \in \RR, \; c > 0$ such that for every $f \in H$ and a fixed $\bm{x} \in X$ we have $\abs{\delta_{\bm{x}} (f)} \leq c \norm{f}_H$ \cite{BerezanskyMakarovich1996FaV1}. This property of uniform convergence implying pointwise convergence is important since it tells us that if functions $f,g \in H$ are close in norm then their evaluation at any point is also similar. The following lemma ties together the definition of an RKHS, reproducing kernel and a kernel.

\begin{lem}[] \label{lem: RKHS_rk_k}
    Let $H$ be a Hilbert function space over $X$ that has a reproducing kernel $k$. Then $H$ is a RKHS and $H$ is also a feature space of $k$ where the feature map $\Phi : X \to H$ is given by
    \[
        \Phi (\bm{x}) = k \left( \cdot , \bm{x}  \right)
    \]
    for some $\bm{x} \in X$. We call $\Phi$ the canonical feature map.
\end{lem}

\begin{proof}
    Since the reproducing property tells us that any Dirac functional can be represented by the reproducing kernel this means
    \[
        \abs{\delta_{\bm{x}} (f)} = \abs{f(\bm{x})} = \abs{\langle f , k \left( \cdot, \bm{x} \right) \rangle} \leq \norm{k \left( \cdot, \bm{x} \right)}_H \cdot \norm{f}_H
    \]
    for all $\bm{x} \in X, \; f \in H$. This shows continuity of $\delta_{\bm{x}}$ for $\bm{x} \in X$. In order to show that $\Phi$ is a feature map, fix an $\bm{x}' \in X$ and set $f = k \left( \cdot, \bm{x}' \right)$. Then for $\bm{x} \in X$, the reproducing property yields
    \[
        \langle \Phi (\bm{x}') , \Phi (\bm{x}) \rangle_H = \langle k \left( \cdot, \bm{x}' \right) , k \left( \cdot, \bm{x} \right) \rangle_H = \langle f , k \left( \cdot, \bm{x} \right) \rangle_H = f(\bm{x}) = k \left( \bm{x}', \bm{x} \right).
    \]
\end{proof}

This tells us that every Hilbert space with a reproducing kernel is a RKHS. We can also show the converse, that is, every RKHS has a unique reproducing kernel seen in theorem \ref{theorem: unique_kernel}.

\begin{thm} \label{theorem: unique_kernel}
    Let $H$ be a RKHS over $X$. Then $k: X \times X \to \RR$ defined by $k \left( \bm{x}', \bm{x} \right) = \langle \delta_{\bm{x}} , \delta_{\bm{x}'} \rangle_H, \; \bm{x} , \bm{x}' \in X$ is the only reproducing kernel of $H$ \cite{SteinwartIngo2008SVMb}.
\end{thm}

Theorem \ref{theorem: unique_kernel} shows that a RKHS is uniquely determined by its kernel. In fact the other direction can also be shown to afford a one-to-one correspondence between kernels and RKHS. This is known as the Moore-Aronszajn theorem presented in thorem \ref{theorem: Moore-Aronszajn}.

\begin{thm}[Moore-Aronszajn] \label{theorem: Moore-Aronszajn}
    Suppose $k$ is a symmertic positive definite kernel on a set $X$. Then there is a unique Hilbert space of functions for which $k$ is the reproducing kernel \cite{BerlinetAlain2003RKHS}.
\end{thm}

The elements of a RKHS will often inherit the analytical properties of its corresponding kernel. This means that kernels provide a mechanism for generating spaces of functions with useful analytical properties.

\subsection{Gaussian Radial Basis Kernel}\label{Section1.3}

We shall now focus on a specific class of kernel that shall be used extensively in upcoming theory and experimentation.

\begin{defe}[Gaussian Radial Basis Kernel] \label{defe: grbfk}
    For $d \in \NN, \; \sigma \in \RR_{>0}$ and $ \bm{z} , \bm{z}' \in \RR^d$ we define
    \[
        k_\sigma \left( \bm{z} , \bm{z}' \right) \triangleq \exp \left( - \sigma^{-2} \sum_{j=1}^{d} \left( \bm{z}_j - {\bm{z}'}_j \right)^2 \right).
    \]
    Then $k_\sigma$ is a real valued kernel called the Gaussian Radial Basis Kernel (RBF) kernel with bandwidth $\sigma$. Moreover $k_\sigma$ can be computed as
    \[
        \exp \left( \frac{- \norm{\bm{z} - {\bm{z}'}}_{2}^{2}}{\sigma^2} \right)
    \]
    \cite{SteinwartIngo2008SVMb}.
\end{defe}
The Gaussian RBF kernel makes for a very simple an intuitive measurement of similarity between its inputs. One geometric interpretation of the Gaussian RBF kernel is that as the radius of the smallest $d$-sphere containing $\bm{z} , \bm{z}' \in \RR^d$ grows the corresponding measurement of similarity decays exponentially. A visual representation of this decay is shown in figure \ref{fig: grbfk_graph_1}.

% \begin{figure}[H]
%     \centering
%     \subfloat{
%         \begin{tikzpicture}

%             \begin{axis}[
%                     width=8cm,height=8cm,
%                     domain=-2:2,
%                     xmax=2.25,
%                     ymax=2.25,
%                     xmin=-2.25,
%                     ymin=-2.25,
%                     zmax=1.1,
%                     axis lines = left,
%                     colormap/hot,
%                 ]

%                 \addplot3[samples = 25, surf] {exp(-(x^2 + y^2)/1)};
%                 \node at (rel axis cs:1,0.5,1) [above] {\(\sigma=1\)};

%             \end{axis}

%         \end{tikzpicture}
%     }%
%     \quad
%     \subfloat{
%         \begin{tikzpicture}

%             \begin{axis}[
%                     width=8cm,height=8cm,
%                     domain=-2:2,
%                     xmax=2.25,
%                     ymax=2.25,
%                     xmin=-2.25,
%                     ymin=-2.25,
%                     zmax=1.1,
%                     axis lines = left,
%                     colormap/hot,
%                 ]

%                 \addplot3[samples = 25, surf] {exp(-(x^2 + y^2)/2)};
%                 \node at (rel axis cs:1,0.5,1) [above] {\(\sigma=2\)};

%             \end{axis}

%         \end{tikzpicture}
%     }%
%     \caption{A graph of the Gaussian RBF from definition \ref{defe: grbfk} for $d=2$. Evidently, a larger value of $\sigma$ slows the rate of decay increasing the similarity between the same pair of samples.}%
%     \label{fig: grbfk_graph_1}
% \end{figure}

This kernel is infinitely differentiable meaning it has mean square derivatives of all orders and is therefore very smooth. In fact, some argue that such strong smoothness makes it unrealistic for modelling natural phenomena \cite{RasmussenCarlEdward2006Gpfm,SteinMichaelL1999IoSD}. Nontheless, Gaussian RBF kernelis remains the one of the most widely used kernels in literature.

\subsection{Kernel Machines}\label{Section1.4}

In this section, we shall look at two different machine learning models that make use of kernels to perform classification and regression. The first kernel machine we shall look at are support vector machines (SVM). SVMs where originally designed for binary classification and as such we shall only present a model for binary classification, although extensions exist that allow regression and multi-class classification.

For the binary classification problem we are tasked with labelling new samples with either one of two classes, $-1$ or $1$. We shall assume our input space consists of vectors from $\RR^d$ and that we provided with a labelled training set $D = \left\{ \left( \bm{x}_1 , y_1 \right), \left( \bm{x}_1 , y_1 \right), \ldots , \left( \bm{x}_n , y_n \right) \right\}$. One simple method to classify samples is by creating an affine linear hyperplane satisfying
\begin{align} \label{eq: linear_sep_hyp}
    \langle \bm{w}, \bm{x}_i \rangle + b > 0, \quad y_i = +1 \nonumber \\
    \langle \bm{w}, \bm{x}_i \rangle + b < 0, \quad y_i = -1
\end{align}
for some $\bm{w} \in \RR^d$ and $b \in \RR$ where $\norm{w}_2 = 1$. Moreover we would like $\bm{w}$ and $b$ to maximise the margin, that is the maximal distance between the separating hyperplane and the points in $D$. The specific $\bm{w}$ and $b$ obtained through the training set is denoted $\bm{w}_D$ and $b_D$ and the resulting descision function is defined as
\[
    f_D \left( \bm{x} \right) \triangleq \operatorname{sign} \left( \langle \bm{w}_D , \bm{x} \rangle + b_D \right).
\]
There are, however, a number of short comings to this model. The most obvious is that our training data may not be linearly separable in $\RR^d$ meaning that no such $\bm{w}_D$ and $b_D$ exist. Moreover, when noise is introduced to the data set this model will prioritize finding a hyperplane that perfectly separates the two classes, making no comprises in misclassifying points, leaving it subject to overfitting. SVMs where introduced by Boser {\it et al.} \cite{BoserBernhard1992Ataf} to address the first issue of separability. Their approach was to lift the input vector into a more malleable Hilbert space $H_0$ using a feature map. The inputs are then classified within the new space. Unfortunately this method does nothing to address the second issue of over fitting and, if anything, actually worsens it. Cortes and Vapnik \cite{CortesCorinna1995SN} attempted to address this second issue by introducing slack variables to equation \ref{eq: linear_sep_hyp} so that we instead need to satisfy $y_i \left( \langle \bm{w} , \Phi \left( \bm{x}_i \right) \rangle + b \right) \geq 1 - \xi_i$ for some $\xi_i \in \RR_{>0}$. These constraints can be re-written as
\[
    \xi_i \geq 1 - y_i \left( \langle \bm{w} , \Phi \left( \bm{x}_i \right) \rangle + b \right)
\]
and combining this this our slack constraints (that is $\xi_i \geq 0$) yields
\[
    \xi_i \geq \max \left\{ 0, 1 - y_i \left( \langle \bm{w} , \Phi \left( \bm{x}_i \right) \rangle + b \right)  \right\} = L_{\text{hinge}} \left( y_i , \langle \bm{w} , \Phi \left( \bm{x}_i \right) \rangle + b \right)
\]
where $L_{\text{hinge}}$ is the hinge loss defined as
\[
    L_{\text{hinge}} \left( y,\eta \right) \triangleq \max \left\{ 0,1-y\eta \right\}.
\]
This optimization problem can be re-written is the form
\[
    \min_{\left( \bm{w} , b \right) \in H_0 \times \RR} \lambda \norm{\bm{w}}_{H_0} + \frac{1}{n} \sum_{i=1}^{n} L_{\text{hinge}} \left( y_i , f_{\left( \bm{w} , b \right)} \right)
\]
where $f_{\left( \bm{w} , b \right)} : X \to \RR$ is defined as
\[
    f_{\left( \bm{w} , b \right)} \triangleq \langle \bm{w} , \Phi \left( x_i \right) \rangle + b.
\]
Unfortunately, this new embedding requires us to solve for optimal parameters in a very high, or even infinite, dimension vector space. To get around this, often the Lagrange approach is used to solve the corresponding dual problem. When the hinge loss is used the dual problem becomes
\begin{align} \label{eq: SVM_dual_1}
     & \max_{\alpha \in \left[ 0,C \right]^n} \sum_{i=1}^{n} \alpha_i - \frac{1}{2} \sum_{i,j=1}^{n} y_i y_j \alpha_i \alpha_j \langle \Phi \left( \bm{x}_i \right), \Phi \left( \bm{x}_j \right) \rangle \nonumber \\
     & \text{subject to} \quad \sum_{i=1}^{n} y_i \alpha_i = 0
\end{align}
Notice that in the dual problem, we find that inner products are only taken with vectors that have the feature map applied to them allowing us to employ the kernel if the corresponding kernel trick described in section \ref{Section1.1} is known for the feature map used so that \ref{eq: SVM_dual_1} becomes
\begin{align*}
     & \max_{\alpha \in \left[ 0,C \right]^n} \sum_{i=1}^{n} \alpha_i - \frac{1}{2} \sum_{i,j=1}^{n} y_i y_j \alpha_i \alpha_j k \left( \bm{x}_i, \bm{x}_j \right) \\
     & \text{subject to} \quad \sum_{i=1}^{n} y_i \alpha_i = 0.
\end{align*}

The next machine learning model of interest that uses kernels are gaussian processes. To motivate this model, consider the time series data in figure \ref{fig: motive_gp_1}.

\begin{figure}[H]
    \centering
    \subfloat[]{
        \begin{tikzpicture}
            \draw[->,thick] (-0.01,0)--(6,0) node[right]{$x$};
            \draw[->,thick] (0,-0.01)--(0,6) node[above]{$y$};

            \draw[-,ultra thick] (0.7,-0.1)--(0.7,0.1) node[below,yshift=-0.3cm]{$x_1$};
            \draw[fill,draw,blue] (0.7,0.5) circle[radius=2.5pt];

            \draw[-,ultra thick] (1.4,-0.1)--(1.4,0.1) node[below,yshift=-0.3cm]{$x_2$};
            \draw[fill,draw,blue] (1.4,0.6) circle[radius=2.5pt];

            \draw[-,ultra thick] (2.7,-0.1)--(2.7,0.1) node[below,yshift=-0.3cm]{$x_3$};
            \draw[fill,draw,blue] (2.7,1.7) circle[radius=2.5pt];

            \draw[-,ultra thick] (3.7,-0.1)--(3.7,0.1) node[below,yshift=-0.2cm]{$x^{\ast}$};
            \draw[dashed,thick,red] (3.7,0)--(3.7,5);

            \draw[-,ultra thick] (5,-0.1)--(5,0.1) node[below,yshift=-0.3cm]{$x_4$};
            \draw[fill,draw,blue] (5,4) circle[radius=2.5pt];
        \end{tikzpicture}
    }%
    \qquad
    \subfloat[]{
        \begin{tikzpicture}
            \draw[->,thick] (-0.01,0)--(6,0) node[right]{$x$};
            \draw[->,thick] (0,-0.01)--(0,6) node[above]{$y$};

            \draw[-,ultra thick] (0.7,-0.1)--(0.7,0.1) node[below,yshift=-0.3cm]{$x_1$};
            \draw[fill,draw,blue] (0.7,0.5) circle[radius=2.5pt];
            \draw[dashed,blue] (0.7,0)--(0.7,4.7);
            \draw[<->,thick] (0.7,4.7)--(3.7,4.7) node[above,xshift=-1.5cm]{$k(x^{\ast},x_1)$};

            \draw[-,ultra thick] (1.4,-0.1)--(1.4,0.1) node[below,yshift=-0.3cm]{$x_2$};
            \draw[fill,draw,blue] (1.4,0.6) circle[radius=2.5pt];
            \draw[dashed,blue] (1.4,0)--(1.4,3.5);
            \draw[<->,thick] (1.4,3.5)--(3.7,3.5) node[above,xshift=-1.1cm]{$k(x^{\ast},x_2)$};

            \draw[-,ultra thick] (2.7,-0.1)--(2.7,0.1) node[below,yshift=-0.3cm]{$x_3$};
            \draw[fill,draw,blue] (2.7,1.7) circle[radius=2.5pt];
            \draw[dashed,blue] (2.7,0)--(2.7,2.3);
            \draw[<->,thick] (2.7,2.3)--(3.7,2.3) node[above,xshift=-0.9cm]{$k(x^{\ast},x_3)$};

            \draw[-,ultra thick] (3.7,-0.1)--(3.7,0.1) node[below,yshift=-0.2cm]{$x^{\ast}$};
            \node[diamond,draw,fill,draw,red,minimum width = 1cm,minimum height = 1.3cm,scale=0.25] (d) at (3.7,3) {};
            \draw[dashed,thick,red] (3.7,0)--(3.7,5);

            \draw[-,ultra thick] (5,-0.1 )--(5,0.1) node[below,yshift=-0.3cm]{$x_4$};
            \draw[fill,draw,blue] (5,4) circle[radius=2.5pt];
            \draw[dashed,blue] (5,0)--(5,4.5);
            \draw[<->,thick] (3.7,4.5)--(5,4.5) node[above,xshift=-0.3cm]{$k(x^{\ast},x_4)$};
        \end{tikzpicture}
    }%
    \caption{A graph of the Gaussian RBF from definition \ref{defe: grbfk} for $d=2$. Evidently, a larger value of $\sigma$ slows the rate of decay increasing the similarity between the same pair of samples.}%
    \label{fig: motive_gp_1}
\end{figure}

\begin{filecontents*}{./data/gp_intro_dat1.csv}
    x,y0,y1,y2
    0.0,2.6341780930873786,4.41685044685407,1.884123117075101
    0.11224489795918367,2.685150340856032,4.351109330694541,2.0933788453347146
    0.22448979591836735,2.758222906849677,4.246986054733594,2.3215583495507697
    0.336734693877551,2.844091796198551,4.111722477611423,2.5621308980348045
    0.4489795918367347,2.9327004798948444,3.954138391914307,2.8093607381581323
    0.5612244897959183,3.0142490315111,3.7837568463624893,3.058296829896802
    0.673469387755102,3.0800293333419133,3.609905398833131,3.304575221937234
    0.7857142857142857,3.1230071846450134,3.4408573216000065,3.5440886761610075
    0.8979591836734694,3.1381275068842287,3.2830844115325393,3.772606115615495
    1.010204081632653,3.1223796951701805,3.1407028040204787,3.985434762859824
    1.1224489795918366,3.074690487867501,3.0151840031490704,4.177213384641246
    1.2346938775510203,2.99572649694323,2.9053883056890326,4.341906534238328
    1.346938775510204,2.887680093346262,2.807934404050683,4.473021654529128
    1.4591836734693877,2.7540712359611135,2.7178753181761683,4.5640359352552755
    1.5714285714285714,2.5995639802641644,2.629596908378246,4.608968085491204
    1.683673469387755,2.4297636797240854,2.537815790076743,4.603008456947798
    1.7959183673469388,2.250945266539242,2.4385306514757277,4.54310273176884
    1.9081632653061225,2.0696826217289814,2.3297825969976156,4.428402138611055
    2.020408163265306,1.8923794707563157,2.2121155584786374,4.26051300065903
    2.13265306122449,1.724750314129252,2.088670901030661,4.043522033936433
    2.2448979591836733,1.5713409480295042,1.9649185604698398,3.78380984825793
    2.357142857142857,1.4351902937232137,1.8480772970914263,3.4896953544368947
    2.4693877551020407,1.3177316537250037,1.7463209648090843,3.170973707657322
    2.5816326530612246,1.2189781885886188,1.667887909534986,2.838411584600058
    2.693877551020408,1.137982180009872,1.6202034591848866,2.503253609595508
    2.806122448979592,1.0734814750833834,1.6091067041507867,2.176777775544475
    2.9183673469387754,1.0246008560812094,1.638242628202587,1.8699164299237567
    3.0306122448979593,0.9914469543413351,1.708653796550625,1.5929463528940093
    3.142857142857143,0.9754532832118712,1.8185871450769515,1.3552308689036683
    3.2551020408163267,0.9793831046385866,1.9635250522558478,1.1649988273056024
    3.36734693877551,1.0069633897912214,2.136445411268843,1.0291341361289352
    3.479591836734694,1.0622134340832736,2.328310102622806,0.9529576266820667
    3.5918367346938775,1.1485904962060034,2.5287676666005745,0.9399816624503701
    3.704081632653061,1.2681182060719118,2.7270230125223396,0.9916380696381344
    3.816326530612245,1.4206777569266578,2.9127943188682153,1.1069890129419622
    3.9285714285714284,1.603600797665261,3.077239529790242,1.282461239660869
    4.040816326530612,1.8116633596458505,3.213718985997705,1.5116563075035667
    4.153061224489796,2.0374945244629954,3.31826875093589,1.7853109321561014
    4.26530612244898,2.272346385818944,3.3896996920659737,2.0914743830547122
    4.377551020408164,2.50710273056381,3.4293088690936613,2.415953795652976
    4.489795918367347,2.7333622652331386,3.440263336916595,2.743037053839629
    4.6020408163265305,2.9444150750492786,3.426788503862636,3.0564529194286605
    4.714285714285714,3.1359501955059153,3.3933321658244284,3.3404756976868217
    4.826530612244898,3.306372509617609,3.343882098004392,3.5810500536442835
    4.938775510204081,3.456677113806286,3.281564998201946,3.766794718394943
    5.051020408163265,3.5899014357864862,3.2085942528948497,3.889767236618262
    5.163265306122449,3.710253447229357,3.1265441378249808,3.9459164775264113
    5.275510204081633,3.822067924064683,3.036856306931041,3.9352029543435254
    5.387755102040816,3.928774395166681,2.9414323013414645,3.8614290770863415
    5.5,4.032057566247807,2.843156622782664,3.7318467351526268
\end{filecontents*}

\begin{filecontents*}{./data/gp_intro_dat2.csv}
    x,mu,y0,y1,y2,bU,bL
    0.0,2.96787109791578,3.646442446207024,2.3796779884645356,3.123887098083909,3.8222957144465917,2.1134464813849685
    0.11224489795918367,3.178846435851742,3.7278533056947927,2.705139783400845,3.3197136652913843,3.8425540443223007,2.515138827381183
    0.22448979591836735,3.373957062754166,3.772122911137243,3.033674383762561,3.4864096188544065,3.8416203985733905,2.906293726934942
    0.336734693877551,3.550670833185919,3.7845706480846375,3.3482893361633472,3.6227538754415383,3.8226477418327116,3.2786939245391267
    0.4489795918367347,3.707387470227142,3.777024821961083,3.6497714076425667,3.734021204276469,3.7908281722625117,3.623946768191772
    0.5612244897959183,3.8435001880813098,3.7587623765632605,3.9224326540622303,3.820415281746996,3.938789896922746,3.7482104792398734
    0.673469387755102,3.959374699757277,3.7401691473547714,4.161761533893699,3.884187912930478,4.2112638754378935,3.70748552407666
    0.7857142857142857,4.056245238332488,3.727886967246087,4.3591514373501585,3.9294944134700884,4.440290931214802,3.6721995454501735
    0.8979591836734694,4.136035742591053,3.7320230213633776,4.516870894977162,3.9667169440013335,4.6225623417878845,3.6495091433942224
    1.010204081632653,4.201122318895088,3.7511336294964686,4.631536079365145,3.9909291094494637,4.756943867968448,3.645300769821728
    1.1224489795918366,4.254059537569402,3.7922886212348805,4.703821106353732,4.014456904551665,4.844164391874928,3.663954683263876
    1.2346938775510203,4.297297268072625,3.8510310641448773,4.736836098027459,4.040882180292408,4.886718646421257,3.7078758897239927
    1.346938775510204,4.332916090251523,3.931146658869461,4.737917134427073,4.0721402987298845,4.8886932140093196,3.7771389664937254
    1.4591836734693877,4.362407665942143,4.019692223676497,4.711145152040588,4.114354849026403,4.855508162034269,3.869307169850018
    1.5714285714285714,4.386522000140943,4.113994631302509,4.667147410140193,4.1675113923767135,4.79358919579317,3.9794548044887157
    1.683673469387755,4.405196773426522,4.211161115648332,4.60858632272203,4.2301483229724735,4.709995085814498,4.100398461038546
    1.7959183673469388,4.417575648333239,4.297237502243545,4.542647493156313,4.298601004588312,4.612045273426419,4.223106023240059
    1.9081632653061225,4.422113551855313,4.370044631852721,4.478504417000914,4.366233750854552,4.507185391473455,4.337041712237171
    2.020408163265306,4.416758358107319,4.4212535412567835,4.413073469318789,4.428720525173312,4.439002309014812,4.394514407199825
    2.13265306122449,4.399191007367498,4.4498214644109035,4.354633815643454,4.472837424274621,4.501454599238494,4.296927415496501
    2.2448979591836733,4.367100601462363,4.44669502086167,4.293220973455457,4.495071115055989,4.529355328572067,4.204845874352659
    2.357142857142857,4.318467882753287,4.4122754488830465,4.2367626348642995,4.480392701402552,4.511998980020114,4.124936785486461
    2.4693877551020407,4.251829947856004,4.344361140161457,4.176480274188317,4.420872228223025,4.444440701959786,4.059219193752223
    2.5816326530612246,4.166501022807468,4.242712341391929,4.110274997773214,4.315673648503417,4.324777879750019,4.008224165864918
    2.693877551020408,4.062728357957119,4.110217875771083,4.032563200602017,4.153918593320322,4.154406402240164,3.9710503136740742
    2.806122448979592,3.9417683270387145,3.9492135726749384,3.9344120355000705,3.9443400697750968,3.9571937560826638,3.926342897994765
    2.9183673469387754,3.805875044263767,3.7642279401126544,3.810832282328448,3.684817080524404,3.9345310053498097,3.6772190831777247
    3.0306122448979593,3.6582015833339296,3.564842924761986,3.6644624038944493,3.3833530964865193,3.9269741902831705,3.3894289763846888
    3.142857142857143,3.5026215163692607,3.3520647108430026,3.488370188345563,3.052946676019309,3.9234237680191817,3.0818192647193396
    3.2551020408163267,3.3434853499709174,3.1402141378498163,3.2837887833202726,2.7002390009588226,3.920293063216916,2.7666776367249186
    3.36734693877551,3.1853319635639767,2.934150639260623,3.057322775625835,2.3472996257286054,3.914233966397971,2.4564299607299827
    3.479591836734694,3.03257891484487,2.738827249921258,2.812094834179371,2.009196676123942,3.902114762761585,2.163043066928155
    3.5918367346938775,2.8892171796754873,2.5662919443248073,2.559707152254641,1.6967015390942675,3.881056263819646,1.8973780955313282
    3.704081632653061,2.758535417727554,2.41434042449297,2.309411800258502,1.4256334316849872,3.848437973565799,1.668632861889309
    3.816326530612245,2.642896256015177,2.2902148380401837,2.075552115650442,1.207890471342624,3.8018786148254615,1.4839138972048924
    3.9285714285714284,2.5435825902305362,2.1965744091641124,1.8662299680038845,1.0467598332277432,3.7392081008329594,1.3479570796281128
    4.040816326530612,2.4607259083694415,2.133954232173078,1.6922651032951084,0.952486845042865,3.6584498086819774,1.2630020080569058
    4.153061224489796,2.393321664023437,2.0881938587934696,1.566133253333461,0.92147130773977,3.5578289289014684,1.2288143991454055
    4.26530612244898,2.339329380270647,2.0632588371219063,1.4903968555363714,0.94987473663854,3.435816329632721,1.2428424309085735
    4.377551020408164,2.295848100924048,2.0605594124527933,1.4690236774181895,1.033187881225276,3.2912091804701387,1.3004870213779576
    4.489795918367347,2.259351656203322,2.064341290936802,1.499164363108256,1.1636813952423448,3.123241143061824,1.3954621693448201
    4.6020408163265305,2.2259635269573055,2.0739080755801877,1.575649936757477,1.3294049093439706,2.931708042701583,1.520219011213028
    4.714285714285714,2.1917482915683912,2.0845379827266677,1.6904928495001776,1.5224847717101069,2.717092081155121,1.6664045019816616
    4.826530612244898,2.1529959577040843,2.0900622587437008,1.8320811599608942,1.733755447747318,2.4806793091811996,1.8253126062269691
    4.938775510204081,2.10647694342231,2.087733094036374,1.98722071801456,1.9517354055585423,2.2248945171189956,1.988059369725624
    5.051020408163265,2.049648890548753,2.0747166503580527,2.1438285289379833,2.167467964578078,2.149507699667685,1.9497900814298208
    5.163265306122449,1.9808014834652514,2.050810263655156,2.283421101970534,2.3742039664111614,2.297682746932975,1.663920219997528
    5.275510204081633,1.8991314699538484,2.006424555553889,2.395089681917207,2.5656201066053885,2.429871221330997,1.3683917185766998
    5.387755102040816,1.8047465055032903,1.9408588788404912,2.4682291512970815,2.7418915989000654,2.5411728761483077,1.068320134858273
    5.5,1.69860261545027,1.8531818596862015,2.488627885130089,2.8888780190386125,2.628636682030301,0.7685685488702394
\end{filecontents*}

\begin{tikzpicture}
    \begin{axis}[
            xmin=-0.0,xmax=6.5,
            ymin=-0.5,ymax=6.5,
            axis line style={draw=none},
            tick style={draw=none},
            yticklabels=\empty,
            xticklabels=\empty,
        ]
        \addplot[smooth, color=black, semithick] table [x=x, y=y0, col sep=comma, mark=none] {./data/gp_intro_dat1.csv};
        \addplot[smooth, color=black, semithick, dashed] table [x=x, y=y1, col sep=comma, mark=none] {./data/gp_intro_dat1.csv};
        \addplot[smooth, color=black, semithick, dotted] table [x=x, y=y2, col sep=comma, mark=none] {./data/gp_intro_dat1.csv};

        \addplot[name path = bU, mark=none, blue!10] coordinates {(0,0.5) (5.5,0.5)};
        \addplot[name path = bL, mark=none, blue!10] coordinates {(0,5.75) (5.5,5.75)};
        \addplot [blue!10] fill between [of = bU and bL, soft clip={domain=0:5.5}];
    \end{axis}
    \draw[->,thick] (0,0.5)--(0,5.5) node[above]{$y$};
    \draw[->,thick] (0,0.5)--(6,0.5) node[right]{$x$};
\end{tikzpicture}

\begin{tikzpicture}
    \begin{axis}[
            xmin=-0.0,xmax=6.5,
            ymin=-0.5,ymax=6.5,
            axis line style={draw=none},
            tick style={draw=none},
            yticklabels=\empty,
            xticklabels=\empty,
        ]
        \addplot[smooth, color=black, semithick] table [x=x, y=y0, col sep=comma, mark=none] {./data/gp_intro_dat2.csv};
        \addplot[smooth, color=black, semithick, dashed] table [x=x, y=y1, col sep=comma, mark=none] {./data/gp_intro_dat2.csv};
        \addplot[smooth, color=black, semithick, dotted] table [x=x, y=y2, col sep=comma, mark=none] {./data/gp_intro_dat2.csv};
        \addplot[smooth, color=red, ultra thick] table [x=x, y=mu, col sep=comma, mark=none] {./data/gp_intro_dat2.csv};

        \addplot[name path = bU, smooth, color=blue!10] table [x=x, y=bU, col sep=comma, mark=none] {./data/gp_intro_dat2.csv};
        \addplot[name path = bL, smooth, color=blue!10] table [x=x, y=bL, col sep=comma, mark=none] {./data/gp_intro_dat2.csv};
        \addplot [blue!10] fill between [of = bU and bL, soft clip={domain=0:5.5}];
    \end{axis}
    \draw[->,thick] (0,0.5)--(0,5.5) node[above]{$y$};
    \draw[->,thick] (0,0.5)--(6,0.5) node[right]{$x$};
\end{tikzpicture}


\subsection{Gaussian Processes for Regression}\label{Section1.5}

We saw in section \ref{Section1.4} that Gaussian processes directly predicts the value function we are seeking to predict instead of parameteric values. To keep this computation tractable we only evalute our predicted function at a finite number of points. The prediction itself is found by taking the mean over all functions with respect to the posterior conditioned on the observed data which is assumed to be jointly Gaussian with the input value. This gives rise to Gaussian Process more formally stated in definition

\begin{defe}[Gaussian Process] \label{defe: GP}
    A Gaussian Process (GP) is a collection of random variables with index set $I$, such that every finite subset of random variables has a joint Gaussian distribution \cite{RasmussenCarlEdward2006Gpfm,MurphyKevinP2012Ml}.
\end{defe}

A GP is completely characterised by a mean function $m(\bm{x})$ and a kernel, which in the context of GPs is sometimes called a covariance function, $k (\bm{x}, \bm{x'})$ on a real process as
\begin{align*}
    m(\bm{x})           & = \EE \left[ f(\bm{x}) \right]                                         \\
    k (\bm{x}, \bm{x'}) & = \EE \left[ (f(\bm{x}) - m(\bm{x})) (f(\bm{x'}) - m(\bm{x'})) \right]
\end{align*}
GPs define a prior over all possible functions which can be used to create a posterior once enough data has been observed. The prior is used to represent the functions we expect to see before any observations are made. Although defining a prior over all possible function may seem computationally intractable, we actually only need to define a distribution over a finite number of points. Before any observations are made, we typically assume that the mean function is the constant zero function, that is $m \left( \bm{x} \right) = 0$. A function $f(\bm{x})$ sampled from a GP with mean $m(\bm{x})$ and covariance $k (\bm{x}, \bm{x'})$ is written as
\[
    f(\bm{x}) \sim \calG \calP \left( m(\bm{x}), k (\bm{x}, \bm{x'}) \right)
\]
Since a GP is a collection of random variables it must satisfy the consistency requirement, that is, an observation of a set of variables should not the distribution of any small sub set of the observed values. More specifically if
\[
    (\bm{y_1}, \bm{y_2}) \sim \calN (\bm{\mu}, \bm{\Sigma})
\]
then
\begin{align*}
    \bm{y_1} & \sim \calN (\bm{\mu_1}, \bm{\Sigma_{1,1}}) \\
    \bm{y_2} & \sim \calN (\bm{\mu_2}, \bm{\Sigma_{2,2}})
\end{align*}

where $\bm{\Sigma_{1,1}}$ and $\bm{\Sigma_{2,2}}$ are the relevant sub matrices. Again, we shall us the notation that for set of data $\bm{W} = \left[ \bm{w}_1 ,\bm{w}_2 , \ldots , \bm{w}_n \right]^{\intercal} \in \RR^{n \times d}$ and $\bm{W}' = \left[ \bm{w}_1' ,\bm{w}_2' , \ldots , \bm{w}_m' \right]^{\intercal} \in \RR^{n' \times d}$ we use the notation
\[
    \left( \bm{K}_{\bm{W} \bm{W}'} \right)_{i,j} \triangleq k \left( \bm{w}_i , \bm{w}_j' \right)
\]
where \( \bm{K}_{\bm{W} \bm{W}'} \in \RR^{n \times n'} \). The convariance function completely characterized by its kernel. To understand this better, as a small exercise we can select a number of inputs $\bm{X}^{\ast} = \left[ \bm{x}_1 , \bm{x}_2 , \ldots , \bm{x}_{n^{\ast}} \right]^{\intercal} \in \RR^{n^{\ast} \times d}$ of compute the corresponding covariance matrix using, as an example, the Gaussian RBF kernel. Gaussian vectors can then be sampled using a joint Gaussian distribution using the covariance matrix from the distribution
\[
    \bm{f} \sim \calN \left( \bm{0}, \bm{K_{XX}} \right)
\]
and its values graphed as a function of its inputs. Figure \ref{fig: motive_gp_1} (C) shows functions drawn from the prior before any observations are made. GPs also allow us to compute the pointwise variance which can provide some measure of variability for predicted values. The blue shaded area of figure \ref{fig: motive_gp_1} (C) represents twice the standard deviation about the mean.

\subsubsection{Noise-free observations}\label{Section1.4.1}
Typically when using GP we would like to incorporate data from observations, or training data, into our predictions on unobserved values.
Let us suppose there is some obsevered data $D = \left\{ (\bm{x}_i, \bm{f}_i) \mid i \in \left\{ 1,2, \ldots , n \right\} \right\}$ which is (unrealistically) noise-free that we would like to model as a GP. In other words, for any sample in our dataset we can be certain that the observed value is the true value of the underlying function we wish to model. Then for the observed data
\[
    \bm{f} \sim \calN \left( \bm{0}, \bm{K_{XX}} \right).
\]
where $\bm{K_{XX}} = k(\bm{X}, \bm{X}) \in \RR^{n \times n}$. We would then like to make a prediction for unobserved values say $X^{\ast} = \left[ \bm{x}_1^{\ast}, \bm{x}_2^{\ast}, \ldots , \bm{x}_{n_\ast}^{\ast} \right]$ with value $f_{\ast}$ as has a distribution of
\[
    \bm{f}_{\ast} \sim \calN \left( \bm{0}, \bm{K_{X^{\ast}X^{\ast}}} \right).
\]
where $\bm{K_{X^{\ast}X^{\ast}}} = k(\bm{X^{\ast}}, \bm{X^{\ast}}) \in \RR^{n_\ast \times n_\ast}$. Here $\bm{f}$ and $\bm{f}_{\ast}$ are independent but we would like to give them some sort of correlation. We can do this by having them originate from the same joint distribution. According to the prior, we can write the joint distribution of the training points $\bm{f}$ and the test points $\bm{f}_{\ast}$ as
\[
    \begin{pmatrix}
        \bm{f} \\
        \bm{f}_{\ast}
    \end{pmatrix}
    \sim \calN
    \begin{pmatrix}
        \bm{0}, &
        {
                \begin{pmatrix}
                    \bm{K_{XX}}                    & \bm{K_{XX^{\ast}}}        \\
                    \bm{K_{XX^{\ast}}}^{\intercal} & \bm{K_{X^{\ast}X^{\ast}}}
                \end{pmatrix}
            }
    \end{pmatrix}
\]
where $\bm{K_{XX^{\ast}}} = k(\bm{X}, \bm{X^{\ast}}) \in \RR^{n \times n_\ast}$.

While the above does give us some information on $\bm{f}_{\ast}$ is related to the observed data and the test inputs, it does not provide any method to evalute $\bm{f}_{\ast}$. To do this we shall need the assistance of the following lemma
\begin{thm}\label{theorem: cond_of_MVN}
    (Marginals and conditionals of an MVN \cite{MurphyKevinP2012Ml}) Suppose $\bm{x} = (\bm{x}_1, \bm{x}_2)$ is jointly Gaussian with parameters
    \[
        \bm{\mu} =
        \begin{pmatrix}
            \bm{\mu}_1 \\
            \bm{\mu}_2
        \end{pmatrix}, \quad
        \bm{\Sigma} =
        \begin{pmatrix}
            \bm{\Sigma}_{1,1} & \bm{\Sigma}_{1,2} \\
            \bm{\Sigma}_{2,1} & \bm{\Sigma}_{2,2}
        \end{pmatrix}
    \]
    then the posterior conditional is given by
    \begin{align*}
        \bm{x}_2 \mid \bm{x}_1 & \sim \calN \left( \bm{x}_2 \mid \bm{\mu}_{2 \mid 1}, \bm{\Sigma}_{2 \mid 1} \right)          \\
        \bm{\mu}_{2 \mid 1}    & = \bm{\mu}_2 + \bm{\Sigma}_{2,1} \bm{\Sigma}_{1,1}^{-1} \left( \bm{x}_1 - \bm{\mu}_1 \right) \\
        \bm{\Sigma}_{2 \mid 1} & = \bm{\Sigma}_{2,2} - \bm{\Sigma}_{2,1} \bm{\Sigma}_{1,1}^{-1} \bm{\Sigma}_{1,2}
    \end{align*}
\end{thm}

Thus finding a mean an covariance for $\bm{f}_{\ast}$ requires a direct application of Theorem \ref{theorem: cond_of_MVN} which gives
\begin{align*}
    \bm{f}_{\ast} \mid \bm{K_{XX^{\ast}}} , \bm{K_{XX}}, \bm{f} \sim \calN \left( \bm{\mu}^{\ast}, \bm{\Sigma}^{\ast} \right)
\end{align*}
where
\begin{align*}
    \bm{\mu}^{\ast} & = \bm{0} + \bm{K_{XX^{\ast}}}^{\intercal} \bm{K_{XX}}^{-1} \left( \bm{f} - \bm{0} \right) \\
                    & = \bm{K_{XX^{\ast}}}^{\intercal} \bm{K_{XX}}^{-1} \bm{f}
\end{align*}
and
\begin{align*}
    \bm{\Sigma}^{\ast} & = \bm{K_{X^{\ast}X^{\ast}}} - \bm{K_{XX^{\ast}}}^{\intercal} \bm{K_{XX}}^{-1} \bm{K_{XX^{\ast}}}
\end{align*}
meaning we can write a distribution for $\bm{f}_{\ast}$ as
\begin{equation}\label{prop:GP_train_distr1}
    \bm{f}_{\ast} \mid \bm{K_{XX^{\ast}}} , \bm{K_{XX}}, \bm{f} \sim \calN \left( \bm{K_{XX^{\ast}}}^{\intercal} \bm{K_{XX}}^{-1} \bm{f},  \bm{K_{X^{\ast}X^{\ast}}} - \bm{K_{XX^{\ast}}}^{\intercal} \bm{K_{XX}}^{-1} \bm{K_{XX^{\ast}}}  \right)
\end{equation}
Function values from the unobserved inputs $\bm{X^{\ast}}$ can be estimated using the mean of $\bm{f}_{\ast}$ evaluted in \ref{prop:GP_train_distr1}.

\subsubsection{Prediction with Noisy observations}\label{Section1.1.2}
When attempting to model our value function we usually do not have access to the value function itself but a noisy version thereof, $y = f(\bm{x}) + \varepsilon$ where $\varepsilon \calN (0, \sigma_n^2)$ meaning the prior on the noisy values becomes
\[
    \operatorname{cov} (\bm{y}) = \bm{K_{XX}} + \sigma_n^2 \bm{I}
\]
The reason why noise is only added along the diagonal follows from the assumption of independence in our data.
We can write out the new distribution of the observed noisy values along the points at which we wish to test the underlying function as
\[
    \begin{pmatrix}
        \bm{f} \\
        \bm{f}_{\ast}
    \end{pmatrix}
    \sim \calN
    \begin{pmatrix}
        \bm{0}, &
        {
                \begin{pmatrix}
                    \bm{K_{XX}} + \sigma_n^2 \bm{I} & \bm{K_{XX^{\ast}}}        \\
                    \bm{K_{XX^{\ast}}}^{\intercal}  & \bm{K_{X^{\ast}X^{\ast}}}
                \end{pmatrix}
            }
    \end{pmatrix}
\]
Using a similar we arrive at a similar condition distribution of $\bm{f}_{\ast} \mid \bm{K_{XX^{\ast}}} , \bm{K_{XX}}, \bm{f}$ we arrive at one of the most fundamental equations for GP regression tasks
\begin{align*}\label{prop:GP_train_distr2}
    \bm{f}_{\ast} \mid \bm{K_{XX^{\ast}}} , \bm{K_{XX}}, \bm{y} \sim & \calN \left( \overline{\bm{f}_{\ast}}, \operatorname{cov} (\bm{f}_{\ast}) \right)                                                   \\
    \overline{\bm{f}_{\ast}}                                         & := \bm{K_{XX^{\ast}}}^{\intercal} \left[ \bm{K_{XX}} + \sigma_n^2 \bm{I} \right]^{-1} \bm{y}                                        \\
    \operatorname{cov} (\bm{f}_{\ast})                               & = \bm{K_{X^{\ast}X^{\ast}}} - \bm{K_{XX^{\ast}}}^{\intercal} \left[ \bm{K_{XX}} + \sigma_n^2 \bm{I} \right]^{-1} \bm{K_{XX^{\ast}}}
\end{align*}

\subsection{Gaussian Processes for Classification}\label{Section1.6}

As with most classification models, the Gaussian processes classifier (GPC) seeks an estimate for the joint probability $p \left( y , \bm{x} \right)$ where $\bm{x} \in \RR^d$ is an input, as in the regression case, but $y$ is now a class taking on a discrete and finite number of values $\left\{ \calC_i \right\}_{i=1}^C$. Using Baye's theorem the joint probability density can be decomposed into either $p \left( y \right) p \left( \bm{x} \mid y \right)$ or $p \left( \bm{x} \right) p \left( \bm{y} \mid \bm{x} \right)$ giving rise to the {\it generative} and {\it discriminative} approaches respectively \cite{RasmussenCarlEdward2006Gpfm}*{page 34}. The generative approach models the prior probabilities of each class, $p \left( \calC_i \right)$, as well as the class conditional probabilities for each input $p \left( \bm{x} \mid \calC_i \right)$ and computes the posterior as
\[
    p \left( y \mid \bm{x} \right) = \frac{ p \left( y \right) p \left( \bm{x} \mid y \right) }{ \sum_{i=1}^{C} p \left( \calC_i \right) p \left( \bm{x} \mid \calC_i \right) }.
\]
On the other hand, the discriminative method focuses on modelling $p \left( y \mid \bm{x} \right)$ directly. With both these paradigms at our disposal, which one would be preferred for our GPC? While there are strengths and weaknesses associated with both models, the discriminative approach is usually chosen as it has a rather attractive property of directly modeling what we require, that is $p \left( y \mid \bm{x} \right)$. Aditionally, the density estimation of $p \left( \bm{x} \mid \calC_i \right)$ using in the generative model presents a number of difficulties, especially for larger values of $d$. If we are only focused on classifying inputs, the generative approach could mean trying to solve a harder problem than what is necessary. For this reason we focus on GPCs that adopt the discriminative approach.

\subsubsection{Linear Models for Classification}\label{Section1.6.1}

We can start by reviewing linear models for the simplist form of classification, that is binary classification. Adopting the notation from SVM (see \Cref{Section1.4.1}) literature, the binary classification problem involves assigning an input $\bm{x}$ to a class of either $-1$ or $+1$. For a linear model likelihood can be formulated as
\begin{equation} \label{eq: GPC-lin-model-1}
    p \left( y=+1 \mid \bm{x} , \bm{w} \right) = \sigma \left( \langle \bm{x} , \bm{w} \rangle \right)
\end{equation}
given a weight vector $\bm{w}$ and where $\sigma (\bm{z})$ is chosen to be any sigmoid function, see \Cref{defe: sigmoid-function}.
\begin{defe}[Sigmoid Function] \label{defe: sigmoid-function}
    A sigmoid function is a monotonically increasing function mapping from $\RR$ to $\left[ 0,1 \right]$ \cite{RasmussenCarlEdward2006Gpfm}*{page 35}.
\end{defe}
In this text, the commonly used logistic function
\begin{equation} \label{eq: logistic-function}
    \sigma (z) = \frac{1}{1 + \exp (-z)}
\end{equation}
will take the role of the sigmoid function in equation \ref{eq: GPC-lin-model-1}, graphed in Figure \ref{fig: logistic-func-and-probit}. This type of model is aptly named the logistic regression.
\begin{figure}[h]
    \centering
    \begin{tikzpicture}[>=latex, scale=1.1]
        \begin{axis}[
                axis line style = thick,
                xlabel={$x$},
                ylabel={$y$},
                every axis x label/.style={
                        at={(ticklabel* cs:1.01)},
                        anchor=west,
                    },
                xmin=-5.5,xmax=5.5,
                ymin=-0.125,ymax=1.125,
                axis y line =middle,
                axis x line =bottom,
                every axis y label/.style={
                        at={(ticklabel* cs:1.01)},
                        anchor=south,
                    },
                % axis line style={->},
                xticklabels={$-5$, $5$},
                xtick={-5,5},
                yticklabels={$0$, $1$},
                ytick={-0.05,0.95},
                % axis line style={draw=none},
                ytick style={draw=none},
            ]
            \addplot[smooth, color=red, thick] {1/(1+exp(-x))};
            \addplot [smooth, blue, thick, dashed] {normcdf((0.6266 * x),0,1)};
            \addplot[smooth, color=black, semithick, dotted] {0};
            \addplot[smooth, color=black, semithick, dotted] {1};
        \end{axis}
    \end{tikzpicture}
    \caption{The logistic function from equation \ref{eq: logistic-function} (solid red) juxtaposed with a close approximation, the scaled probit function (dashed blue).}
    \label{fig: logistic-func-and-probit}
\end{figure}
Unlike GPR, the likelihood is no longer a Gaussian distribution. Instead it follows the Bernoulli distribution
\begin{equation*}
    p \left( y \mid \bm{x} , \bm{w} \right) = \sigma \left( \langle \bm{x} , \bm{w} \rangle \right)^{y} \left( 1 - \sigma \left( \langle \bm{x} , \bm{w} \rangle \right) \right)^{\frac{1 - y}{2}}
\end{equation*}
which for symmeteric likelihood functions can be written more concisely as
\begin{equation*}
    p \left( y_i \mid \bm{x}_i , \bm{w} \right) = \sigma \left( y_i f_i \right)
\end{equation*}
where
\begin{equation} \label{eq: GPC-lin-latent-func}
    f_i \triangleq f \left( \bm{x}_i \right) = \langle \bm{x} , \bm{w} \rangle .
\end{equation}
Thus, the logistic regression model can be written as the log ratio of the likelihoods of the input belonging to either class, that is
\begin{equation*}
    \operatorname{logit} \left( \bm{x} \right) \triangleq \langle \bm{x} , \bm{w} \rangle = \log \left( \frac{p \left( y = +1 \right)}{p \left( y = -1 \right)} \right)
\end{equation*}
where $\operatorname{logit}$ is commonly referred to as the logit transformation \cite{RasmussenCarlEdward2006Gpfm}*{page 37}. For a given dataset $\calD = \left\{ \left( x_i , y_i \right) \right\}_{i=1}^{n}$ we assume each observation is independently generated conditioned over $f \left( \bm{x} \right)$. Similar to GPR, a Gaussian prior is used for the weights so that $\bm{w} \sim \calN \left( \bm{0} , \sigma_p \right)$ giving an un-normalised log posterior of
\begin{equation*}
    \log p \left( \bm{w} \mid \bm{X} , \bm{y} \right) \propto - \frac{1}{2} \bm{w}^{\intercal} \Sigma_p^{-1} \bm{w} + \sum_{i=1}^{n} \log \sigma \left( y_i f_i \right).
\end{equation*}

However, unlike GPR an analytic form for the mean and variance for the posterior is not available due to the non-Gaussian nature of the likelihood, although, when using the logistic function it is easy enough to show that the log likelihood is concave as a function of $\bm{w}$ for a fixed dataset. This means a number of numerical optimization techniques, such as Newton's method or the Broyden-Fletcher-Goldfarb-Shanno (BFGS) algorithm \cite{FletcherR2000PMoO} can be used to solve these values.

The idea behind Gaussian process classification for binary classes is that a Gaussian process prior is placed over a latent function $f \left( \bm{x} \right)$ where the output is then "squashed" through a sigmoid function to obtain a prior on
\begin{equation*}
    \pi \left( \bm{x} \right) \triangleq p \left( y=+1 \mid \bm{x} \right) = \sigma \left( f \left( \bm{x} \right) \right).
\end{equation*}
This construction is illustrated in Figure \ref{fig: latent-func-and-sig-trans} and provides a natural extension to the linear logistic regression model.

\begin{filecontents*}{./data/gpr_latent_fig1.csv}
    x,f,sig
    0.0,-0.716636450452456,0.3281340890410802
    0.05555555555555555,-0.3872677518321085,0.4043752064908497
    0.1111111111111111,-0.04169388251654786,0.48957803910434766
    0.16666666666666666,0.30748545475025907,0.5762713707991717
    0.2222222222222222,0.6464740783985357,0.6562154650063955
    0.2777777777777778,0.961561586658563,0.7234343518272932
    0.3333333333333333,1.2404054048798008,0.7756345729728175
    0.38888888888888884,1.4731593678003678,0.8135371194857097
    0.4444444444444444,1.653300565910448,0.8393366325299798
    0.5,1.77804434211388,0.8554552137616092
    0.5555555555555556,1.8482947284219486,0.8639267602259872
    0.611111111111111,1.8681477880300634,0.8662438156048382
    0.6666666666666666,1.844024635176618,0.8634239990093878
    0.7222222222222222,1.7835778456043871,0.8561380953611345
    0.7777777777777777,1.6945293075464642,0.8448188796360699
    0.8333333333333333,1.5836227815051651,0.8297169804398394
    0.8888888888888888,1.4558461620643777,0.8108965338173831
    0.9444444444444444,1.3140273314869115,0.7881862938214962
    1.0,1.1588551707012498,0.7611246311602722
    1.0555555555555556,0.9893037805660674,0.7289503840440278
    1.1111111111111112,0.8033735795719339,0.690695659627058
    1.1666666666666665,0.5990205172125318,0.6454321840431138
    1.222222222222222,0.37510931242033974,0.5926929891152688
    1.2777777777777777,0.13223323848313873,0.5330102232552971
    1.3333333333333333,-0.12672784735928108,0.46836037106151224
    1.3888888888888888,-0.3964103237476957,0.40217510289406855
    1.4444444444444444,-0.6691517706610377,0.33868679954793734
    1.5,-0.9355408280853343,0.28180195391419693
    1.5555555555555554,-1.1852457987537584,0.23411029892781032
    1.611111111111111,-1.408010271880984,0.196548078418454
    1.6666666666666665,-1.5946766330839104,0.16872694438891356
    1.722222222222222,-1.7381092175623238,0.1495532583157629
    1.7777777777777777,-1.8339034326105361,0.13777391959330848
    1.8333333333333333,-1.8808066358911115,0.13229624915295843
    1.8888888888888888,-1.8808148704868526,0.13229530387403557
    1.9444444444444444,-1.8389575421436446,0.1371746288812035
    2.0,-1.762808219338528,0.14643897890898566
    2.0555555555555554,-1.6617924461232692,0.15952152909513476
    2.111111111111111,-1.5463718876794204,0.17561089816790565
    2.1666666666666665,-1.4271799785358956,0.1935384565574492
    2.2222222222222223,-1.3141764578300106,0.21178881080529863
    2.2777777777777777,-1.2158759418012686,0.22866302240021388
    2.333333333333333,-1.1386790564398872,0.24256297036803995
    2.388888888888889,-1.0863393391090705,0.2523082312707144
    2.444444444444444,-1.0595738916176571,0.2573908929832913
    2.5,-1.0558426697384702,0.25810472715213173
    2.5555555555555554,-1.0693030523966423,0.2555356469279933
    2.611111111111111,-1.0909711640146635,0.25143544599371426
    2.6666666666666665,-1.1090929165853411,0.24804003562062307
    2.722222222222222,-1.1097459725310912,0.24791825016606753
    2.7777777777777777,-1.0776532758488337,0.25395036925471315
    2.833333333333333,-0.997190360006935,0.26949418860886254
    2.888888888888889,-0.853523705726157,0.2986942026351552
    2.944444444444444,-0.6338078056665624,0.34664763152253814
    3.0,-0.3283465008205865,0.4186429992846845
    3.0555555555555554,0.06838484205618314,0.5170895511116412
    3.1111111111111107,0.5569529512258093,0.6357472192793857
    3.1666666666666665,1.132487738026538,0.7562977100837649
    3.222222222222222,1.7845801849827891,0.8562615050851906
    3.2777777777777777,2.49755753518847,0.9239704166376049
    3.333333333333333,3.2511324784312694,0.9627137853569252
    3.388888888888889,4.021376394857483,0.9823874902884194
    3.444444444444444,4.781943394168808,0.9916899377360553
    3.5,5.505458755235512,0.995951929944337
    3.5555555555555554,6.164984744717579,0.9979026576630112
    3.6111111111111107,6.735466982435726,0.9988133894314339
    3.6666666666666665,7.1950796833818895,0.9992502941814919
    3.722222222222222,7.526406822593906,0.9994616196879862
    3.7777777777777777,7.717385130530989,0.9995551751969357
    3.833333333333333,7.761974187627159,0.9995745655656185
    3.888888888888889,7.660512373107247,0.9995291558112748
    3.944444444444444,7.419735623560392,0.9994010513870635
    4.0,7.052442559921735,0.999135455240024
    4.055555555555555,6.5768311951578315,0.9986096815828182
    4.111111111111111,6.0155205200837125,0.9975653642534463
    4.166666666666666,5.394318761388515,0.9954782255735908
    4.222222222222222,4.740803814055081,0.991343956562032
    4.277777777777778,4.082820560277176,0.9834196970094687
    4.333333333333333,3.446989970396544,0.9691412482325584
    4.388888888888888,2.857339498962172,0.9456968331619086
    4.444444444444445,2.334160434608953,0.911666952206589
    4.5,1.8931622621329902,0.8691156694032487
    4.555555555555555,1.544984634180779,0.8241881763838215
    4.611111111111111,1.295071693838633,0.7850043888236089
    4.666666666666666,1.1439101905755478,0.7583968311600912
    4.722222222222222,1.087555379614477,0.7479211041915879
    4.777777777777778,1.1183780186296264,0.7536877316301696
    4.833333333333333,1.2259536925758674,0.7731095912803282
    4.888888888888888,1.3979718665845051,0.8018618571626841
    4.944444444444444,1.6211061459040699,0.8349476243586275
    5.0,1.8817742817340295,0.8678147912195052
    5.055555555555555,2.1667472795664557,0.8972234089235315
    5.111111111111111,2.4636123746267415,0.9215512159171637
    5.166666666666666,2.761092162073636,0.9405367451793998
    5.222222222222222,3.049269824821768,0.9547509923004637
    5.277777777777778,3.3197364525186464,0.9650997156829392
    5.333333333333333,3.5657178376098226,0.9725009030361754
    5.388888888888888,3.7821806293058837,0.9777340834061939
    5.444444444444444,3.9659265896685927,0.9814019721867842
    5.5,4.115652418864887,0.983946623009968
\end{filecontents*}

\begin{figure}[h]
    \centering
    \subfloat[]{
        \begin{adjustbox}{width=0.48\textwidth}
            \begin{tikzpicture}[>=latex]
                \begin{axis}[
                        xmin=-0.0,xmax=6.5,
                        ymin=-4.0,ymax=10.0,
                        axis line style={draw=none},
                        tick style={draw=none},
                        yticklabels=\empty,
                        xticklabels=\empty,
                    ]
                    \addplot[smooth, color=red, thick] table [x=x, y=f, col sep=comma, mark=none] {./data/gpr_latent_fig1.csv};
                \end{axis}
                \draw[->,thick] (0,0.5)--(0,5.5) node[above]{$f(x)$};
                \draw[->,thick] (0,0.5)--(6,0.5) node[right]{$x$};

                \draw[dotted] (5.8,0.7)--(0,0.7) node[left]{$-4$};
                \draw[dotted] (5.8,5.1)--(0,5.1) node[left]{$9$};
            \end{tikzpicture}
        \end{adjustbox}
    }
    \subfloat[]{
        \begin{adjustbox}{width=0.48\textwidth}
            \begin{tikzpicture}[>=latex]
                \begin{axis}[
                        xmin=-0.0,xmax=6.5,
                        ymin=-0.05,ymax=1.15,
                        axis line style={draw=none},
                        tick style={draw=none},
                        yticklabels=\empty,
                        xticklabels=\empty,
                    ]
                    \addplot[smooth, color=red, thick] table [x=x, y=sig, col sep=comma, mark=none] {./data/gpr_latent_fig1.csv};
                \end{axis}
                \draw[->,thick] (0,0.5)--(0,5.5) node[above]{$\sigma(f(x)$)};
                \draw[->,thick] (0,0.5)--(6,0.5) node[right]{$x$};

                \draw[dotted] (5.8,0.7)--(0,0.7) node[left]{$0$};
                \draw[dotted] (5.8,5.1)--(0,5.1) node[left]{$1$};
            \end{tikzpicture}
        \end{adjustbox}
    }
    \caption{The latent function $f$, panel (A), is transformed using a sigmoid function, panel (B), to provide a probabilistic interpretation of $x$ belonging to the class $+1$.}
    \label{fig: latent-func-and-sig-trans}
\end{figure}
Specifically, the linear model from equation \ref{eq: GPC-lin-latent-func} is replaced with a GPR model and the Gaussian prior on the weights with a GPR weight prior with
\begin{equation*}
    p \left(
    \begin{bmatrix}
            \bm{f} \\
            f_{\star}
        \end{bmatrix}
    \right)
    =
    \calN \left( \bm{0} ,
    \begin{bmatrix}
        \bm{K_{XX}}         & \bm{K_{x^{\star}X}}^{\intercal}                 \\
        \bm{K_{x^{\star}X}} & k \left( \bm{x}_{\star}, \bm{x}_{\star} \right)
    \end{bmatrix}
    \right)
\end{equation*}
where $f_{\star} = f ( \bm{x}_{\star} )$ and $\bm{f} = f \left( \bm{X} \right)$. For classification tasks, we assume that each observation has received the correct label which is why no noise is added to the covariance matrix.

Note that values of $f$ are also never observed within the phenomena we are modelling, nor are we particularly interested in them. The function $f$ serves the role of a {\it nuisance function} and acts solely as a convenience tool within our formulations. The ultimate goal is to make predictions for $\pi$, not $f$, and the goal of the coming sections will be to eventually integrate out $f$.

Subsequently, predictions for $\pi_{\star} = \pi \left( \bm{x}_{\star} \right)$ are made by average over all possible latent functions weighted by the posterior giving the prediction
\begin{equation} \label{eq: GP-pred-1}
    \overline{\pi_{\star}} \triangleq p \left( y_{\star} = +1 \mid \bm{X} , \bm{y} , \bm{x}_{\star} \right) = \int \sigma \left( f_{\star} \right) p \left( f_{\star} \mid \bm{X} , \bm{y} , \bm{x}_{\star} \right) \; d f_{\star}
\end{equation}
While this is a sound model, computing predictions is not so straight forward since the integral in \ref{eq: GP-pred-1} is not analytically tractable for the same reason as the linear binary classifier. Later on we will see how we can make use of our numerical toolbox to derive a good approximation for $\overline{\pi_{\star}}$.

\subsubsection{Lapace Approximation for Posterior}\label{Section1.6.2}

We saw that the integral in \ref{eq: GP-pred-1} could not be used to make predictions for $\overline{\pi_{\star}}$ analytically. In this section we shall address how the distribution for the latent process, $p \left( f_{\star} \mid \bm{X} , \bm{y} , \bm{x}_{\star} \right)$, can be approximated to provide a numerically tractable succedaneum. Using Baye's theorem
\begin{align*}
    p \left( f_{\star} \mid \bm{X} , \bm{y} , \bm{x}_{\star} \right)
     & = \int p \left( f_{\star} , \bm{f} \mid \bm{X} , \bm{y} , \bm{x}_{\star} \right) \; d \bm{f}                                                                                                                                                       \\
     & = \frac{1}{p \left( \bm{y} \mid \bm{X} , \bm{x}_{\star} \right)} \int p \left( f_{\star} \mid \bm{X} , \bm{x}_{\star}, \bm{f} \right) p \left( \bm{f} \mid \bm{X} \right) p \left( \bm{y} \mid \bm{X} , \bm{x}_{\star}, \bm{f} \right) \; d \bm{f} \\
     & = \int p \left( f_{\star} \mid \bm{X} , \bm{x}_{\star}, \bm{f} \right) p \left( \bm{f} \mid \bm{X} , \bm{y} \right) \; d \bm{f}
\end{align*}
using the fact that $p \left( \bm{y} \mid \bm{X} , \bm{x}_{\star}, \bm{f}, f_{\star} \right) = p \left( \bm{y} \mid \bm{X} , \bm{x}_{\star}, \bm{f} \right)$ \cite{BishopChristopherM2006Pram, RasmussenCarlEdward2006Gpfm}. The conditional distribution $p \left( f_{\star} \mid \bm{X} , \bm{x}_{\star}, \bm{f} \right)$ can be derived as
\begin{equation*}
    p \left( f_{\star} \mid \bm{X} , \bm{x}_{\star}, \bm{f} \right) = \calN \left( f_{\star} \mid \bm{K}_{\bm{x}_{\star}\bm{X}} \bm{K}_{\bm{X} \bm{X}}^{-1} \bm{y}, k \left( \bm{x}_{\star}, \bm{x}_{\star} \right) - \bm{K}_{\bm{x}_{\star}\bm{X}} \bm{K}_{\bm{X} \bm{X}}^{-1} \bm{K}_{\bm{x}_{\star} \bm{X}}^{\intercal} \right)
\end{equation*}
through the use of equation \ref{eq: GP_train_distr2_mean} and \ref{eq: GP_train_distr2_var}. Unfortunately
\begin{equation*}
    p \left( \bm{f} \mid \bm{X} , \bm{y} \right) = \frac{p \left( \bm{y} \mid \bm{f} \right) p \left( \bm{f} \mid \bm{X} \right) }{p \left( \bm{y} \mid \bm{X} \right)}
\end{equation*}
does not follow a Gaussian distribution. Instead we can use a Lapace approximation to estimate $p \left( \bm{f} \mid \bm{X} , \bm{y} \right)$ as a Gaussian distribution. Breifly, the Lapace approximation works by assuming the distribution at hand, $p \left( \bm{z} \right)$, can be modelled as
\begin{equation*}
    p \left( \bm{z} \right) = \frac{1}{c} q \left( \bm{z} \right)
\end{equation*}
where $q \left( \bm{z} \right)$ is multivariate Gaussian and $c$ is some normalization constant \cite{BishopChristopherM2006Pram}*{page 214}. To do this, first the centre of $q \left( \bm{z} \right)$ is placed at the mode of $p \left( \bm{z} \right)$. The mode of $p \left( \bm{z} \right)$ is
\begin{equation*}
    \bm{z}_0 = \argmin_{\bm{z}} p \left( \bm{z} \right)
\end{equation*}
which can be computed by solving
\begin{equation} \label{eq: lapace-grad-zero}
    \nabla p \left( \bm{z}_0 \right) = \bm{0}.
\end{equation}
To ensure the covariance of the synthesized multivariate Gaussian behaves similar to the original distribution we can make use of an important property of the Gaussian distribution which is its logarithm being is a quadratic function of its inputs. Taking the Taylor series expansion of $\ln q \left( \bm{z} \right)$ centered at $\bm{z}_0$ yields
\begin{equation*}
    \ln q \left( \bm{z} \right) \simeq \ln q \left( \bm{z}_0 \right) - \frac{1}{2} \left( \bm{z} - \bm{z}_0 \right)^{\intercal} \bm{A} \left( \bm{z} - \bm{z}_0 \right)
\end{equation*}
where
\begin{equation*}
    \bm{A} = - \nabla \nabla \left. \ln q \left( \bm{z} \right) \right|_{\bm{z} = \bm{z}_0}.
\end{equation*}
Expotentiating both sides gives
\begin{align}
    q \left( \bm{z} \right)
     & \simeq q \left( \bm{z}_0 \right) \exp \left( - \frac{1}{2} \left( \bm{z} - \bm{z}_0 \right)^{\intercal} \bm{A} \left( \bm{z} - \bm{z}_0 \right) \right) \nonumber \\
     & \propto \calN \left( \bm{z} \mid \bm{z}_0 , \bm{A}^{-1} \right) \label{eq: lapace-gauss-apprx} .
\end{align}
Returning to our original problem of estimating $p \left( \bm{f} \mid \bm{X} , \bm{y} \right) \propto p \left( \bm{y} \mid \bm{f} \right) p \left( \bm{f} \mid \bm{X} \right)$ as a Gaussian distribution, the prior $p \left( \bm{f} \mid \bm{X} \right)$ follows a Gaussian distribution with zero mean and covariance $\bm{K}_{\bm{X} \bm{X}}$ and the distribution of $p \left( \bm{y} \mid \bm{f} \right)$ (assuming independence of samples) can be written as
\begin{equation*}
    p \left( \bm{y} \mid \bm{f} \right) = \prod_{i=1}^{n} \sigma \left( y_i f_i \right).
\end{equation*}
To find a Laplace approximation for $p \left( \bm{f} \mid \bm{X} , \bm{y} \right)$ we only need to consider an unnormalized posterior when maximizing with respect to $\bm{f}$ since $p \left( \bm{y} \mid \bm{f} \right)$ does not depend on $\bm{f}$. Thus, the log of the unnormalized posterior is
\begin{align*}
    \Psi \left( \bm{f} \right)
     & \triangleq \ln p \left( \bm{y} \mid \bm{f} \right) + \ln p \left( \bm{f} \mid \bm{X} \right)                                                                                                                                \\
     & = - \sum_{i=1}^{n} \ln \left( 1 + \exp \left( y_i f_i \right) \right) - \frac{1}{2} \bm{f}^{\intercal} \bm{K}_{\bm{X} \bm{X}}^{-1} \bm{f} - \frac{1}{2} \ln \left| \bm{K}_{\bm{X} \bm{X}} \right| - \frac{n}{2} \ln 2 \pi .
\end{align*}
The gradient and Hessian of the unnormalized posterior then becomes
\begin{align*}
    \nabla \Psi \left( \bm{f} \right)        & = \nabla \ln p \left( \bm{y} \mid \bm{f} \right) - \bm{K}_{\bm{X} \bm{X}}^{-1} \bm{f} = \left( \bm{t} - \bm{\pi} \right) - \bm{K}_{\bm{X} \bm{X}}^{-1} \bm{f} \\
    \nabla \nabla \Psi \left( \bm{f} \right) & = \nabla \nabla \ln p \left( \bm{y} \mid \bm{f} \right) - \bm{K}_{\bm{X} \bm{X}}^{-1} = - \bm{W} - \bm{K}_{\bm{X} \bm{X}}^{-1}
\end{align*}
where $\pi_i = p \left( y_i = +1 \mid f_i \right) = \sigma ( f_i )$, $\bm{t} = \left( \bm{y} + \bm{1} \right) / 2 \in \RR^{n}$ and $\bm{W} \triangleq - \nabla \nabla \ln p \left( \bm{y} \mid \bm{f} \right)$ is a diagonal matrix (since the distribution of $y_i$ only depends on $f_i$ and not $f_{j \neq i}$) with entries $\bm{W}_{ii} = \sigma \left( y_i f_i \right)$ \cite{BishopChristopherM2006Pram, RasmussenCarlEdward2006Gpfm}. From equation \ref{eq: lapace-grad-zero}, the mode of $\hat{\bm{f}}$ of $\bm{\Psi}$ can be computed as
\begin{align}
    \nabla \Psi \left( \hat{\bm{f}} \right) & = \bm{0} = \left( \bm{t} - \bm{\pi} \right) - \bm{K}_{\bm{X} \bm{X}}^{-1} \hat{\bm{f}} \nonumber \\
    \iff \hat{\bm{f}}                       & = \bm{K}_{\bm{X} \bm{X}} \left( \bm{t} - \bm{\pi} \right) \label{eq: expr-for-mode-lapace} .
\end{align}
Since $\bm{t} - \bm{\pi}$ is a non-linear function, a non-linear optimization technique method is required to solve $\hat{\bm{f}}$ in \ref{eq: expr-for-mode-lapace}. Since the Hessian of $\Psi \left( \bm{f} \right)$ is available, Newton's method is typically employed as fast iterative method to approximate $\hat{\bm{f}}$ where $\hat{\bm{f}}$ is updated as
\begin{equation*}
    \hat{\bm{f}}^{\; \text{new}} = \bm{K}_{\bm{X} \bm{X}} \left( \Id_{n \times n} + \bm{W} \bm{K}_{\bm{X} \bm{X}} \right)^{-1} \left( \bm{W} \hat{\bm{f}}^{\; \text{old}} + \nabla \ln \left( \bm{y} \mid \hat{\bm{f}}^{\; \text{old}} \right) \right).
\end{equation*}
Once a suitable mode is found, using equation \ref{eq: lapace-gauss-apprx}, the Lapacian approximation for $p \left( \bm{f} \mid \bm{X} , \bm{y} \right)$ becomes
\begin{equation} \label{eq: lapace-apprx}
    p \left( \bm{f} \mid \bm{X} , \bm{y} \right) \simeq q \left( \bm{f} \mid \bm{X} , \bm{y} \right) = \calN \left( \hat{\bm{f}} , \left( \bm{K}_{\bm{X} \bm{X}}^{-1} + \bm{W} \right)^{-1} \right).
\end{equation}

\subsubsection{Predictions}\label{Section1.6.3}

With the Lapace approximation for $p \left( \bm{f} \mid \bm{X} , \bm{y} \right)$ (equation \ref{eq: lapace-apprx}) and an exact probability distribution for $p \left( f_{\star} \mid \bm{X} , \bm{x}_{\star}, \bm{f} \right)$, a mean for the latent process, $p \left( f_{\star} \mid \bm{X} , \bm{y} , \bm{x}_{\star} \right)$, can now be computed by invoking \ref{eq: GP_train_distr2_mean} to give
\begin{align}
    \mu_{f_{\star}} = \EE \left[ f_{\star} \mid \bm{X} , \bm{y} , \bm{x}_{\star} \right]
     & = \bm{K}_{\bm{x}_{\star} \bm{X}} \bm{K}_{\bm{X} \bm{X}}^{-1} \hat{\bm{f}}           \nonumber               \\
     & = \bm{K}_{\bm{x}_{\star} \bm{X}} \nabla \ln \left( \bm{y} \mid \hat{\bm{f}} \right) \nonumber               \\
     & = \bm{K}_{\bm{x}_{\star} \bm{X}} \left( \bm{t} - \bm{\pi} \right) \label{eq: latent-process-mean-apprx-1} .
\end{align}
Similarly, the variance can be computed using equation \ref{eq: GP_train_distr2_var} to give
\begin{align} \label{eq: latent-process-var-apprx-1}
    \sigma_{f_{\star}}^2 = \VV \left[ f_{\star} \mid \bm{X} , \bm{y} , \bm{x}_{\star} \right]
     & = k \left( \bm{x}_{\star} , \bm{x}_{\star} \right) - \bm{K}_{\bm{x}_{\star} \bm{X}} \left( \bm{K}_{\bm{X} \bm{X}} + \bm{W}^{-1} \right)^{-1} \bm{K}_{\bm{x}_{\star} \bm{X}}^{\intercal}.
\end{align}
Using equation \ref{eq: GP-pred-1}, predictions can now be made as
\begin{equation} \label{eq: pred-apprx-1}
    \overline{\pi_{\star}} \simeq \int \sigma \left( f_{\star} \right) q \left( f_{\star} \mid \bm{X} , \bm{y} , \bm{x}_{\star} \right) \; d f_{\star}
\end{equation}
where $q \left( f_{\star} \mid \bm{X} , \bm{y} , \bm{x}_{\star} \right)$ is a multivariate Gaussian distribution with mean and variance given by equations \ref{eq: latent-process-mean-apprx-1} and \ref{eq: latent-process-var-apprx-1} respectively. Notice that the prediction given in \ref{eq: pred-apprx-1} is a convolution of a Gaussian and logistic function which unfortunately cannot be evaluated analytically. However, Spiegelhalter and Lauritzen \cite{spiegelhalter1990sequential} show that a good approximation can be found by replacing the sigmoid function with the probit function $\Phi \left( \lambda a \right)$ which is simply the cumulative distribution function (CDF) of the standard Gaussian distribution. To get the best approximation using the probit function, the constant factor $\lambda$ is adjusted to equate their slopes at the origin. The value of $\lambda$ that gives this equality is $\lambda = \sqrt{\pi / 8}$. The similarity between the sigmoid function and probit function rescaled by a factor of $\sqrt{\pi / 8}$ is illustrated in Figure \ref{fig: logistic-func-and-probit}. The reason for replacing the sigmoid function with a probit function is that the convolution of a Gaussian distribution and probit function can be analytically evaluated as
\begin{equation} \label{eq: probit-int-apprx-1}
    \int \Phi \left( \lambda a \right) \calN \left( a \mid \mu , \sigma^2 \right) \; da = \Phi \left( \frac{\mu}{\left( \lambda^{-2} + \sigma^2 \right)^{\frac{1}{2}}} \right).
\end{equation}
Again apply the approximation $\sigma \left( a \right) \simeq \Phi \left( \lambda a \right)$ to left hand side of \ref{eq: probit-int-apprx-1} gives the following estimate for the convolution of a Gaussian and sigmoid function
\begin{equation} \label{eq: pred-apprx-2}
    \int \sigma \left( a \right) \calN \left( a \mid \mu , \sigma^2 \right) \; da \simeq \sigma \left( \frac{\mu}{\left( 1 + \pi \sigma^2 / 8 \right)^{\frac{1}{2}}} \right)
\end{equation}
\cite{BishopChristopherM2006Pram}*{page 219}. The integral used to approximate $\overline{\pi_{\star}}$ in \ref{eq: pred-apprx-1} can now be estimated using \ref{eq: pred-apprx-2} to give
\begin{equation*} \label{eq: pred-apprx-3}
    \overline{\pi_{\star}} = \sigma \left( \frac{\mu_{f_{\star}}}{\left( 1 + \pi \sigma_{f_{\star}}^2 / 8 \right)^{\frac{1}{2}}} \right).
\end{equation*}
This theory justifies Algorithm \ref{alg: Unoptimized_GPC} which creates predictions based on the GPC method.

    {\centering
        \begin{minipage}{.85\linewidth}
            \begin{algorithm}[H]
                \caption{Unoptimized GPC}
                \label{alg: Unoptimized_GPC}
                \SetAlgoLined
                \DontPrintSemicolon
                \SetKwInOut{Input}{input}\SetKwInOut{Output}{output}

                \Input{Observations $\bm{X}, \bm{y}$ and a test input $\bm{x}^{\star}$.}
                \Output{A prediction $\overline{f_{\star}} $ with its corresponding variance $ \VV \left[ f_{\star} \right]$.}
                \BlankLine
                $\bm{t} = \left( \bm{y} + \bm{1} \right) / 2$\;
                $\bm{f} = \bm{0}$\;
                \Repeat{convergence}{
                    $\bm{W} = \operatorname{diag} \left( \sigma \left( \bm{y} .^{\ast} \bm{f} \right) \right)$\;
                    $\bm{\alpha} = \operatorname{lin-solve} \left( \Id_{n \times n} + \bm{W} \bm{K}_{\bm{X} \bm{X}}, \bm{K}_{\bm{X} \bm{X}} \right)$\;
                    $\bm{f} = \bm{\alpha} \left( \bm{t} - \sigma (\bm{f}) + \bm{W} \bm{f} \right)$\;
                }
                $\mu_{f_{\star}} = \bm{K}_{\bm{x}_{\star} \bm{X}} \left( \bm{t} - \sigma (\bm{f}) \right)$\;
                $\sigma_{f_{\star}}^2 = k \left( \bm{x}_{\star} , \bm{x}_{\star} \right) - \bm{K}_{\bm{x}_{\star} \bm{X}} \left( \bm{K}_{\bm{X} \bm{X}} + \bm{W}^{-1} \right)^{-1} \bm{K}_{\bm{x}_{\star} \bm{X}}^{\intercal}$\;
                $\overline{\pi_{\star}} = \sigma \left( \mu_{f_{\star}} / {\left( 1 + \pi \sigma_{f_{\star}}^2 / 8 \right)^{\frac{1}{2}}} \right)$\;
                \Return{$\overline{\pi_{\star}} , \mu_{f_{\star}} , \sigma_{f_{\star}}^2$}
                \BlankLine
            \end{algorithm}
        \end{minipage}
        \par
    }
\newpage

\section{Krylov Subspace Methods}\label{Chapter4}
In this section we will focus on how iterative methods, in particular a class of iterative methods called Krylov Subspace methods, may be used to solve a linear system $\bm{A} \bm{x} = \bm{b}$. While non-iterative methods exist to solve such systems virtually all of them carry an unwieldy runtime of $\mathcal{O} \left( n^3 \right)$ for a system of $n$ parameters. Even for current computer systems, this renders many common matrix problems untractable. Consequently the focus of solving linear systems has shifted towards iterative methods. While iterative methods typically demand certain structural properties of the matrices, such as symmetry and positive definiteness, this generally is not a problem since the majority of large matrix problems that, by mature, endow these systems with the desired properties. For example, in the context of this paper the Gram matrices used to solve linear systems in Gaussian Processes possess both symmetry and positive definiteness. There are also a number of other properties of iterative methods which make them rather attractive to users. To start, iterative Krylov subspace methods are guranteed to converge to an exact solution within a finite number of iterations and even if the method is prematurely stopped before reaching an exact solution, the approximation obtained on the final iteration will in some sense be a good enough estimate of our exact solution. Furthermore, unlike most non-iterative methods, Krylov subspace methods do not require an explicit form of the matrix $\bm{A}$ and instead only requires some routine or process for computing $\bm{A} \bm{x}$.


\subsection{Krylov Subspaces}\label{Section4.1}

We will motivate the Krylov subspaces by observing their usefullness in solving linear systems. To this end, consider the problem of solving the linear system
\begin{equation}\label{eq: lin_sys_1}
    \bm{A} \bm{x^{\star}} = \bm{b}
\end{equation}
where no explicit form of $\bm{A}$ is available and instead one must draw information from $\bm{A}$ solely through a routine that can evaluate $\bm{A} \bm{v}$ for any $\bm{v}$. How could this routine be utilized in such a manner to provide with a solution to equation \ref{eq: lin_sys_1}? Before answering this, consider the following theorem

\begin{thm} \label{theorem: invert_mat_norm}
    For $\bm{A} \in \CC^{n \times n}$ if $\| \bm{A} \| = q < 1$ then $\Id - \bm{A}$ is invertible and its inverse admits the following representation
    \begin{equation*} \label{eq: convg-of-part}
        \left( \Id - \bm{A} \right)^{-1} = \sum_{k=0}^{\infty} \bm{A}^k
    \end{equation*}
    \cite{BerezanskyMakarovich1996FaV1}*{page 287}.
\end{thm}

\begin{proof}
    Let us show that the sequence $\bm{S}_{n} = \sum_{k=0}^{n} \bm{A}^k$ (where $n$ is some natural number) of partial sums of the series on the right-hand side of \Cref{eq: convg-of-part} is fundamental in $\CC^{n \times n}$. Indeed, by using the fact that $\| \bm{A B} \| \leq \| \bm{A} \| \| \bm{B} \|$ for any $\bm{A} , \bm{B} \in \CC^{n \times n}$, we conclude that
    \begin{equation*}
        \| \bm{S}_{n+p} - \bm{S}_{n} \| \leq \| \bm{A}^{n+1} \| + \cdots + \| \bm{A}^{n+p} \| \leq {q}^{n+1} + \cdots + {q}^{n+p}
    \end{equation*}
    for all $n,p \in \NN$. Since $q < 1$, this yields the desired result. Since $\CC^{n \times n}$ is a Banach space, the sequence $\left\{ \bm{S}_{n} \right\}_{n}^{\infty}$ uniformly converges to an operator $\bm{S} \in \CC^{n \times n}$. Let us show that $\bm{S} \left( \Id_{n \times n} - \bm{A} \right) = \left( \Id_{n \times n} - \bm{A} \right) \bm{S} = \Id_{n \times n}$. To this end, we note that
    \[
        \| \left( \Id_{n \times n} - \bm{A} \right) \bm{S}_n - \left( \Id_{n \times n} - \bm{A} \right) \bm{S} \| \leq \|  \Id_{n \times n} - \bm{A} \| \cdot \| \bm{S} - \bm{S}_{n} \| \to 0
    \]
    as $n \to \infty$. Therefore, it suffices to show that the sequence $\left\{ \left( \Id_{n \times n} - \bm{A} \right) \bm{S}_n \right\}_{n=1}^{\infty}$ uniformly converges to the identity operator. We have
    \[
        \bm{S}_{n} \left( \Id_{n \times n} - \bm{A} \right) = \left( \Id_{n \times n} - \bm{A} \right) \bm{S}_{n} = \sum_{k=0}^{n} \bm{A}^k - \sum_{k=1}^{n+1} \bm{A}^k = \Id_{n \times n} - \bm{A}^{n+1},
    \]
    that is, $\| \left( \Id_{n \times n} - \bm{A} \right) \bm{S}_{n} - \Id_{n \times n} \| = \| \bm{A}^{n+1} \| \leq q^{n+1}$, which vanishes as $n \to \infty$. Thus $\bm{S} = \left( \Id_{n \times n} - \bm{A} \right)^{-1}$.
\end{proof}

Consider a matrix for which $\| \bm{A} \| < 2$, it follows that $\| \Id_{n \times n} - \bm{A} \| < 1$ meaning $\Id_{n \times n} - \left( \Id_{n \times n} - \bm{A} \right)$ is invertible and $\bm{A}^{-1} = \left( \Id_{n \times n} - \left( \Id_{n \times n} - \bm{A} \right) \right)^{-1} = \sum_{k=0}^{\infty} \left( \Id_{n \times n} - \bm{A} \right)^{k}$. Thinking back to equation \ref{eq: lin_sys_1} for any $\bm{x}_0 \in \RR^{n}$ we have
\begin{align*}
    \bm{x^{\star}} & = \bm{A}^{-1} \bm{b} = \bm{A}^{-1} \left( \bm{A} \bm{x^{\star}} - \bm{A} \bm{x_0} + \bm{A} \bm{x_0} \right) \\
                   & = \bm{x_0} + \bm{A}^{-1} \bm{r_0}                                                                           \\
                   & = \bm{x_0} + \sum_{k=0}^{\infty} \left( \Id_{n \times n} - \bm{A} \right)^k \bm{r_0}
\end{align*}
where $\bm{r_0} = \bm{A} \bm{x^{\star}} - \bm{A} \bm{x_0}$. A natural question that arises is that can we find a closed form solution of the above equation? To answer this question we need to enlist the help of the Cayley-Hamilton theorem.
\begin{thm}[Cayley-Hamilton] \label{theorem: cayley_amilton}
    Let $p_n \left( \lambda \right) = \sum_{i=0}^{n} c_i \lambda^{i}$ be the characteristic polynomial of the matrix $\bm{A} \in \CC^{n \times n}$, then $p_n \left( \bm{A} \right) = \bm{0}$. {\color{red} \textbf{THIS NEEDS A CITATION}}
\end{thm}
The Cayley-Hamilton theorem implies that
\begin{align*}
    \bm{0}      & = c_0 + c_1 \bm{A} + \ldots + c_{n-1} \bm{A}^{n-1} + c_{n} \bm{A}^{n}                      \\
    \bm{0}      & = \bm{A}^{-1} c_0 + c_1 + \ldots + c_{n-1} \bm{A}^{n-2} + c_{n} \bm{A}^{n-1}               \\
    \bm{A}^{-1} & = \alpha_1 \Id_{n \times n} + \ldots + \alpha_{n-1} \bm{A}^{n-2} + \alpha_{n} \bm{A}^{n-1}
\end{align*}
where $\alpha_i = -c_i / c_0$. The above demonstrates that $\bm{A}^{-1}$ can be represented as a matrix polynomial of degree $n-1$. This means that $\sum_{k=0}^{\infty} \left( \Id - \bm{A} \right)^k$ indeed possess a closed form solution namely
\begin{equation} \label{eq: x_ast_via_cayley}
    \bm{x^{\star}} = \bm{x}_0 + \bm{A}^{-1} \bm{r}_0 = \bm{x}_0 + \alpha_1 \bm{r}_0 + \ldots + \alpha_{n-1} \bm{A}^{n-2} \bm{r}_0 + \alpha_{n} \bm{A}^{n-1} \bm{r}_0.
\end{equation}
This also shows that $\bm{x^{\star}} \in \bm{x}_0 + \operatorname{l.s} \left\{ \bm{r}_0, \bm{A} \bm{r}_0, \bm{A}^2 \bm{r}_0, \ldots , \bm{A}^{n-1} \bm{r}_0 \right\}$. One idea for finding a solution to equation \ref{eq: lin_sys_1} is to use our routine for evaluting $\bm{A} \bm{v}$ to iteratively compute new basis elements for the space generated by $\left\{ \bm{r}_0, \bm{A} \bm{r}_0, \bm{A}^2 \bm{r}_0, \ldots , \bm{A}^{n-1} \bm{r}_0 \right\}$ and at each step carefully choosing a $\bm{x_k}$ such that $\bm{x_k}$ approaches $\bm{x^{\star}}$, in some form. The subspace constructed using this technique is so important that is has its own name.
\begin{defe}[Krylov Subspace] \label{defe: krylov_subspace}
    The Krylov Subspace of order $k$ generated by the matrix $\bm{A} \in \CC^{n \times n}$ and the vector $\bm{v} \in \CC^{n}$ is defined as
    \[
        \calK_{k} \left( \bm{A},\bm{v} \right) = \operatorname{l.s} \left\{ \bm{r}_0, \bm{A} \bm{r}_0, \bm{A}^2 \bm{r}_0, \ldots , \bm{A}^{k-1} \bm{r}_0 \right\}
    \]
    for $k \geq 1$ and $\calK_{0} \left( \bm{A},\bm{v} \right) = \left\{ \bm{0} \right\}$ \cite{TrefethenLloydN.LloydNicholas1997Nla/}*{page 245}.
\end{defe}
For the purposes of solving equation \ref{eq: lin_sys_1} it is of much interest to understand how $\calK_{k} \left( \bm{A},\bm{v} \right)$ grows for larger and larger $k$ since a solution for equation \ref{eq: lin_sys_1} will be present in a Krylov Subspace that cannot be grown any larger, according to equation \ref{eq: x_ast_via_cayley}. In other words, an exact solution can be constructed once we have extracted all the information from $\bm{A}$ through multiplication of $\bm{r_0}$. The following theorem provides information on how exactly the Krylov Subspace grows as $k$ increases.
\begin{thm} \label{theorem: grade_of_v}
    There is a positive called the grade of $\bm{v}$ with respect to $\bm{A}$, denoted $t_{\bm{v}, \bm{A}}$, where
    \[
        \operatorname{dim} \left( \calK_{k} \left( \bm{A} , \bm{v} \right) \right) = \left\{
        \begin{matrix}
            k,                  & k \leq t_{\bm{v}, \bm{A}} \\
            t_{\bm{v}, \bm{A}}, & k \geq t_{\bm{v}, \bm{A}}
        \end{matrix}
        \right.
    \]
\end{thm}
Theorem \ref{theorem: grade_of_v} essentially tells us for $k \leq t_{\bm{v}, \bm{A}}$ that $\bm{A}^k \bm{v}$ is linearly independent to $\bm{A}^i \bm{v}$ for $0 \leq i \leq k-1$ meaning $\left\{ \bm{v}, \bm{A} \bm{v}, \bm{A}^2 \bm{v}, \ldots , \bm{A}^{k-1} \bm{v} \right\}$ serves as a basis for $\calK_{k} \left( \bm{A},\bm{v} \right)$ and $\calK_{k-1} \left( \bm{A},\bm{v} \right) \subsetneq \calK_{k} \left( \bm{A},\bm{v} \right)$. Conversely, any new vectors formed beyond $t_{\bm{v}, \bm{A}}$ will be linearly independent meaning $\calK_{k} \left( \bm{A},\bm{v} \right) \subsetneq \calK_{k+1} \left( \bm{A},\bm{v} \right)$ for $k \geq t_{\bm{v}, \bm{A}}$. While $t_{\bm{v}, \bm{A}}$ obviously plays a central role in determining a suitable basis whose span contains $\bm{A}^{-1} \bm{b}$, its importance is made abundantly clear in the following corollary.
\begin{cor} \label{theorem: grade_as_min}
    \[
        t_{\bm{v}, \bm{A}} = \min \left\{k \mid \bm{A}^{-1} \bm{v} \in \calK_{k} \left( \bm{A},\bm{v} \right) \right\}
    \]
\end{cor}
\begin{proof}
    Recall from Cayley-Hamilton (theorem \ref{theorem: cayley_amilton}) that
    \[
        \bm{A}^{-1} \bm{v} = \sum_{i=0}^{n-1} \alpha_{i} \bm{A}^{i} \bm{v}
    \]
    But since $\calK_{k} \left( \bm{A},\bm{v} \right) = \calK_{k+1} \left( \bm{A},\bm{v} \right)$ for $k \geq t_{\bm{v}, \bm{A}}$
    \[
        \bm{A}^{-1} \bm{v} = \sum_{i=0}^{t-1} \beta_{i} \bm{A}^{i} \bm{v}
    \]
    meaing $\bm{A}^{-1} \bm{v} \in \calK_{k} \left( \bm{A},\bm{v} \right)$ for $k \geq t_{\bm{v}, \bm{A}}$. Suppose for the sake of contradiction that this also holds for $k = t_{\bm{v}, \bm{A}} - 1$, that is, $\bm{A}^{-1} \bm{v} = \sum_{i=0}^{t-2} \gamma_{i} \bm{A}^{i} \bm{v}$. However, this gives
    \[
        \bm{v} = \sum_{i=0}^{t-2} \gamma_{i} \bm{A}^{i+1} \bm{v} = \sum_{i=0}^{t-1} \gamma_{i-1} \bm{A}^{i} \bm{v}
    \]
    implying $\left\{ \bm{v}, \bm{A} \bm{v}, \bm{A}^2 \bm{v}, \ldots , \bm{A}^{t-1} \bm{v} \right\}$ are linearly dependent which means that $\operatorname{dim} \left( \calK_{k} \left( \bm{A} , \bm{v} \right) \right) < t$, which provides us with our contradiction.
\end{proof}
This allows us to make a much stronger statement on the where abouts of $\bm{x^{\star}}$ in relation to the Krylov Subspaces.
\begin{cor} \label{theorem: sol_in_krylov}
    For any $\bm{x_0}$ we have
    \[
        \bm{x^{\star}} \in \bm{x_0} + \calK_{t_{\bm{r_0}, \bm{A}}} \left( \bm{A},\bm{r_0} \right)
    \]
    where $\bm{r_0} = \bm{b} - \bm{A} \bm{x_0}$.
\end{cor}

\subsection{Gram-Schmidt Process and QR factorisations}\label{Section4.2}

Many areas of linear algebra involving studing the column space of matrices. The $QR$ factorisation provides us with a powerful tool to better understand the column space of a matrix as well as serving as an important factorisation mechanism for many numerical methods. Suppose that a matrix $\bm{A} = \left[ \bm{a}_1 , \bm{a}_2 , \ldots , \bm{a}_n \right] \in \KK^{n \times n}$ has full rank. The idea of a $QR$ factorisation is to find an alternative orthornormal basis for $\left( \bm{a}_i \right)_{i=1}^{n}$, say $\left( \bm{q}_i \right)_{i=1}^{n}$, and to somehow relate the original matrix $\bm{A}$ to a new matrix whose columns are $\left( \bm{q}_i \right)_{i=1}^{n}$. Consider the following procedure that allows us to find an orthornormal basis $\left( \bm{q}_i \right)_{i=1}^{n}$ for which $\operatorname{l.s} \left\{ \left( \bm{a}_i \right)_{i=1}^{n} \right\} = \operatorname{l.s} \left\{ \left( \bm{q}_i \right)_{i=1}^{n} \right\}$. First set $\bm{q}_1 = \frac{\bm{a}_1}{\| \bm{a}_i \|}$, clearly $\operatorname{l.s} \left\{ \bm{a}_1 \right\} = \operatorname{l.s} \left\{ \bm{q}_1 \right\}$. Next, construct a vector $\bm{q}_2' = \bm{a}_2 - r_{1,2} \cdot \bm{q}_1$ so that $\bm{q}_2' \perp \bm{q}_1$. This means
\begin{align*}
    0       & = \langle \bm{q}_1, \bm{q}_2' \rangle                                                   \\
    0       & = \langle \bm{q}_1, \bm{a}_2 - r_{1,2} \cdot \bm{q}_1 \rangle                           \\
    0       & = \langle \bm{q}_1, \bm{a}_2 \rangle - r_{1,2} \cdot \langle \bm{q}_1, \bm{q}_1 \rangle \\
    r_{1,2} & = \langle \bm{q}_1, \bm{a}_2 \rangle
\end{align*}
Since $\bm{q}_2'$ may not be a unit vector we set $\bm{q}_2 = \frac{\bm{q}_2'}{\| \bm{q}_2' \|}$ where $\operatorname{l.s} \left( \left\{ \bm{a}_1, \bm{a}_2 \right\} \right) = \operatorname{l.s} \left( \left\{ \bm{q}_1, \bm{q}_2 \right\} \right)$. Continuing the vector $\bm{q}_3'$ is constructed so that
\[
    \bm{q}_3' = \bm{a}_3 - \bm{r}_{1,3} \bm{q}_1 - \bm{r}_{2,3} \bm{q}_2
\]
are chosen so that $\bm{q}_3'$ is orthogonal to both $\bm{q}_2$ and $\bm{q}_1$. This amounts to setting $r_{1,3} = \langle \bm{q}_1, \bm{a}_3 \rangle$ and $r_{2,3} = \langle \bm{q}_2, \bm{a}_{3} \rangle$. Similarly, $\bm{q}_3'$ is normalized so that $\bm{q}_3 = \frac{\bm{q}_3'}{\| \bm{q}_3' \|}$ and $\operatorname{l.s} \left( \left\{ \bm{a}_1, \bm{a}_2, \bm{a}_3 \right\} \right) = \operatorname{l.s} \left( \left\{ \bm{q}_1, \bm{q}_2, \bm{q}_3 \right\} \right)$. Continuing in this fashion the $k^{th}$ vector in our orthornormal basis is computed as
\begin{equation}\label{eq: comp_orth_basis}
    \bm{q}_k = \frac{\bm{a}_k - \sum_{i=1}^{k-1} r_{i,k} \cdot \bm{q}_i}{r_{k,k}}
\end{equation}
where $r_{i,k} = \langle \bm{q}_i, \bm{a}_k \rangle$, $r_{k,k} = \| \bm{a}_k - \sum_{i=1}^{k-1} r_{i,k} \cdot \bm{q}_i \|$ and $\operatorname{l.s} \left( \left\{ \bm{a}_1, \bm{a}_2, \ldots , \bm{a}_k \right\} \right) = \operatorname{l.s} \left( \left\{ \bm{q}_1, \bm{q}_2, \ldots , \bm{q}_k \right\} \right)$. This procedure is famiously known as the Gram-Schmidt process \cite{BerezanskyMakarovich1996FaV1,TrefethenLloydN.LloydNicholas1997Nla/,DemmelJamesW1997Anla} and is summarized in the following algorithm.

    % https://tex.stackexchange.com/questions/463359/algorithm-inside-a-tcolorbox-how-to-put-a-label-to-the-algorithm-but-the-captio
    % http://cfrgtkky.blogspot.com/2018/12/algorithm-inside-tcolorbox-how-to-put.html
    % {\centering
    %     \begin{minipage}{.85\linewidth}
    %         \begin{tcolorbox}[colback=white!100,colframe=black!100]
    %             \begin{algorithm}[H]
    %                 \caption{Classical Gram-Schmidt}
    % \label{alg: Classical_Gram-Schmidt}
    % \SetAlgoLined
    % \DontPrintSemicolon
    % \SetKwInOut{Input}{input}\SetKwInOut{Output}{output}

    % \Input{A basis $\left( \bm{a}_i \right)_{i=1}^{n}$.}
    % \Output{An orthornormal basis $\left( \bm{q}_i \right)_{i=1}^{n}$ such that $\operatorname{l.s} \left\{ \left( \bm{a}_i \right)_{i=1}^{n} \right\} = \operatorname{l.s} \left\{ \left( \bm{q}_i \right)_{i=1}^{n} \right\}$}
    % \BlankLine
    % \For{$k = 1$ \KwTo $n$}{
    %     $\bm{q}_k' = \bm{a}_k$\;
    %     \For{$i = 1$ \KwTo $k-1$}{
    %         $r_{i,k} = \langle \bm{q}_i, \bm{a}_k \rangle$\;
    %         $\bm{q}_k' = \bm{q}_k' - r_{i,k} \bm{q}_i$\;
    %     }
    %     $r_{k,k} = \| \bm{q}_k' \|$\;
    %     $\bm{q}_k = \bm{q}_k' / r_{k,k}$\;
    % }
    % \Return{$\left( \bm{q}_i \right)_{i=1}^{n}$}
    % \BlankLine
    %             \end{algorithm}
    %         \end{tcolorbox}
    %     \end{minipage}
    %     \par
    % }


    {\centering
        \begin{minipage}{.85\linewidth}
            \begin{algorithm}[H]
                \caption{Classical Gram-Schmidt}
                \label{alg: Classical_Gram-Schmidt}
                \SetAlgoLined
                \DontPrintSemicolon
                \SetKwInOut{Input}{input}\SetKwInOut{Output}{output}

                \Input{A basis $\left( \bm{a}_i \right)_{i=1}^{n}$.}
                \Output{An orthornormal basis $\left( \bm{q}_i \right)_{i=1}^{n}$ such that $\operatorname{l.s} \left\{ \left( \bm{a}_i \right)_{i=1}^{n} \right\} = \operatorname{l.s} \left\{ \left( \bm{q}_i \right)_{i=1}^{n} \right\}$}
                \BlankLine
                \For{$k = 1$ \KwTo $n$}{
                    $\bm{q}_k' = \bm{a}_k$\;
                    \For{$i = 1$ \KwTo $k-1$}{
                        $r_{i,k} = \langle \bm{q}_i, \bm{a}_k \rangle$\;
                        $\bm{q}_k' = \bm{q}_k' - r_{i,k} \bm{q}_i$\;
                    }
                    $r_{k,k} = \| \bm{q}_k' \|$\;
                    $\bm{q}_k = \bm{q}_k' / r_{k,k}$\;
                }
                \Return{$\left( \bm{q}_i \right)_{i=1}^{n}$}
                \BlankLine
            \end{algorithm}
        \end{minipage}
        \par
    }

Relating the column space of $\bm{A}$ to the orthornormal basis $\left( \bm{q}_{i} \right)_{i=1}^{n}$ in a matrix form
\[
    \left[ \bm{a}_1 , \bm{a}_2 , \ldots \bm{a}_n \right] =
    \left[ \bm{q}_1 , \bm{q}_2 , \ldots \bm{q}_n \right]
    \begin{bmatrix}
        r_{1,1} & r_{1,2} & \cdots & r_{1,n} \\
                & r_{2,2} &        & \vdots  \\
                &         & \ddots & \vdots  \\
                &         &        & r_{n,n}
    \end{bmatrix}
\]
or more succinctly
\begin{equation}\label{eq: QR_factorisation}
    \bm{A} = \bm{Q} \bm{R}
\end{equation}
where $\bm{Q} = \left[ \bm{q}_1 , \bm{q}_2 , \ldots \bm{q}_n \right]$ and $\left( \bm{R} \right)_{i,j} = r_{i,j}$ for $i \leq j$ and $\left( \bm{R} \right)_{i,j} = 0$ for $i > j$. This is exactly the $QR$ factorisation for a full rank matrix. Note that $\operatorname{Range} \left( \bm{A} \right) = \operatorname{Range} \left( \bm{Q} \right)$. In general, any square matrix  $\bm{A} \in \KK^{m \times n}$ may be decomposed as $\bm{A} = \bm{Q} \bm{R}$ where $\bm{Q} \in \KK^{m \times m}$ is an orthogonal matrix and $\bm{R} \in \KK^{m \times n}$ is an upper triangular matrix. This is known as a full $QR$ factorisation. Since bottom $(m-n)$ rows of this $\bm{R}$ consists entirely of zeros, it is often useful to partition the full $QR$ factorisation in the following manner to shed vacuous entries
\[
    \bm{A} = \bm{Q} \bm{R} = \bm{Q}
    \begin{bmatrix}
        \hat{\bm{R}} \\
        \bm{0}_{(m-n) \times n}
    \end{bmatrix}
    =
    \begin{bmatrix}
        \hat{\bm{Q}} & \bm{Q}'
    \end{bmatrix}
    \begin{bmatrix}
        \hat{\bm{R}} \\
        \bm{0}_{(m-n) \times n}
    \end{bmatrix}
    = \hat{\bm{Q}} \hat{\bm{R}}.
\]
This alternate decomposition is called the reduced (or somtimes the thin) $QR$ factorization. We shall state the following two theorems on the $QR$ factorization are stated without proof.

\begin{thm} \label{theorem: QR_general_existence}
    Every $\bm{A} \in \KK^{m \times n}, \; (m \geq n)$ has a full $QR$ factorisation, hence also a reduced $QR$ factorisation.
    \cite{TrefethenLloydN.LloydNicholas1997Nla/}
\end{thm}

\begin{thm} \label{theorem: QR_full_rank_unique}
    Each $\bm{A} \in \KK^{m \times n}, \; (m \geq n)$ of full rank has a unique reduced $QR$ factorisation $\bm{A} = \hat{\bm{Q}} \hat{\bm{R}}$ with $r_{k,k} > 0$.
    \cite{TrefethenLloydN.LloydNicholas1997Nla/}
\end{thm}

In practice the classical Gram-Schmidt process described in algorithm \ref{alg: Classical_Gram-Schmidt} is rarely used as the procedure becomes numerically unstable if $\left( \bm{a}_i \right)_{i=1}^{n}$ are almost linearly dependent. Before looking for ways to resolve these numerical instabilities a quick recap of projectors has been devised. A square matrix $\bm{P}_{G}$ acting on a Hilbert space $H$ that sends $\bm{x} \in H$ to its projection onto a subspace $G$ is called the projector onto $G$. If $\left( \bm{q}_k \right)_{k=1}^{m}$ is an orthornormal basis in $G$ then
\[
    \bm{P}_{G} = \bm{Q} \bm{Q}^{\ast}
\]
where $\bm{Q} = \left[ \bm{q}_1 , \bm{q}_2 , \ldots \bm{q}_m, 0 , \ldots , 0 \right] \in \KK^{n \times n}$. A special class of projectors which isolates the components of a given vector onto a one dimensional subspace spanned by a single unit vector $\bm{q}$ called a rank one orthogonal projector, denoted as $\bm{P}_{q}$. Each $k$ in the classical Gram-Schmidt process $\bm{q}_k'$ using the following orthogonal projection
\begin{equation}\label{eq: classical_GS_proj}
    \bm{q}_k' = \bm{P}_{A_{k}^{\perp}} \bm{a}_k
\end{equation}
where $A_k = \operatorname{l.s} \left\{ \bm{a}_i \right\}_{i=1}^{k}$ and $\bm{P}_{A_{1}^{\perp}} = \Id$ for convenience. A modified version of the Gram-Schmidt process performs the same orthogonal projection broken up as $k-1$ orthogonal projections of rank $n-1$ as so
\begin{align*}
    \bm{q}_k' & = \bm{P}_{A_{k}^{\perp}} \bm{a}_k                                                                                                                                   \\
              & = \left( \Id - \bm{Q}_{k} \bm{Q}_{k}^{\ast} \right) \bm{a}_k                                                                                                        \\
              & = \left( \prod_{i=1}^{k-1} \left( \Id - \bm{q}_i \bm{q}_i^{\ast} \right) \right)\bm{a}_k                                                                            \\
              & = \left( \Id - \bm{q}_1 \bm{q}_1^{\ast} \right) \left( \Id - \bm{q}_1 \bm{q}_1^{\ast} \right) \cdots \left( \Id - \bm{q}_{k-1} \bm{q}_{k-1}^{\ast} \right) \bm{a}_k \\
              & = \bm{P}_{\bm{q}_{k}^{\perp}} \cdots \bm{P}_{\bm{q}_{1}^{\perp}} \bm{a}_k
\end{align*}

While its clear that $\bm{P}_{A_{k}^{\perp}} \bm{a} $ and $\bm{P}_{\bm{q}_{k}^{\perp}} \cdots \bm{P}_{\bm{q}_{1}^{\perp}} \bm{a}_k$ used for computing $\bm{q}_k'$ are algebraically, they differ arithmetically as the latter expression evaluates $\bm{q}_k'$ using the follow procedure

\begin{align*}
    \bm{q}_k^{(1)}             & = \bm{a}_k                                       \\
    \bm{q}_k^{(2)}             & = \bm{P}_{\bm{q}_{1}^{\perp}} \bm{q}_k^{(1)}     \\
    \bm{q}_k^{(3)}             & = \bm{P}_{\bm{q}_{2}^{\perp}} \bm{q}_k^{(2)}     \\
                               & \vdots                                           \\
    \bm{q}_k' = \bm{q}_k^{(k)} & = \bm{P}_{\bm{q}_{k-1}^{\perp}} \bm{q}_k^{(k-1)}
\end{align*}

Applying projections sequentially in this manner produces smaller numerical errors. The modified Gram-Schmidt process \cite{TrefethenLloydN.LloydNicholas1997Nla/,DemmelJamesW1997Anla} is summarized in the following algorithm.

    {\centering
        \begin{minipage}{.85\linewidth}
            \begin{algorithm}[H]
                \caption{Modified Gram-Schmidt}
                \label{alg: Modified_Gram-Schmidt}
                \SetAlgoLined
                \DontPrintSemicolon
                \SetKwInOut{Input}{input}\SetKwInOut{Output}{output}

                \Input{A basis $\left\{ \bm{a}_i \right\}_{i=1}^{n}$.}
                \Output{An orthornormal basis $\left\{ \bm{q}_i \right\}_{i=1}^{n}$ such that $\operatorname{l.s} \left\{ \bm{a}_i \right\}_{i=1}^{n} = \operatorname{l.s} \left\{  \bm{q}_i \right\}_{i=1}^{n}$}
                \BlankLine
                \For{$k = 1$ \KwTo $n$}{
                    $\bm{q}_k' = \bm{a}_k$\;
                }
                \For{$k = 1$ \KwTo $n$}{
                    $r_{k,k} = \| \bm{q}_k' \|$\;
                    $\bm{q}_k = \bm{q}_k' / r_{k,k}$\;
                    \For{$i = k+1$ \KwTo $n$}{
                        $r_{i,k} = \langle \bm{q}_k, \bm{q}_i' \rangle$\;
                        $\bm{q}_i = \bm{q}_i - r_{i,k} \bm{q}_i$\;
                    }
                }
                \Return{$\left\{ \bm{q}_i \right\}_{i=1}^{n}$}
                \BlankLine
            \end{algorithm}
        \end{minipage}
        \par
    }

\subsection{Arnoldi and Lanczos Algorithm}\label{Section4.3}

As a quick reminder, we are in search of an iterative process to solve the linear system $\bm{A} \bm{x}^{\star} = \bm{b}$ where no explicit form of $\bm{A}$ is available and we may only rely on a routine that computes $\bm{A} \bm{v}$ for any $\bm{v}$ to extract information on $\bm{A}$. In \Cref{Section4.1} it was shown that $\bm{x}^{\star} \in \calK_{t_{\bm{r}_0}, \bm{A}} \left( \bm{A}, \bm{r}_0 \right)$. With many iterative methods, computing an exact value for $\bm{x}^{\star}$ is out the question with the view that $t_{\bm{r}_0, \bm{A}}$ is impractically large. We must instead resort to approximating $\bm{x}^{\star}$ by $\bm{x}_k$ for which $\bm{x}^{k} \in \calK_{k} \left( \bm{A}, \bm{r}_0 \right)$ where $k \ll t_{\bm{r}_0}$. To find an appropriate value for $\bm{x}_k$, a good start would be to find a basis $\calK_{k} \left( \bm{A}, \bm{r}_0 \right)$. \Cref{defe: krylov_subspace} indicated that $\left\{ \bm{A}^{i-1} \bm{r}_0 \right\}_{i=1}^{k}$ serves as a basis for $\calK_{k} \left( \bm{A}, \bm{r}_0 \right)$. However, for numerical reasons this is a poor choice of basis since each consecutive term becomes closer and closer to being linearly dependent. From now on, for more convenient notation we shall set $n = t_{\bm{r}_0, \bm{A}}$ so that $\bm{x}^{\star} \in \calK_{n} \left( \bm{A}, \bm{r}_0 \right)$. To search for a more appropriate basis let $\bm{K} \in \KK^{n \times n}$ be the invertible matrix
\[
    \bm{K} = \left[ \bm{r}_0 , \bm{A} \bm{r}_0, \ldots , \bm{A}^{n-1} \bm{r}_0 \right].
\]
Since $\bm{K}$ is invertible we can compute $\bm{c} = - \bm{K}^{-1} \bm{A}^{n} \bm{r}_0$ so that
\begin{align*}
    \bm{A} \bm{K} & = \left[ \bm{A} \bm{r}_0, \bm{A}^{2} \bm{r}_0, \ldots , \bm{A}^{n} \bm{r}_0 \right]                     \\
    \bm{A} \bm{K} & = \bm{K} \cdot \left[ \bm{e}_2, \bm{e}_3, \ldots , \bm{e}_n, - \bm{c}  \right] \triangleq \bm{K} \bm{C}
\end{align*}
or, in other terms
\[
    \bm{K}^{-1} \bm{A} \bm{K} = \bm{C} =
    \begin{bmatrix}
        0      & 0      & \cdots & 0      & -c_1   \\
        1      & 0      & \cdots & 0      & -c_2   \\
        0      & 1      & \cdots & 0      & \vdots \\
        \vdots & \vdots & \cdots & \vdots & \vdots \\
        0      & 0      & \cdots & 1      & -c_n
    \end{bmatrix}.
\]
Note here that $\bm{C}$ is upper Hessenberg. While this form is simple, it is of little practical use since the matrix $\bm{K}$ is very likely to be ill-conditioned. To remedy this we can replace $\bm{K}$ with an orthogonal matrix which spans the same space. These are exactly the properties that the $\bm{V}$ matrix offers in the $QR$ factorisation of $\bm{K}$. With this in mind let $\bm{K} = \bm{V} \bm{R}$ be the full $QR$ factorisation of $\bm{K}$. Then
\begin{align*}
    \bm{A} \bm{V} \bm{R} & = \bm{A} \bm{K}                    \\
    \bm{A} \bm{V}        & = \bm{A} \bm{K} \bm{R}^{-1}        \\
    \bm{A} \bm{V}        & = \bm{K} \bm{C} \bm{R}^{-1}        \\
    \bm{A} \bm{V}        & = \bm{V} \bm{R} \bm{C} \bm{R}^{-1} \\
    \bm{A} \bm{V}        & \triangleq \bm{V} \bm{H}.
\end{align*}
Since $\bm{R}$ and $\bm{R}^{-1}$ and both upper triangular and $\bm{C}$ is upper Hessenberg, $\bm{H}$ is also upper Hessenberg. This form provides us with a $\bm{V}$ such that the range of $\bm{V}$ is $\calK_{n} \left( \bm{A}, \bm{r}_0 \right)$ and
\begin{equation}\label{eq: QTAQ_eq_H}
    \bm{V}^{\intercal} \bm{A} \bm{V} = \bm{H}.
\end{equation}
Again, in practice, it may be very difficult to compute this entire expression forcing us to search for approximative alternatives. Consider \Cref{eq: QTAQ_eq_H} for which the only first $k$ columns of $\bm{V}$ have been computed. Let $\bm{V}_k = \left[ \bm{v}_1 , \bm{v}_2 , \ldots , \bm{v}_k \right]$ and $\bm{V}_u = \left[ \bm{v}_{k+1} , \bm{v}_{k+2} , \ldots , \bm{v}_{n} \right]$. Then
\begin{align*}
    \bm{V}^{\intercal} \bm{A} \bm{V}                                                         & = \bm{H} \\
    \left[ \bm{V}_k , \bm{V}_u \right]^{\intercal} \bm{A} \left[ \bm{V}_k , \bm{V}_u \right] & =
    \begin{bmatrix}
        \bm{H}_k     & \bm{H}_{u,k} \\
        \bm{H}_{k,u} & \bm{H}_{u}
    \end{bmatrix}                                                                           \\
    \begin{bmatrix}
        \bm{V}_{k}^{\intercal} \bm{A} \bm{V}_{k} & \bm{V}_{k}^{\intercal} \bm{A} \bm{V}_{u} \\
        \bm{V}_{u}^{\intercal} \bm{A} \bm{V}_{k} & \bm{V}_{u}^{\intercal} \bm{A} \bm{V}_{u}
    \end{bmatrix}
                                                                                             & =
    \begin{bmatrix}
        \bm{H}_k     & \bm{H}_{u,k} \\
        \bm{H}_{k,u} & \bm{H}_{u}
    \end{bmatrix}
\end{align*}
where $\bm{H}_k , \bm{H}_{u,k}, \bm{H}_{k,u}$ and $\bm{H}_u$ are the relevant sub matrices. This provides us with the equality
\begin{equation}\label{eq: QTkAQk_eq_Hk}
    \bm{V}_{k}^{\intercal} \bm{A} \bm{V}_{k} = \bm{H}_k
\end{equation}
noting that $\bm{H}_{k}$ is upper Hessenberg for the same reason that $\bm{H}$ is. We know that when $n = t_{\bm{r}_0, \bm{A}}$ we can find a $\bm{V} \in \KK^{n \times n}$ and $\bm{H} \in \KK^{n \times n}$ that satisfies $\bm{A} \bm{V} = \bm{V} \bm{H}$. However, in general, we may not be so fortunate in finding a $\bm{V}_{k} \in \KK^{n \times k}$ and $\bm{H}_{k} \in \KK^{n \times k}$ to satisfy $\bm{A} \bm{V}_{k} = \bm{V}_{k} \bm{H}_k$ for any $k < n$. Instead we can adjust this equality by adding an error $\bm{E}_k \in \KK^{n \times k}$ to force an equality. Our expression now becomes
\begin{equation}\label{eq: QTkAQk_eq_HkEk}
    \bm{V}_{k}^{\intercal} \bm{A} \bm{V}_{k} = \bm{H}_k + \bm{E}_k.
\end{equation}
A judicious choice of $\bm{E}_k$ must be made to also retain equality in \Cref{eq: QTkAQk_eq_Hk}, meaning $\bm{V}_{k}^{\intercal} \bm{E}_k = \bm{0}$. Since $\left\{ \bm{v}_i \right\}_{i=1}^{k}$ forms an orthornormal basis for $\calK_{n} \left( \bm{A}, \bm{r}_0 \right)$, consider the following choice of $\bm{E}_k$,
\[
    \bm{E}_k = \bm{v}_{k+1} \bm{h}_{k}^{\intercal}
\]
where $\bm{h}_k$ is any vector in $\KK^{k}$. Notice that
\[
    \bm{V}_{k}^{\intercal} \bm{E} = \bm{V}^{\intercal} \left( \bm{v}_{k+1} \bm{h}_k \right) = \left( \bm{V}^{\intercal} \bm{v}_{k+1} \right) \bm{h}_{k}^{\intercal} = \bm{0}.
\]
Since this holds for any $\bm{h}_k \in \KK^{k}$, to preserve sparsity and to keep this form as simple as possible we can set $\bm{h}_k = \left[ 0,0, \ldots , h_{k+1,k} \right]^{\intercal}$. This means $\bm{A} \bm{V}_k$ can be written as
\begin{equation}\label{eq: QTkAQk_eq_Hk_p_qkhk}
    \bm{A} \bm{V}_k =  \bm{V}_k \bm{H}_k + \bm{v}_{k+1} \bm{h}_{k}^{\intercal}
\end{equation}
where
\[
    \bm{V}_k \bm{H}_k =
    \left[ \bm{v}_1 , \bm{v}_2 , \ldots , \bm{v}_k \right]
    \begin{bmatrix}
        h_{1,1} & \cdots & \cdots & \cdots    & h_{1,k} \\
        h_{2,1} & \cdots & \cdots & \cdots    & \vdots  \\
        0       & \ddots & \ddots & \ddots    & \vdots  \\
        \vdots  & \ddots & \ddots & \ddots    & \vdots  \\
        0       & \cdots & 0      & h_{k,k-1} & h_{k,k} \\
        0       & \cdots & 0      & 0         & 0
    \end{bmatrix}.
\]
Equating the $j^{th}$ columns of \Cref{eq: QTkAQk_eq_Hk_p_qkhk} yields
\[
    \bm{A} \bm{v}_j = \sum_{i=1}^{j+1} h_{i,j} \bm{v}_{i}.
\]
Again since $\left\{ \bm{v}_i \right\}_{i=1}^{n}$ form an orthornormal basis, multiplying both sides by $\bm{v}_m$ for $1 \leq m \leq j$ gives
\[
    \bm{v}_m^{\intercal} \bm{A} \bm{v}_j = \sum_{i=1}^{j+1} h_{i,j} \bm{v}_m^{\intercal} \bm{v}_{i} = h_{m,j}
\]
and so
\begin{equation}\label{eq: arn_eq_1}
    h_{j+1,j} \bm{v}_{j+1} = \bm{A} \bm{v}_j - \sum_{i=1}^{j} h_{i,j} \bm{v}_{i}.
\end{equation}
From \Cref{eq: arn_eq_1} we find that $\bm{v}_{j+1}$ can be computed using a recurrance involving its previous Krylov factors. This bears a striking resemblance to \Cref{eq: comp_orth_basis} having a virtually identical setup to computing an orthornormal basis using the modified Gram-Schmidt process (\Cref{alg: Modified_Gram-Schmidt}). As such, values for $\bm{v}_{j+1}$ and $h_{j+1,j}$ can be evaluted using a procedure very similar to the modified Gram-Schmidt process better known as the Arnoldi Algorithm \cite{TrefethenLloydN.LloydNicholas1997Nla/,DemmelJamesW1997Anla}, presented in \Cref{alg: Arnoldi_Algorithm}.

{\centering
\begin{minipage}{.85\linewidth}
    \begin{algorithm}[H]
        \caption{Arnoldi Algorithm}
        \label{alg: Arnoldi_Algorithm}
        \SetAlgoLined
        \DontPrintSemicolon
        \SetKwInOut{Input}{input}\SetKwInOut{Output}{output}

        \Input{$\bm{A}, \bm{r}_0$ and $k$, the number of columns of $\bm{V}$ to compute.}
        \Output{$\bm{V}_k , \bm{H}_k$.}
        \BlankLine
        $\bm{v}_1 = \bm{r}_0 / \| \bm{r}_0 \|$\;
        \For{$j = 1$ \KwTo $k$}{
            $\bm{z} = \bm{A} \bm{v}_j$\;
            \For{$i = 1$ \KwTo $j$}{
                $h_{i,j} = \langle \bm{v}_{i} , \bm{z} \rangle$\;
                $\bm{z} = \bm{z} - h_{i,j} \bm{v}_{i}$\;
            }
            $h_{j+1,j} = \| \bm{z} \|$\;
            \If{$h_{j+1,j} = 0$}{
                \Return{$\bm{V}_k , \bm{H}_k$}
            }
            $\bm{v}_{j+1} = \bm{z} / h_{j+1,j}$\;
        }
        \Return{$\bm{V}_k , \bm{H}_k$}
        \BlankLine
    \end{algorithm}
\end{minipage}
\par
}

When $\bm{A}$ is symmertic then $\bm{H} = \bm{T}$ becomes a tridiagonal matrix, simplifying a large amount of the Arnoldi algorithm since the matrix elements from $\bm{T}$ can be written as
\[
    \bm{T} =
    \begin{bmatrix}
        \alpha_1 & \beta_1 &        &             &             \\
        \beta_1  & \ddots  & \ddots &             &             \\
                 & \ddots  & \ddots & \ddots      &             \\
                 &         & \ddots & \ddots      & \beta_{n-1} \\
                 &         &        & \beta_{n-1} & \alpha_{n}
    \end{bmatrix}.
\]
As before, equating the $j^{th}$ columns of $\bm{A} \bm{V} = \bm{V} \bm{T}$ yields
\begin{equation}\label{eq: lancz_orth_basis}
    \bm{A} \bm{v}_{j} = \beta_{j-1} \bm{v}_{j-1} + \alpha_{j} \bm{v}_j + \beta_j \bm{v}_{j+1}.
\end{equation}
Again since $\left\{ \bm{v}_{i} \right\}_{i=1}^{n}$ form an orthornormal basis, multiplying both sides of \Cref{eq: lancz_orth_basis} by $\bm{v}_j$ gives $\bm{v}_j \bm{A} \bm{v}_j = \alpha_j$. This simplified version of the Arnoldi algorithm used for computing $\left\{ \bm{v}_{i} \right\}_{i=1}^{n}$ and $\bm{T}$ for symmetric matrices is better known as the Lanczos algorithm \cite{DemmelJamesW1997Anla}, presented more explicitly in \Cref{alg: Lanczos_Algorithm}.

{\centering
\begin{minipage}{.85\linewidth}
    \begin{algorithm}[H]
        \caption{Lanczos Algorithm}
        \label{alg: Lanczos_Algorithm}
        \SetAlgoLined
        \DontPrintSemicolon
        \SetKwInOut{Input}{input}\SetKwInOut{Output}{output}

        \Input{$\bm{A}, \bm{r}_0$ and $k$, the number of columns of $\bm{V}$ to compute.}
        \Output{$\bm{V}_k , \bm{T}_k$.}
        \BlankLine
        $\bm{v}_1 = \bm{r}_0 / \| \bm{r}_0 \|$, $\beta_0 = 0$, $\bm{v}_0 = 0$\;
        \For{$j = 1$ \KwTo $k$}{
            $\bm{z} = \bm{A} \bm{v}_j$\;
            $\alpha_j = \langle \bm{v}_{j}, \bm{z} \rangle$\;
            $\bm{z} = \bm{z} - \alpha_j \bm{v}_{j} - \beta_{j-1} \bm{v}_{j-1}$\;
            $\beta_j = \| z \|$\;
            \If{$\beta_{j} = 0$}{
                \Return{$\bm{V}_k , \bm{T}_k$}
            }
            $\bm{v}_{j+1} = \bm{z} / \beta_{j}$\;
        }
        \Return{$\bm{V}_k , \bm{T}_k$}
        \BlankLine
    \end{algorithm}
\end{minipage}
\par
}

For the Lanczos algorithm, \Cref{eq: QTkAQk_eq_Hk_p_qkhk} can be re-written in the a more compact form as
\begin{equation}\label{eq: AVk_eq_VkTk1k}
    \bm{A} \bm{V}_{k} \triangleq \bm{V}_{k} \bm{T}_{k+1,k}
\end{equation}
where $\bm{T}_{k+1,k} = \bm{T}_{k} + \bm{v}_{k+1} \bm{t}_{k}^{\intercal}$.

\subsection{Optimality Conditions}\label{Section4.4}

So far we have shown that $\bm{x}^{\star} \in \calK_{t_{\bm{r}_0}, \bm{A}} \left( \bm{A}, \bm{r}_0 \right)$ where $n = t_{\bm{r}_0}$ is the grade of $\bm{r}_0$ with respect to $\bm{A}$. Moreover from section \ref{Section4.3} we found ways to construct a basis for $\calK_{t_{\bm{r}_0}, \bm{A}} \left( \bm{A}, \bm{r}_0 \right)$ allowing us to generate vectors with these affine spaces, namely the Arnoldi algorithm (algorithm \ref{alg: Arnoldi_Algorithm}) and Lanczos algorithm (algorithm \ref{alg: Lanczos_Algorithm}) for non-symmertic and symmertic systems respectively. From now on $\calK_{t_{\bm{r}_0}, \bm{A}} \left( \bm{A}, \bm{r}_0 \right)$ will be abbreviated to $\calK_{t_{\bm{r}_0}, \bm{A}}$ when the context is clear. The question still remains however, how should one choose an $\bm{x}_k$ that best approximates $\bm{x}^{\ast}$ satisfying equation \ref{eq: lin_sys_1}? Here are a few of the most well known methods for selecting a suitable $\bm{x}_k$.

\begin{enumerate}

    \item Select an $\bm{x}_k \in \bm{x}_0 + \calK_k$ which minimizes $\| \bm{x}_k - \bm{x}^{\ast} \|_2$. While this method seems like the most intuitive and natural way to select $\bm{x}_k$, it is unfortunately of no practical use since there is not enough information in the Krylov subspace to find an $\bm{x}_k$ which matches this profile.

    \item Select an $\bm{x}_k \in \bm{x}_0 + \calK_k$ which minimizes $\| \bm{r}_k \|_2$ (recall this is the residual of $\bm{x}_k$, that is, $\bm{r}_k = \bm{b} - \bm{A} \bm{x}_k$). This method is possible to implement. Two well known algorithms stem from this class of methods, namely MINRES (minimum residual) and GMRES (general minimum residual) which solve linear systems for symmetric and non-symmertic $\bm{A}$ respectively.

    \item When $\bm{A}$ is a positive definite matrix it defines a norm $\| \bm{r} \|_{\bm{A}} = \left( \bm{r}^{\intercal} \bm{A} \bm{r} \right)^{\frac{1}{2}}$, called the energy norm. Select an $\bm{x}_k \in \bm{x}_0 + \calK_k$ which minimizes $\| \bm{r} \|_{\bm{A}^{-1}}$ which is equivalent to minimizing $\| \bm{x}_k - \bm{x} \|_{A}$. This technique is known as the CG (conjugate gradient) algorithm.

    \item Select an $\bm{x}_k \in \bm{x}_0 + \calK_k$ for which $\bm{r}_{k} \perp \calW_k$ where $\calW_k$ is some $k$-dimensional subspace. Two well known algorithms that belong to this family of methods are SYMMLQ (Symmetric LQ Method) and a variant of GMRES used for solving symmetric and non-symmetric methods respectively.

\end{enumerate}

Interestingly, when $\bm{A}$ is symmetric positive definite and $\calW_k = \calK_k$ the last two selection methods are equivalent. This is stated more precisely in theorem \ref{theorem: 3_4_method_eq} without proof.

\begin{thm} \label{theorem: 3_4_method_eq}
    In the context of the above selection method, if $\bm{A} \succ \bm{0}$ and $\calW_k = \calK_k$ in method (4) then it produces the same $\bm{x}_k$ in method (3) \cite{DemmelJamesW1997Anla}.
\end{thm}

In fact the very last method can be used to bring together a number of different analytical aspects and unify them in a general framework known as projection methods. Selecting an $\bm{x}_k$ from our Krylov subspace allows $k$ degrees of freedom meaning $k$ constraints must be used to determine a unique $\bm{x}_k$ for selection. As seen in method (4) already, typically orthogonality constraints are imposed on the residual $\bm{r}_k$. Specifically we would like to find a $\bm{x}_k \in \bm{x}_0 + \calK_k$ where $\bm{r}_k \perp \calW_k$. This is sometimes referred to as the Petrov-Galerkin (or just Galerkin) conditions. Projection methods for which $\calW_k = \calK_k$ are known as orthogonal projections while methods for which $\calW_k = \bm{A} \calK_k$ are known as oblique projections. If we set $\bm{x}_k = \bm{x}_0 + \bm{z}_k$ for some $\bm{z}_k \in \calK_k$ then the Petrov-Galerkin conditions imply $\bm{r}_0 - \bm{A} \bm{z}_k \perp \calW_k$, or alternatively $\langle \bm{r}_0 - \bm{A} \bm{z}_k , \bm{w} \rangle = 0$ for every $\bm{w} \in \calW_k$. To impose these conditions it will help to have an appropriate basis for $\calK$ and $\calW$. Suppose we have access to such a basis where $\left\{ \bm{v}_i \right\}_{i=1}^{k}$ and $\left\{ \bm{w}_i \right\}_{i=1}^{k}$ are basis elements for $\calK$ and $\calW$ respectively. Let
\begin{align*}
    \bm{K}_k & \triangleq \left[ \bm{v}_1 , \bm{v}_2 , \ldots , \bm{v}_k \right] \in \KK^{n \times k} \\
    \bm{W}_k & \triangleq \left[ \bm{w}_1 , \bm{w}_2 , \ldots , \bm{w}_k \right] \in \KK^{n \times k}
\end{align*}
then the Petrov-Galerkin conditions can be imposed as follows
\begin{align*}
    \bm{K}_k \bm{y}_k                                                       & = \bm{z}_k , \quad \text{for some} \; \bm{y}_k \in \KK^k \\
    \bm{W}_k^{\intercal} \left( \bm{r}_0 - \bm{A} \bm{K}_k \bm{y}_k \right) & = \bm{0}.
\end{align*}
Moreover if $\bm{W}_k^{\intercal} \bm{A} \bm{K}_k$ is invertible then $\bm{x}_k$ can be expressed as
\begin{equation} \label{eq: expr_x_Petrov_Galerkin_1}
    \bm{x}_k = \bm{x}_0 + \bm{K}_k \left( \bm{W}_k^{\intercal} \bm{A} \bm{K}_k \right)^{-1} \bm{W}_k \bm{r}_0.
\end{equation}
This justifies a general form of the projection method algorithm presented in algorithm \ref{alg: General_Projection}.

{\centering
\begin{minipage}{.85\linewidth}
    \begin{algorithm}[H]
        \caption{General Projection Method}
        \label{alg: General_Projection}
        \SetAlgoLined
        \DontPrintSemicolon
        \SetKwInOut{Input}{input}\SetKwInOut{Output}{output}

        \Output{An approximation of $\bm{x}^{\ast}$, $\bm{x}_k$.}
        \BlankLine
        \For{$k = 1 , \ldots $ \Until convergence}{
        Select $\calK_k$ and $\calW_k$\;
        Form $\bm{K}_k$ and $\bm{W}_k$\;
        Solve $\left( \bm{W}_k^{\intercal} \bm{A} \bm{K}_k \right) \bm{y}_k = \bm{W}_k^{\intercal} \bm{r}_0$\;
        $\bm{x}_k = \bm{x}_0 + \bm{K}_k \bm{y}_k$\;
        }
        \Return{$\bm{x}_k$}
        \BlankLine
    \end{algorithm}
\end{minipage}
\par
}

\subsection{Conjugate Gradient Algorithm}\label{Section4.5}

From \Cref{Section4.4} that the Petrov-Galerkin conditions for the CG algorithm used an orthogonal projection and the matrix $\bm{A}$ was assumed to be positive definite. To derive the CG algorithm we can start be using some machinery that the Lanczos algorithm provides us with. Recall, the Lanczos algorithm produces the form $\bm{A}\bm{V}_{k} = \bm{V}_{k} \bm{T}_k + \bm{v}_{k+1} \bm{t}_{k}^{\intercal}$ where $\bm{t}_{k} \triangleq \left[ 0,0, \ldots , 0, \beta_k \right]^{\intercal} \in \KK^k$ and the columns of $\bm{V}_{k}$ span $\calK_k$. Recall that $\bm{x}_k$ can be expressed as $\bm{x}_k = \bm{x}_0 + \bm{K}_k \left( \bm{W}_k^{\intercal} \bm{A} \bm{K}_k \right)^{-1} \bm{W}_k \bm{r}_0$ (\Cref{eq: expr_x_Petrov_Galerkin_1}) when $\bm{W}_k^{\intercal} \bm{A} \bm{K}_k$ is invertible. For the CG algorithm $\calK = \calW$ and $\bm{A} \succ \bm{0}$. Under these conditions we can easily show that $\bm{W}_k^{\intercal} \bm{A} \bm{K}_k$ is indeed invertible. Thus the approximate vector can be expressed as $\bm{x}_k = \bm{x}_0 + \bm{z}_k$ where $\bm{z}_k \in \calK_k$. In terms of the Petrov-Galerkin conditions this means that $\bm{z}_k$ must satisfy $\bm{r}_0 - \bm{A} \bm{z}_k \perp \calW_k$. Furthermore since $\calK_k = \operatorname{Range} \left( \bm{V}_{k} \right)$ where $\bm{V}_{k}$ has full column rank then $\bm{z}_k$ can be represented as $\bm{z}_k = \bm{V}_{k} \bm{y}$ for a unique $\bm{y} \in \RR^k$ so that
\begin{equation} \label{eq: x_eq_Qky}
    \bm{x}_k = \bm{x}_0 + \bm{V}_{k} \bm{y}.
\end{equation}
Coupling this with the Petrov-Galerkin conditions ensures
\begin{align} \label{eq: Tky_eq_normr0e1}
    \bm{V}_{k}^{\intercal} \left( \bm{r}_0 - \bm{A} \bm{V}_{k} \bm{y} \right) & = \bm{0}                        \nonumber   \\
    \bm{V}_{k}^{\intercal} \bm{A} \bm{V}_{k} \bm{y}                           & = \bm{V}_{k}^{\intercal} \bm{r}_0 \nonumber \\
    \bm{T}_k \bm{y}                                                           & = \| \bm{r}_0 \| \bm{e}_1.
\end{align}
In the CG algorithm $\bm{x}_{k+1}$ is computed as the recurrance of the following three sets of vectors
\begin{enumerate}
    \item The approximate solutions $\bm{x}_{k}$
    \item The residual vectors $\bm{r}_{k}$
    \item The conjugate gradient vectors $\bm{p}_k$
\end{enumerate}
The conjugate gradient vectors include the word 'gradient' since the attempt to find the direction of steepest descent that minimizes $\| \bm{r}_{k} \|_{\bm{A}^{-1}}$. It also includes the word 'conjugate' since $\langle \bm{p}_k, \bm{A} \bm{p}_j \rangle = 0$ for $i \neq j$, that is, vectors $\bm{p}_i$ and $\bm{p}_j$ are mutally $A$-conjugate (shown in \Cref{lemma: Pk_cols_A_conj}).

Since $\bm{A}$ is symmetric positive definite then so is $\bm{T}_k  = \bm{V}_{k} \bm{A} \bm{V}_{k}$. We can take the Cholesky decomposition of $\bm{T}_k$ to get
\begin{equation} \label{eq: Tk_Cholesky}
    \bm{T}_k = \bm{L}_k \bm{D}_k \bm{L}_k^{\intercal}
\end{equation}
where $\bm{L}_k$ is a unit lower bidiagonal matrix and $\bm{D}_k$ is diagonal written as
\[
    \bm{L}_k =
    \begin{bmatrix}
        1   &        &         &   \\
        l_1 & \ddots &         &   \\
            & \ddots & \ddots  &   \\
            &        & l_{k-1} & 1
    \end{bmatrix}, \quad
    \bm{D}_k =
    \begin{bmatrix}
        d_1 &     &        &     \\
            & d_2 &        &     \\
            &     & \ddots &     \\
            &     &        & d_k
    \end{bmatrix}.
\]
Combining equations \Cref{eq: x_eq_Qky}, \Cref{eq: Tky_eq_normr0e1} and \Cref{eq: Tk_Cholesky}
\begin{align*}
    \bm{x}_k & = \bm{x}_0 + \bm{V}_{k} \bm{y}                                                                                                  \\
    \bm{x}_k & = \bm{x}_0 + \| \bm{r}_0 \| \bm{V}_{k} \bm{T}_k^{-1} \bm{e}_1                                                                   \\
    \bm{x}_k & = \bm{x}_0 + \| \bm{r}_0 \| \bm{V}_{k} \left( \bm{L}_k \bm{D}_k \bm{L}_k^{\intercal} \right)^{-1} \bm{e}_1                      \\
    \bm{x}_k & = \bm{x}_0 + \left( \bm{V}_{k} \bm{L}_k^{-\intercal} \right) \left( \| \bm{r}_0 \| \bm{D}_k^{-1} \bm{L}_k^{-1} \bm{e}_1 \right) \\
    \bm{x}_k & \triangleq \bm{x}_0 + \tilde{\bm{P}}_k \tilde{\bm{y}}_k
\end{align*}
where $\tilde{\bm{P}}_k = \bm{V}_{k} \bm{L}_k^{-\intercal}$ and $\tilde{\bm{y}}_k = \| \bm{r}_0 \| \bm{D}_k^{-1} \bm{L}_k^{-1} \bm{e}_1$. The matrix $\tilde{\bm{P}}_k$ can be written as
$\tilde{\bm{P}}_k = \left[ \tilde{\bm{p}}_1 , \tilde{\bm{p}}_2 , \ldots , \tilde{\bm{p}}_k \right]$. \Cref{lemma: Pk_cols_A_conj} shows that the columns of $\tilde{\bm{P}}_k$ are $A$-conjugate.

\begin{lem} \label{lemma: Pk_cols_A_conj}
    The columns of $\tilde{\bm{P}}_k$ are $A$-conjugate, in otherwords $\tilde{\bm{P}}_k^{\intercal} \bm{A} \tilde{\bm{P}}_k$ is diagonal.
\end{lem}

\begin{proof}
    We compute
    \begin{align*}
        \tilde{\bm{P}}_k^{\intercal} \bm{A} \tilde{\bm{P}}_k
         & = \left( \bm{V}_{k} \bm{L}_k^{-\intercal} \right)^{\intercal} \bm{A} \left( \bm{V}_{k} \bm{L}_k^{-\intercal} \right)      \\
         & = \bm{L}_k^{-1} \left( \bm{V}_{k}^{\intercal} \bm{A} \bm{V}_{k} \right) \bm{L}_k^{-\intercal}                             \\
         & = \bm{L}_k^{-1} \left( \bm{T}_k \right) \bm{L}_k^{-\intercal}                                                             \\
         & = \bm{L}_k^{-1} \left( \bm{L}_k \bm{D}_k \bm{L}_k^{\intercal} \right) \bm{L}_k^{-\intercal} \tag*{\Cref{eq: Tk_Cholesky}} \\
         & = \bm{D}_k
    \end{align*}
    as wanted.
\end{proof}

Since $\bm{L}_k$ is a lower bidiagonal, setting $\bm{a} \triangleq l_{k-1} \bm{e}_{k-1}$, it can be written in the form
\[
    \bm{L}_k =
    \begin{bmatrix}
        \bm{L}_{k-1}       & \bm{0} \\
        \bm{a}^{\intercal} & 1
    \end{bmatrix}
\]
meaning
\[
    \bm{L}_k^{-1} =
    \begin{bmatrix}
        \bm{L}_{k-1}^{-1} & \bm{0} \\
        \star             & 1
    \end{bmatrix}
\]

where the elements of $\star$ are of no importance. With this a recurrance for the columns of $\tilde{\bm{P}}_k$ can now be derived in terms of $\bm{y}_k$. To start we can show that the first $k-1$ entries of $\tilde{\bm{y}}_{k}$ shares the first $k-1$ entires with $\tilde{\bm{y}}_{k-1}$ and that $\tilde{\bm{P}}_k$ and $\tilde{\bm{P}}_{k-1}$ share the same first $k-1$ columns. We can begin by computing a recurrance for $\tilde{\bm{y}}_{k}$ as follows
\begin{align*}
    \tilde{\bm{y}}_{k} & = \| \bm{r}_0 \| \bm{D}_k^{-1} \bm{L}_k^{-1} \bm{e}_1^k \\
                       & = \| \bm{r}_0 \|
    \begin{bmatrix}
        \bm{D}_{k-1}^{-1} & \bm{0}   \\
        \bm{0}            & d_k^{-1}
    \end{bmatrix}
    \begin{bmatrix}
        \bm{L}_{k-1}^{-1} & \bm{0} \\
        \star             & 1
    \end{bmatrix}
    \bm{e}_1^k                                                                   \\
                       & = \| \bm{r}_0 \|
    \begin{bmatrix}
        \bm{D}_{k-1}^{-1} \bm{L}_{k-1}^{-1} & \bm{0}   \\
        \star                               & d_k^{-1}
    \end{bmatrix}
    \begin{bmatrix}
        \bm{e}_1^k \\
        0
    \end{bmatrix}                                                   \\
                       & =
    \begin{bmatrix}
        \tilde{\bm{y}}_{k-1} \\
        \eta_k
    \end{bmatrix}
\end{align*}

where $\bm{e}_i^k$ is the $i^{th}$ unit vector with $k$ dimensions. To get a recurrance for the columns of $\tilde{\bm{P}}_{k-1} = \left[ \tilde{\bm{p}}_1 , \tilde{\bm{p}}_2 , \ldots , \tilde{\bm{p}}_k \right]$ since $\bm{L}_{k-1}^{\intercal}$ is upper triangular then so is $\bm{L}_{k-1}^{-\intercal}$, thus forming the leading $(k-1) \times (k-1)$ submatrix of $\bm{L}_{k}^{-\intercal}$. In effect, $\tilde{\bm{P}}_{k-1}$ is identical to the leading $k-1$ columns of

\[
    \tilde{\bm{P}}_{k} = \bm{V}_{k} \bm{L}_k^{-\intercal} = \left[ \bm{V}_{k-1} , \bm{v}_k \right]
    \begin{bmatrix}
        \bm{L}_{k-1}^{-1} & \bm{0} \\
        \star             & 1
    \end{bmatrix}
    = \left[ \bm{V}_{k-1} \bm{L}_{k-1}^{-1} , \tilde{\bm{p}}_{k} \right]
    = \left[ \tilde{\bm{P}}_{k-1} , \tilde{\bm{p}}_{k} \right].
\]
Moreover rearranging $\tilde{\bm{P}}_{k} = \bm{V}_{k} \bm{L}_k^{-\intercal}$ we get $\tilde{\bm{P}}_{k} \bm{L}_k^{\intercal} = \bm{V}_{k}$. Equating the $k^{th}$ column yields
\begin{equation} \label{eq: pk_rec}
    \tilde{\bm{p}}_{k} = \bm{v}_k - l_{k-1} \tilde{\bm{p}}_{k-1}.
\end{equation}
Finally we can use
\begin{equation} \label{eq: xk_rec}
    \bm{x}_k = \bm{x}_0 + \tilde{\bm{P}}_{k} \tilde{\bm{y}}_{k}                                 \\
    = \bm{x}_0 + \left[ \tilde{\bm{P}}_{k-1} , \tilde{\bm{p}}_{k} \right]
    \begin{bmatrix}
        \tilde{\bm{y}}_{k-1} \\
        \eta_k
    \end{bmatrix}                                                                    \\
    = \bm{x}_0 + \tilde{\bm{P}}_{k-1} \tilde{\bm{y}}_{k-1} + \eta_k \tilde{\bm{p}}_{k} \\
    = \bm{x}_{k-1} + \eta_k \tilde{\bm{p}}_{k}
\end{equation}
as a recurrance for $\bm{x}_k$. A recurrance for $\bm{r}_k$ is easily computed as
\begin{equation} \label{eq: rk_rec}
    \bm{r}_{k} = b - \bm{A} \bm{x}_k = b - \bm{A} \left( \bm{x}_{k-1} + \eta_k \tilde{\bm{p}}_{k} \right) = \left( b - \bm{A} \bm{x}_{k-1} \right) - \eta_k \bm{A} \tilde{\bm{p}}_{k} = \bm{r}_{k-1} - \eta_k \bm{A} \tilde{\bm{p}}_{k}
\end{equation}
Altogether we are left with recurrences for $\bm{v}_k$ from Lanczos, $\tilde{\bm{p}}_{k}$ \Cref{eq: pk_rec}, the residual $\bm{r}_k$ \Cref{eq: pk_rec},  and for the approximate solution $\bm{x}_k$ \Cref{eq: xk_rec}. However, futher simplifications can be made to bring about a more efficient algorithm. Recall from \Cref{Section4.3} that $\bm{A} \bm{V}_{k} =  \bm{V}_{k} \bm{T}_k + \bm{v}_{k+1} \bm{t}_{k}^{\intercal}$ where $\bm{t}_k = \left[ 0,0, \ldots , 0, \beta_k \right]^{\intercal} \in \RR^k$ meaning
\[
    \bm{r}_k = \bm{r}_0 - \bm{A} \bm{V}_{k} \bm{y}_k = \bm{r}_0 - \bm{V}_{k} \bm{T}_k \bm{y}_k - \langle \bm{t}_k , \bm{y} \rangle \bm{v}_{k+1} = - \beta_k y_k \bm{v}_{k+1}.
\]
This tells us that $\bm{r}_k$ is parallel to $\bm{v}_{k+1}$ and orthogonal to all $\bm{v}_{i}, \; 1 \leq i \leq k$. This further implies that $\bm{r}_k$ is orthogonal to all $\bm{r}_i, \; 1 \leq i \leq k-1$ since they are just $\bm{v}_{i}$ scaled by some constant factor. So replacing $\bm{r}_{k-1}$ with $\bm{v}_k / \eta_k$ and defining $\bm{p}_k \triangleq \tilde{\bm{p}}_k / \gamma_k$ gives us a new set of recurrences
\begin{align*}
    \bm{x}_k & = \bm{x}_{k-1} + \alpha_k \bm{p}_k        \\
    \bm{r}_k & = \bm{r}_{k-1} - \alpha_k \bm{A} \bm{p}_k \\
    \bm{p}_k & = \bm{r}_{k-1} + \beta_k \bm{p}_{k-1}
\end{align*}
where $\alpha_k = \eta_k / \gamma_k$. From \Cref{lemma: Pk_cols_A_conj} we have shown that the columns of $\tilde{\bm{P}}_k$ are $A$-conjugate (that is $\langle \tilde{\bm{p}}_i , \bm{A} \tilde{\bm{p}}_j \rangle = 0, \; i \neq j$) and that $\tilde{\bm{P}}_k^{\intercal} \bm{A} \tilde{\bm{P}}_k = \bm{D}_k$. This also means that $\langle \bm{r}_i , \bm{r}_j \rangle = 0, \; i \neq j$. Note that from our recurrence for $\bm{p}_k = \bm{r}_{k-1} + \beta_k \bm{p}_{k-1}$ we have
\[
    \langle \bm{A} \bm{p}_k ,\bm{p}_k \rangle = \langle \bm{A} \bm{p}_k , \bm{r}_{k-1} + \beta_k \bm{p}_{k-1} \rangle = \langle \bm{A} \bm{p}_k , \bm{r}_{k-1} \rangle.
\]
We can now find an expression for $\alpha_k$ as
\begin{align*}
    \langle \bm{r}_{k-1} , \bm{r}_{k} \rangle & = \langle \bm{r}_{k-1} , \bm{r}_{k-1} - \alpha_k \bm{A} \bm{p}_k \rangle                           \\
    \langle \bm{r}_{k-1} -1 \rangle           & = \langle \bm{r}_{k-1} , \bm{r}_{k-1} \rangle - \alpha_k \langle \bm{p}_k, \bm{A} \bm{p}_k \rangle \\
    \alpha_k                                  & = \frac{\langle \bm{r}_{k-1} , \bm{r}_{k-1} \rangle}{\langle \bm{p}_k, \bm{A} \bm{p}_k \rangle}.
\end{align*}
Similarly, using the recurrence for $\bm{p}_k$, an expression for $\beta_k$ can be computed as
\begin{align*}
    \langle \bm{A} \bm{p}_{k-1} , \bm{p}_k \rangle & = \langle \bm{A} \bm{p}_{k-1}, \bm{r}_{k-1} + \beta_k \bm{p}_{k-1} \rangle                                       \\
    \langle \bm{A} \bm{p}_{k-1} , \bm{p}_k \rangle & = \langle \bm{A} \bm{p}_{k-1}, \bm{r}_{k-1} \rangle + \beta_k \langle \bm{A} \bm{p}_{k-1}, \bm{p}_{k-1} \rangle  \\
    \beta_k                                        & = - \frac{\langle \bm{A} \bm{p}_{k-1}, \bm{r}_{k-1} \rangle}{\langle \bm{A} \bm{p}_{k-1}, \bm{p}_{k-1} \rangle}.
\end{align*}
This formula requires an additional dot product which was not present before. Fortunately, this dot product can be eliminated using our recurrence for $\bm{r}_k$
\begin{align*}
    \langle \bm{r}_k , \bm{r}_k \rangle & = \langle \bm{r}_k , \bm{r}_{k-1} - \alpha_k \bm{A} \bm{p}_k \rangle                            \\
    \langle \bm{r}_k , \bm{r}_k \rangle & = \langle \bm{r}_k , \bm{r}_{k-1} \rangle - \alpha_k \langle \bm{r}_k , \bm{A} \bm{p}_k \rangle \\
    \alpha_k                            & = - \frac{\langle \bm{r}_k , \bm{r}_k \rangle}{\langle \bm{r}_k , \bm{A} \bm{p}_k \rangle}.
\end{align*}
Equating the two expressions for $\alpha_k$ affords
\begin{align*}
    - \frac{\langle \bm{r}_k , \bm{r}_k \rangle}{\langle \bm{r}_k , \bm{A} \bm{p}_k \rangle}  & = \frac{\langle \bm{r}_{k-1} , \bm{r}_{k-1} \rangle}{\langle \bm{p}_k, \bm{A} \bm{p}_k \rangle} \\
    - \frac{\langle \bm{r}_k , \bm{r}_k \rangle}{\langle \bm{r}_{k-1} , \bm{r}_{k-1} \rangle} & = \frac{\langle \bm{r}_k , \bm{A} \bm{p}_k \rangle}{\langle \bm{p}_k, \bm{A} \bm{p}_k \rangle}.
\end{align*}
This means that
\[
    \beta_k = \frac{\langle \bm{r}_{k-1} , \bm{r}_{k-1} \rangle}{\langle \bm{r}_{k-2} , \bm{r}_{k-2} \rangle}.
\]
These recurrences are computed iteratively to form the basis of the CG algorithm, seen in Algorithm \ref{alg: CG}.

{\centering
\begin{minipage}{.85\linewidth}
    \begin{algorithm}[H]
        \caption{CG Algorithm}
        \label{alg: CG}
        \SetAlgoLined
        \DontPrintSemicolon
        \SetKwInOut{Input}{input}\SetKwInOut{Output}{output}

        \Input{$\bm{A} \succ \bm{0}$, $\bm{b}$ and an initial guess $\bm{x}_0$.}
        \Output{An approximation of $\bm{x}^{\ast}$, $\bm{x}_k$.}
        \BlankLine
        $\bm{r}_0 = \bm{b} - \bm{A} \bm{x}_0$, $\bm{p}_1 = \bm{r}_0$\;
        \For{$k = 1 , \ldots $ \Until $\| r_{k-1} \| \leq \tau$}{
            $\alpha_k = \frac{\langle \bm{r}_{k-1} , \bm{r}_{k-1} \rangle}{\langle \bm{p}_k, \bm{A} \bm{p}_k \rangle}$ \;
            $\bm{x}_k = \bm{x}_{k-1} + \alpha_k \bm{p}_k$ \;
            $\bm{r}_k = \bm{r}_{k-1} - \alpha_k \bm{A} \bm{p}_k$ \;
            $\beta_{k+1} = \frac{\langle \bm{r}_{k} , \bm{r}_{k} \rangle}{\langle \bm{r}_{k-1} , \bm{r}_{k-1} \rangle}$ \;
            $\bm{p}_{k+1} = \bm{r}_{k} + \beta_{k+1} \bm{p}_k$ \;
        }
        \Return{$\bm{x}_k$}
        \BlankLine
    \end{algorithm}
\end{minipage}
\par
}

The next few theorems pertain to the rate of convergence of the CG algorithm.

\begin{thm} \label{theorem: equiv-CG-spaces}
    Let the CG iteration (\Cref{alg: CG}) be applied to a symmetric positive definite matrix problem $\bm{A} \bm{x} = \bm{b}$. As long as the iteration has not yet converged (that is, $\bm{r}_{n-1} \neq 0$), the algorithm proceeds without divisions by zero, and we have the following identities of subspaces:
    \begin{equation} \label{eq: equiv-CG-spaces}
        \calK_n                                                  = \operatorname{l.s} \left\{ \bm{x}_i \right\}_{i=1}^{n} = \operatorname{l.s} \left\{ \bm{p}_i \right\}_{i=0}^{n-1} = \operatorname{l.s} \left\{ \bm{r}_i \right\}_{i=0}^{n-1} = \operatorname{l.s} \left\{ \bm{A}^{i} \bm{b} \right\}_{i=0}^{n-1}.
    \end{equation}
    Moreover the residuals are orthogonal,
    \begin{equation*}
        \langle \bm{r}_{n} , \bm{r}_{j} \rangle = 0 \quad (j<n)
    \end{equation*}
    and the search directions are $\bm{A}-$conjugate \cite{TrefethenLloydN.LloydNicholas1997Nla/}*{page 295}.
\end{thm}

\begin{proof}
    The orthogonality of the residuals has already been shown within the derivation of the CG algorithm. Seeing that the search directions are $\bm{A}-$conjugate is a very straight forward application of \Cref{lemma: Pk_cols_A_conj}. To show \Cref{eq: equiv-CG-spaces} we can induct on $k$. From the initial guess $\bm{x}_0$ and the expression $\bm{x}_k = \bm{x}_{k-1} + \alpha_k \bm{p}_k$, using an induction argument it follows that $\bm{x}_k$ belongs to $\operatorname{l.s} \left\{ \bm{p}_i \right\}_{i=0}^{k-1}$. Similarly, from $\bm{p}_k = \bm{r}_{k-1} + \beta_k \bm{p}_{k-1}$ it follows that this belongs to $\operatorname{l.s} \left\{ \bm{r}_i \right\}_{i=0}^{n-1}$. Finally, from $\bm{r}_k = \bm{r}_{k-1} - \alpha_k \bm{A} \bm{p}_k$ that this belongs to $\operatorname{l.s} \left\{ \bm{A}^{i} \bm{b} \right\}_{i=0}^{n-1}$.
\end{proof}

It is fairly straight forward to confirm that CG minimizes error at each step.

\begin{thm} \label{theorem: CG-error-min}
    Let the CG iteration be applied to a symmertic positive definite matrix $\bm{A} \bm{x} = \bm{b}$. If the iteration has not already converged (that is, $\| \bm{r}_{k-1} \| > 0 $), then $\bm{x}_{k}$ is the unique point in $\calK_{k}$ that minimizes $\| \bm{x}^{\star} - \bm{x}_{k} \|_{\bm{A}}$. The convergence is the monotonic,
    \[
        \| \bm{x}^{\star} - \bm{x}_{k} \|_{\bm{A}} \leq \| \bm{x}^{\star} - \bm{x}_{k-1} \|_{\bm{A}}
    \]
    and $\| \bm{x}^{\star} - \bm{x}_{k} \|_{\bm{A}} = 0$ is achieved for some $k \leq n$ \cite{TrefethenLloydN.LloydNicholas1997Nla/}*{page 296}.
\end{thm}

\begin{proof}
    From \Cref{theorem: equiv-CG-spaces} we know that $\bm{x}_{k}$ belongs to $\calK_{k}$. To show that it is the unique point in $\calK_{k}$ that minimizes the error, consider an arbitrary point $\bm{z} = \bm{x}_k - \Delta \bm{z} \in \calK_{k}$, with error $\bm{e} = \bm{x}_{\star} - \bm{z} = \bm{e}_{k} + \Delta \bm{x}$. Note in this context $\bm{e}_{k} = \bm{x}_{\star} - \bm{x}_{k}$, and does {\it not} represent standard basis vectors. We calculate
    \begin{align*}
        \| \bm{e} \|_{\bm{A}}^2 \
         & = \left( \bm{e}_{k} + \Delta \bm{x} \right)^{\intercal} \bm{A} \left( \bm{e}_{k} + \Delta \bm{x} \right)                                                                                  \\
         & = \bm{e}_{k}^{\intercal} \bm{A} \bm{e}_{k} + \left( \Delta \bm{x} \right)^{\intercal} \bm{A} \left( \Delta \bm{x} \right) + 2 \bm{e}_{k}^{\intercal} \bm{A} \left( \Delta \bm{x} \right).
    \end{align*}
    The final term of this expression is $2 \bm{r}_{k}^{\intercal} \left( \Delta \bm{x} \right)$, an inner product of $\bm{r}_{k}$ with a vector in $\calK_k$, and by \Cref{theorem: CG-error-min}, any such inner product is zero. Effectively, this means
    \[
        \| \bm{e} \|_{\bm{A}}^{2} = \bm{e}_{k}^{\intercal} \bm{A} \bm{e}_{k} + \left( \Delta \bm{x} \right)^{\intercal} \bm{A} \left( \Delta \bm{x} \right).
    \]
    Only the second of these terms depends on $\Delta \bm{x}$, since $\bm{A}$ is positive definite, that term is greater than or equal to $0$, attaining the value of $0$ if and only if $\Delta \bm{x} = \bm{0}$, that is, $\bm{x}_k = \bm{x}$. Thus $\| \bm{e} \|_{\bm{A}}$ is minimal if and only if $\bm{x}_{k} = \bm{x}$, as claimed.
\end{proof}

As seen in \Cref{eq: x_ast_via_cayley} Krylov subspace methods, in some sense, aim to find a polynomial $p_n \in P_n$ which minimizes $\| p_n \left( \bm{A} \right) \| = \| \sum_{k=0}^{n-1} \alpha_{k + 1} \bm{A}^{k} \bm{r}_{0} \|$, where $P_n$ is the set of polynomials with degree less than or equal to $n$ with $p(\bm{0}) = \Id$. In the context of the CG algorithm, this is equivalent to finding a polynomial $p_n \in P_n$ which instead minimizes
\begin{equation} \label{eq: CG-poly-min}
    \| p_n \left( \bm{A} \right) \bm{e}_{0} \|_{\bm{A}}.
\end{equation}
From \Cref{theorem: equiv-CG-spaces}, we can derive the following convergence theorem.

\begin{thm} \label{theorem: CG-convg}
    If the CG iteration has not already converged before step $k$ (that is, $\| \bm{r}_{k-1} \| > 0 $), then \Cref{eq: CG-poly-min} has a unique solution $p_n \in P_n$, and the iterate $\bm{x}_{k}$ has error $\bm{e}_k = p_n \left( \bm{A} \right) \bm{e}_{0}$ for this same polynomial $p_n$. Consequently we have
    \begin{equation} \label{eq: CG-err-bounds}
        \frac{\| \bm{e}_k \|_{\bm{A}}}{\| \bm{e}_{0} \|_{\bm{A}}} = \inf_{p \in P_n} \frac{\| p \left( \bm{A} \right) \bm{e}_{0} \|_{\bm{A}}}{\| \bm{e}_{0} \|_{\bm{A}}} \leq \inf_{p \in P_n} \max_{\lambda \in \Lambda \left( \bm{A} \right)} \left| p (\lambda) \right|
    \end{equation}
    \cite{TrefethenLloydN.LloydNicholas1997Nla/}*{page 298}.
\end{thm}

\begin{proof}
    From \Cref{theorem: equiv-CG-spaces} it follows that $\bm{e}_{k} = p \left( \bm{A} \right) \bm{e}_{0}$ for some $p \in P_n$. The equality in \Cref{eq: CG-err-bounds} is a consequence of this and \Cref{theorem: CG-error-min}. As for the inequality in \Cref{eq: CG-err-bounds}, if $\bm{e}_{0} = \sum_{j=1}^{m} a_j \bm{v}_j$ is an expansion of $\bm{e}_{0}$ in orthonormal eigenvectors of $\bm{A}$, then we have $p \left( \bm{A} \right) \bm{e}_{0} = \sum_{j=1}^{m} a_j p \left( \lambda_j \right) \bm{v}_{j}$ and thus
    \[
        \| \bm{e}_{0} \|_{\bm{A}}^{2} = \sum_{j=1}^{m} a_{j}^{2} \lambda_j, \qquad \| p \left( \bm{A} \right) \bm{e}_0 \|_{\bm{A}}^{2} = \sum_{j=1}^{m} a_{j}^{2} \lambda_j \left( p \left( \lambda_j \right) \right)^{2}.
    \]
    These identities indicate $\| p \left( \bm{A} \right) \bm{e}_{0} \|_{\bm{A}}^{2} / \| \bm{e}_{0} \|_{\bm{A}}^{2} \leq \max_{\lambda \in \Lambda \left( \bm{A} \right)} \left| p (\lambda) \right|^{2}$, which implies the inequality in question.
\end{proof}
\newpage

\section{Gaussian Processes}\label{Chapter1}
The aim of this chapter is to build up the theory behind GPs from the ground up. First, we shall review some essential theory from functional analysis on kernels and reproducing kernel Hilbert spaces which are not used in GPs but are used in a vast array of machine learning models, aptly named kernel machines. Afterward, we shall go through the underlying statistics that drive GP prediction and use it to form algorithms for both regression and classification tasks. Note that most of the theory presented here is only for real-values data sets although most the time complex-valued generalizations do exist.

\subsection{Kernels}\label{Section1.1}

Often in machine learning we are often met with the challenge of how to best represent data instances as fixed size feature vectors $\bm{x}_i \in X$. For certain objects it might not be obvious at all how to represent the data as a fixed length vector. Good examples of variable length data include textual documents and genomic data. For these data types we can define a method of measuring similarity between objects which requires them to first be converted to a fixed length feature vector first \cite{MurphyKevinP2012Ml}. To do this we begin by mapping the feature vectors into a Hilbert space $H$ which enriches the vector space with an inner product $\langle \cdot , \cdot \rangle_H : H \times H \to \RR$ and a norm $\| \cdot \|_H : H \to \RR$. Input data is transformed into feature space vectors via a non-linear feature mapping $\Phi : X \to H$. The benefit of using feature maps in this way is that a non-linear descision boundary can be constructed using linear models. In some instances a similarity measure can be computed directly using a function $k : X \times X \to \RR$, instead of needing to construct a $\Phi$ and then computing the inner product of the transformed instances. Functions that act directly on our data instances are known as kernel functions and using them to avoid computation associated with the underlying feature space is known as the kernel trick \cite{SteinwartIngo2008SVMb}. These ideas are stated more formally in definition \ref{defe: kernel}.

\begin{defe}[Kernel] \label{defe: kernel}
    Let $X$ be a non-empty set. Then a function $k : X \times X \to \RR$ is called a kernel on $X$ if there exists a Hilbert space and a map $\Phi : X \to H$ such that for all $\bm{x} , \bm{x}' \in X$ we have $k \left( \bm{x} , \bm{x}' \right) = \langle \Phi \left( \bm{x} \right), \Phi \left( \bm{x}' \right) \rangle_H$. We call the $\Phi$ the feature map and $H$ the feature space of $k$.
\end{defe}

It is worth noting that almost no conditions are placed on the set $X$, allowing it to accommodate virtually any form of data. It is not surprising then that neither the feature map nor the feature space are uniquely determined by the kernel. As shown by the example from Steinwart and Christmann \cite{SteinwartIngo2008SVMb}, when $X = \RR$ and $k \left( x , x' \right) = x \cdot x'$ where $x , x' \in X$, we can see that $k$ is a kernel using the feature map $\Phi \left( x \right) = x$ and $H = \RR$. However, another suitable feature map for this particular kernel is $\Phi' \left( x \right) = \left( x / \sqrt{2} , x / \sqrt{2} \right)$ with a corresponding feature space of $H = \RR^2$ since
\[
    \langle \Phi' \left( x \right), \Phi' \left( x' \right) \rangle_{\RR^2} = \frac{x'}{\sqrt{2}} \cdot \frac{x}{\sqrt{2}} + \frac{x'}{\sqrt{2}} \cdot \frac{x}{\sqrt{2}} = x \cdot x'
\]
for $x,x' \in X$. While their might be numerous functions that provide some notion of similarity between data entries, these functions might not be valid kernels. Instead of needing to construct a feature map and feature space to verify that a chosen function is a valid kernel using definition \ref{defe: kernel}, we can make use of a much simpler set of criteria. Before embarking on this train of thought, we need to define the following.

\begin{defe}[Positive Definite and Positive Semidefinite] \label{defe: PD}
    A function $k : K \times K \to \RR$ is positive semidefinite if for all $n \in \NN$ and $\alpha_1 , \ldots , \alpha_n \in \RR$ and all $\bm{x}_1 ,\ldots , \bm{x}_n \in X$ we have
    \begin{equation}\label{eq: PSD}
        \sum_{i=1}^{n} \sum_{j=1}^{n} \alpha_i \alpha_j k \left( \bm{x}_j , \bm{x}_i \right) \geq 0.
    \end{equation}
    Furthermore, $k$ is said to be positive definite if for mutually distinct $\bm{x}_1 ,\ldots , \bm{x}_n \in X$ equality \ref{eq: PSD} only holds for $\alpha_1 = \ldots = \alpha_n = 0$ \cite{SteinwartIngo2008SVMb}.
\end{defe}

\begin{defe}[Symmetric] \label{defe: Symmetric_function}
    A function $k : K \times K \to \RR$ is called symmetric if $k \left( \bm{x} , \bm{x}' \right) = k \left( \bm{x}' , \bm{x} \right)$ for any inputs $\bm{x}' , \bm{x} \in X$ \cite{SteinwartIngo2008SVMb}.
\end{defe}

\begin{defe}[Gram Matrix] \label{defe: Gram_Matrix}
    For fixed $\bm{x}_1 ,\ldots , \bm{x}_n \in X$ the matrix $\bm{K} \in \RR^{n \times n}$ where $\bm{K}_{i,j} \triangleq k \left( \bm{x}_j , \bm{x}_i \right)$ is the Gram matrix \cite{SteinwartIngo2008SVMb}.
\end{defe}

Note that checking if a function is positive (semi) definite is equivalent to checking that any Gram matrix produced by a function is positive (semi) definite. If $k$ is a real valued kernel corresponding to the feature map $\Phi$, then $k$ is symmertic by virtue of the fact that the inner product of a real Hilbert space is symmetric. Moreover $k$ is positive definite since for $\alpha_1 , \ldots , \alpha_n \in \RR$ and $\bm{x}_1 ,\ldots , \bm{x}_n \in X$ we have
\begin{align*}
     & \sum_{i=1}^{n} \sum_{j=1}^{n} \alpha_i \alpha_j k \left( \bm{x}_j , \bm{x}_i \right)                                           \\
     & = \sum_{i=1}^{n} \sum_{j=1}^{n} \alpha_i \alpha_j \langle \Phi \left( \bm{x}_i \right), \Phi \left( \bm{x}_j \right) \rangle_H \\
     & = \norm{ \sum_i^n \alpha_i \Phi \left( \bm{x}_i \right) }_{H}^{2}                                                              \\
     & \geq 0.
\end{align*}

The following theorems tell us that it is not only necessary for a kernel to be positive semi definite but it is also a sufficient condition.

\begin{thm} \label{theorem: nec_and_suf_kernel_1}
    A function $k : K \times K \to \RR$ is a kernel if and only if it is symmertic and positive semidefinite \cite{SteinwartIngo2008SVMb}.
\end{thm}

\subsection{Reproducing Kernel Hilbert Spaces}\label{Section1.2}

We shall now shift our attention towards reproducing kernel Hilbert spaces (RKHS) and describe their relation to kernels, and see that in some sense, the RKHS of a kernel $k$ is the smallest feature space for a kernel. The formal definition of a RKHS is stated in definition \ref{defe: RKHS}.

\begin{defe}[RKHS] \label{defe: RKHS}
    Let $X \neq \emptyset$ and $H$ be a real Hilbert space over $X$
    \begin{enumerate}
        \item A function $k : X \times X \to \RR$ is called a reproducing kernel if we have $k \left( \cdot, \bm{x} \right) \in H$ for all $\bm{x} \in X$ and the reproducing property
              \[
                  f(\bm{x}) = \langle f , k \left( \cdot, \bm{x} \right) \rangle
              \]
              holds for all $f \in H$ and $x \in X$.
        \item The space $H$ is called a reproducing kernel Hilbert space over $X$ if for all $\bm{x} \in X$ the Dirac functional $\delta_{\bm{x}} : H \to \RR$ defined by $\delta_{\bm{x}} (f) \triangleq f(x), \; f \in H$ is continuous.
    \end{enumerate}
    \cite{SteinwartIngo2008SVMb}
\end{defe}

An important property of the RKHS is that the convergence in the norm implies pointwise convergence. Specifically, in a RKHS for any sequence of functions $\left\{ f_n \right\} \subset H$ where $\norm{f_n - f} \to 0$ we have $\abs{\delta_{\bm{x}} (f_n) - \delta_{\bm{x}} (f)} = \abs{f_n (x) - f (x)} \to 0$. Note that because the evaluation function is both linear and continuous, then it is also bounded in the sense that there is an $c \in \RR, \; c > 0$ such that for every $f \in H$ and a fixed $\bm{x} \in X$ we have $\abs{\delta_{\bm{x}} (f)} \leq c \norm{f}_H$ \cite{BerezanskyMakarovich1996FaV1}. This property of uniform convergence implying pointwise convergence is important since it tells us that if functions $f,g \in H$ are close in norm then their evaluation at any point is also similar. The following lemma ties together the definition of an RKHS, reproducing kernel and a kernel.

\begin{lem}[] \label{lem: RKHS_rk_k}
    Let $H$ be a Hilbert function space over $X$ that has a reproducing kernel $k$. Then $H$ is a RKHS and $H$ is also a feature space of $k$ where the feature map $\Phi : X \to H$ is given by
    \[
        \Phi (\bm{x}) = k \left( \cdot , \bm{x}  \right)
    \]
    for some $\bm{x} \in X$. We call $\Phi$ the canonical feature map.
\end{lem}

\begin{proof}
    Since the reproducing property tells us that any Dirac functional can be represented by the reproducing kernel this means
    \[
        \abs{\delta_{\bm{x}} (f)} = \abs{f(\bm{x})} = \abs{\langle f , k \left( \cdot, \bm{x} \right) \rangle} \leq \norm{k \left( \cdot, \bm{x} \right)}_H \cdot \norm{f}_H
    \]
    for all $\bm{x} \in X, \; f \in H$. This shows continuity of $\delta_{\bm{x}}$ for $\bm{x} \in X$. In order to show that $\Phi$ is a feature map, fix an $\bm{x}' \in X$ and set $f = k \left( \cdot, \bm{x}' \right)$. Then for $\bm{x} \in X$, the reproducing property yields
    \[
        \langle \Phi (\bm{x}') , \Phi (\bm{x}) \rangle_H = \langle k \left( \cdot, \bm{x}' \right) , k \left( \cdot, \bm{x} \right) \rangle_H = \langle f , k \left( \cdot, \bm{x} \right) \rangle_H = f(\bm{x}) = k \left( \bm{x}', \bm{x} \right).
    \]
\end{proof}

This tells us that every Hilbert space with a reproducing kernel is a RKHS. We can also show the converse, that is, every RKHS has a unique reproducing kernel seen in theorem \ref{theorem: unique_kernel}.

\begin{thm} \label{theorem: unique_kernel}
    Let $H$ be a RKHS over $X$. Then $k: X \times X \to \RR$ defined by $k \left( \bm{x}', \bm{x} \right) = \langle \delta_{\bm{x}} , \delta_{\bm{x}'} \rangle_H, \; \bm{x} , \bm{x}' \in X$ is the only reproducing kernel of $H$ \cite{SteinwartIngo2008SVMb}.
\end{thm}

Theorem \ref{theorem: unique_kernel} shows that a RKHS is uniquely determined by its kernel. In fact the other direction can also be shown to afford a one-to-one correspondence between kernels and RKHS. This is known as the Moore-Aronszajn theorem presented in thorem \ref{theorem: Moore-Aronszajn}.

\begin{thm}[Moore-Aronszajn] \label{theorem: Moore-Aronszajn}
    Suppose $k$ is a symmertic positive definite kernel on a set $X$. Then there is a unique Hilbert space of functions for which $k$ is the reproducing kernel \cite{BerlinetAlain2003RKHS}.
\end{thm}

The elements of a RKHS will often inherit the analytical properties of its corresponding kernel. This means that kernels provide a mechanism for generating spaces of functions with useful analytical properties.

\subsection{Gaussian Radial Basis Kernel}\label{Section1.3}

We shall now focus on a specific class of kernel that shall be used extensively in upcoming theory and experimentation.

\begin{defe}[Gaussian Radial Basis Kernel] \label{defe: grbfk}
    For $d \in \NN, \; \sigma \in \RR_{>0}$ and $ \bm{z} , \bm{z}' \in \RR^d$ we define
    \[
        k_\sigma \left( \bm{z} , \bm{z}' \right) \triangleq \exp \left( - \sigma^{-2} \sum_{j=1}^{d} \left( \bm{z}_j - {\bm{z}'}_j \right)^2 \right).
    \]
    Then $k_\sigma$ is a real valued kernel called the Gaussian Radial Basis Kernel (RBF) kernel with bandwidth $\sigma$. Moreover $k_\sigma$ can be computed as
    \[
        \exp \left( \frac{- \norm{\bm{z} - {\bm{z}'}}_{2}^{2}}{\sigma^2} \right)
    \]
    \cite{SteinwartIngo2008SVMb}.
\end{defe}
The Gaussian RBF kernel makes for a very simple an intuitive measurement of similarity between its inputs. One geometric interpretation of the Gaussian RBF kernel is that as the radius of the smallest $d$-sphere containing $\bm{z} , \bm{z}' \in \RR^d$ grows the corresponding measurement of similarity decays exponentially. A visual representation of this decay is shown in figure \ref{fig: grbfk_graph_1}.

% \begin{figure}[H]
%     \centering
%     \subfloat{
%         \begin{tikzpicture}

%             \begin{axis}[
%                     width=8cm,height=8cm,
%                     domain=-2:2,
%                     xmax=2.25,
%                     ymax=2.25,
%                     xmin=-2.25,
%                     ymin=-2.25,
%                     zmax=1.1,
%                     axis lines = left,
%                     colormap/hot,
%                 ]

%                 \addplot3[samples = 25, surf] {exp(-(x^2 + y^2)/1)};
%                 \node at (rel axis cs:1,0.5,1) [above] {\(\sigma=1\)};

%             \end{axis}

%         \end{tikzpicture}
%     }%
%     \quad
%     \subfloat{
%         \begin{tikzpicture}

%             \begin{axis}[
%                     width=8cm,height=8cm,
%                     domain=-2:2,
%                     xmax=2.25,
%                     ymax=2.25,
%                     xmin=-2.25,
%                     ymin=-2.25,
%                     zmax=1.1,
%                     axis lines = left,
%                     colormap/hot,
%                 ]

%                 \addplot3[samples = 25, surf] {exp(-(x^2 + y^2)/2)};
%                 \node at (rel axis cs:1,0.5,1) [above] {\(\sigma=2\)};

%             \end{axis}

%         \end{tikzpicture}
%     }%
%     \caption{A graph of the Gaussian RBF from definition \ref{defe: grbfk} for $d=2$. Evidently, a larger value of $\sigma$ slows the rate of decay increasing the similarity between the same pair of samples.}%
%     \label{fig: grbfk_graph_1}
% \end{figure}

This kernel is infinitely differentiable meaning it has mean square derivatives of all orders and is therefore very smooth. In fact, some argue that such strong smoothness makes it unrealistic for modelling natural phenomena \cite{RasmussenCarlEdward2006Gpfm,SteinMichaelL1999IoSD}. Nontheless, Gaussian RBF kernelis remains the one of the most widely used kernels in literature.

\subsection{Kernel Machines}\label{Section1.4}

In this section, we shall look at two different machine learning models that make use of kernels to perform classification and regression. The first kernel machine we shall look at are support vector machines (SVM). SVMs where originally designed for binary classification and as such we shall only present a model for binary classification, although extensions exist that allow regression and multi-class classification.

For the binary classification problem we are tasked with labelling new samples with either one of two classes, $-1$ or $1$. We shall assume our input space consists of vectors from $\RR^d$ and that we provided with a labelled training set $D = \left\{ \left( \bm{x}_1 , y_1 \right), \left( \bm{x}_1 , y_1 \right), \ldots , \left( \bm{x}_n , y_n \right) \right\}$. One simple method to classify samples is by creating an affine linear hyperplane satisfying
\begin{align} \label{eq: linear_sep_hyp}
    \langle \bm{w}, \bm{x}_i \rangle + b > 0, \quad y_i = +1 \nonumber \\
    \langle \bm{w}, \bm{x}_i \rangle + b < 0, \quad y_i = -1
\end{align}
for some $\bm{w} \in \RR^d$ and $b \in \RR$ where $\norm{w}_2 = 1$. Moreover we would like $\bm{w}$ and $b$ to maximise the margin, that is the maximal distance between the separating hyperplane and the points in $D$. The specific $\bm{w}$ and $b$ obtained through the training set is denoted $\bm{w}_D$ and $b_D$ and the resulting descision function is defined as
\[
    f_D \left( \bm{x} \right) \triangleq \operatorname{sign} \left( \langle \bm{w}_D , \bm{x} \rangle + b_D \right).
\]
There are, however, a number of short comings to this model. The most obvious is that our training data may not be linearly separable in $\RR^d$ meaning that no such $\bm{w}_D$ and $b_D$ exist. Moreover, when noise is introduced to the data set this model will prioritize finding a hyperplane that perfectly separates the two classes, making no comprises in misclassifying points, leaving it subject to overfitting. SVMs where introduced by Boser {\it et al.} \cite{BoserBernhard1992Ataf} to address the first issue of separability. Their approach was to lift the input vector into a more malleable Hilbert space $H_0$ using a feature map. The inputs are then classified within the new space. Unfortunately this method does nothing to address the second issue of over fitting and, if anything, actually worsens it. Cortes and Vapnik \cite{CortesCorinna1995SN} attempted to address this second issue by introducing slack variables to equation \ref{eq: linear_sep_hyp} so that we instead need to satisfy $y_i \left( \langle \bm{w} , \Phi \left( \bm{x}_i \right) \rangle + b \right) \geq 1 - \xi_i$ for some $\xi_i \in \RR_{>0}$. These constraints can be re-written as
\[
    \xi_i \geq 1 - y_i \left( \langle \bm{w} , \Phi \left( \bm{x}_i \right) \rangle + b \right)
\]
and combining this this our slack constraints (that is $\xi_i \geq 0$) yields
\[
    \xi_i \geq \max \left\{ 0, 1 - y_i \left( \langle \bm{w} , \Phi \left( \bm{x}_i \right) \rangle + b \right)  \right\} = L_{\text{hinge}} \left( y_i , \langle \bm{w} , \Phi \left( \bm{x}_i \right) \rangle + b \right)
\]
where $L_{\text{hinge}}$ is the hinge loss defined as
\[
    L_{\text{hinge}} \left( y,\eta \right) \triangleq \max \left\{ 0,1-y\eta \right\}.
\]
This optimization problem can be re-written is the form
\[
    \min_{\left( \bm{w} , b \right) \in H_0 \times \RR} \lambda \norm{\bm{w}}_{H_0} + \frac{1}{n} \sum_{i=1}^{n} L_{\text{hinge}} \left( y_i , f_{\left( \bm{w} , b \right)} \right)
\]
where $f_{\left( \bm{w} , b \right)} : X \to \RR$ is defined as
\[
    f_{\left( \bm{w} , b \right)} \triangleq \langle \bm{w} , \Phi \left( x_i \right) \rangle + b.
\]
Unfortunately, this new embedding requires us to solve for optimal parameters in a very high, or even infinite, dimension vector space. To get around this, often the Lagrange approach is used to solve the corresponding dual problem. When the hinge loss is used the dual problem becomes
\begin{align} \label{eq: SVM_dual_1}
     & \max_{\alpha \in \left[ 0,C \right]^n} \sum_{i=1}^{n} \alpha_i - \frac{1}{2} \sum_{i,j=1}^{n} y_i y_j \alpha_i \alpha_j \langle \Phi \left( \bm{x}_i \right), \Phi \left( \bm{x}_j \right) \rangle \nonumber \\
     & \text{subject to} \quad \sum_{i=1}^{n} y_i \alpha_i = 0
\end{align}
Notice that in the dual problem, we find that inner products are only taken with vectors that have the feature map applied to them allowing us to employ the kernel if the corresponding kernel trick described in section \ref{Section1.1} is known for the feature map used so that \ref{eq: SVM_dual_1} becomes
\begin{align*}
     & \max_{\alpha \in \left[ 0,C \right]^n} \sum_{i=1}^{n} \alpha_i - \frac{1}{2} \sum_{i,j=1}^{n} y_i y_j \alpha_i \alpha_j k \left( \bm{x}_i, \bm{x}_j \right) \\
     & \text{subject to} \quad \sum_{i=1}^{n} y_i \alpha_i = 0.
\end{align*}

The next machine learning model of interest that uses kernels are gaussian processes. To motivate this model, consider the time series data in figure \ref{fig: motive_gp_1}.

\begin{figure}[H]
    \centering
    \subfloat[]{
        \begin{tikzpicture}
            \draw[->,thick] (-0.01,0)--(6,0) node[right]{$x$};
            \draw[->,thick] (0,-0.01)--(0,6) node[above]{$y$};

            \draw[-,ultra thick] (0.7,-0.1)--(0.7,0.1) node[below,yshift=-0.3cm]{$x_1$};
            \draw[fill,draw,blue] (0.7,0.5) circle[radius=2.5pt];

            \draw[-,ultra thick] (1.4,-0.1)--(1.4,0.1) node[below,yshift=-0.3cm]{$x_2$};
            \draw[fill,draw,blue] (1.4,0.6) circle[radius=2.5pt];

            \draw[-,ultra thick] (2.7,-0.1)--(2.7,0.1) node[below,yshift=-0.3cm]{$x_3$};
            \draw[fill,draw,blue] (2.7,1.7) circle[radius=2.5pt];

            \draw[-,ultra thick] (3.7,-0.1)--(3.7,0.1) node[below,yshift=-0.2cm]{$x^{\ast}$};
            \draw[dashed,thick,red] (3.7,0)--(3.7,5);

            \draw[-,ultra thick] (5,-0.1)--(5,0.1) node[below,yshift=-0.3cm]{$x_4$};
            \draw[fill,draw,blue] (5,4) circle[radius=2.5pt];
        \end{tikzpicture}
    }%
    \qquad
    \subfloat[]{
        \begin{tikzpicture}
            \draw[->,thick] (-0.01,0)--(6,0) node[right]{$x$};
            \draw[->,thick] (0,-0.01)--(0,6) node[above]{$y$};

            \draw[-,ultra thick] (0.7,-0.1)--(0.7,0.1) node[below,yshift=-0.3cm]{$x_1$};
            \draw[fill,draw,blue] (0.7,0.5) circle[radius=2.5pt];
            \draw[dashed,blue] (0.7,0)--(0.7,4.7);
            \draw[<->,thick] (0.7,4.7)--(3.7,4.7) node[above,xshift=-1.5cm]{$k(x^{\ast},x_1)$};

            \draw[-,ultra thick] (1.4,-0.1)--(1.4,0.1) node[below,yshift=-0.3cm]{$x_2$};
            \draw[fill,draw,blue] (1.4,0.6) circle[radius=2.5pt];
            \draw[dashed,blue] (1.4,0)--(1.4,3.5);
            \draw[<->,thick] (1.4,3.5)--(3.7,3.5) node[above,xshift=-1.1cm]{$k(x^{\ast},x_2)$};

            \draw[-,ultra thick] (2.7,-0.1)--(2.7,0.1) node[below,yshift=-0.3cm]{$x_3$};
            \draw[fill,draw,blue] (2.7,1.7) circle[radius=2.5pt];
            \draw[dashed,blue] (2.7,0)--(2.7,2.3);
            \draw[<->,thick] (2.7,2.3)--(3.7,2.3) node[above,xshift=-0.9cm]{$k(x^{\ast},x_3)$};

            \draw[-,ultra thick] (3.7,-0.1)--(3.7,0.1) node[below,yshift=-0.2cm]{$x^{\ast}$};
            \node[diamond,draw,fill,draw,red,minimum width = 1cm,minimum height = 1.3cm,scale=0.25] (d) at (3.7,3) {};
            \draw[dashed,thick,red] (3.7,0)--(3.7,5);

            \draw[-,ultra thick] (5,-0.1 )--(5,0.1) node[below,yshift=-0.3cm]{$x_4$};
            \draw[fill,draw,blue] (5,4) circle[radius=2.5pt];
            \draw[dashed,blue] (5,0)--(5,4.5);
            \draw[<->,thick] (3.7,4.5)--(5,4.5) node[above,xshift=-0.3cm]{$k(x^{\ast},x_4)$};
        \end{tikzpicture}
    }%
    \caption{A graph of the Gaussian RBF from definition \ref{defe: grbfk} for $d=2$. Evidently, a larger value of $\sigma$ slows the rate of decay increasing the similarity between the same pair of samples.}%
    \label{fig: motive_gp_1}
\end{figure}

\begin{filecontents*}{./data/gp_intro_dat1.csv}
    x,y0,y1,y2
    0.0,2.6341780930873786,4.41685044685407,1.884123117075101
    0.11224489795918367,2.685150340856032,4.351109330694541,2.0933788453347146
    0.22448979591836735,2.758222906849677,4.246986054733594,2.3215583495507697
    0.336734693877551,2.844091796198551,4.111722477611423,2.5621308980348045
    0.4489795918367347,2.9327004798948444,3.954138391914307,2.8093607381581323
    0.5612244897959183,3.0142490315111,3.7837568463624893,3.058296829896802
    0.673469387755102,3.0800293333419133,3.609905398833131,3.304575221937234
    0.7857142857142857,3.1230071846450134,3.4408573216000065,3.5440886761610075
    0.8979591836734694,3.1381275068842287,3.2830844115325393,3.772606115615495
    1.010204081632653,3.1223796951701805,3.1407028040204787,3.985434762859824
    1.1224489795918366,3.074690487867501,3.0151840031490704,4.177213384641246
    1.2346938775510203,2.99572649694323,2.9053883056890326,4.341906534238328
    1.346938775510204,2.887680093346262,2.807934404050683,4.473021654529128
    1.4591836734693877,2.7540712359611135,2.7178753181761683,4.5640359352552755
    1.5714285714285714,2.5995639802641644,2.629596908378246,4.608968085491204
    1.683673469387755,2.4297636797240854,2.537815790076743,4.603008456947798
    1.7959183673469388,2.250945266539242,2.4385306514757277,4.54310273176884
    1.9081632653061225,2.0696826217289814,2.3297825969976156,4.428402138611055
    2.020408163265306,1.8923794707563157,2.2121155584786374,4.26051300065903
    2.13265306122449,1.724750314129252,2.088670901030661,4.043522033936433
    2.2448979591836733,1.5713409480295042,1.9649185604698398,3.78380984825793
    2.357142857142857,1.4351902937232137,1.8480772970914263,3.4896953544368947
    2.4693877551020407,1.3177316537250037,1.7463209648090843,3.170973707657322
    2.5816326530612246,1.2189781885886188,1.667887909534986,2.838411584600058
    2.693877551020408,1.137982180009872,1.6202034591848866,2.503253609595508
    2.806122448979592,1.0734814750833834,1.6091067041507867,2.176777775544475
    2.9183673469387754,1.0246008560812094,1.638242628202587,1.8699164299237567
    3.0306122448979593,0.9914469543413351,1.708653796550625,1.5929463528940093
    3.142857142857143,0.9754532832118712,1.8185871450769515,1.3552308689036683
    3.2551020408163267,0.9793831046385866,1.9635250522558478,1.1649988273056024
    3.36734693877551,1.0069633897912214,2.136445411268843,1.0291341361289352
    3.479591836734694,1.0622134340832736,2.328310102622806,0.9529576266820667
    3.5918367346938775,1.1485904962060034,2.5287676666005745,0.9399816624503701
    3.704081632653061,1.2681182060719118,2.7270230125223396,0.9916380696381344
    3.816326530612245,1.4206777569266578,2.9127943188682153,1.1069890129419622
    3.9285714285714284,1.603600797665261,3.077239529790242,1.282461239660869
    4.040816326530612,1.8116633596458505,3.213718985997705,1.5116563075035667
    4.153061224489796,2.0374945244629954,3.31826875093589,1.7853109321561014
    4.26530612244898,2.272346385818944,3.3896996920659737,2.0914743830547122
    4.377551020408164,2.50710273056381,3.4293088690936613,2.415953795652976
    4.489795918367347,2.7333622652331386,3.440263336916595,2.743037053839629
    4.6020408163265305,2.9444150750492786,3.426788503862636,3.0564529194286605
    4.714285714285714,3.1359501955059153,3.3933321658244284,3.3404756976868217
    4.826530612244898,3.306372509617609,3.343882098004392,3.5810500536442835
    4.938775510204081,3.456677113806286,3.281564998201946,3.766794718394943
    5.051020408163265,3.5899014357864862,3.2085942528948497,3.889767236618262
    5.163265306122449,3.710253447229357,3.1265441378249808,3.9459164775264113
    5.275510204081633,3.822067924064683,3.036856306931041,3.9352029543435254
    5.387755102040816,3.928774395166681,2.9414323013414645,3.8614290770863415
    5.5,4.032057566247807,2.843156622782664,3.7318467351526268
\end{filecontents*}

\begin{filecontents*}{./data/gp_intro_dat2.csv}
    x,mu,y0,y1,y2,bU,bL
    0.0,2.96787109791578,3.646442446207024,2.3796779884645356,3.123887098083909,3.8222957144465917,2.1134464813849685
    0.11224489795918367,3.178846435851742,3.7278533056947927,2.705139783400845,3.3197136652913843,3.8425540443223007,2.515138827381183
    0.22448979591836735,3.373957062754166,3.772122911137243,3.033674383762561,3.4864096188544065,3.8416203985733905,2.906293726934942
    0.336734693877551,3.550670833185919,3.7845706480846375,3.3482893361633472,3.6227538754415383,3.8226477418327116,3.2786939245391267
    0.4489795918367347,3.707387470227142,3.777024821961083,3.6497714076425667,3.734021204276469,3.7908281722625117,3.623946768191772
    0.5612244897959183,3.8435001880813098,3.7587623765632605,3.9224326540622303,3.820415281746996,3.938789896922746,3.7482104792398734
    0.673469387755102,3.959374699757277,3.7401691473547714,4.161761533893699,3.884187912930478,4.2112638754378935,3.70748552407666
    0.7857142857142857,4.056245238332488,3.727886967246087,4.3591514373501585,3.9294944134700884,4.440290931214802,3.6721995454501735
    0.8979591836734694,4.136035742591053,3.7320230213633776,4.516870894977162,3.9667169440013335,4.6225623417878845,3.6495091433942224
    1.010204081632653,4.201122318895088,3.7511336294964686,4.631536079365145,3.9909291094494637,4.756943867968448,3.645300769821728
    1.1224489795918366,4.254059537569402,3.7922886212348805,4.703821106353732,4.014456904551665,4.844164391874928,3.663954683263876
    1.2346938775510203,4.297297268072625,3.8510310641448773,4.736836098027459,4.040882180292408,4.886718646421257,3.7078758897239927
    1.346938775510204,4.332916090251523,3.931146658869461,4.737917134427073,4.0721402987298845,4.8886932140093196,3.7771389664937254
    1.4591836734693877,4.362407665942143,4.019692223676497,4.711145152040588,4.114354849026403,4.855508162034269,3.869307169850018
    1.5714285714285714,4.386522000140943,4.113994631302509,4.667147410140193,4.1675113923767135,4.79358919579317,3.9794548044887157
    1.683673469387755,4.405196773426522,4.211161115648332,4.60858632272203,4.2301483229724735,4.709995085814498,4.100398461038546
    1.7959183673469388,4.417575648333239,4.297237502243545,4.542647493156313,4.298601004588312,4.612045273426419,4.223106023240059
    1.9081632653061225,4.422113551855313,4.370044631852721,4.478504417000914,4.366233750854552,4.507185391473455,4.337041712237171
    2.020408163265306,4.416758358107319,4.4212535412567835,4.413073469318789,4.428720525173312,4.439002309014812,4.394514407199825
    2.13265306122449,4.399191007367498,4.4498214644109035,4.354633815643454,4.472837424274621,4.501454599238494,4.296927415496501
    2.2448979591836733,4.367100601462363,4.44669502086167,4.293220973455457,4.495071115055989,4.529355328572067,4.204845874352659
    2.357142857142857,4.318467882753287,4.4122754488830465,4.2367626348642995,4.480392701402552,4.511998980020114,4.124936785486461
    2.4693877551020407,4.251829947856004,4.344361140161457,4.176480274188317,4.420872228223025,4.444440701959786,4.059219193752223
    2.5816326530612246,4.166501022807468,4.242712341391929,4.110274997773214,4.315673648503417,4.324777879750019,4.008224165864918
    2.693877551020408,4.062728357957119,4.110217875771083,4.032563200602017,4.153918593320322,4.154406402240164,3.9710503136740742
    2.806122448979592,3.9417683270387145,3.9492135726749384,3.9344120355000705,3.9443400697750968,3.9571937560826638,3.926342897994765
    2.9183673469387754,3.805875044263767,3.7642279401126544,3.810832282328448,3.684817080524404,3.9345310053498097,3.6772190831777247
    3.0306122448979593,3.6582015833339296,3.564842924761986,3.6644624038944493,3.3833530964865193,3.9269741902831705,3.3894289763846888
    3.142857142857143,3.5026215163692607,3.3520647108430026,3.488370188345563,3.052946676019309,3.9234237680191817,3.0818192647193396
    3.2551020408163267,3.3434853499709174,3.1402141378498163,3.2837887833202726,2.7002390009588226,3.920293063216916,2.7666776367249186
    3.36734693877551,3.1853319635639767,2.934150639260623,3.057322775625835,2.3472996257286054,3.914233966397971,2.4564299607299827
    3.479591836734694,3.03257891484487,2.738827249921258,2.812094834179371,2.009196676123942,3.902114762761585,2.163043066928155
    3.5918367346938775,2.8892171796754873,2.5662919443248073,2.559707152254641,1.6967015390942675,3.881056263819646,1.8973780955313282
    3.704081632653061,2.758535417727554,2.41434042449297,2.309411800258502,1.4256334316849872,3.848437973565799,1.668632861889309
    3.816326530612245,2.642896256015177,2.2902148380401837,2.075552115650442,1.207890471342624,3.8018786148254615,1.4839138972048924
    3.9285714285714284,2.5435825902305362,2.1965744091641124,1.8662299680038845,1.0467598332277432,3.7392081008329594,1.3479570796281128
    4.040816326530612,2.4607259083694415,2.133954232173078,1.6922651032951084,0.952486845042865,3.6584498086819774,1.2630020080569058
    4.153061224489796,2.393321664023437,2.0881938587934696,1.566133253333461,0.92147130773977,3.5578289289014684,1.2288143991454055
    4.26530612244898,2.339329380270647,2.0632588371219063,1.4903968555363714,0.94987473663854,3.435816329632721,1.2428424309085735
    4.377551020408164,2.295848100924048,2.0605594124527933,1.4690236774181895,1.033187881225276,3.2912091804701387,1.3004870213779576
    4.489795918367347,2.259351656203322,2.064341290936802,1.499164363108256,1.1636813952423448,3.123241143061824,1.3954621693448201
    4.6020408163265305,2.2259635269573055,2.0739080755801877,1.575649936757477,1.3294049093439706,2.931708042701583,1.520219011213028
    4.714285714285714,2.1917482915683912,2.0845379827266677,1.6904928495001776,1.5224847717101069,2.717092081155121,1.6664045019816616
    4.826530612244898,2.1529959577040843,2.0900622587437008,1.8320811599608942,1.733755447747318,2.4806793091811996,1.8253126062269691
    4.938775510204081,2.10647694342231,2.087733094036374,1.98722071801456,1.9517354055585423,2.2248945171189956,1.988059369725624
    5.051020408163265,2.049648890548753,2.0747166503580527,2.1438285289379833,2.167467964578078,2.149507699667685,1.9497900814298208
    5.163265306122449,1.9808014834652514,2.050810263655156,2.283421101970534,2.3742039664111614,2.297682746932975,1.663920219997528
    5.275510204081633,1.8991314699538484,2.006424555553889,2.395089681917207,2.5656201066053885,2.429871221330997,1.3683917185766998
    5.387755102040816,1.8047465055032903,1.9408588788404912,2.4682291512970815,2.7418915989000654,2.5411728761483077,1.068320134858273
    5.5,1.69860261545027,1.8531818596862015,2.488627885130089,2.8888780190386125,2.628636682030301,0.7685685488702394
\end{filecontents*}

\begin{tikzpicture}
    \begin{axis}[
            xmin=-0.0,xmax=6.5,
            ymin=-0.5,ymax=6.5,
            axis line style={draw=none},
            tick style={draw=none},
            yticklabels=\empty,
            xticklabels=\empty,
        ]
        \addplot[smooth, color=black, semithick] table [x=x, y=y0, col sep=comma, mark=none] {./data/gp_intro_dat1.csv};
        \addplot[smooth, color=black, semithick, dashed] table [x=x, y=y1, col sep=comma, mark=none] {./data/gp_intro_dat1.csv};
        \addplot[smooth, color=black, semithick, dotted] table [x=x, y=y2, col sep=comma, mark=none] {./data/gp_intro_dat1.csv};

        \addplot[name path = bU, mark=none, blue!10] coordinates {(0,0.5) (5.5,0.5)};
        \addplot[name path = bL, mark=none, blue!10] coordinates {(0,5.75) (5.5,5.75)};
        \addplot [blue!10] fill between [of = bU and bL, soft clip={domain=0:5.5}];
    \end{axis}
    \draw[->,thick] (0,0.5)--(0,5.5) node[above]{$y$};
    \draw[->,thick] (0,0.5)--(6,0.5) node[right]{$x$};
\end{tikzpicture}

\begin{tikzpicture}
    \begin{axis}[
            xmin=-0.0,xmax=6.5,
            ymin=-0.5,ymax=6.5,
            axis line style={draw=none},
            tick style={draw=none},
            yticklabels=\empty,
            xticklabels=\empty,
        ]
        \addplot[smooth, color=black, semithick] table [x=x, y=y0, col sep=comma, mark=none] {./data/gp_intro_dat2.csv};
        \addplot[smooth, color=black, semithick, dashed] table [x=x, y=y1, col sep=comma, mark=none] {./data/gp_intro_dat2.csv};
        \addplot[smooth, color=black, semithick, dotted] table [x=x, y=y2, col sep=comma, mark=none] {./data/gp_intro_dat2.csv};
        \addplot[smooth, color=red, ultra thick] table [x=x, y=mu, col sep=comma, mark=none] {./data/gp_intro_dat2.csv};

        \addplot[name path = bU, smooth, color=blue!10] table [x=x, y=bU, col sep=comma, mark=none] {./data/gp_intro_dat2.csv};
        \addplot[name path = bL, smooth, color=blue!10] table [x=x, y=bL, col sep=comma, mark=none] {./data/gp_intro_dat2.csv};
        \addplot [blue!10] fill between [of = bU and bL, soft clip={domain=0:5.5}];
    \end{axis}
    \draw[->,thick] (0,0.5)--(0,5.5) node[above]{$y$};
    \draw[->,thick] (0,0.5)--(6,0.5) node[right]{$x$};
\end{tikzpicture}


\subsection{Gaussian Processes for Regression}\label{Section1.5}

We saw in section \ref{Section1.4} that Gaussian processes directly predicts the value function we are seeking to predict instead of parameteric values. To keep this computation tractable we only evalute our predicted function at a finite number of points. The prediction itself is found by taking the mean over all functions with respect to the posterior conditioned on the observed data which is assumed to be jointly Gaussian with the input value. This gives rise to Gaussian Process more formally stated in definition

\begin{defe}[Gaussian Process] \label{defe: GP}
    A Gaussian Process (GP) is a collection of random variables with index set $I$, such that every finite subset of random variables has a joint Gaussian distribution \cite{RasmussenCarlEdward2006Gpfm,MurphyKevinP2012Ml}.
\end{defe}

A GP is completely characterised by a mean function $m(\bm{x})$ and a kernel, which in the context of GPs is sometimes called a covariance function, $k (\bm{x}, \bm{x'})$ on a real process as
\begin{align*}
    m(\bm{x})           & = \EE \left[ f(\bm{x}) \right]                                         \\
    k (\bm{x}, \bm{x'}) & = \EE \left[ (f(\bm{x}) - m(\bm{x})) (f(\bm{x'}) - m(\bm{x'})) \right]
\end{align*}
GPs define a prior over all possible functions which can be used to create a posterior once enough data has been observed. The prior is used to represent the functions we expect to see before any observations are made. Although defining a prior over all possible function may seem computationally intractable, we actually only need to define a distribution over a finite number of points. Before any observations are made, we typically assume that the mean function is the constant zero function, that is $m \left( \bm{x} \right) = 0$. A function $f(\bm{x})$ sampled from a GP with mean $m(\bm{x})$ and covariance $k (\bm{x}, \bm{x'})$ is written as
\[
    f(\bm{x}) \sim \calG \calP \left( m(\bm{x}), k (\bm{x}, \bm{x'}) \right)
\]
Since a GP is a collection of random variables it must satisfy the consistency requirement, that is, an observation of a set of variables should not the distribution of any small sub set of the observed values. More specifically if
\[
    (\bm{y_1}, \bm{y_2}) \sim \calN (\bm{\mu}, \bm{\Sigma})
\]
then
\begin{align*}
    \bm{y_1} & \sim \calN (\bm{\mu_1}, \bm{\Sigma_{1,1}}) \\
    \bm{y_2} & \sim \calN (\bm{\mu_2}, \bm{\Sigma_{2,2}})
\end{align*}

where $\bm{\Sigma_{1,1}}$ and $\bm{\Sigma_{2,2}}$ are the relevant sub matrices. Again, we shall us the notation that for set of data $\bm{W} = \left[ \bm{w}_1 ,\bm{w}_2 , \ldots , \bm{w}_n \right]^{\intercal} \in \RR^{n \times d}$ and $\bm{W}' = \left[ \bm{w}_1' ,\bm{w}_2' , \ldots , \bm{w}_m' \right]^{\intercal} \in \RR^{n' \times d}$ we use the notation
\[
    \left( \bm{K}_{\bm{W} \bm{W}'} \right)_{i,j} \triangleq k \left( \bm{w}_i , \bm{w}_j' \right)
\]
where \( \bm{K}_{\bm{W} \bm{W}'} \in \RR^{n \times n'} \). The convariance function completely characterized by its kernel. To understand this better, as a small exercise we can select a number of inputs $\bm{X}^{\ast} = \left[ \bm{x}_1 , \bm{x}_2 , \ldots , \bm{x}_{n^{\ast}} \right]^{\intercal} \in \RR^{n^{\ast} \times d}$ of compute the corresponding covariance matrix using, as an example, the Gaussian RBF kernel. Gaussian vectors can then be sampled using a joint Gaussian distribution using the covariance matrix from the distribution
\[
    \bm{f} \sim \calN \left( \bm{0}, \bm{K_{XX}} \right)
\]
and its values graphed as a function of its inputs. Figure \ref{fig: motive_gp_1} (C) shows functions drawn from the prior before any observations are made. GPs also allow us to compute the pointwise variance which can provide some measure of variability for predicted values. The blue shaded area of figure \ref{fig: motive_gp_1} (C) represents twice the standard deviation about the mean.

\subsubsection{Noise-free observations}\label{Section1.4.1}
Typically when using GP we would like to incorporate data from observations, or training data, into our predictions on unobserved values.
Let us suppose there is some obsevered data $D = \left\{ (\bm{x}_i, \bm{f}_i) \mid i \in \left\{ 1,2, \ldots , n \right\} \right\}$ which is (unrealistically) noise-free that we would like to model as a GP. In other words, for any sample in our dataset we can be certain that the observed value is the true value of the underlying function we wish to model. Then for the observed data
\[
    \bm{f} \sim \calN \left( \bm{0}, \bm{K_{XX}} \right).
\]
where $\bm{K_{XX}} = k(\bm{X}, \bm{X}) \in \RR^{n \times n}$. We would then like to make a prediction for unobserved values say $X^{\ast} = \left[ \bm{x}_1^{\ast}, \bm{x}_2^{\ast}, \ldots , \bm{x}_{n_\ast}^{\ast} \right]$ with value $f_{\ast}$ as has a distribution of
\[
    \bm{f}_{\ast} \sim \calN \left( \bm{0}, \bm{K_{X^{\ast}X^{\ast}}} \right).
\]
where $\bm{K_{X^{\ast}X^{\ast}}} = k(\bm{X^{\ast}}, \bm{X^{\ast}}) \in \RR^{n_\ast \times n_\ast}$. Here $\bm{f}$ and $\bm{f}_{\ast}$ are independent but we would like to give them some sort of correlation. We can do this by having them originate from the same joint distribution. According to the prior, we can write the joint distribution of the training points $\bm{f}$ and the test points $\bm{f}_{\ast}$ as
\[
    \begin{pmatrix}
        \bm{f} \\
        \bm{f}_{\ast}
    \end{pmatrix}
    \sim \calN
    \begin{pmatrix}
        \bm{0}, &
        {
                \begin{pmatrix}
                    \bm{K_{XX}}                    & \bm{K_{XX^{\ast}}}        \\
                    \bm{K_{XX^{\ast}}}^{\intercal} & \bm{K_{X^{\ast}X^{\ast}}}
                \end{pmatrix}
            }
    \end{pmatrix}
\]
where $\bm{K_{XX^{\ast}}} = k(\bm{X}, \bm{X^{\ast}}) \in \RR^{n \times n_\ast}$.

While the above does give us some information on $\bm{f}_{\ast}$ is related to the observed data and the test inputs, it does not provide any method to evalute $\bm{f}_{\ast}$. To do this we shall need the assistance of the following lemma
\begin{thm}\label{theorem: cond_of_MVN}
    (Marginals and conditionals of an MVN \cite{MurphyKevinP2012Ml}) Suppose $\bm{x} = (\bm{x}_1, \bm{x}_2)$ is jointly Gaussian with parameters
    \[
        \bm{\mu} =
        \begin{pmatrix}
            \bm{\mu}_1 \\
            \bm{\mu}_2
        \end{pmatrix}, \quad
        \bm{\Sigma} =
        \begin{pmatrix}
            \bm{\Sigma}_{1,1} & \bm{\Sigma}_{1,2} \\
            \bm{\Sigma}_{2,1} & \bm{\Sigma}_{2,2}
        \end{pmatrix}
    \]
    then the posterior conditional is given by
    \begin{align*}
        \bm{x}_2 \mid \bm{x}_1 & \sim \calN \left( \bm{x}_2 \mid \bm{\mu}_{2 \mid 1}, \bm{\Sigma}_{2 \mid 1} \right)          \\
        \bm{\mu}_{2 \mid 1}    & = \bm{\mu}_2 + \bm{\Sigma}_{2,1} \bm{\Sigma}_{1,1}^{-1} \left( \bm{x}_1 - \bm{\mu}_1 \right) \\
        \bm{\Sigma}_{2 \mid 1} & = \bm{\Sigma}_{2,2} - \bm{\Sigma}_{2,1} \bm{\Sigma}_{1,1}^{-1} \bm{\Sigma}_{1,2}
    \end{align*}
\end{thm}

Thus finding a mean an covariance for $\bm{f}_{\ast}$ requires a direct application of Theorem \ref{theorem: cond_of_MVN} which gives
\begin{align*}
    \bm{f}_{\ast} \mid \bm{K_{XX^{\ast}}} , \bm{K_{XX}}, \bm{f} \sim \calN \left( \bm{\mu}^{\ast}, \bm{\Sigma}^{\ast} \right)
\end{align*}
where
\begin{align*}
    \bm{\mu}^{\ast} & = \bm{0} + \bm{K_{XX^{\ast}}}^{\intercal} \bm{K_{XX}}^{-1} \left( \bm{f} - \bm{0} \right) \\
                    & = \bm{K_{XX^{\ast}}}^{\intercal} \bm{K_{XX}}^{-1} \bm{f}
\end{align*}
and
\begin{align*}
    \bm{\Sigma}^{\ast} & = \bm{K_{X^{\ast}X^{\ast}}} - \bm{K_{XX^{\ast}}}^{\intercal} \bm{K_{XX}}^{-1} \bm{K_{XX^{\ast}}}
\end{align*}
meaning we can write a distribution for $\bm{f}_{\ast}$ as
\begin{equation}\label{prop:GP_train_distr1}
    \bm{f}_{\ast} \mid \bm{K_{XX^{\ast}}} , \bm{K_{XX}}, \bm{f} \sim \calN \left( \bm{K_{XX^{\ast}}}^{\intercal} \bm{K_{XX}}^{-1} \bm{f},  \bm{K_{X^{\ast}X^{\ast}}} - \bm{K_{XX^{\ast}}}^{\intercal} \bm{K_{XX}}^{-1} \bm{K_{XX^{\ast}}}  \right)
\end{equation}
Function values from the unobserved inputs $\bm{X^{\ast}}$ can be estimated using the mean of $\bm{f}_{\ast}$ evaluted in \ref{prop:GP_train_distr1}.

\subsubsection{Prediction with Noisy observations}\label{Section1.1.2}
When attempting to model our value function we usually do not have access to the value function itself but a noisy version thereof, $y = f(\bm{x}) + \varepsilon$ where $\varepsilon \calN (0, \sigma_n^2)$ meaning the prior on the noisy values becomes
\[
    \operatorname{cov} (\bm{y}) = \bm{K_{XX}} + \sigma_n^2 \bm{I}
\]
The reason why noise is only added along the diagonal follows from the assumption of independence in our data.
We can write out the new distribution of the observed noisy values along the points at which we wish to test the underlying function as
\[
    \begin{pmatrix}
        \bm{f} \\
        \bm{f}_{\ast}
    \end{pmatrix}
    \sim \calN
    \begin{pmatrix}
        \bm{0}, &
        {
                \begin{pmatrix}
                    \bm{K_{XX}} + \sigma_n^2 \bm{I} & \bm{K_{XX^{\ast}}}        \\
                    \bm{K_{XX^{\ast}}}^{\intercal}  & \bm{K_{X^{\ast}X^{\ast}}}
                \end{pmatrix}
            }
    \end{pmatrix}
\]
Using a similar we arrive at a similar condition distribution of $\bm{f}_{\ast} \mid \bm{K_{XX^{\ast}}} , \bm{K_{XX}}, \bm{f}$ we arrive at one of the most fundamental equations for GP regression tasks
\begin{align*}\label{prop:GP_train_distr2}
    \bm{f}_{\ast} \mid \bm{K_{XX^{\ast}}} , \bm{K_{XX}}, \bm{y} \sim & \calN \left( \overline{\bm{f}_{\ast}}, \operatorname{cov} (\bm{f}_{\ast}) \right)                                                   \\
    \overline{\bm{f}_{\ast}}                                         & := \bm{K_{XX^{\ast}}}^{\intercal} \left[ \bm{K_{XX}} + \sigma_n^2 \bm{I} \right]^{-1} \bm{y}                                        \\
    \operatorname{cov} (\bm{f}_{\ast})                               & = \bm{K_{X^{\ast}X^{\ast}}} - \bm{K_{XX^{\ast}}}^{\intercal} \left[ \bm{K_{XX}} + \sigma_n^2 \bm{I} \right]^{-1} \bm{K_{XX^{\ast}}}
\end{align*}

\subsection{Gaussian Processes for Classification}\label{Section1.6}

As with most classification models, the Gaussian processes classifier (GPC) seeks an estimate for the joint probability $p \left( y , \bm{x} \right)$ where $\bm{x} \in \RR^d$ is an input, as in the regression case, but $y$ is now a class taking on a discrete and finite number of values $\left\{ \calC_i \right\}_{i=1}^C$. Using Baye's theorem the joint probability density can be decomposed into either $p \left( y \right) p \left( \bm{x} \mid y \right)$ or $p \left( \bm{x} \right) p \left( \bm{y} \mid \bm{x} \right)$ giving rise to the {\it generative} and {\it discriminative} approaches respectively \cite{RasmussenCarlEdward2006Gpfm}*{page 34}. The generative approach models the prior probabilities of each class, $p \left( \calC_i \right)$, as well as the class conditional probabilities for each input $p \left( \bm{x} \mid \calC_i \right)$ and computes the posterior as
\[
    p \left( y \mid \bm{x} \right) = \frac{ p \left( y \right) p \left( \bm{x} \mid y \right) }{ \sum_{i=1}^{C} p \left( \calC_i \right) p \left( \bm{x} \mid \calC_i \right) }.
\]
On the other hand, the discriminative method focuses on modelling $p \left( y \mid \bm{x} \right)$ directly. With both these paradigms at our disposal, which one would be preferred for our GPC? While there are strengths and weaknesses associated with both models, the discriminative approach is usually chosen as it has a rather attractive property of directly modeling what we require, that is $p \left( y \mid \bm{x} \right)$. Aditionally, the density estimation of $p \left( \bm{x} \mid \calC_i \right)$ using in the generative model presents a number of difficulties, especially for larger values of $d$. If we are only focused on classifying inputs, the generative approach could mean trying to solve a harder problem than what is necessary. For this reason we focus on GPCs that adopt the discriminative approach.

\subsubsection{Linear Models for Classification}\label{Section1.6.1}

We can start by reviewing linear models for the simplist form of classification, that is binary classification. Adopting the notation from SVM (see \Cref{Section1.4.1}) literature, the binary classification problem involves assigning an input $\bm{x}$ to a class of either $-1$ or $+1$. For a linear model likelihood can be formulated as
\begin{equation} \label{eq: GPC-lin-model-1}
    p \left( y=+1 \mid \bm{x} , \bm{w} \right) = \sigma \left( \langle \bm{x} , \bm{w} \rangle \right)
\end{equation}
given a weight vector $\bm{w}$ and where $\sigma (\bm{z})$ is chosen to be any sigmoid function, see \Cref{defe: sigmoid-function}.
\begin{defe}[Sigmoid Function] \label{defe: sigmoid-function}
    A sigmoid function is a monotonically increasing function mapping from $\RR$ to $\left[ 0,1 \right]$ \cite{RasmussenCarlEdward2006Gpfm}*{page 35}.
\end{defe}
In this text, the commonly used logistic function
\begin{equation} \label{eq: logistic-function}
    \sigma (z) = \frac{1}{1 + \exp (-z)}
\end{equation}
will take the role of the sigmoid function in equation \ref{eq: GPC-lin-model-1}, graphed in Figure \ref{fig: logistic-func-and-probit}. This type of model is aptly named the logistic regression.
\begin{figure}[h]
    \centering
    \begin{tikzpicture}[>=latex, scale=1.1]
        \begin{axis}[
                axis line style = thick,
                xlabel={$x$},
                ylabel={$y$},
                every axis x label/.style={
                        at={(ticklabel* cs:1.01)},
                        anchor=west,
                    },
                xmin=-5.5,xmax=5.5,
                ymin=-0.125,ymax=1.125,
                axis y line =middle,
                axis x line =bottom,
                every axis y label/.style={
                        at={(ticklabel* cs:1.01)},
                        anchor=south,
                    },
                % axis line style={->},
                xticklabels={$-5$, $5$},
                xtick={-5,5},
                yticklabels={$0$, $1$},
                ytick={-0.05,0.95},
                % axis line style={draw=none},
                ytick style={draw=none},
            ]
            \addplot[smooth, color=red, thick] {1/(1+exp(-x))};
            \addplot [smooth, blue, thick, dashed] {normcdf((0.6266 * x),0,1)};
            \addplot[smooth, color=black, semithick, dotted] {0};
            \addplot[smooth, color=black, semithick, dotted] {1};
        \end{axis}
    \end{tikzpicture}
    \caption{The logistic function from equation \ref{eq: logistic-function} (solid red) juxtaposed with a close approximation, the scaled probit function (dashed blue).}
    \label{fig: logistic-func-and-probit}
\end{figure}
Unlike GPR, the likelihood is no longer a Gaussian distribution. Instead it follows the Bernoulli distribution
\begin{equation*}
    p \left( y \mid \bm{x} , \bm{w} \right) = \sigma \left( \langle \bm{x} , \bm{w} \rangle \right)^{y} \left( 1 - \sigma \left( \langle \bm{x} , \bm{w} \rangle \right) \right)^{\frac{1 - y}{2}}
\end{equation*}
which for symmeteric likelihood functions can be written more concisely as
\begin{equation*}
    p \left( y_i \mid \bm{x}_i , \bm{w} \right) = \sigma \left( y_i f_i \right)
\end{equation*}
where
\begin{equation} \label{eq: GPC-lin-latent-func}
    f_i \triangleq f \left( \bm{x}_i \right) = \langle \bm{x} , \bm{w} \rangle .
\end{equation}
Thus, the logistic regression model can be written as the log ratio of the likelihoods of the input belonging to either class, that is
\begin{equation*}
    \operatorname{logit} \left( \bm{x} \right) \triangleq \langle \bm{x} , \bm{w} \rangle = \log \left( \frac{p \left( y = +1 \right)}{p \left( y = -1 \right)} \right)
\end{equation*}
where $\operatorname{logit}$ is commonly referred to as the logit transformation \cite{RasmussenCarlEdward2006Gpfm}*{page 37}. For a given dataset $\calD = \left\{ \left( x_i , y_i \right) \right\}_{i=1}^{n}$ we assume each observation is independently generated conditioned over $f \left( \bm{x} \right)$. Similar to GPR, a Gaussian prior is used for the weights so that $\bm{w} \sim \calN \left( \bm{0} , \sigma_p \right)$ giving an un-normalised log posterior of
\begin{equation*}
    \log p \left( \bm{w} \mid \bm{X} , \bm{y} \right) \propto - \frac{1}{2} \bm{w}^{\intercal} \Sigma_p^{-1} \bm{w} + \sum_{i=1}^{n} \log \sigma \left( y_i f_i \right).
\end{equation*}

However, unlike GPR an analytic form for the mean and variance for the posterior is not available due to the non-Gaussian nature of the likelihood, although, when using the logistic function it is easy enough to show that the log likelihood is concave as a function of $\bm{w}$ for a fixed dataset. This means a number of numerical optimization techniques, such as Newton's method or the Broyden-Fletcher-Goldfarb-Shanno (BFGS) algorithm \cite{FletcherR2000PMoO} can be used to solve these values.

The idea behind Gaussian process classification for binary classes is that a Gaussian process prior is placed over a latent function $f \left( \bm{x} \right)$ where the output is then "squashed" through a sigmoid function to obtain a prior on
\begin{equation*}
    \pi \left( \bm{x} \right) \triangleq p \left( y=+1 \mid \bm{x} \right) = \sigma \left( f \left( \bm{x} \right) \right).
\end{equation*}
This construction is illustrated in Figure \ref{fig: latent-func-and-sig-trans} and provides a natural extension to the linear logistic regression model.

\begin{filecontents*}{./data/gpr_latent_fig1.csv}
    x,f,sig
    0.0,-0.716636450452456,0.3281340890410802
    0.05555555555555555,-0.3872677518321085,0.4043752064908497
    0.1111111111111111,-0.04169388251654786,0.48957803910434766
    0.16666666666666666,0.30748545475025907,0.5762713707991717
    0.2222222222222222,0.6464740783985357,0.6562154650063955
    0.2777777777777778,0.961561586658563,0.7234343518272932
    0.3333333333333333,1.2404054048798008,0.7756345729728175
    0.38888888888888884,1.4731593678003678,0.8135371194857097
    0.4444444444444444,1.653300565910448,0.8393366325299798
    0.5,1.77804434211388,0.8554552137616092
    0.5555555555555556,1.8482947284219486,0.8639267602259872
    0.611111111111111,1.8681477880300634,0.8662438156048382
    0.6666666666666666,1.844024635176618,0.8634239990093878
    0.7222222222222222,1.7835778456043871,0.8561380953611345
    0.7777777777777777,1.6945293075464642,0.8448188796360699
    0.8333333333333333,1.5836227815051651,0.8297169804398394
    0.8888888888888888,1.4558461620643777,0.8108965338173831
    0.9444444444444444,1.3140273314869115,0.7881862938214962
    1.0,1.1588551707012498,0.7611246311602722
    1.0555555555555556,0.9893037805660674,0.7289503840440278
    1.1111111111111112,0.8033735795719339,0.690695659627058
    1.1666666666666665,0.5990205172125318,0.6454321840431138
    1.222222222222222,0.37510931242033974,0.5926929891152688
    1.2777777777777777,0.13223323848313873,0.5330102232552971
    1.3333333333333333,-0.12672784735928108,0.46836037106151224
    1.3888888888888888,-0.3964103237476957,0.40217510289406855
    1.4444444444444444,-0.6691517706610377,0.33868679954793734
    1.5,-0.9355408280853343,0.28180195391419693
    1.5555555555555554,-1.1852457987537584,0.23411029892781032
    1.611111111111111,-1.408010271880984,0.196548078418454
    1.6666666666666665,-1.5946766330839104,0.16872694438891356
    1.722222222222222,-1.7381092175623238,0.1495532583157629
    1.7777777777777777,-1.8339034326105361,0.13777391959330848
    1.8333333333333333,-1.8808066358911115,0.13229624915295843
    1.8888888888888888,-1.8808148704868526,0.13229530387403557
    1.9444444444444444,-1.8389575421436446,0.1371746288812035
    2.0,-1.762808219338528,0.14643897890898566
    2.0555555555555554,-1.6617924461232692,0.15952152909513476
    2.111111111111111,-1.5463718876794204,0.17561089816790565
    2.1666666666666665,-1.4271799785358956,0.1935384565574492
    2.2222222222222223,-1.3141764578300106,0.21178881080529863
    2.2777777777777777,-1.2158759418012686,0.22866302240021388
    2.333333333333333,-1.1386790564398872,0.24256297036803995
    2.388888888888889,-1.0863393391090705,0.2523082312707144
    2.444444444444444,-1.0595738916176571,0.2573908929832913
    2.5,-1.0558426697384702,0.25810472715213173
    2.5555555555555554,-1.0693030523966423,0.2555356469279933
    2.611111111111111,-1.0909711640146635,0.25143544599371426
    2.6666666666666665,-1.1090929165853411,0.24804003562062307
    2.722222222222222,-1.1097459725310912,0.24791825016606753
    2.7777777777777777,-1.0776532758488337,0.25395036925471315
    2.833333333333333,-0.997190360006935,0.26949418860886254
    2.888888888888889,-0.853523705726157,0.2986942026351552
    2.944444444444444,-0.6338078056665624,0.34664763152253814
    3.0,-0.3283465008205865,0.4186429992846845
    3.0555555555555554,0.06838484205618314,0.5170895511116412
    3.1111111111111107,0.5569529512258093,0.6357472192793857
    3.1666666666666665,1.132487738026538,0.7562977100837649
    3.222222222222222,1.7845801849827891,0.8562615050851906
    3.2777777777777777,2.49755753518847,0.9239704166376049
    3.333333333333333,3.2511324784312694,0.9627137853569252
    3.388888888888889,4.021376394857483,0.9823874902884194
    3.444444444444444,4.781943394168808,0.9916899377360553
    3.5,5.505458755235512,0.995951929944337
    3.5555555555555554,6.164984744717579,0.9979026576630112
    3.6111111111111107,6.735466982435726,0.9988133894314339
    3.6666666666666665,7.1950796833818895,0.9992502941814919
    3.722222222222222,7.526406822593906,0.9994616196879862
    3.7777777777777777,7.717385130530989,0.9995551751969357
    3.833333333333333,7.761974187627159,0.9995745655656185
    3.888888888888889,7.660512373107247,0.9995291558112748
    3.944444444444444,7.419735623560392,0.9994010513870635
    4.0,7.052442559921735,0.999135455240024
    4.055555555555555,6.5768311951578315,0.9986096815828182
    4.111111111111111,6.0155205200837125,0.9975653642534463
    4.166666666666666,5.394318761388515,0.9954782255735908
    4.222222222222222,4.740803814055081,0.991343956562032
    4.277777777777778,4.082820560277176,0.9834196970094687
    4.333333333333333,3.446989970396544,0.9691412482325584
    4.388888888888888,2.857339498962172,0.9456968331619086
    4.444444444444445,2.334160434608953,0.911666952206589
    4.5,1.8931622621329902,0.8691156694032487
    4.555555555555555,1.544984634180779,0.8241881763838215
    4.611111111111111,1.295071693838633,0.7850043888236089
    4.666666666666666,1.1439101905755478,0.7583968311600912
    4.722222222222222,1.087555379614477,0.7479211041915879
    4.777777777777778,1.1183780186296264,0.7536877316301696
    4.833333333333333,1.2259536925758674,0.7731095912803282
    4.888888888888888,1.3979718665845051,0.8018618571626841
    4.944444444444444,1.6211061459040699,0.8349476243586275
    5.0,1.8817742817340295,0.8678147912195052
    5.055555555555555,2.1667472795664557,0.8972234089235315
    5.111111111111111,2.4636123746267415,0.9215512159171637
    5.166666666666666,2.761092162073636,0.9405367451793998
    5.222222222222222,3.049269824821768,0.9547509923004637
    5.277777777777778,3.3197364525186464,0.9650997156829392
    5.333333333333333,3.5657178376098226,0.9725009030361754
    5.388888888888888,3.7821806293058837,0.9777340834061939
    5.444444444444444,3.9659265896685927,0.9814019721867842
    5.5,4.115652418864887,0.983946623009968
\end{filecontents*}

\begin{figure}[h]
    \centering
    \subfloat[]{
        \begin{adjustbox}{width=0.48\textwidth}
            \begin{tikzpicture}[>=latex]
                \begin{axis}[
                        xmin=-0.0,xmax=6.5,
                        ymin=-4.0,ymax=10.0,
                        axis line style={draw=none},
                        tick style={draw=none},
                        yticklabels=\empty,
                        xticklabels=\empty,
                    ]
                    \addplot[smooth, color=red, thick] table [x=x, y=f, col sep=comma, mark=none] {./data/gpr_latent_fig1.csv};
                \end{axis}
                \draw[->,thick] (0,0.5)--(0,5.5) node[above]{$f(x)$};
                \draw[->,thick] (0,0.5)--(6,0.5) node[right]{$x$};

                \draw[dotted] (5.8,0.7)--(0,0.7) node[left]{$-4$};
                \draw[dotted] (5.8,5.1)--(0,5.1) node[left]{$9$};
            \end{tikzpicture}
        \end{adjustbox}
    }
    \subfloat[]{
        \begin{adjustbox}{width=0.48\textwidth}
            \begin{tikzpicture}[>=latex]
                \begin{axis}[
                        xmin=-0.0,xmax=6.5,
                        ymin=-0.05,ymax=1.15,
                        axis line style={draw=none},
                        tick style={draw=none},
                        yticklabels=\empty,
                        xticklabels=\empty,
                    ]
                    \addplot[smooth, color=red, thick] table [x=x, y=sig, col sep=comma, mark=none] {./data/gpr_latent_fig1.csv};
                \end{axis}
                \draw[->,thick] (0,0.5)--(0,5.5) node[above]{$\sigma(f(x)$)};
                \draw[->,thick] (0,0.5)--(6,0.5) node[right]{$x$};

                \draw[dotted] (5.8,0.7)--(0,0.7) node[left]{$0$};
                \draw[dotted] (5.8,5.1)--(0,5.1) node[left]{$1$};
            \end{tikzpicture}
        \end{adjustbox}
    }
    \caption{The latent function $f$, panel (A), is transformed using a sigmoid function, panel (B), to provide a probabilistic interpretation of $x$ belonging to the class $+1$.}
    \label{fig: latent-func-and-sig-trans}
\end{figure}
Specifically, the linear model from equation \ref{eq: GPC-lin-latent-func} is replaced with a GPR model and the Gaussian prior on the weights with a GPR weight prior with
\begin{equation*}
    p \left(
    \begin{bmatrix}
            \bm{f} \\
            f_{\star}
        \end{bmatrix}
    \right)
    =
    \calN \left( \bm{0} ,
    \begin{bmatrix}
        \bm{K_{XX}}         & \bm{K_{x^{\star}X}}^{\intercal}                 \\
        \bm{K_{x^{\star}X}} & k \left( \bm{x}_{\star}, \bm{x}_{\star} \right)
    \end{bmatrix}
    \right)
\end{equation*}
where $f_{\star} = f ( \bm{x}_{\star} )$ and $\bm{f} = f \left( \bm{X} \right)$. For classification tasks, we assume that each observation has received the correct label which is why no noise is added to the covariance matrix.

Note that values of $f$ are also never observed within the phenomena we are modelling, nor are we particularly interested in them. The function $f$ serves the role of a {\it nuisance function} and acts solely as a convenience tool within our formulations. The ultimate goal is to make predictions for $\pi$, not $f$, and the goal of the coming sections will be to eventually integrate out $f$.

Subsequently, predictions for $\pi_{\star} = \pi \left( \bm{x}_{\star} \right)$ are made by average over all possible latent functions weighted by the posterior giving the prediction
\begin{equation} \label{eq: GP-pred-1}
    \overline{\pi_{\star}} \triangleq p \left( y_{\star} = +1 \mid \bm{X} , \bm{y} , \bm{x}_{\star} \right) = \int \sigma \left( f_{\star} \right) p \left( f_{\star} \mid \bm{X} , \bm{y} , \bm{x}_{\star} \right) \; d f_{\star}
\end{equation}
While this is a sound model, computing predictions is not so straight forward since the integral in \ref{eq: GP-pred-1} is not analytically tractable for the same reason as the linear binary classifier. Later on we will see how we can make use of our numerical toolbox to derive a good approximation for $\overline{\pi_{\star}}$.

\subsubsection{Lapace Approximation for Posterior}\label{Section1.6.2}

We saw that the integral in \ref{eq: GP-pred-1} could not be used to make predictions for $\overline{\pi_{\star}}$ analytically. In this section we shall address how the distribution for the latent process, $p \left( f_{\star} \mid \bm{X} , \bm{y} , \bm{x}_{\star} \right)$, can be approximated to provide a numerically tractable succedaneum. Using Baye's theorem
\begin{align*}
    p \left( f_{\star} \mid \bm{X} , \bm{y} , \bm{x}_{\star} \right)
     & = \int p \left( f_{\star} , \bm{f} \mid \bm{X} , \bm{y} , \bm{x}_{\star} \right) \; d \bm{f}                                                                                                                                                       \\
     & = \frac{1}{p \left( \bm{y} \mid \bm{X} , \bm{x}_{\star} \right)} \int p \left( f_{\star} \mid \bm{X} , \bm{x}_{\star}, \bm{f} \right) p \left( \bm{f} \mid \bm{X} \right) p \left( \bm{y} \mid \bm{X} , \bm{x}_{\star}, \bm{f} \right) \; d \bm{f} \\
     & = \int p \left( f_{\star} \mid \bm{X} , \bm{x}_{\star}, \bm{f} \right) p \left( \bm{f} \mid \bm{X} , \bm{y} \right) \; d \bm{f}
\end{align*}
using the fact that $p \left( \bm{y} \mid \bm{X} , \bm{x}_{\star}, \bm{f}, f_{\star} \right) = p \left( \bm{y} \mid \bm{X} , \bm{x}_{\star}, \bm{f} \right)$ \cite{BishopChristopherM2006Pram, RasmussenCarlEdward2006Gpfm}. The conditional distribution $p \left( f_{\star} \mid \bm{X} , \bm{x}_{\star}, \bm{f} \right)$ can be derived as
\begin{equation*}
    p \left( f_{\star} \mid \bm{X} , \bm{x}_{\star}, \bm{f} \right) = \calN \left( f_{\star} \mid \bm{K}_{\bm{x}_{\star}\bm{X}} \bm{K}_{\bm{X} \bm{X}}^{-1} \bm{y}, k \left( \bm{x}_{\star}, \bm{x}_{\star} \right) - \bm{K}_{\bm{x}_{\star}\bm{X}} \bm{K}_{\bm{X} \bm{X}}^{-1} \bm{K}_{\bm{x}_{\star} \bm{X}}^{\intercal} \right)
\end{equation*}
through the use of equation \ref{eq: GP_train_distr2_mean} and \ref{eq: GP_train_distr2_var}. Unfortunately
\begin{equation*}
    p \left( \bm{f} \mid \bm{X} , \bm{y} \right) = \frac{p \left( \bm{y} \mid \bm{f} \right) p \left( \bm{f} \mid \bm{X} \right) }{p \left( \bm{y} \mid \bm{X} \right)}
\end{equation*}
does not follow a Gaussian distribution. Instead we can use a Lapace approximation to estimate $p \left( \bm{f} \mid \bm{X} , \bm{y} \right)$ as a Gaussian distribution. Breifly, the Lapace approximation works by assuming the distribution at hand, $p \left( \bm{z} \right)$, can be modelled as
\begin{equation*}
    p \left( \bm{z} \right) = \frac{1}{c} q \left( \bm{z} \right)
\end{equation*}
where $q \left( \bm{z} \right)$ is multivariate Gaussian and $c$ is some normalization constant \cite{BishopChristopherM2006Pram}*{page 214}. To do this, first the centre of $q \left( \bm{z} \right)$ is placed at the mode of $p \left( \bm{z} \right)$. The mode of $p \left( \bm{z} \right)$ is
\begin{equation*}
    \bm{z}_0 = \argmin_{\bm{z}} p \left( \bm{z} \right)
\end{equation*}
which can be computed by solving
\begin{equation} \label{eq: lapace-grad-zero}
    \nabla p \left( \bm{z}_0 \right) = \bm{0}.
\end{equation}
To ensure the covariance of the synthesized multivariate Gaussian behaves similar to the original distribution we can make use of an important property of the Gaussian distribution which is its logarithm being is a quadratic function of its inputs. Taking the Taylor series expansion of $\ln q \left( \bm{z} \right)$ centered at $\bm{z}_0$ yields
\begin{equation*}
    \ln q \left( \bm{z} \right) \simeq \ln q \left( \bm{z}_0 \right) - \frac{1}{2} \left( \bm{z} - \bm{z}_0 \right)^{\intercal} \bm{A} \left( \bm{z} - \bm{z}_0 \right)
\end{equation*}
where
\begin{equation*}
    \bm{A} = - \nabla \nabla \left. \ln q \left( \bm{z} \right) \right|_{\bm{z} = \bm{z}_0}.
\end{equation*}
Expotentiating both sides gives
\begin{align}
    q \left( \bm{z} \right)
     & \simeq q \left( \bm{z}_0 \right) \exp \left( - \frac{1}{2} \left( \bm{z} - \bm{z}_0 \right)^{\intercal} \bm{A} \left( \bm{z} - \bm{z}_0 \right) \right) \nonumber \\
     & \propto \calN \left( \bm{z} \mid \bm{z}_0 , \bm{A}^{-1} \right) \label{eq: lapace-gauss-apprx} .
\end{align}
Returning to our original problem of estimating $p \left( \bm{f} \mid \bm{X} , \bm{y} \right) \propto p \left( \bm{y} \mid \bm{f} \right) p \left( \bm{f} \mid \bm{X} \right)$ as a Gaussian distribution, the prior $p \left( \bm{f} \mid \bm{X} \right)$ follows a Gaussian distribution with zero mean and covariance $\bm{K}_{\bm{X} \bm{X}}$ and the distribution of $p \left( \bm{y} \mid \bm{f} \right)$ (assuming independence of samples) can be written as
\begin{equation*}
    p \left( \bm{y} \mid \bm{f} \right) = \prod_{i=1}^{n} \sigma \left( y_i f_i \right).
\end{equation*}
To find a Laplace approximation for $p \left( \bm{f} \mid \bm{X} , \bm{y} \right)$ we only need to consider an unnormalized posterior when maximizing with respect to $\bm{f}$ since $p \left( \bm{y} \mid \bm{f} \right)$ does not depend on $\bm{f}$. Thus, the log of the unnormalized posterior is
\begin{align*}
    \Psi \left( \bm{f} \right)
     & \triangleq \ln p \left( \bm{y} \mid \bm{f} \right) + \ln p \left( \bm{f} \mid \bm{X} \right)                                                                                                                                \\
     & = - \sum_{i=1}^{n} \ln \left( 1 + \exp \left( y_i f_i \right) \right) - \frac{1}{2} \bm{f}^{\intercal} \bm{K}_{\bm{X} \bm{X}}^{-1} \bm{f} - \frac{1}{2} \ln \left| \bm{K}_{\bm{X} \bm{X}} \right| - \frac{n}{2} \ln 2 \pi .
\end{align*}
The gradient and Hessian of the unnormalized posterior then becomes
\begin{align*}
    \nabla \Psi \left( \bm{f} \right)        & = \nabla \ln p \left( \bm{y} \mid \bm{f} \right) - \bm{K}_{\bm{X} \bm{X}}^{-1} \bm{f} = \left( \bm{t} - \bm{\pi} \right) - \bm{K}_{\bm{X} \bm{X}}^{-1} \bm{f} \\
    \nabla \nabla \Psi \left( \bm{f} \right) & = \nabla \nabla \ln p \left( \bm{y} \mid \bm{f} \right) - \bm{K}_{\bm{X} \bm{X}}^{-1} = - \bm{W} - \bm{K}_{\bm{X} \bm{X}}^{-1}
\end{align*}
where $\pi_i = p \left( y_i = +1 \mid f_i \right) = \sigma ( f_i )$, $\bm{t} = \left( \bm{y} + \bm{1} \right) / 2 \in \RR^{n}$ and $\bm{W} \triangleq - \nabla \nabla \ln p \left( \bm{y} \mid \bm{f} \right)$ is a diagonal matrix (since the distribution of $y_i$ only depends on $f_i$ and not $f_{j \neq i}$) with entries $\bm{W}_{ii} = \sigma \left( y_i f_i \right)$ \cite{BishopChristopherM2006Pram, RasmussenCarlEdward2006Gpfm}. From equation \ref{eq: lapace-grad-zero}, the mode of $\hat{\bm{f}}$ of $\bm{\Psi}$ can be computed as
\begin{align}
    \nabla \Psi \left( \hat{\bm{f}} \right) & = \bm{0} = \left( \bm{t} - \bm{\pi} \right) - \bm{K}_{\bm{X} \bm{X}}^{-1} \hat{\bm{f}} \nonumber \\
    \iff \hat{\bm{f}}                       & = \bm{K}_{\bm{X} \bm{X}} \left( \bm{t} - \bm{\pi} \right) \label{eq: expr-for-mode-lapace} .
\end{align}
Since $\bm{t} - \bm{\pi}$ is a non-linear function, a non-linear optimization technique method is required to solve $\hat{\bm{f}}$ in \ref{eq: expr-for-mode-lapace}. Since the Hessian of $\Psi \left( \bm{f} \right)$ is available, Newton's method is typically employed as fast iterative method to approximate $\hat{\bm{f}}$ where $\hat{\bm{f}}$ is updated as
\begin{equation*}
    \hat{\bm{f}}^{\; \text{new}} = \bm{K}_{\bm{X} \bm{X}} \left( \Id_{n \times n} + \bm{W} \bm{K}_{\bm{X} \bm{X}} \right)^{-1} \left( \bm{W} \hat{\bm{f}}^{\; \text{old}} + \nabla \ln \left( \bm{y} \mid \hat{\bm{f}}^{\; \text{old}} \right) \right).
\end{equation*}
Once a suitable mode is found, using equation \ref{eq: lapace-gauss-apprx}, the Lapacian approximation for $p \left( \bm{f} \mid \bm{X} , \bm{y} \right)$ becomes
\begin{equation} \label{eq: lapace-apprx}
    p \left( \bm{f} \mid \bm{X} , \bm{y} \right) \simeq q \left( \bm{f} \mid \bm{X} , \bm{y} \right) = \calN \left( \hat{\bm{f}} , \left( \bm{K}_{\bm{X} \bm{X}}^{-1} + \bm{W} \right)^{-1} \right).
\end{equation}

\subsubsection{Predictions}\label{Section1.6.3}

With the Lapace approximation for $p \left( \bm{f} \mid \bm{X} , \bm{y} \right)$ (equation \ref{eq: lapace-apprx}) and an exact probability distribution for $p \left( f_{\star} \mid \bm{X} , \bm{x}_{\star}, \bm{f} \right)$, a mean for the latent process, $p \left( f_{\star} \mid \bm{X} , \bm{y} , \bm{x}_{\star} \right)$, can now be computed by invoking \ref{eq: GP_train_distr2_mean} to give
\begin{align}
    \mu_{f_{\star}} = \EE \left[ f_{\star} \mid \bm{X} , \bm{y} , \bm{x}_{\star} \right]
     & = \bm{K}_{\bm{x}_{\star} \bm{X}} \bm{K}_{\bm{X} \bm{X}}^{-1} \hat{\bm{f}}           \nonumber               \\
     & = \bm{K}_{\bm{x}_{\star} \bm{X}} \nabla \ln \left( \bm{y} \mid \hat{\bm{f}} \right) \nonumber               \\
     & = \bm{K}_{\bm{x}_{\star} \bm{X}} \left( \bm{t} - \bm{\pi} \right) \label{eq: latent-process-mean-apprx-1} .
\end{align}
Similarly, the variance can be computed using equation \ref{eq: GP_train_distr2_var} to give
\begin{align} \label{eq: latent-process-var-apprx-1}
    \sigma_{f_{\star}}^2 = \VV \left[ f_{\star} \mid \bm{X} , \bm{y} , \bm{x}_{\star} \right]
     & = k \left( \bm{x}_{\star} , \bm{x}_{\star} \right) - \bm{K}_{\bm{x}_{\star} \bm{X}} \left( \bm{K}_{\bm{X} \bm{X}} + \bm{W}^{-1} \right)^{-1} \bm{K}_{\bm{x}_{\star} \bm{X}}^{\intercal}.
\end{align}
Using equation \ref{eq: GP-pred-1}, predictions can now be made as
\begin{equation} \label{eq: pred-apprx-1}
    \overline{\pi_{\star}} \simeq \int \sigma \left( f_{\star} \right) q \left( f_{\star} \mid \bm{X} , \bm{y} , \bm{x}_{\star} \right) \; d f_{\star}
\end{equation}
where $q \left( f_{\star} \mid \bm{X} , \bm{y} , \bm{x}_{\star} \right)$ is a multivariate Gaussian distribution with mean and variance given by equations \ref{eq: latent-process-mean-apprx-1} and \ref{eq: latent-process-var-apprx-1} respectively. Notice that the prediction given in \ref{eq: pred-apprx-1} is a convolution of a Gaussian and logistic function which unfortunately cannot be evaluated analytically. However, Spiegelhalter and Lauritzen \cite{spiegelhalter1990sequential} show that a good approximation can be found by replacing the sigmoid function with the probit function $\Phi \left( \lambda a \right)$ which is simply the cumulative distribution function (CDF) of the standard Gaussian distribution. To get the best approximation using the probit function, the constant factor $\lambda$ is adjusted to equate their slopes at the origin. The value of $\lambda$ that gives this equality is $\lambda = \sqrt{\pi / 8}$. The similarity between the sigmoid function and probit function rescaled by a factor of $\sqrt{\pi / 8}$ is illustrated in Figure \ref{fig: logistic-func-and-probit}. The reason for replacing the sigmoid function with a probit function is that the convolution of a Gaussian distribution and probit function can be analytically evaluated as
\begin{equation} \label{eq: probit-int-apprx-1}
    \int \Phi \left( \lambda a \right) \calN \left( a \mid \mu , \sigma^2 \right) \; da = \Phi \left( \frac{\mu}{\left( \lambda^{-2} + \sigma^2 \right)^{\frac{1}{2}}} \right).
\end{equation}
Again apply the approximation $\sigma \left( a \right) \simeq \Phi \left( \lambda a \right)$ to left hand side of \ref{eq: probit-int-apprx-1} gives the following estimate for the convolution of a Gaussian and sigmoid function
\begin{equation} \label{eq: pred-apprx-2}
    \int \sigma \left( a \right) \calN \left( a \mid \mu , \sigma^2 \right) \; da \simeq \sigma \left( \frac{\mu}{\left( 1 + \pi \sigma^2 / 8 \right)^{\frac{1}{2}}} \right)
\end{equation}
\cite{BishopChristopherM2006Pram}*{page 219}. The integral used to approximate $\overline{\pi_{\star}}$ in \ref{eq: pred-apprx-1} can now be estimated using \ref{eq: pred-apprx-2} to give
\begin{equation*} \label{eq: pred-apprx-3}
    \overline{\pi_{\star}} = \sigma \left( \frac{\mu_{f_{\star}}}{\left( 1 + \pi \sigma_{f_{\star}}^2 / 8 \right)^{\frac{1}{2}}} \right).
\end{equation*}
This theory justifies Algorithm \ref{alg: Unoptimized_GPC} which creates predictions based on the GPC method.

    {\centering
        \begin{minipage}{.85\linewidth}
            \begin{algorithm}[H]
                \caption{Unoptimized GPC}
                \label{alg: Unoptimized_GPC}
                \SetAlgoLined
                \DontPrintSemicolon
                \SetKwInOut{Input}{input}\SetKwInOut{Output}{output}

                \Input{Observations $\bm{X}, \bm{y}$ and a test input $\bm{x}^{\star}$.}
                \Output{A prediction $\overline{f_{\star}} $ with its corresponding variance $ \VV \left[ f_{\star} \right]$.}
                \BlankLine
                $\bm{t} = \left( \bm{y} + \bm{1} \right) / 2$\;
                $\bm{f} = \bm{0}$\;
                \Repeat{convergence}{
                    $\bm{W} = \operatorname{diag} \left( \sigma \left( \bm{y} .^{\ast} \bm{f} \right) \right)$\;
                    $\bm{\alpha} = \operatorname{lin-solve} \left( \Id_{n \times n} + \bm{W} \bm{K}_{\bm{X} \bm{X}}, \bm{K}_{\bm{X} \bm{X}} \right)$\;
                    $\bm{f} = \bm{\alpha} \left( \bm{t} - \sigma (\bm{f}) + \bm{W} \bm{f} \right)$\;
                }
                $\mu_{f_{\star}} = \bm{K}_{\bm{x}_{\star} \bm{X}} \left( \bm{t} - \sigma (\bm{f}) \right)$\;
                $\sigma_{f_{\star}}^2 = k \left( \bm{x}_{\star} , \bm{x}_{\star} \right) - \bm{K}_{\bm{x}_{\star} \bm{X}} \left( \bm{K}_{\bm{X} \bm{X}} + \bm{W}^{-1} \right)^{-1} \bm{K}_{\bm{x}_{\star} \bm{X}}^{\intercal}$\;
                $\overline{\pi_{\star}} = \sigma \left( \mu_{f_{\star}} / {\left( 1 + \pi \sigma_{f_{\star}}^2 / 8 \right)^{\frac{1}{2}}} \right)$\;
                \Return{$\overline{\pi_{\star}} , \mu_{f_{\star}} , \sigma_{f_{\star}}^2$}
                \BlankLine
            \end{algorithm}
        \end{minipage}
        \par
    }
\newpage

% \section{Hecke Algebras Generated by a Coxeter Group}\label{Chapter3}
% The purpose of this section is to investigate the Hecke algebra obtained when we choose $G=\SL_n(\FF_q)$ and $K=B(\FF_q)$, the Borel subgroup of $G$, i.e.\ the subgroup of upper-triangular matrices in $G$.
% For this chapter, our convolution product associated to $\calH$ is modified with a normalising factor $\frac{1}{|B|}$.
% We begin in Section \ref{Section3.1} by investigating the Weyl group of $G$.
% We give its general definition and show that it reduces to the symmetric group $S_n$ for our purposes.
% Next, in Section \ref{Section3.2}, we thoroughly describe the algebra $\calH(G,B)$ in the simple case of $G=\SL_2(\FF_q)$.
% We move to the more difficult cases of $G=\SL_3(\FF_q)$ and $G=\SL_4(\FF_q)$ in Sections \ref{Section3.3} and \ref{Section3.4} respectively.
% These examples lead us to the definition of a Coxeter group, given in Section \ref{Section3.5}.
% This chapter is concluded with the definition of a Hecke algebra of a finite Coxeter group and a complete description of $\calH(G,B)$ in the general case.
% This is done so in Section \ref{Section3.6}.

% %%%%%%%%%%%%%%%%%%%%%%%%%%%%%%%%%%%%%%%%%%%%%%%%%%%%%%%%%%%%%%%%%%%%%%%%%%%%%%%%%%%%%%%%%%%%%%

% \subsection{The Weyl group $W$}\label{Section3.1}
% Fix the group $G=\GL_n(\FF_q)$.
% Consider the subgroup of diagonal matrices
% \[
% 	T := \{\diag(a_1,\ldots,a_n)\ |\ a_i\in\FF_q^\times\}\subseteq G.
% \]
% This is a \emph{maximal torus} of $G$.
% The \emph{Weyl group} of $G$ is defined as the quotient group $W := N_G(T)/T$, where $N_G(T) := \{t\in T\ |\ gTg^{-1}=T\}$ is the \emph{normaliser} of $T$ in $G$.
% The Weyl group may be understood as the reflection group of the \emph{root system} associated to a Lie group $G$.
% It is a useful fact in \cite{Brocker85} that all maximal tori are conjugate to each other over the algebraic closure of $\FF_q$.
% As a result, $W$ is independent of the choice of maximal tori, so we only need to consider the one we have chosen.

% Our goal in this section is to show that $W\cong S_n$.
% To do this, we need a lemma.

% \begin{lem}\label{lemma: normaliser_is_generalised_permutation_matrices}
% 	$N_G(T)$ is the set of monomial matrices in $G$.
% \end{lem}

% Before we prove the lemma, we need a definition and a small fact about matrices.
% The \emph{spectrum} of a matrix is the multiset of its eigenvalues.
% The fact we need is that conjugation preserves the spectrum of a matrix.
% More specifically, for any $A,B\in G$, the spectrums of $A$ and $BAB^{-1}$ coincide.
% To see this, notice
% \[
% 	\det(BAB^{-1}-\lambda I) = \det(BAB^{-1}-\lambda BB^{-1}) = \det(B)\det(A-\lambda I)\det(B^{-1}) = \det(A-\lambda I).
% \]
% We see that $A$ and $BAB^{-1}$ have the same characteristic polynomial which proves the fact.

% \begin{proof}[Proof of Lemma \ref{lemma: normaliser_is_generalised_permutation_matrices}]
% 	First, we show that the monomial matrices lie in $N_G(T)$.
% 	To see this, fix a monomial matrix $p = \sum_{i=1}^n a_i E_{i,\sigma(i)}$ for some $\sigma\in S_n$.
% 	It is not difficult to see that $p^{-1} = \sum_{i=1}^n a_i^{-1} E_{i,\sigma^{-1}(i)}$.
% 	To verify this, one can compute $(pp^{-1})_{ij} = \delta_{ij}$, so $pp^{-1}=I$.
% 	Now, take an arbitrary element $t=\sum_{i=1}^n b_i E_{ii}\in T$.
% 	We compute
% 	\[
% 		(ptp^{-1})_{ij} = \sum_{k=1}^n (pt)_{ik}(p^{-1})_{kj} = \sum_{k=1}^n \bigg(\sum_{l=1}^n p_{il}t_{lk}\bigg)(p^{-1})_{kj} = \sum_{k=1}^n \sum_{l=1}^n p_{il}t_{lk}(p^{-1})_{kj}.
% 	\]
% 	Notice $p_{il}$ is non-zero if and only if $l=\sigma(i)$, $t_{lk}$ is non-zero if and only if $l=k$ and $(p^{-1})_{kj}$ is non-zero if and only if $j=\sigma^{-1}(k)$.
% 	Then $(ptp^{-1})_{ij}$ is non-zero if and only if $\sigma(i)=\sigma(j)$, i.e.\ $i=j$.
% 	Thus, $ptp^{-1}=t'$ and $pt=t'p$ for some diagonal matrix $t'\in T$.
% 	This means $p$ normalizes $T$ in $G$.

% 	Conversely, we show that a normaliser of $T$ in $G$ must be a monomial matrix.
% 	Let $s\in N_G(T)$.
% 	By definition of the normaliser, we know $sts^{-1}$ is a diagonal matrix for each $t\in T$.
% 	Choose $t=\sum_{i=1}^n iE_{ii}$.
% 	Then $sts^{-1} = w$ for some $w\in T$.
% 	The matrix $t$ clearly has the spectrum $\{1,\ldots,n\}$ and we know that conjugation preserves the spectrum, so $sts^{-1}=w$ has the same spectrum, up to permutation.
% 	We write $w = \sum_{i=1}^n \sigma(i) E_{ii}$ for some $\sigma\in S_n$ permuting the spectrum of $t$.
% 	The equation $sts^{-1}=w$ tells us that $st-ws=0$.
% 	We compute
% 	\[
% 		(st)_{ij} = \sum_{k=1}^n s_{ik}t_{kj} = js_{ij},\quad (ws)_{ij} = \sum_{k=1}^n w_{ik}s_{kj} = \sigma(i)s_{ij}.
% 	\]
% 	We put this together and see that
% 	\[
% 		0 = (st-ws)_{ij} = (st)_{ij} - (ws)_{ij} = js_{ij}-\sigma(i)s_{ij} = s_{ij}(j-\sigma(i)).
% 	\]
% 	This tells us that $s_{ij}$ must be $0$ if $j\neq \sigma(i)$ and $s_{ij}$ can be arbitrary when $j=\sigma(i)$.
% 	This is exactly what it means for $s$ to be a monomial matrix associated to $\sigma$.
% \end{proof}

% \begin{prop}
% 	$N_G(T)/T \cong S_n$.
% \end{prop}

% \begin{proof}
% 	Consider the map $f\colon N_G(T)\to G$ defined on matrix entries by
% 	\[
% 		(f(g))_{ij} := \begin{cases}
% 			1,\  & \text{if}\ g_{ij}\neq 0, \\
% 			0,\  & \text{if}\ g_{ij}=0.
% 		\end{cases}
% 	\]
% 	This sends the non-zero entries of $g\in N_G(T)$ to $1$.
% 	Notice this is a group homomorphism with $\ker f = T$ and $\im f = \{\text{permutation matrices}\}\subseteq G$.
% 	Here a permutation matrix is a monomial matrix with its non-zero entries all equal to $1$.
% 	The group of permutation matrices is isomorphic to $S_n$.
% 	Thus, the first isomorphism theorem tells us that $N_G(T)/T \cong S_n$.
% \end{proof}

% The above proof can be followed for the choice $G=\SL_n(\FF_q)$ to see that its Weyl group is also $S_n$.
% Then, similarly to Lemma \ref{lemma: bruhat}, we see that $\SL_n(\FF_q)$ has its own Bruhat decomposition in terms of $S_n$.
% This is explicitly proven in \cite{Bump13}.

% %%%%%%%%%%%%%%%%%%%%%%%%%%%%%%%%%%%%%%%%%%%%%%%%%%%%%%%%%%%%%%%%%%%%%%%%%%%%%%%%%%%%%%%%%%%%%%

% \subsection{A simple case, $G=\SL_2(\FF_q)$}\label{Section3.2}
% We investigate the structure of the Hecke algebra $\calH(G,K)$ when $G=\SL_2(\FF_q)$ and $K=B(\FF_q)$.
% Recall the Hecke algebra $\calH(G,B) = \{f\colon G\to\CC\ |\ f(bgb') = f(g)\}$.
% However, instead of the convolution product used throughout Chapter \ref{Chapter1}, we define the \emph{normalised convolution product}
% \[
% 	(f\star g)(x) := \frac{1}{|B|} \sum_{yz=x} f(y)g(z).
% \]
% Notice that this normalising factor only affects the identity in $\calH$.
% Specifically, it rescales the identity.
% In Section \ref{Section1.3}, we saw that $\calH(G,K)$ has the identity $\iota_K$.
% Under the normalised convolution product, it is easy to see that $\calH(G,B)$ has identity $\chi_B$, the characteristic function of $B$.

% Consider the symmetric group $S_2 = \{1,s\}$.
% Note that $s$ may be represented by the permutation matrix $\left(\begin{smallmatrix}0 & 1 \\ 1 & 0\end{smallmatrix}\right)$.
% However, this has determinant $-1$ so it is not an element of $\SL_2(\FF_q)$.
% However, we can swap a non-zero element from $1$ to $-1$ so that $s$ has the permutation matrix $\left(\begin{smallmatrix}0 & 1 \\ -1 & 0\end{smallmatrix}\right)$ with determinant $1$.
% In general, given the natural map $n\colon N_G(T)\to N_G(T)/T\cong S_n$, we say that $g\in N_G(T)$ is a \emph{lift} of $w\in S_n$ if $n(g)=w$.
% We see that $\left(\begin{smallmatrix}0 & 1 \\ -1 & 0\end{smallmatrix}\right)$ is a lift of $s\in S_2$, but $\left(\begin{smallmatrix}0 & 1 \\ 1 & 0\end{smallmatrix}\right)$ is not.

% We have the Bruhat decomposition
% \[
% 	\SL_2(\FF_q)=\bigsqcup_{w\in S_2} BwB = B \sqcup BsB = \underbrace{\begin{pmatrix} \FF_q & \FF_q \\ 0 & \FF_q \end{pmatrix}}_{B} \sqcup \underbrace{\begin{pmatrix} \FF_q & \FF_q \\ \FF_q^\times & \FF_q \end{pmatrix}}_{BsB}.
% \]
% The Bruhat decomposition above and Section \ref{Section1.3} tell us that $\calH(G,B)$ has a basis $\{\chi_B,\chi_{BsB}\}$.
% For brevity, we write $I:=\chi_B$ and $T:= \chi_{BsB}$.
% We are interested in the products of these basis elements.
% The objective of this section is to show that
% \[
% 	\calH(G,B) = \langle T\ |\ T^2 = (q-1)T + qI\rangle.
% \]
% To do this, we compute all possible convolution products: $I\star I$, $I\star T$, $T\star I$ and $T\star T$.
% The first three computations are clear, since $I$ is the identity.
% The fourth computation requires more work.
% The basis of $\calH(G,B)$ tells us that $T\star T = \alpha I+\beta T$ for some $\alpha,\beta\in\CC$.
% Notice that
% \[
% 	\alpha = (\alpha I+\beta T)(1) = (T\star T)(1), \quad \beta = (\alpha I+\beta T)(s) = (T\star T)(s).
% \]
% Thus, it suffices to evaluate $T\star T$ at $1$ and $s$.
% We compute
% \[
% 	(T\star T)(1) = \frac{1}{|B|}\sum_{x\in G} \chi_{BsB}(x)\chi_{BsB}(x^{-1}) = \frac{1}{|B|}\sum_{x\in G} \chi_{BsB}(x) = \frac{1}{|B|}\sum_{x\in BsB} 1 = \frac{|BsB|}{|B|},
% \]
% where the second equality comes from the observation that $G=B\sqcup BsB$ so $x\in BsB$ if and only if $x^{-1}\in BsB$.
% Notice that $B=\{\left(\begin{smallmatrix} a & b \\ 0 & a^{-1}\end{smallmatrix}\right)\ |\ a\neq 0\}$.
% Thus, we have $q-1$ choices for $a$ and $q$ choices for $b$, so $|B| = (q-1)q$.
% To compute $|BsB|$, notice that
% \[
% 	BsB = \SL_2(\FF_q)-B = \Bigg\{\begin{pmatrix} a & b \\ c & d\end{pmatrix}\ \Bigg|\ c\neq 0,\ ad-bc=1\Bigg\} = \Bigg\{\begin{pmatrix} a & \frac{ad-1}{c} \\ c & d\end{pmatrix}\ \Bigg|\ c\neq 0\Bigg\}.
% \]
% Thus, we have $q-1$ choices for $c$ and $q$ choices for both $a$ and $d$, so $|BsB| = (q-1)q^2$.
% Then $\alpha = |BsB|/|B| = q$.
% Next, we compute
% \begin{multline*}
% 	(T\star T)(s) = \frac{1}{|B|}\sum_{x\in G} \chi_{BsB}(sx)\chi_{BsB}(x^{-1}) = \frac{1}{|B|}\sum_{x\in BsB} \chi_{BsB}(sx)\chi_{BsB}(x^{-1}) \\
% 	= \frac{|\{x\in BsB\ |\ sx\in BsB\}|}{|B|} = \frac{|\{x\in BsB\ |\ x\in s^{-1}BsB\}|}{|B|} = \frac{|BsB\cap s^{-1}BsB|}{|B|}
% \end{multline*}
% Notice that, for arbitrary $g = \left(\begin{smallmatrix} a & b \\ 0 & a^{-1}\end{smallmatrix}\right)\in B$, we have $s^{-1}gs = \left(\begin{smallmatrix} a^{-1} & 0 \\ -b & a \end{smallmatrix}\right)$ which is the general form of a lower-triangular matrix in $\SL_2(\FF_q)$.
% Thus, $s^{-1}Bs =: B^-$, the subgroup of lower-triangular matrices in $\SL_2(\FF_q)$.
% Then $s^{-1}BsB = B^-B$.

% We wish to characterise $B^-B$.
% Consider the product
% \[
% 	\underbrace{\begin{pmatrix}
% 			a & 0      \\
% 			b & a^{-1}
% 		\end{pmatrix}}_{B^-}
% 	\underbrace{\begin{pmatrix}
% 			c & d      \\
% 			0 & c^{-1}
% 		\end{pmatrix}}_{B} =
% 	\begin{pmatrix}
% 		ac & ad                \\
% 		bc & bd + a^{-1}c^{-1}
% 	\end{pmatrix}.
% \]
% The entry $ac$ is invertible and the entries $bc$, $ad$ and $bd + a^{-1}c^{-1}$ are not necessarily invertible.
% Thus
% \[
% 	B^-B = \Bigg\{\begin{pmatrix} a & b \\ c & d\end{pmatrix}\in \SL_2(\FF_q)\ \Bigg|\ a\neq 0\Bigg\}
% \]
% and
% \[
% 	BsB\cap B^-B = \Bigg\{\begin{pmatrix} a & b \\ c & d\end{pmatrix}\in \SL_2(\FF_q)\ \Bigg|\ a\neq 0, c\neq 0\Bigg\} = \Bigg\{\begin{pmatrix} a & b \\ c & \frac{1+bc}{a}\end{pmatrix}\in \SL_2(\FF_q)\ \Bigg|\ a\neq 0, c\neq 0\Bigg\}.
% \]
% Thus, we have $q-1$ choices for both $a$ and $c$, and $q$ choices for $b$, so $|BsB\cap B^-B| = (q-1)^2q$.
% Then $\beta = |BsB\cap B^-B|/|B| = q-1$.

% We conclude that $T^2 = (q-1)T + qI$ and we may write
% \[
% 	\calH(G,B) = \langle T\ |\ T^2 = (q-1)T + qI\rangle.
% \]
% This is known as the \emph{quadratic relation}.
% The relation lets us prove the existence of the inverse of $T$.
% Define the function $X := q^{-1}T + (q^{-1}-1)I \in \calH(G,B)$.
% Observe that
% \begin{multline*}
% 	T\star X = T\star (q^{-1}T + (q^{-1}-1)I) = q^{-1}T^2 + (q^{-1}-1)T \\
% 	= q^{-1}((q-1)T+qI) + (q^{-1}-1)T = (1-q^{-1})T+I + (q^{-1}-1)T = I.
% \end{multline*}
% Similarly, $X\star T = I$.
% Thus, $X$ is the inverse of $T$ and we may write $T^{-1} = q^{-1}T + (q^{-1}-1)I$.

% %%%%%%%%%%%%%%%%%%%%%%%%%%%%%%%%%%%%%%%%%%%%%%%%%%%%%%%%%%%%%%%%%%%%%%%%%%%%%%%%%%%%%%%%%%%%%%

% \subsection{A less simple case, $G=\SL_3(\FF_q)$}\label{Section3.3}
% Consider the symmetric group
% \[
% 	S_3 = \{1,s_1,s_2,s_1s_2,s_2s_1,s_1s_2s_1\} = \left\langle s_1,s_2\ \Bigg|\ \begin{array}{c} s_1^2=s_2^2=1, \\ s_1s_2s_1=s_2s_1s_2 \end{array}\right\rangle.
% \]
% The elements of $S_3$ have multiple permutation matrix representations in $\SL_3(\FF_q)$, i.e.\ there are multiple lifts of each permutation.
% We fix the following representations
% \begin{multline*}
% 	1 =
% 	\begin{pmatrix}
% 		1 & 0 & 0 \\
% 		0 & 1 & 0 \\
% 		0 & 0 & 1
% 	\end{pmatrix}, \
% 	s_1 =
% 	\begin{pmatrix}
% 		0 & 1 & 0  \\
% 		1 & 0 & 0  \\
% 		0 & 0 & -1
% 	\end{pmatrix}, \
% 	s_2 =
% 	\begin{pmatrix}
% 		-1 & 0 & 0 \\
% 		0  & 0 & 1 \\
% 		0  & 1 & 0
% 	\end{pmatrix}, \\
% 	s_1s_2 =
% 	\begin{pmatrix}
% 		0  & 0  & 1 \\
% 		-1 & 0  & 0 \\
% 		0  & -1 & 0
% 	\end{pmatrix}, \ s_2s_1 =
% 	\begin{pmatrix}
% 		0 & -1 & 0  \\
% 		0 & 0  & -1 \\
% 		1 & 0  & 0
% 	\end{pmatrix}, \ s_1s_2s_1 =
% 	\begin{pmatrix}
% 		0  & 0  & -1 \\
% 		0  & -1 & 0  \\
% 		-1 & 0  & 0
% 	\end{pmatrix}.
% \end{multline*}
% Note that our choice of lifts of $s_1$ and $s_2$ do not matter, but these choices affect the matrix representations of $s_1s_2$, $s_2s_1$ and $s_1s_2s_1$.
% As in Section \ref{Section3.2}, we have the Bruhat decomposition
% \[
% 	SL_3(\FF_q)=\bigsqcup_{w\in S_3} BwB = B \sqcup Bs_1B \sqcup Bs_2B \sqcup Bs_1s_2B \sqcup Bs_2s_1B \sqcup Bs_1s_2s_1B.
% \]
% The Bruhat decomposition tells us that $\calH(G,K)$ has a basis $\{\chi_{BwB}\}_{w\in S_3}$.
% For brevity, we write $I:=\chi_B$, $T:= \chi_{Bs_1B}$ and $S:= \chi_{Bs_2B}$.
% We are interested in the products of these elements.
% The objective of this section is to show that
% \[
% 	\calH(G,B) = \left\langle T,S\ \Bigg|\ \begin{array}{c}
% 		T^2 = (q-1)T + qI, \\
% 		S^2=(q-1)S+qI,
% 	\end{array}TST=STS\right\rangle.
% \]
% To do this, we compute the convolution products $T\star T$, $S\star S$, $T\star S\star T$ and $S\star T\star S$.
% Recall that $\star$ is associative so the last two products are unambiguous.
% The basis of $\calH(G,B)$ tells us that
% \begin{align*}
% 	(T\star T)(x)        & = (\alpha_1 \chi_B + \alpha_2\chi_{Bs_1B} + \alpha_3 \chi_{Bs_2B} + \alpha_4\chi_{Bs_1s_2B} + \alpha_5\chi_{Bs_2s_1B} + \alpha_6\chi_{Bs_1s_2s_1B})(x), \\
% 	(S\star S)(x)        & = (\beta_1 \chi_B + \beta_2\chi_{Bs_1B} + \beta_3 \chi_{Bs_2B} + \beta_4\chi_{Bs_1s_2B} + \beta_5\chi_{Bs_2s_1B} + \beta_6\chi_{Bs_1s_2s_1B})(x),       \\
% 	(T\star S\star T)(x) & = (\gamma_1 \chi_B + \gamma_2\chi_{Bs_1B} + \gamma_3 \chi_{Bs_2B} + \gamma_4\chi_{Bs_1s_2B} + \gamma_5\chi_{Bs_2s_1B} + \gamma_6\chi_{Bs_1s_2s_1B})(x), \\
% 	(S\star T\star S)(x) & = (\delta_1 \chi_B + \delta_2\chi_{Bs_1B} + \delta_3 \chi_{Bs_2B} + \delta_4\chi_{Bs_1s_2B} + \delta_5\chi_{Bs_2s_1B} + \delta_6\chi_{Bs_1s_2s_1B})(x),
% \end{align*}
% for some $\alpha_i, \beta_i, \gamma_i, \delta_i \in\CC$.
% Notice that
% \begin{multline*}
% 	\alpha_1 = (T\star T)(1),\ \alpha_2 = (T\star T)(s_1),\ \alpha_3 = (T\star T)(s_2), \\
% 	\alpha_4 = (T\star T)(s_1s_2),\ \alpha_5 = (T\star T)(s_2s_1),\ \alpha_6 = (T\star T)(s_1s_2s_1).
% \end{multline*}
% The analogous statements are true for the other convolution products and their coefficients.

% We begin by showing that $\alpha_i=0$ for all $i\neq1,2$.
% For $\alpha_3$, we compute
% \[
% 	(T\star T)(s_2) = \frac{1}{|B|}\sum_{xy=s_2} \chi_{Bs_1B}(x)\chi_{Bs_1B}(y) = \frac{|\{(x,y)\in Bs_1B\times Bs_1B\ |\ xy=s_2\}|}{|B|}.
% \]
% To compute the numerator, notice that
% \[
% 	\underbrace{\begin{pmatrix}
% 			\FF_q^\times & \FF_q        & \FF_q        \\
% 			0            & \FF_q^\times & \FF_q        \\
% 			0            & 0            & \FF_q^\times
% 		\end{pmatrix}}_{B}
% 	\underbrace{\begin{pmatrix}
% 			0 & 1 & 0 \\ 1 & 0 & 0 \\ 0 & 0 & -1
% 		\end{pmatrix}}_{s_1}
% 	\underbrace{\begin{pmatrix}
% 			\FF_q^\times & \FF_q        & \FF_q        \\
% 			0            & \FF_q^\times & \FF_q        \\
% 			0            & 0            & \FF_q^\times
% 		\end{pmatrix}}_{B} =
% 	\begin{pmatrix}
% 		\FF_q        & \FF_q & \FF_q        \\
% 		\FF_q^\times & \FF_q & \FF_q        \\
% 		0            & 0     & \FF_q^\times
% 	\end{pmatrix},
% \]
% which tells us that $Bs_1B$ is a subset of all matrices in $\SL_3(\FF_q)$ with $0$'s in the bottom left and bottom middle entries.
% In fact, this is exactly $Bs_1B$.
% To see this, we compute:
% \[
% 	\left|\left\{\begin{pmatrix}
% 		\FF_q        & \FF_q & \FF_q        \\
% 		\FF_q^\times & \FF_q & \FF_q        \\
% 		0            & 0     & \FF_q^\times
% 	\end{pmatrix}\right\}\right| = q^4\cdot (q-1)^2,\quad \left|\left\{\begin{pmatrix}
% 		\FF_q^\times & \FF_q        & \FF_q        \\
% 		0            & \FF_q^\times & \FF_q        \\
% 		0            & 0            & \FF_q^\times
% 	\end{pmatrix}\right\}\right| = q^3\cdot (q-1)^2.
% \]
% It is a proposition of Bump in \cite{Bump10} that $|Bs_1B|/|B|=q$, so $|Bs_1B|=q|B|$.
% This tells us that $Bs_1B$ consists of all matrices of the form above.

% We multiply two of these matrices together to see that
% \[
% 	\underbrace{\begin{pmatrix}
% 			\FF_q        & \FF_q & \FF_q        \\
% 			\FF_q^\times & \FF_q & \FF_q        \\
% 			0            & 0     & \FF_q^\times
% 		\end{pmatrix}}_{Bs_1B}
% 	\underbrace{\begin{pmatrix}
% 			\FF_q        & \FF_q & \FF_q        \\
% 			\FF_q^\times & \FF_q & \FF_q        \\
% 			0            & 0     & \FF_q^\times
% 		\end{pmatrix}}_{Bs_1B} =
% 	\begin{pmatrix}
% 		\FF_q        & \FF_q & \FF_q        \\
% 		\FF_q^\times & \FF_q & \FF_q        \\
% 		0            & 0     & \FF_q^\times
% 	\end{pmatrix}.
% \]
% Then the product of two elements in $Bs_1B$ will always have a $0$ in the bottom left entry.
% However, $s_2$ has a $1$ in the bottom left entry.
% Then $|\{(x,y)\in Bs_1B\times Bs_1B\ |\ xy=s_2\}| = 0$ and $\alpha_3=0$.
% Similarly, notice that $s_1s_2$, $s_2s_1$ and $s_1s_2s_1$ all have non-zero entries in the bottom left or bottom middle entry.
% This shows that $\alpha_4=\alpha_5=\alpha_6=0$ as well.
% Thus, $T^2 = \alpha_1 I + \alpha_2 T$.

% Similarly, we show that $\beta_i=0$ for all $i\neq 1,3$.
% For $\beta_2$, we compute
% \[
% 	(S\star S)(s_1) = \frac{1}{|B|}\sum_{xy=s_1} \chi_{Bs_2B}(x)\chi_{Bs_2B}(y) = \frac{|\{(x,y)\in Bs_2B\times Bs_2B\ |\ xy=s_1\}|}{|B|}.
% \]
% To compute the numerator, notice that
% \[
% 	\underbrace{\begin{pmatrix}
% 			\FF_q^\times & \FF_q        & \FF_q        \\
% 			0            & \FF_q^\times & \FF_q        \\
% 			0            & 0            & \FF_q^\times
% 		\end{pmatrix}}_{B}
% 	\underbrace{\begin{pmatrix}
% 			-1 & 0 & 0 \\ 0 & 0 & 1 \\ 0 & 1 & 0
% 		\end{pmatrix}}_{s_2}
% 	\underbrace{\begin{pmatrix}
% 			\FF_q^\times & \FF_q        & \FF_q        \\
% 			0            & \FF_q^\times & \FF_q        \\
% 			0            & 0            & \FF_q^\times
% 		\end{pmatrix}}_{B} =
% 	\begin{pmatrix}
% 		\FF_q^\times & \FF_q        & \FF_q \\
% 		0            & \FF_q        & \FF_q \\
% 		0            & \FF_q^\times & \FF_q
% 	\end{pmatrix},
% \]
% which tells us that $Bs_2B$ is a subset of all matrices in $\SL_3(\FF_q)$ with $0$'s in the bottom left and middle left entries.
% As before, $Bs_2B$ consists of all matrices of the form above.
% We multiply two of these matrices together to see that
% \[
% 	\underbrace{\begin{pmatrix}
% 			\FF_q^\times & \FF_q        & \FF_q \\
% 			0            & \FF_q        & \FF_q \\
% 			0            & \FF_q^\times & \FF_q
% 		\end{pmatrix}}_{Bs_2B}
% 	\underbrace{\begin{pmatrix}
% 			\FF_q^\times & \FF_q        & \FF_q \\
% 			0            & \FF_q        & \FF_q \\
% 			0            & \FF_q^\times & \FF_q
% 		\end{pmatrix}}_{Bs_2B}  =
% 	\begin{pmatrix}
% 		\FF_q^\times & \FF_q        & \FF_q \\
% 		0            & \FF_q        & \FF_q \\
% 		0            & \FF_q^\times & \FF_q
% 	\end{pmatrix}.
% \]
% Then the product of two elements in $Bs_2B$ will always have a $0$ in the middle left entry.
% However, $s_1$ has a $1$ in the middle left entry.
% Then $|\{(x,y)\in Bs_2B\times Bs_2B\ |\ xy=s_1\}| = 0$ and $\beta_2=0$.
% Similarly, notice that $s_1s_2$, $s_2s_1$ and $s_1s_2s_1$ all have non-zero entries in the middle left or bottom left entry.
% This shows that $\beta_4=\beta_5=\beta_6=0$ as well.
% Thus, $S^2 = \beta_1 I + \beta_3 S$.

% All that is left is to determine $\alpha_1, \alpha_2, \beta_1$ and $\beta_3$.
% Notice that if $x\in BwB$ for some $w\in W$, then $x=b_1wb_2$ and $x^{-1} = b_2^{-1}w^{-1}b_1^{-1}\in Bw^{-1}B$.
% If $w$ is a simple reflection then $w^{-1}=w$.
% Thus $x\in Bs_iB$ if and only if $x^{-1}\in Bs_iB$ for $i=1,2$.
% We compute
% \[
% 	\alpha_1 = (T\star T)(1) = \frac{1}{|B|}\sum_{x\in G} \chi_{Bs_1B}(x)\chi_{Bs_1B}(x^{-1}) = \frac{1}{|B|}\sum_{x\in G} \chi_{Bs_1B}(x) = \frac{|Bs_1B|}{|B|}.
% \]
% To count $|Bs_1B|$, notice that the bottom rows and right columns of $Bs_1B$ and $B$ are the same (highlighted in {\color{red} red} below):
% \[
% 	\underbrace{\begin{pmatrix}
% 			\color{blue}\FF_q^\times & \color{blue}\FF_q        & \color{red}\FF_q        \\
% 			\color{blue}0            & \color{blue}\FF_q^\times & \color{red}\FF_q        \\
% 			\color{red}0             & \color{red}0             & \color{red}\FF_q^\times
% 		\end{pmatrix}}_{B}\quad \underbrace{\begin{pmatrix}
% 			\color{blue}\FF_q        & \color{blue}\FF_q & \color{red}\FF_q        \\
% 			\color{blue}\FF_q^\times & \color{blue}\FF_q & \color{red}\FF_q        \\
% 			\color{red}0             & \color{red}0      & \color{red}\FF_q^\times
% 		\end{pmatrix}}_{Bs_1B}
% \]
% This tells us that, when computing $|Bs_1B|/|B|$, the terms resulting from these columns and rows will cancel out.
% Thus, we only need to consider the submatrices obtained by deleting the bottom row and right column.
% Explicitly, we have
% \[
% 	\frac{|Bs_1B|}{|B|} = \frac{\left|\left\{\begin{pmatrix} \color{blue}\FF_q & \color{blue}\FF_q \\ \color{blue}\FF_q^\times & \color{blue}\FF_q \end{pmatrix}\right\}\right|}{\left|\left\{\begin{pmatrix} \color{blue}\FF_q^\times & \color{blue}\FF_q \\ \color{blue}0 & \color{blue}\FF_q^\times \end{pmatrix}\right\}\right|}.
% \]
% However, we have already performed this calculation in Section \ref{Section3.2}.
% We evaluated this to be $q$, so $\alpha_1 = q$.
% Similarly, for $\beta_1$, we compute
% \[
% 	\beta_1 = (S\star S)(1) = \frac{1}{|B|}\sum_{x\in G} \chi_{Bs_2B}(x)\chi_{Bs_2B}(x^{-1}) = \frac{|\{x\in Bs_2B\ |\ x^{-1}\in Bs_2 B\}|}{|B|}.
% \]
% As before, $s_2$ is a simple reflection so $\beta_1 = |Bs_2B|/|B|$.
% To count $|Bs_2B|$, again, notice that the top rows and left columns of $Bs_2B$ and $B$ are the same (highlighted in {\color{red} red} below):
% \[
% 	\underbrace{\begin{pmatrix}
% 			\color{red}\FF_q^\times & \color{red}\FF_q         & \color{red}\FF_q         \\
% 			\color{red}0            & \color{blue}\FF_q^\times & \color{blue}\FF_q        \\
% 			\color{red}0            & \color{blue}0            & \color{blue}\FF_q^\times
% 		\end{pmatrix}}_{B}\quad \underbrace{\begin{pmatrix}
% 			\color{red}\FF_q^\times & \color{red}\FF_q         & \color{red}\FF_q  \\
% 			\color{red}0            & \color{blue}\FF_q        & \color{blue}\FF_q \\
% 			\color{red}0            & \color{blue}\FF_q^\times & \color{blue}\FF_q
% 		\end{pmatrix}}_{Bs_2B}
% \]
% This tells us that we only need to consider the submatrices obtained by deleting the top row and left column.
% Explicitly, we have
% \[
% 	\frac{|Bs_2B|}{|B|} = \frac{\left|\left\{\begin{pmatrix} \color{blue}\FF_q & \color{blue}\FF_q \\ \color{blue}\FF_q^\times & \color{blue}\FF_q \end{pmatrix}\right\}\right|}{\left|\left\{\begin{pmatrix} \color{blue}\FF_q^\times & \color{blue}\FF_q \\ \color{blue}0 & \color{blue}\FF_q^\times \end{pmatrix}\right\}\right|}.
% \]
% We've already evaluated this to be $q$, so $\beta_1 = q$.
% Now we compute $\alpha_2$.
% Notice
% \[
% 	\alpha_2 = (T\star T)(s_1) = \frac{|\{x\in G\ |\ s_1x\in Bs_1B,\ x^{-1}\in Bs_1B\}|}{|B|} = \frac{|\{x\in s_1^{-1}Bs_1B\ |\ x\in Bs_1B\}|}{|B|}.
% \]
% We want to describe $s_1^{-1}Bs_1$.
% Notice $s_1^{-1}=s_1$.
% Then
% \[
% 	\underbrace{\begin{pmatrix}
% 			0 & 1 & 0  \\
% 			1 & 0 & 0  \\
% 			0 & 0 & -1
% 		\end{pmatrix}}_{s_1^{-1}}
% 	\underbrace{\begin{pmatrix}
% 			\FF_q^\times & \FF_q        & \FF_q        \\
% 			0            & \FF_q^\times & \FF_q        \\
% 			0            & 0            & \FF_q^\times
% 		\end{pmatrix}}_{B}
% 	\underbrace{\begin{pmatrix}
% 			0 & 1 & 0  \\
% 			1 & 0 & 0  \\
% 			0 & 0 & -1
% 		\end{pmatrix}}_{s_1}
% 	\underbrace{\begin{pmatrix}
% 			\FF_q^\times & \FF_q        & \FF_q        \\
% 			0            & \FF_q^\times & \FF_q        \\
% 			0            & 0            & \FF_q^\times
% 		\end{pmatrix}}_{B} =
% 	\begin{pmatrix}
% 		\FF_q^\times & \FF_q & \FF_q        \\
% 		\FF_q        & \FF_q & \FF_q        \\
% 		0            & 0     & \FF_q^\times
% 	\end{pmatrix}
% \]
% We see that
% \[
% 	s_1^{-1}Bs_1B\cap Bs_1B = \left\{\begin{pmatrix}
% 		\FF_q^\times & \FF_q & \FF_q        \\
% 		\FF_q        & \FF_q & \FF_q        \\
% 		0            & 0     & \FF_q^\times
% 	\end{pmatrix}\right\}
% 	\cap
% 	\left\{\begin{pmatrix}
% 		\FF_q        & \FF_q & \FF_q        \\
% 		\FF_q^\times & \FF_q & \FF_q        \\
% 		0            & 0     & \FF_q^\times
% 	\end{pmatrix}
% 	\right\} =
% 	\left\{
% 	\begin{pmatrix}
% 		\color{blue}\FF_q^\times & \color{blue}\FF_q & \color{red}\FF_q        \\
% 		\color{blue}\FF_q^\times & \color{blue}\FF_q & \color{red}\FF_q        \\
% 		\color{red}0             & \color{red}0      & \color{red}\FF_q^\times
% 	\end{pmatrix}
% 	\right\}.
% \]
% Notice that the bottom rows and right columns of $s_1^{-1}Bs_1B\cap Bs_1B$ and $B$ are the same.
% This tells us that, when computing $|s_1^{-1}Bs_1B\cap Bs_1B|/|B|$, the terms resulting from this column and row will cancel out.
% Thus, we only need to consider the submatrices obtained by deleting the bottom row and right column.
% More specifically, we have
% \[
% 	\frac{|s_1^{-1}Bs_1B\cap Bs_1B|}{|B|} = \frac{\left|\left\{\begin{pmatrix}\color{blue}\FF_q^\times & \color{blue}\FF_q \\ \color{blue}\FF_q^\times & \color{blue}\FF_q \end{pmatrix}\right\}\right|}{\left|\left\{\begin{pmatrix} \color{blue}\FF_q^\times & \color{blue}\FF_q \\ \color{blue}0 & \color{blue}\FF_q^\times \end{pmatrix}\right\}\right|}.
% \]
% However, we have already performed this calculation in Section \ref{Section3.3}.
% We evaluated this to be $q-1$.
% Thus, $\alpha_2 = q-1$.
% Similarly, notice
% \[
% 	\beta_3 = (S\star S)(s_2) = \frac{|\{x\in G\ |\ s_2x\in Bs_2B,\ x^{-1}\in Bs_2B\}|}{|B|} = \frac{|\{x\in s_2^{-1}Bs_2B\ |\ x\in Bs_2B\}|}{|B|}.
% \]
% We want to describe $s_2^{-1}Bs_2$.
% Notice $s_2^{-1}=s_2$.
% Then
% \[
% 	\underbrace{\begin{pmatrix}
% 			-1 & 0 & 0 \\
% 			0  & 0 & 1 \\
% 			0  & 1 & 0
% 		\end{pmatrix}}_{s_2^{-1}}
% 	\underbrace{\begin{pmatrix}
% 			\FF_q^\times & \FF_q        & \FF_q        \\
% 			0            & \FF_q^\times & \FF_q        \\
% 			0            & 0            & \FF_q^\times
% 		\end{pmatrix}}_{B}
% 	\underbrace{\begin{pmatrix}
% 			-1 & 0 & 0 \\
% 			0  & 0 & 1 \\
% 			0  & 1 & 0
% 		\end{pmatrix}}_{s_2}
% 	\underbrace{\begin{pmatrix}
% 			\FF_q^\times & \FF_q        & \FF_q        \\
% 			0            & \FF_q^\times & \FF_q        \\
% 			0            & 0            & \FF_q^\times
% 		\end{pmatrix}}_{B} =
% 	\begin{pmatrix}
% 		\FF_q^\times & \FF_q        & \FF_q \\
% 		0            & \FF_q^\times & \FF_q \\
% 		0            & \FF_q        & \FF_q
% 	\end{pmatrix}.
% \]
% We see that
% \[
% 	s_2^{-1}Bs_2B\cap Bs_2B = \left\{\begin{pmatrix}
% 		\FF_q^\times & \FF_q        & \FF_q \\
% 		0            & \FF_q^\times & \FF_q \\
% 		0            & \FF_q        & \FF_q
% 	\end{pmatrix}\right\}
% 	\cap
% 	\left\{\begin{pmatrix}
% 		\FF_q^\times & \FF_q        & \FF_q \\
% 		0            & \FF_q        & \FF_q \\
% 		0            & \FF_q^\times & \FF_q
% 	\end{pmatrix}
% 	\right\} =
% 	\left\{
% 	\begin{pmatrix}
% 		\color{red}\FF_q^\times & \color{red}\FF_q         & \color{red}\FF_q  \\
% 		\color{red}0            & \color{blue}\FF_q^\times & \color{blue}\FF_q \\
% 		\color{red}0            & \color{blue}\FF_q^\times & \color{blue}\FF_q
% 	\end{pmatrix}
% 	\right\}.
% \]
% Again, notice that the top rows and left columns of $s_2^{-1}Bs_2B\cap Bs_2B$ and $B$ are the same.
% We only need to consider the submatrices obtained by deleting the top row and left column.
% More specifically, we have
% \[
% 	\frac{|s_2^{-1}Bs_2B\cap Bs_2B|}{|B|} = \frac{\left|\left\{\begin{pmatrix} \color{blue}\FF_q^\times & \color{blue}\FF_q \\ \color{blue}\FF_q^\times & \color{blue}\FF_q \end{pmatrix}\right\}\right|}{\left|\left\{\begin{pmatrix} \color{blue}\FF_q^\times & \color{blue}\FF_q \\ \color{blue}0 & \color{blue}\FF_q^\times \end{pmatrix}\right\}\right|}.
% \]
% We've already evaluated this to be $q-1$.
% Thus, $\beta_3 = q-1$.
% This lets us conclude that
% \[
% 	T^2 = (q-1)T + qI,\quad S^2 = (q-1)S + qI.
% \]
% Lastly, we wish to show that $TST=STS$.
% To do this, we show that $\gamma_i=\delta_i$ for each $i$.
% Notice that
% \begin{align*}
% 	\gamma_i-\delta_i & =  ((T\star S)\star T)(w)-((S\star T)\star S)(w)                                                                    \\
% 	                  & = \frac{1}{|B|^2}\sum_{x,y\in G} T(wxy)S(y^{-1})T(x^{-1}) - \frac{1}{|B|^2}\sum_{x,y\in G} S(wyx)T(x^{-1})S(y^{-1}) \\
% 	                  & = \frac{1}{|B|^2}\sum_{x,y\in G} T(x^{-1})S(y^{-1})[T(wxy)-S(wyx)],
% \end{align*}
% where $w\in S_3$ is chosen appropriately.
% For instance, to compute $\gamma_4-\delta_4$, we choose $w=s_1s_2$.

% For $\gamma_1-\delta_1$, we want to compute
% \[
% 	\gamma_1-\delta_1 = \frac{1}{|B|^2}\sum_{x,y\in G} T(x^{-1})S(y^{-1})[T(xy)-S(yx)].
% \]
% For the summand to be non-zero, we require that $x\in Bs_1B$ and $y\in Bs_2B$.
% We look at the products $xy$ and $yx$:
% \begin{align*}
% 	\underbrace{\begin{pmatrix}
% 			\FF_q        & \FF_q & \FF_q        \\
% 			\FF_q^\times & \FF_q & \FF_q        \\
% 			0            & 0     & \FF_q^\times
% 		\end{pmatrix}}_{x}
% 	\underbrace{\begin{pmatrix}
% 			\FF_q^\times & \FF_q        & \FF_q \\
% 			0            & \FF_q        & \FF_q \\
% 			0            & \FF_q^\times & \FF_q
% 		\end{pmatrix}}_{y} & =
% 	\begin{pmatrix}
% 		\FF_q        & \FF_q        & \FF_q \\
% 		\FF_q^\times & \FF_q        & \FF_q \\
% 		0            & \FF_q^\times & \FF_q
% 	\end{pmatrix} \notin Bs_1B,        \\
% 	\underbrace{\begin{pmatrix}
% 			\FF_q^\times & \FF_q        & \FF_q \\
% 			0            & \FF_q        & \FF_q \\
% 			0            & \FF_q^\times & \FF_q
% 		\end{pmatrix}}_{y}
% 	\underbrace{\begin{pmatrix}
% 			\FF_q        & \FF_q & \FF_q        \\
% 			\FF_q^\times & \FF_q & \FF_q        \\
% 			0            & 0     & \FF_q^\times
% 		\end{pmatrix}}_{x} & =
% 	\begin{pmatrix}
% 		\FF_q        & \FF_q & \FF_q \\
% 		\FF_q        & \FF_q & \FF_q \\
% 		\FF_q^\times & \FF_q & \FF_q
% 	\end{pmatrix}\notin Bs_2B.
% \end{align*}
% We see that $T(xy)-S(yx)$ is always zero whenever $T(x^{-1})$ and $S(y^{-1})$ are non-zero, so this sum is always $0$.
% Thus, $\gamma_1=\delta_1$.

% For $\gamma_2-\delta_2$, we want to compute
% \[
% 	\gamma_2-\delta_2 = \frac{1}{|B|^2}\sum_{x,y\in G} T(x^{-1})S(y^{-1})[T(s_1xy)-S(s_1yx)].
% \]
% For the summand to be non-zero, we require that $x\in Bs_1B$ and $y\in Bs_2B$.
% We look at the products $s_1xy$ and $s_1yx$:
% \begin{align*}
% 	\underbrace{\begin{pmatrix}
% 			0 & 1 & 0  \\
% 			1 & 0 & 0  \\
% 			0 & 0 & -1
% 		\end{pmatrix}}_{s_1}
% 	\underbrace{\begin{pmatrix}
% 			\FF_q        & \FF_q & \FF_q        \\
% 			\FF_q^\times & \FF_q & \FF_q        \\
% 			0            & 0     & \FF_q^\times
% 		\end{pmatrix}}_{x}
% 	\underbrace{\begin{pmatrix}
% 			\FF_q^\times & \FF_q        & \FF_q \\
% 			0            & \FF_q        & \FF_q \\
% 			0            & \FF_q^\times & \FF_q
% 		\end{pmatrix}}_{y} & =
% 	\begin{pmatrix}
% 		\FF_q^\times & \FF_q        & \FF_q \\
% 		\FF_q        & \FF_q        & \FF_q \\
% 		0            & \FF_q^\times & \FF_q
% 	\end{pmatrix} \notin Bs_1B,        \\
% 	\underbrace{\begin{pmatrix}
% 			0 & 1 & 0  \\
% 			1 & 0 & 0  \\
% 			0 & 0 & -1
% 		\end{pmatrix}}_{s_1}
% 	\underbrace{\begin{pmatrix}
% 			\FF_q^\times & \FF_q        & \FF_q \\
% 			0            & \FF_q        & \FF_q \\
% 			0            & \FF_q^\times & \FF_q
% 		\end{pmatrix}}_{y}
% 	\underbrace{\begin{pmatrix}
% 			\FF_q        & \FF_q & \FF_q        \\
% 			\FF_q^\times & \FF_q & \FF_q        \\
% 			0            & 0     & \FF_q^\times
% 		\end{pmatrix}}_{x} & =
% 	\begin{pmatrix}
% 		\FF_q        & \FF_q & \FF_q \\
% 		\FF_q        & \FF_q & \FF_q \\
% 		\FF_q^\times & \FF_q & \FF_q
% 	\end{pmatrix} \notin Bs_2B.
% \end{align*}
% We see that $T(s_1xy)-S(s_1yx)$ is always zero whenever $T(x^{-1})$ and $S(y^{-1})$ are non-zero, so this sum is always $0$.
% Thus, $\gamma_2=\delta_2$.

% For $\gamma_3-\delta_3$, we want to compute
% \[
% 	\gamma_3-\delta_3 = \frac{1}{|B|^2}\sum_{x,y\in G} T(x^{-1})S(y^{-1})[T(s_2xy)-S(s_2yx)].
% \]
% For the summand to be non-zero, we require that $x\in Bs_1B$ and $y\in Bs_2B$.
% We look at the products $s_2xy$ and $s_2yx$:
% \begin{align*}
% 	\underbrace{\begin{pmatrix}
% 			-1 & 0 & 0 \\
% 			0  & 0 & 1 \\
% 			0  & 1 & 0
% 		\end{pmatrix}}_{s_2}
% 	\underbrace{\begin{pmatrix}
% 			\FF_q        & \FF_q & \FF_q        \\
% 			\FF_q^\times & \FF_q & \FF_q        \\
% 			0            & 0     & \FF_q^\times
% 		\end{pmatrix}}_{x}
% 	\underbrace{\begin{pmatrix}
% 			\FF_q^\times & \FF_q        & \FF_q \\
% 			0            & \FF_q        & \FF_q \\
% 			0            & \FF_q^\times & \FF_q
% 		\end{pmatrix}}_{y} & =
% 	\begin{pmatrix}
% 		\FF_q        & \FF_q        & \FF_q \\
% 		0            & \FF_q^\times & \FF_q \\
% 		\FF_q^\times & \FF_q        & \FF_q
% 	\end{pmatrix}\notin Bs_1B,         \\
% 	\underbrace{\begin{pmatrix}
% 			-1 & 0 & 0 \\
% 			0  & 0 & 1 \\
% 			0  & 1 & 0
% 		\end{pmatrix}}_{s_2}
% 	\underbrace{\begin{pmatrix}
% 			\FF_q^\times & \FF_q        & \FF_q \\
% 			0            & \FF_q        & \FF_q \\
% 			0            & \FF_q^\times & \FF_q
% 		\end{pmatrix}}_{y}
% 	\underbrace{\begin{pmatrix}
% 			\FF_q        & \FF_q & \FF_q        \\
% 			\FF_q^\times & \FF_q & \FF_q        \\
% 			0            & 0     & \FF_q^\times
% 		\end{pmatrix}}_{x} & =
% 	\begin{pmatrix}
% 		\FF_q        & \FF_q & \FF_q \\
% 		\FF_q^\times & \FF_q & \FF_q \\
% 		\FF_q        & \FF_q & \FF_q
% 	\end{pmatrix}\notin Bs_2B.
% \end{align*}
% We see that $T(s_2xy)-S(s_2yx)$ is always zero whenever $T(x^{-1})$ and $S(y^{-1})$ are non-zero, so this sum is always $0$.
% Thus, $\gamma_3=\delta_3$.

% For $\gamma_4-\delta_4$, we want to compute
% \[
% 	\gamma_4-\delta_4 = \frac{1}{|B|^2}\sum_{x,y\in G} T(x^{-1})S(y^{-1})[T(s_1s_2xy)-S(s_1s_2yx)].
% \]
% For the summand to be non-zero, we require that $x\in Bs_1B$ and $y\in Bs_2B$.
% We look at the products $s_1s_2xy$ and $s_1s_2yx$:
% \[
% 	\underbrace{\begin{pmatrix}
% 			0  & 0  & 1 \\
% 			-1 & 0  & 0 \\
% 			0  & -1 & 0
% 		\end{pmatrix}}_{s_1s_2}
% 	\underbrace{\begin{pmatrix}
% 			\FF_q        & \FF_q & \FF_q        \\
% 			\FF_q^\times & \FF_q & \FF_q        \\
% 			0            & 0     & \FF_q^\times
% 		\end{pmatrix}}_{x}
% 	\underbrace{\begin{pmatrix}
% 			\FF_q^\times & \FF_q        & \FF_q \\
% 			0            & \FF_q        & \FF_q \\
% 			0            & \FF_q^\times & \FF_q
% 		\end{pmatrix}}_{y} =
% 	\begin{pmatrix}
% 		0            & \FF_q^\times & \FF_q \\
% 		\FF_q        & \FF_q        & \FF_q \\
% 		\FF_q^\times & \FF_q        & \FF_q
% 	\end{pmatrix}\notin Bs_1B.
% \]
% \[
% 	\underbrace{\begin{pmatrix}
% 			0  & 0  & 1 \\
% 			-1 & 0  & 0 \\
% 			0  & -1 & 0
% 		\end{pmatrix}}_{s_1s_2}
% 	\underbrace{\begin{pmatrix}
% 			a & b & c \\
% 			0 & d & e \\
% 			0 & f & g
% 		\end{pmatrix}}_{y}
% 	\underbrace{\begin{pmatrix}
% 			h & i & j \\
% 			k & l & m \\
% 			0 & 0 & n
% 		\end{pmatrix}}_{x} =
% 	\begin{pmatrix}
% 		fk     & fl     & fm+gn     \\
% 		-ah-bk & -ai-bl & -aj-bm-cn \\
% 		-dk    & -dl    & -dm-en
% 	\end{pmatrix}.
% \]
% For $x$ to lie in $Bs_1B$ and $y\in Bs_2B$, we must have $afkn\neq 0$.
% For the product $s_1s_2yx$ to lie in $Bs_2B$, we require $-dk=0$.
% Since $k\neq 0$, we must have $d=0$.
% However, this means $-dl=0$, but we also require $-dl\neq 0$ for this product to lie in $Bs_2B$.
% Then this product can never lie in $Bs_2B$.
% We see that $T(s_2xy)-S(s_2yx)$ is always zero whenever $T(x^{-1})$ and $S(y^{-1})$ are non-zero, so this sum is always $0$.
% Thus, $\gamma_4 =\delta_4$.

% For $\gamma_5-\delta_5$, we want to compute
% \[
% 	\gamma_5-\delta_5 = \frac{1}{|B|^2}\sum_{x,y\in G} T(x^{-1})S(y^{-1})[T(s_2s_1xy)-S(s_2s_1yx)].
% \]
% For the summand to be non-zero, we require that $x\in Bs_1B$ and $y\in Bs_2B$.
% We look at the products $s_2s_1xy$ and $s_2s_1yx$:
% \begin{align*}
% 	\underbrace{\begin{pmatrix}
% 			0 & -1 & 0  \\
% 			0 & 0  & -1 \\
% 			1 & 0  & 0
% 		\end{pmatrix}}_{s_2s_1}
% 	\underbrace{\begin{pmatrix}
% 			\FF_q        & \FF_q & \FF_q        \\
% 			\FF_q^\times & \FF_q & \FF_q        \\
% 			0            & 0     & \FF_q^\times
% 		\end{pmatrix}}_{x}
% 	\underbrace{\begin{pmatrix}
% 			\FF_q^\times & \FF_q        & \FF_q \\
% 			0            & \FF_q        & \FF_q \\
% 			0            & \FF_q^\times & \FF_q
% 		\end{pmatrix}}_{y} & =
% 	\begin{pmatrix}
% 		\FF_q^\times & \FF_q        & \FF_q \\
% 		0            & \FF_q^\times & \FF_q \\
% 		\FF_q        & \FF_q        & \FF_q
% 	\end{pmatrix}\notin Bs_1B,         \\
% 	\underbrace{\begin{pmatrix}
% 			0 & -1 & 0  \\
% 			0 & 0  & -1 \\
% 			1 & 0  & 0
% 		\end{pmatrix}}_{s_2s_1}
% 	\underbrace{\begin{pmatrix}
% 			\FF_q^\times & \FF_q        & \FF_q \\
% 			0            & \FF_q        & \FF_q \\
% 			0            & \FF_q^\times & \FF_q
% 		\end{pmatrix}}_{y}
% 	\underbrace{\begin{pmatrix}
% 			\FF_q        & \FF_q & \FF_q        \\
% 			\FF_q^\times & \FF_q & \FF_q        \\
% 			0            & 0     & \FF_q^\times
% 		\end{pmatrix}}_{x} & =
% 	\begin{pmatrix}
% 		\FF_q        & \FF_q & \FF_q \\
% 		\FF_q^\times & \FF_q & \FF_q \\
% 		\FF_q        & \FF_q & \FF_q
% 	\end{pmatrix}\notin Bs_2B.
% \end{align*}
% We see that $T(s_2s_1xy)-S(s_2s_1yx)$ is always zero whenever $T(x^{-1})$ and $S(y^{-1})$ are non-zero, so this sum is always $0$.
% Thus, $\gamma_5=\delta_5$.

% For $\gamma_6-\delta_6$, we will compute $\gamma_6$ and $\delta_6$ seperately and show that they are equal.
% We want to compute
% \[
% 	\gamma_6 = \frac{1}{|B|^2}\sum_{x,y\in G} T(x^{-1})S(y^{-1})T(s_1s_2s_1xy).
% \]
% We look at the product $s_1s_2s_1xy$:
% \[
% 	\underbrace{\begin{pmatrix}
% 			0  & 0  & -1 \\
% 			0  & -1 & 0  \\
% 			-1 & 0  & 0
% 		\end{pmatrix}}_{s_1s_2s_1}
% 	\underbrace{\begin{pmatrix}
% 			h & i & j               \\
% 			k & l & m               \\
% 			0 & 0 & \frac{1}{hl-ik}
% 		\end{pmatrix}}_{x}
% 	\underbrace{\begin{pmatrix}
% 			\frac{1}{dg-ef} & b & c \\
% 			0               & d & e \\
% 			0               & f & g
% 		\end{pmatrix}}_{y} =
% 	\begin{pmatrix}
% 		0                & \frac{-f}{hl-ik} & \frac{-g}{hl-ik} \\
% 		\frac{-k}{dg-ef} & -bk-dl-fm        & -ck-el-gm        \\
% 		\frac{-h}{dg-ef} & -bh-di-fj        & -ch-ei-gj
% 	\end{pmatrix}.
% \]
% For $x$ to lie in $Bs_1B$, we require that $k\neq 0$ and $hl-ik\neq 0$ which can be restated at $i\neq \frac{hl}{k}$.
% Similarly, for $y$ to lie in $Bs_2B$, we require that $f\neq 0$ and $e\neq\frac{dg}{f}$.

% We now investigate when the product $s_1s_2s_1xy$ lies in $Bs_1B$ which will lead to an easy evaluation of $\gamma_6$.
% First, we require that $\frac{-k}{dg-ef}\neq 0$ which is already true since $k\neq 0$.
% Secondly, we require that $\frac{-h}{dg-ef}=0$ which tells us that $h=0$.
% Next, we require that $-bh-di-fj=-di-fj=0$.
% We arrange to rewrite this condition as $j=\frac{-di}{f}$.
% Lastly, we require that $-ch-ei-gj=-ei-gj\neq 0$.
% In other words, we require that $-ei-g(\frac{-di}{f})=i(-e+\frac{dg}{f})\neq 0$.
% This tell us that $i\neq 0$ and $-e+\frac{dg}{f}\neq 0$, which is already true since $y\in Bs_2B$.

% We see that $s_1s_2s_1xy$ lies in $Bs_1B$ when $f,i,k\neq 0$, $h=0$, $j=\frac{-di}{f}$ and $b,c,d,e,g,l,m\in\FF_q$ are arbitrary.
% Thus,
% \[
% 	|B|\gamma_6 = \underbrace{q^7}_{b,c,d,e,g,l,m\in\FF_q}\cdot\underbrace{(q-1)^3}_{f,i,k\neq 0}\cdot \underbrace{1}_{h=0}\cdot \underbrace{1}_{j=\frac{-di}{f}} = q^7(q-1)^3.
% \]
% Now we want to compute
% \[
% 	\delta_6 = \frac{1}{|B|^2}\sum_{x,y\in G} S(y^{-1})T(x^{-1})S(s_1s_2s_1yx).
% \]
% We look at the product $s_1s_2s_1yx$:
% \[
% 	\underbrace{\begin{pmatrix}
% 			0  & 0  & -1 \\
% 			0  & -1 & 0  \\
% 			-1 & 0  & 0
% 		\end{pmatrix}}_{s_1s_2s_1}
% 	\underbrace{\begin{pmatrix}
% 			\frac{1}{dg-ef} & b & c \\
% 			0               & d & e \\
% 			0               & f & g
% 		\end{pmatrix}}_{y}
% 	\underbrace{\begin{pmatrix}
% 			h & i & j               \\
% 			k & l & m               \\
% 			0 & 0 & \frac{1}{hl-ik}
% 		\end{pmatrix}}_{x}
% 	=
% 	\begin{pmatrix}
% 		-fk                 & -fl                   & -fm-\frac{g}{hl-ik}                 \\
% 		-dk                 & -dl                   & -dm-\frac{e}{hl-ik}                 \\
% 		\frac{-h}{dg-ef}-bk & \frac{-i}{dg-ef} - bl & \frac{-j}{dg-ef}-bm-\frac{c}{hl-ik}
% 	\end{pmatrix}.
% \]
% Now we investigate when the product $s_1s_2s_1yx$ lies in $Bs_2B$ which will lead to an easy evaluation of $\delta_6$.
% First, we require that $-fk\neq 0$ which is already true since $f,k\neq 0$.
% Secondly, we require that $-dk=0$ which tells us that $d=0$.
% Next, we require that $\frac{-h}{dg-ef}-bk=\frac{h}{ef}=0$ which tells us that $e\neq 0$ and we can rearrange to see that $b=\frac{h}{efk}$.
% Lastly, we require that $\frac{-i}{dg-ef}-bl\neq 0$ which we can rearrange to see that $i\neq befl = (\frac{h}{efk})efl = \frac{hl}{k}$, which is already true since $x\in Bs_1B$.

% We see that $s_1s_2s_1yx$ lies in $Bs_2B$ when $e,f,k\neq 0$, $d=0$, $b=\frac{h}{efk}$ and $c,g,h,i,j,l,m\in\FF_q$ are arbitrary.
% Thus,
% \[
% 	|B|\delta_6 = \underbrace{q^7}_{c,g,h,i,j,l,m\in\FF_q}\cdot\underbrace{(q-1)^3}_{e,f,k\neq 0}\cdot \underbrace{1}_{d=0}\cdot \underbrace{1}_{b=\frac{h}{efk}}  = q^7(q-1)^3.
% \]
% Then $|B|(\gamma_6-\delta_6)=0$ and $\gamma_6=\delta_6$.
% Thus, $TST=STS$.
% This completes our example of $G=\SL_3(\FF_q)$ and we conclude that
% \[
% 	\calH(G,B) = \left\langle T,S\ \Bigg|\ \begin{array}{c}
% 		T^2 = (q-1)T + qI, \\
% 		S^2=(q-1)S+qI,
% 	\end{array}TST=STS\right\rangle.
% \]
% As in Section \ref{Section3.2}, we can write down the inverses of $T$ and $S$:
% \[
% 	T^{-1} = q^{-1}T + (q^{-1}-1)I,\quad S^{-1} = q^{-1}S + (q^{-1}-1)I.
% \]

% %%%%%%%%%%%%%%%%%%%%%%%%%%%%%%%%%%%%%%%%%%%%%%%%%%%%%%%%%%%%%%%%%%%%%%%%%%%%%%%%%%%%%%%%%%%%%%

% \subsection{An even less simple case, $G=\SL_4(\FF_q)$}\label{Section3.4}
% Consider the symmetric group
% \[
% 	S_4 = \left\langle s_1,s_2,s_3\ \Bigg|\ \begin{array}{c} s_i^2=1,\\ s_is_{i+1}s_i=s_{i+1}s_is_{i+1},\\ s_1s_3=s_3s_1\end{array}\right\rangle.
% \]
% We fix a choice of permutation matrices in $\SL_4(\FF_q)$ that generate $S_4$:
% \[
% 	s_1 =
% 	\begin{pmatrix}
% 		0 & 1 & 0  & 0 \\
% 		1 & 0 & 0  & 0 \\
% 		0 & 0 & -1 & 0 \\
% 		0 & 0 & 0  & 1
% 	\end{pmatrix}, \
% 	s_2 =
% 	\begin{pmatrix}
% 		-1 & 0 & 0 & 0 \\
% 		0  & 0 & 1 & 0 \\
% 		0  & 1 & 0 & 0 \\
% 		0  & 0 & 0 & 1
% 	\end{pmatrix}, \
% 	s_3 =
% 	\begin{pmatrix}
% 		1 & 0  & 0 & 0 \\
% 		0 & -1 & 0 & 0 \\
% 		0 & 0  & 0 & 1 \\
% 		0 & 0  & 1 & 0
% 	\end{pmatrix}.
% \]
% As before, $\calH(G,K)$ has a basis $\{\chi_{BwB}\}_{w\in S_4}$.
% For brevity, we write $I:=\chi_B$ and $T_i:= \chi_{Bs_iB}$.
% The objective of this section is to show that
% \[
% 	\calH(G,B) = \left\langle T_1,T_2,T_3\ \Bigg|\ \begin{array}{c}
% 		T_i^2 = (q-1)T_i + qI,             \\
% 		T_iT_{i+1}T_i = T_{i+1}T_iT_{i+1}, \\
% 		T_1T_3=T_3T_1
% 	\end{array}\right\rangle.
% \]
% We begin by investigating the structure of the double cosets $Bs_1B$, $Bs_2B$ and $Bs_3B$.
% Observe that
% \begin{align*}
% 	\underbrace{\begin{pmatrix}
% 			\FF_q^\times & \FF_q        & \FF_q        & \FF_q        \\
% 			0            & \FF_q^\times & \FF_q        & \FF_q        \\
% 			0            & 0            & \FF_q^\times & \FF_q        \\
% 			0            & 0            & 0            & \FF_q^\times \\
% 		\end{pmatrix}}_{B}
% 	\underbrace{\begin{pmatrix}
% 			0 & 1 & 0  & 0 \\
% 			1 & 0 & 0  & 0 \\
% 			0 & 0 & -1 & 0 \\
% 			0 & 0 & 0  & 1
% 		\end{pmatrix}}_{s_1}
% 	\underbrace{\begin{pmatrix}
% 			\FF_q^\times & \FF_q        & \FF_q        & \FF_q        \\
% 			0            & \FF_q^\times & \FF_q        & \FF_q        \\
% 			0            & 0            & \FF_q^\times & \FF_q        \\
% 			0            & 0            & 0            & \FF_q^\times \\
% 		\end{pmatrix}}_{B} & =
% 	\begin{pmatrix}
% 		\color{blue}\FF_q        & \color{blue}\FF_q & \color{blue}\FF_q        & \FF_q        \\
% 		\color{blue}\FF_q^\times & \color{blue}\FF_q & \color{blue}\FF_q        & \FF_q        \\
% 		\color{blue}0            & \color{blue}0     & \color{blue}\FF_q^\times & \FF_q        \\
% 		0                        & 0                 & 0                        & \FF_q^\times
% 	\end{pmatrix},                     \\
% 	\underbrace{\begin{pmatrix}
% 			\FF_q^\times & \FF_q        & \FF_q        & \FF_q        \\
% 			0            & \FF_q^\times & \FF_q        & \FF_q        \\
% 			0            & 0            & \FF_q^\times & \FF_q        \\
% 			0            & 0            & 0            & \FF_q^\times \\
% 		\end{pmatrix}}_{B}
% 	\underbrace{\begin{pmatrix}
% 			-1 & 0 & 0 & 0 \\
% 			0  & 0 & 1 & 0 \\
% 			0  & 1 & 0 & 0 \\
% 			0  & 0 & 0 & 1
% 		\end{pmatrix}}_{s_2}
% 	\underbrace{\begin{pmatrix}
% 			\FF_q^\times & \FF_q        & \FF_q        & \FF_q        \\
% 			0            & \FF_q^\times & \FF_q        & \FF_q        \\
% 			0            & 0            & \FF_q^\times & \FF_q        \\
% 			0            & 0            & 0            & \FF_q^\times \\
% 		\end{pmatrix}}_{B} & =
% 	\begin{pmatrix}
% 		\color{blue}\FF_q^\times & \color{blue}\FF_q          & \color{blue}\FF_q   & \FF_q                   \\
% 		\color{blue}0            & \color{violet}\FF_q        & \color{violet}\FF_q & \color{red}\FF_q        \\
% 		\color{blue}0            & \color{violet}\FF_q^\times & \color{violet}\FF_q & \color{red}\FF_q        \\
% 		0                        & \color{red}0               & \color{red}0        & \color{red}\FF_q^\times
% 	\end{pmatrix},                     \\
% 	\underbrace{\begin{pmatrix}
% 			\FF_q^\times & \FF_q        & \FF_q        & \FF_q        \\
% 			0            & \FF_q^\times & \FF_q        & \FF_q        \\
% 			0            & 0            & \FF_q^\times & \FF_q        \\
% 			0            & 0            & 0            & \FF_q^\times \\
% 		\end{pmatrix}}_{B}
% 	\underbrace{\begin{pmatrix}
% 			1 & 0  & 0 & 0 \\
% 			0 & -1 & 0 & 0 \\
% 			0 & 0  & 0 & 1 \\
% 			0 & 0  & 1 & 0
% 		\end{pmatrix}}_{s_3}
% 	\underbrace{\begin{pmatrix}
% 			\FF_q^\times & \FF_q        & \FF_q        & \FF_q        \\
% 			0            & \FF_q^\times & \FF_q        & \FF_q        \\
% 			0            & 0            & \FF_q^\times & \FF_q        \\
% 			0            & 0            & 0            & \FF_q^\times \\
% 		\end{pmatrix}}_{B} & =
% 	\begin{pmatrix}
% 		\FF_q^\times & \FF_q                   & \FF_q                   & \FF_q            \\
% 		0            & \color{red}\FF_q^\times & \color{red}\FF_q        & \color{red}\FF_q \\
% 		0            & \color{red}0            & \color{red}\FF_q        & \color{red}\FF_q \\
% 		0            & \color{red}0            & \color{red}\FF_q^\times & \color{red}\FF_q
% 	\end{pmatrix}.
% \end{align*}
% Notice that copies of $Bs_1B\subseteq\SL_3(\FF_q)$ and $Bs_2B\subseteq\SL_3(\FF_q)$ from the previous example (highlighted in {\color{blue}blue} and {\color{red}red} with their common entries in {\color{violet}purple}) lie in these three double cosets in $\SL_4(\FF_q)$.
% We multiply two matrices belonging to $Bs_1B$ together to see that
% \[
% 	\underbrace{\begin{pmatrix}
% 			\FF_q        & \FF_q & \FF_q        & \FF_q        \\
% 			\FF_q^\times & \FF_q & \FF_q        & \FF_q        \\
% 			0            & 0     & \FF_q^\times & \FF_q        \\
% 			0            & 0     & 0            & \FF_q^\times
% 		\end{pmatrix}}_{Bs_1B}
% 	\underbrace{\begin{pmatrix}
% 			\FF_q        & \FF_q & \FF_q        & \FF_q        \\
% 			\FF_q^\times & \FF_q & \FF_q        & \FF_q        \\
% 			0            & 0     & \FF_q^\times & \FF_q        \\
% 			0            & 0     & 0            & \FF_q^\times
% 		\end{pmatrix}}_{Bs_1B} =
% 	\begin{pmatrix}
% 		\FF_q        & \FF_q & \FF_q        & \FF_q        \\
% 		\FF_q^\times & \FF_q & \FF_q        & \FF_q        \\
% 		0            & 0     & \FF_q^\times & \FF_q        \\
% 		0            & 0     & 0            & \FF_q^\times
% 	\end{pmatrix}.
% \]
% Then the product of two elements in $Bs_1B$ lies in $Bs_1B$.
% The same is true for $Bs_2B$ and $Bs_3B$.

% After making these observations, it is clear that the $\SL_4(\FF_q)$ case will reduce to the $\SL_3(\FF_q)$ case when verifying the relations $T_i^2 = (q-1)T_i + qI$ and $T_iT_{i+1}T_i = T_{i+1}T_iT_{i+1}$, just as verifying these conditions reduced to the $\SL_2(\FF_q)$ case.
% However, we must still manually verify the relation $T_1T_3=T_3T_1$.
% First, we write
% \[
% 	T_1T_3 = \sum_{w\in S_4} \alpha_w \chi_{BwB},\quad T_3T_1 = \sum_{w\in S_4} \beta_w \chi_{BwB}.
% \]
% Then we may write
% \[
% 	\alpha_w-\beta_w = \frac{1}{|B|}\sum_{x\in G} T_3(x^{-1})[T_1(wx)-T_1(xw)].
% \]
% For the summand to be non-zero, we require that $x\in Bs_3B$.
% Fix $x\in Bs_3B$, so it has the form
% \[
% 	x = \begin{pmatrix}
% 		\frac{1}{e(hk-ij)} & b & c & d \\
% 		0                  & e & f & g \\
% 		0                  & 0 & h & i \\
% 		0                  & 0 & j & k
% 	\end{pmatrix},
% \]
% with $ej\neq 0$ and $i\neq\frac{hk}{j}$.
% Notice that if $wx\notin Bs_1B$ and $xw\notin Bs_1B$ then $T_1(wx)-T_1(xw)=0$.
% We show that if $w\neq s_1s_3$ then $wx\notin Bs_1B$ and $xw\notin Bs_1B$ (hence, $\alpha_w-\beta_w=0$ for all $w\in S_4-\{s_1s_3\}$).
% The contrapositive of this statement is that if $wx\in Bs_1B$ or $xw\in Bs_1B$ then $w=s_1s_3$, which we now prove.

% Suppose that $wx\in Bs_1B$.
% Note that left-multiplication by $w$ permutes the rows of $x$ according to $w$.
% Recall that elements of $Bs_1B$ have the form
% \[
% 	\begin{pmatrix}
% 		\FF_q        & \FF_q & \FF_q        & \FF_q        \\
% 		\FF_q^\times & \FF_q & \FF_q        & \FF_q        \\
% 		0            & 0     & \FF_q^\times & \FF_q        \\
% 		0            & 0     & 0            & \FF_q^\times
% 	\end{pmatrix}.
% \]
% We see that, for $wx$ to lie in $Bs_1B$, $w$ must perform the row operation $R_1\mapsto R_2$.
% We also see that $w$ must not perform the row operations $R_2\mapsto R_3$ or $R_2\mapsto R_4$ since $e\neq 0$.
% This means $w$ must perform the row operation $R_2\mapsto R_1$.
% Lastly, $w$ cannot perform the row operation $R_4\mapsto R_4$ since $j\neq 0$.
% This means $w$ must be the permutation matrix $s_1s_3 = (1\ 2)(3\ 4)\in S_4$.

% Similarly, suppose that $xw\in Bs_1B$.
% Note that right-multiplication by $w$ permutes the columns of $x$ according to $w$.
% We see that $w$ cannot perform the column operation $C_1\mapsto C_1$, $C_1\mapsto C_3$ or $C_1\mapsto C_4$ since the bottom three entries of $C_1$ are $0$.
% Then $w$ must perform the column operation $C_1\mapsto C_2$.
% We also see that $w$ cannot perform the column operation $C_2\mapsto C_3$ or $C_2\mapsto C_4$ since the bottom two entries of $C_2$ are $0$.
% Then $w$ must perform the column operation $C_2\mapsto C_1$.
% Lastly, $w$ cannot perform the column operation $C_3\mapsto C_3$ since $j\neq 0$.
% This means $w$ must be the permutation matrix $s_1s_3 = (1\ 2)(3\ 4)\in S_4$.

% We have shown that $\alpha_w=\beta_w$ for all $w\in S_4-\{s_1s_3\}$.
% Now we show that $\alpha_{s_1s_3}=\beta_{s_1s_3}$ by manually computing both coefficients.
% Observe that
% \begin{align*}
% 	\alpha_{s_1s_3} & = (T_1\star T_3)(s_1s_3) = \frac{1}{|B|}\sum_{x\in G} T_1(s_1s_3x)T_3(x^{-1}), \\
% 	\beta_{s_1s_3}  & = (T_3\star T_1)(s_1s_3) = \frac{1}{|B|}\sum_{x\in G} T_3(x^{-1})T_1(xs_1s_3).
% \end{align*}
% To count $\alpha_{s_1s_3}$, we count how many ways $s_1s_3x$ can lie in $Bs_1B$.
% We compute
% \[
% 	\underbrace{
% 		\begin{pmatrix}
% 			0 & 1 & 0  & 0 \\
% 			1 & 0 & 0  & 0 \\
% 			0 & 0 & -1 & 0 \\
% 			0 & 0 & 0  & 1
% 		\end{pmatrix}}_{s_1}
% 	\underbrace{\begin{pmatrix}
% 			1 & 0  & 0 & 0 \\
% 			0 & -1 & 0 & 0 \\
% 			0 & 0  & 0 & 1 \\
% 			0 & 0  & 1 & 0
% 		\end{pmatrix}}_{s_3}
% 	\underbrace{\begin{pmatrix}
% 			\frac{1}{e(hk-ij)} & b & c & d \\
% 			0                  & e & f & g \\
% 			0                  & 0 & h & i \\
% 			0                  & 0 & j & k
% 		\end{pmatrix}}_{x} =
% 	\begin{pmatrix}
% 		0                  & -e & -f & -g \\
% 		\frac{1}{e(hk-ij)} & b  & c  & d  \\
% 		0                  & 0  & -j & -k \\
% 		0                  & 0  & h  & i
% 	\end{pmatrix}.
% \]
% For $s_1s_3x$ to lie in $Bs_1B$, we require that $h=0$.
% We also require that $i\neq 0$ but this is already true since we require that $i\neq \frac{hk}{j}=0$.
% We also require that $j\neq 0$ and $\frac{1}{e(hk-ij)}\neq 0$ but these are already true since $x\in Bs_3B$.
% Overall, we require that $e,i,j\neq 0$, $h=0$ and $b,c,d,f,g,k\in\FF_q$.
% Thus,
% \[
% 	|B|\alpha_{s_1s_3} = \underbrace{q^6}_{b,c,d,f,g,k\in\FF_q}\cdot\underbrace{(q-1)^3}_{e,i,j\neq 0} \cdot \underbrace{1}_{h=0} = q^6(q-1)^3.
% \]
% To count $\beta_{s_1s_3}$, we count how many ways $xs_1s_3$ can lie in $Bs_1B$.
% We compute
% \[
% 	\underbrace{\begin{pmatrix}
% 			\frac{1}{e(hk-ij)} & b & c & d \\
% 			0                  & e & f & g \\
% 			0                  & 0 & h & i \\
% 			0                  & 0 & j & k
% 		\end{pmatrix}}_{x}
% 	\underbrace{
% 		\begin{pmatrix}
% 			0 & 1 & 0  & 0 \\
% 			1 & 0 & 0  & 0 \\
% 			0 & 0 & -1 & 0 \\
% 			0 & 0 & 0  & 1
% 		\end{pmatrix}}_{s_1}
% 	\underbrace{\begin{pmatrix}
% 			1 & 0  & 0 & 0 \\
% 			0 & -1 & 0 & 0 \\
% 			0 & 0  & 0 & 1 \\
% 			0 & 0  & 1 & 0
% 		\end{pmatrix}}_{s_3} =
% 	\begin{pmatrix}
% 		b & \frac{-1}{e(hk-ij)} & d & -c \\
% 		e & 0                   & g & -f \\
% 		0 & 0                   & i & -h \\
% 		0 & 0                   & k & -j
% 	\end{pmatrix}.
% \]
% For $xs_1s_3$ to lie in $Bs_1B$, we require that $k=0$.
% We also require that $i\neq 0$ but this is already true since we require that $i\neq \frac{hk}{j}=0$.
% We also require that $j\neq 0$ and $\frac{1}{e(hk-ij)}\neq 0$ but these are already true since $x\in Bs_3B$.
% Overall, we require that $e,i,j\neq 0$, $k=0$ and $b,c,d,f,g,h\in\FF_q$.
% Thus,
% \[
% 	|B|\beta_{s_1s_3} = \underbrace{q^6}_{b,c,d,f,g,h\in\FF_q}\cdot\underbrace{(q-1)^3}_{e,i,j\neq 0} \cdot \underbrace{1}_{k=0} = q^6(q-1)^3.
% \]
% Then $|B|(\alpha_{s_1s_3}-\beta_{s_1s_3}) = 0$ and $\alpha_{s_1s_3}=\beta_{s_1s_3}$.
% Thus $T_1T_3=T_3T_1$.
% This completes our example of $G=\SL_4(\FF_q)$ and we conclude that
% \[
% 	\calH(G,B) = \left\langle T_1,T_2,T_3\ \Bigg|\ \begin{array}{c}
% 		T_i^2 = (q-1)T_i + qI,             \\
% 		T_iT_{i+1}T_i = T_{i+1}T_iT_{i+1}, \\
% 		T_1T_3=T_3T_1
% 	\end{array}\right\rangle.
% \]
% As in the previous cases, we can write down the inverse of $T_i$:
% \[
% 	T_i^{-1} = q^{-1}T_i + (q^{-1}-1)I.
% \]

% %%%%%%%%%%%%%%%%%%%%%%%%%%%%%%%%%%%%%%%%%%%%%%%%%%%%%%%%%%%%%%%%%%%%%%%%%%%%%%%%%%%%%%%%%%%%%%

% \subsection{Coxeter groups}\label{Section3.5}
% The previous sections illustrate a relationship between the Weyl group $W\cong S_n$ and the relations generating $\calH(G,B)$.
% In particular, generators of $S_n$ are subject to the braid relation $s_is_{i+1}s_i=s_{i+1}s_is_{i+1}$, whereas generators of $\calH(G,B)$ are subject to the relation $T_iT_{i+1}T_i=T_{i+1}T_iT_{i+1}$.
% Similarly, generators of $S_n$ are subject to the commuting relation $s_is_j=s_js_i$ for $|i-j|>1$, whereas the generators of $\calH(G,B)$ are subject to a similiar relation $T_iT_j=T_jT_i$ for $|i-j|>1$.

% We make some observations about $S_n$.
% Firstly, note that $S_n$ may be generated by a finite set of elements $\{s_1,s_2,\ldots,s_{n-1}\}$, where each element is the simple transposition $s_i = (i\ i+1)$.
% Next, notice that each generating element satisfies $s_i^2=1$, so they are involutions.
% The braid relation may be rewritten as $(s_is_{i+1})^3 = 1$ since $s_i$ and $s_{i+1}$ are involutions.
% Similarly, the commuting relation may be rewritten as $(s_is_j)^2=1$.
% Notice that each of these three properties may be expressed as $(s_is_j)^{m_{ij}}=1$ for some sequence $(m_{ij})_{i,j=1,\ldots,n-1}$.
% Here $m_{ii}=1$, $m_{i,i+1}=3$ and $m_{ij}=2$ for $|i-j|>1$.
% This information can be encoded in the matrix
% \[
% 	M = \begin{pmatrix}
% 		1      & 3      & 2      & \hdots & 2      \\
% 		3      & 1      & 3      & \ddots & \vdots \\
% 		2      & 3      & 1      & \ddots & 2      \\
% 		\vdots & \ddots & \ddots & \ddots & 3      \\
% 		2      & \hdots & 2      & 3      & 1
% 	\end{pmatrix}.
% \]
% Generally we say that $W$ is a \emph{Coxeter group}, and $(W,S)$ is a \emph{Coxeter system} with \emph{Coxeter matrix} $M=(m_{ij})$ if $W$ is generated by the finite set $S=\{s_1,\ldots,s_n\}$ subject to the relations $(s_is_j)^{m_{ij}}=1$, where $m_{ii}=1$ and $m_{ij}\in\{2,3,4,\ldots,\infty\}$ with the convention that $m_{ij}=\infty$ means no relation is imposed \cite{Humphreys90}.

% A few important consequences are immediate from this definition.
% It is easy to see that $s_i$ and $s_j$ commute if and only if $m_{ij}=2$.
% Furthermore, $M$ must always be a symmetric matrix.
% To see this, notice $(s_js_i)^{m_{ij}} = (s_js_i)^{m_{ij}}s_js_j = s_j(s_is_j)^{m_{ij}}s_j = s_js_j = 1$.
% Thus, $m_{ji}=m_{ij}$.

% We will restrict our attention to finite Coxeter groups. Note that all finite Weyl groups are examples of finite Coxeter groups.

% %%%%%%%%%%%%%%%%%%%%%%%%%%%%%%%%%%%%%%%%%%%%%%%%%%%%%%%%%%%%%%%%%%%%%%%%%%%%%%%%%%%%%%%%%%%%%%

% \subsection{The Hecke algebra $\calH_q(W,S)$}\label{Section3.6}
% Fix the field $\FF_q$.
% The Hecke algebra generated by the finite Coxeter group $(W,S)$ over $\FF_q$ is denoted $\calH_q(W,S)$, and is the algebra generated by an identity $I$ and $\{T_s\}_{s\in S}$ subject to the quadratic relation $T_s^2 = (q-1)T_s + qI$ and the braid relations
% \[
% 	\underbrace{T_s\ T_t\ T_s \cdots}_{m_{st}\ \text{terms}} = \underbrace{T_t\ T_s\ T_t \cdots}_{m_{st}\ \text{terms}}.
% \]
% Let $G=\SL_n(\FF_q)$ and $B=B(\FF_q)$.
% Sections \ref{Section3.2}, \ref{Section3.3} and \ref{Section3.4} illustrate that $\calH(G,B)$ is actually a Hecke algebra generated by the Coxeter group $S_n$.
% Then we may write
% \[
% 	\calH(G,B) = \left\langle T_1, T_2, \ldots, T_3\ \Bigg|\ \begin{array}{c}
% 		T_i^2 = (q-1)T_i + qI, \\
% 		\underbrace{T_iT_jT_i\ldots}_{m_{ij}\ \text{terms}} = \underbrace{T_jT_iT_j\ldots}_{m_{ij}\ \text{terms}}
% 	\end{array}\right\rangle.
% \]
% More generally, if $G$ is a group of Lie type over a finite field, $B$ is its Borel subgroup, and $(W,S)$ is the Coxeter system of its Weyl group, then $\calH(G,B) = \calH_q(W,S)$.

% In \cite{Coxeter35}, Coxeter presented a classification of the finite Coxeter groups.
% This contains four infinite families of groups, and eight other groups.
% One of these infinite families are the dihedral groups, $D_{2n}$.
% Recall that the dihedral group has the familiar presentation $\langle r,f\ |\ r^m =f^2=(rf)^2=1\rangle$, where $r$ represents a rotation of the $n$-gon by $\frac{2\pi}{n}$ radians and $f$ represents a flip of the $n$-gon.
% Notice that there is another presentation $\langle s,t\ |\ s^2=t^2=(st)^n =1\rangle$.
% Then $D_{2n}$ has the Coxeter matrix $\left(\begin{smallmatrix}1&n\\ n&1\end{smallmatrix}\right)$.
% We conclude that the Hecke algebra generated by $D_{2n}$ over $\FF_q$ is given by
% \[
% 	\calH_q(D_{2n},\{s,t\}) = \left\langle T_s,T_t\ \Bigg|\ \begin{array}{c}
% 		T_s^2 = (q-1)T_s+qI, \\
% 		T_t^2 = (q-1)T_t+qI,
% 	\end{array}
% 	\underbrace{T_sT_tT_s\ldots}_{n\ \text{terms}} = \underbrace{T_tT_sT_t\ldots}_{n\ \text{terms}}\right\rangle.
% \]

% %%%%%%%%%%%%%%%%%%%%%%%%%%%%%%%%%%%%%%%%%%%%%%%%%%%%%%%%%%%%%%%%%%%%%%%%%%%%%%%%%%%%%%%%%%%%%%

% \newpage
% \section{Hecke Algebras of Locally Compact Groups}\label{Chapter4}
% This chapter is devoted to generalising the notion of the Hecke algebra of a finite group.
% In particular, we weaken the condition that $G$ and $K$ must be finite to the condition that $G$ must be a locally compact topological group and $K$ must be a compact subgroup.
% We add the additional assumption that the functions in the Hecke algebra must have compact support.
% This is done for technical purposes, namely so that the functions of the Hecke algebra may be expressed as finite sums of basis elements.
% Since our groups are not longer finite, the convolution product on $\Fun(G)$ defined in Section \ref{Section1.1} no longer results in a finite sum of $\CC$-valued functions, so it cannot be used.
% To deal with this, we must introduce the notion of the Haar measure.
% This allows us to define the convolution product in terms of an integral with respect to the Haar measure.
% We see that this new Hecke algebra reduces to the Hecke algebra of Chapter \ref{Chapter1} when $G$ is finite.

% %%%%%%%%%%%%%%%%%%%%%%%%%%%%%%%%%%%%%%%%%%%%%%%%%%%%%%%%%%%%%%%%%%%%%%%%%%%%%%%%%%%%%%%%%%%%%%

% \subsection{Locally compact groups and Haar measures}\label{Section4.1}
% We say that a group $G$ is a \emph{topological group} if it is equipped with a topology such that the group multiplication map $\mu\colon G\times G\to G$ and the group inversion map $\iota\colon G\to G$ are continuous.
% Here the product space $G\times G$ is equipped with the product topology.

% We can equip any group with the discrete topology (i.e.\ the topology where all sets are open) to form a topological group.
% Some more interesting examples to keep in mind are:
% \begin{enumerate}[\itshape(i)]
% 	\item $(\RR^n,+)$ equipped with the Euclidean topology.
% 	\item $(\QQ,+)$ equipped with the subspace topology from $\RR^n$.
% 	\item $\GL_n(\RR)$.
% 	      This may be viewed as a subspace of $\RR^{n^2}$ via an `obvious' injection (see below).
% 	\item $\mO_n(\RR)$.
% 	      This is a subgroup of $\GL_n(\RR)$ and is equipped with the subspace topology.
% 	\item $\GL_n(\FF_q[\![t]\!])$ and $\GL_n(\FF_q(\!(t)\!))$.
% 	      Here $\FF_q[\![t]\!]$ is the field of formal power series over $\FF_q$ in the variable $t$ and $\FF_q(\!(t)\!)$ is the field of formal Laurent series over $\FF_q$ in variable $t$.
% 	      More generally, we may take $\GL_n$ over any non-archimedian local field $k$ or its ring of integers $\calO$ (see Chapter \ref{Chapter5}).
% \end{enumerate}
% Recall that a topological space is \emph{compact} if every open cover contains a finite subcover.
% Furthermore, a topological space is \emph{locally compact} if every point has a compact neighbourhood.
% We say that a topological group $G$ is (locally) compact if its associated topological space is (locally) compact.
% Some examples are $\mO_n(\RR)$ (compact), $\GL_n(\FF_q[\![t]\!])$ (compact) and $\GL_n(\FF_q(\!(t)\!))$ (locally compact).

% In order to define the Hecke algebra of a locally compact group, we want to define a notion of integration over topological groups.
% Those familiar with measure theory should already know that this requires the notion of measurable sets, measurable functions and measures.
% To this end, we present some concepts from measure theory.
% See \cite{Folland84} for details.

% The topology of $G$ can be used to generate a \emph{$\sigma$-algebra} on $G$.
% Here a $\sigma$-algebra on $G$ means a collection of subsets of $G$ that includes $G$, is closed under set-complements in $G$, and is closed under countable unions.
% De Morgan's laws tell us that this implies closure under countable intersections as well.
% The $\sigma$-algebra induced by the topology of $G$ is called the \emph{Borel $\sigma$-algebra} is the $\sigma$-algebra generated by the open sets of $G$.
% That is, every open set is contained in the Borel $\sigma$-algebra, along with every complement, countable union and countable intersection as well.

% We call the elements of a $\sigma$-algebra \emph{measurable sets}.
% Given a $\sigma$-algebra, we can define a notion of length on the measurable sets.
% Suppose that $\Sigma$ is a $\sigma$-algebra.
% The function $\mu\colon\Sigma\to[0,\infty]$ is a \emph{measure} if $\mu(\varnothing)=0$, and whenever the sets $E_1,E_2,\ldots\in\Sigma$ are disjoint, there holds the countable additivity property
% \[
% 	\mu\Bigg(\bigcup_{i=1}^\infty E_i\Bigg) = \sum_{i=1}^n \mu(E_i).
% \]
% A consequence of the above property is the countable sub-additivity property: if $E_1,E_2,\ldots\in\Sigma$ is any sequence (not necessarily disjoint), then there holds
% \[
% 	\mu\Bigg(\bigcup_{i=1}^\infty E_i\Bigg) \leq \sum_{i=1}^n \mu(E_i).
% \]
% Now fix a locally compact topological group $G$ and fix $\Sigma$ to be the Borel $\sigma$-algebra of $G$.
% A \emph{left Haar measure} on $G$ is a non-zero measure $\mu_l\colon \Sigma\to [0,\infty]$ such that, for all $g\in G$ and $S\in\Sigma$, there holds $\mu_l(gS)=\mu_l(S)$, $\mu_l(K)<\infty$ for all compact $K\subseteq G$, and
% \[
% 	\mu_l(S)=\inf\{\mu_l(U)\ |\ \text{open}\ U\supseteq S\} = \sup\{\mu_l(K)\ |\ \text{compact}\ K\subseteq S\}.
% \]
% This last condition is called \emph{regularity} of $\mu_l$.
% Here the set $gS$ is the \emph{left-translation} of $S$ by $g$.
% A \emph{right Haar measure} $\mu_r$ is defined similarly, instead with the right-translation invariance property that $\mu_r(Sg)=\mu_r(S)$.

% In \cite{Folland84}, it is shown that all locally compact topological groups possess a left/right Haar measure.
% Furthermore, this Haar measure is unique up to a positive scalar multiple.
% Then we do not need to concern ourselves with the question of existence and uniqueness of Haar measures.

% A familiar example of a left Haar measure is the Lebesgue measure on $\RR^n$.
% This is an abelian group, so the left Haar measure is also clearly a right Haar measure.
% We investigate the relationship between left and right Haar measures now.

% Let $\mu$ be a left Haar measure on $G$.
% Take $x\in G$ and some measurable subset $E\subseteq G$.
% Then the measure $\mu_x$ defined by $\mu_x(E) := \mu(Ex)$ is also a left Haar measure by the associativity of $G$.
% Then, since all left Haar measures are a scalar multiple of each other, there exists a scalar $\Delta(x)$ such that $\mu_x = \Delta(x)\mu$.
% This defined a function $\Delta\colon G\to (0,\infty)$ called the \emph{modular function} of $G$.
% This is a continuous group homomorphism from $G$ to $(0,\infty)$.

% The modular function describes the relationship between left and right Haar measures.
% Specifically, left and right Haar measures coincide when $\Delta\equiv 1$.
% A locally compact group for which every left Haar measure is also a right Haar measure is called \emph{unimodular}.

% We present some propositions in \cite{Folland84}.
% \begin{prop}
% 	If $G/[G,G]$ is finite the $G$ is unimodular.
% \end{prop}
% \begin{proof}
% 	The modular function is a continuous group homomorphism from $G$ to $(0,\infty)$, the latter of which is abelian.
% 	Then this map $\Delta$ must annihilate the commutators $[x,y]$ for all $x$ and $y$, so it must factor through $G/[G,G]$.
% 	Since $G/[G,G]$ is finite, we know $\Delta(G)$ is a finite subgroup of $(0,\infty)$.
% 	However, the only finite subgroup of $(0,\infty)$ is the trivial subgroup.
% \end{proof}
% \begin{prop}
% 	If $G$ is compact then $G$ is unimodular.
% \end{prop}
% \begin{proof}
% 	It's clear that $G=Gx$, for any $x\in G$.
% 	Now let $\mu$ be a left Haar measure and notice
% 	\[
% 		\mu(G) = \mu(Gx) = \Delta(x)\mu(G).
% 	\]
% 	Since $G$ is compact, we have $0<\mu(G)<\infty$.
% 	Then we must have $\Delta(x)=1$ for any choice of $x$.
% \end{proof}
% We sketch an example of a Haar measure for the case $G=\GL_2(\RR)^+$, the subgroup of $\GL_2(\RR)$ of matrices with positive determinants.
% We may identify $\GL_2(\RR)^+$ with an open subset of $\RR^4$.
% Specifically, we make the identification
% \[
% 	\begin{pmatrix}
% 		x_1 & x_2 \\ x_3 & x_4
% 	\end{pmatrix} \longleftrightarrow (x_1,x_2,x_3,x_4).
% \]
% Now take a measurable set $E\subseteq \GL_2(\RR)^+$.
% To define the measure of $E$, we employ the Lebesgue measure on $\RR^4$.
% Specifically, define the measure of $E$ by
% \[
% 	\mu(E) := \int_E \frac{1}{(x_1x_4-x_2x_3)^2}\ dx_1dx_2dx_3dx_4 = \int_E \frac{1}{(\det x)^2}\ dx,
% \]
% where the left integral is easily understood as the Lebesgue integral of a real-valued function on $\RR^4$.
% To show that $\mu$ is a Haar measure, we note that the Lebesgue integral endows $\mu$ with many properties.
% It is clear that $\mu$ will be non-negative, positive definite, countably additive, finite on compact sets and regular.
% The only property left to check is left-translation invariance.
% To check this, fix $g\in \GL_2(\RR)^+$ and we consider the translation map $T_g\colon \RR^4\to\RR^4$ given by $T_g(x):=gx$.
% Then we have
% \[
% 	\mu(gE) = \int_{T(E)} \frac{1}{(\det x)^2} \ dx_1 dx_2 dx_3 dx_4.
% \]
% We make the change of variables $x=T_g(y) = gy$ which yields
% \[
% 	\mu(gE) = \int_E \frac{1}{(\det gy)^2} \Jac T_g\ dy_1dy_2dy_3dy_4,
% \]
% where $\Jac T_g$ is the Jacobian of the transformation $T_g$.
% One can compute $\Jac T_g = (\det g)^2$ which allows us to compute
% \[
% 	\mu(gE) = \int_E \frac{1}{(\det gy)^2} \Jac T_g\ dy_1dy_2dy_3dy_4 = \int_E \frac{1}{(\det y)^2}\ dy_1dy_2dy_3dy_4 = \mu(E).
% \]
% We see that $\mu$ is left-translation invariant and $\mu$ is indeed a left Haar measure.

% %%%%%%%%%%%%%%%%%%%%%%%%%%%%%%%%%%%%%%%%%%%%%%%%%%%%%%%%%%%%%%%%%%%%%%%%%%%%%%%%%%%%%%%%%%%%%%

% \subsection{The Hecke algebra of a locally compact group $C_c(K\backslash G/K)$}\label{Section4.2}
% Consider a unimodular locally compact topological group $G$ and some open and compact subgroup $K$.
% Note that a non-trivial $K$ might not exist for some choices of $G$.
% However, we will see in Chapter \ref{Chapter5} that non-trivial choices for $K$ exist when $G$ is a matrix group over a non-archimedian local field.

% Then the Hecke algebra $C_c(K\backslash G/K)$ is defined as
% \[
% 	C_c(K\backslash G/K) := \{f\colon G\to\CC\ |\ \supp f\ \text{is compact, and}\ f(k_1gk_2) = f(g),\ \forall g\in G,\ \forall k_1,k_2\in K\}.
% \]
% Notice that, since $K$ is open and compact, the topological space $K\backslash G/K$ is discrete.
% This means that a function on $K\backslash G/K$ having compact support is equivalent to it having finite support.

% Clearly $C_c(K\backslash G/K)$ has the structure of a complex vector space.
% We wish to endow it with the structure of an algebra.
% To this end, we equip the space with the convolution product defined by
% \[
% 	(f\star f')(x) = \int_G f(xg)f'(g^{-1})\ d\mu(g),
% \]
% where $d\mu(g)$ denotes integration with respect to the Haar measure $\mu$ in the variable $g$.
% Recall that $\mu$ is unique up to a scalar multiple.
% It is common to choose this scalar so that $\mu(K)=1$, which uniquely fixes $\mu$.
% As in Chapter \ref{Chapter1}, the Hecke algebra here is an associative algebra.

% It is natural to ask whether or not this construction of the Hecke algebra is a true generalisation of the Hecke algebra throughout Chapter \ref{Chapter1}.
% In other words, we want to recover the usual definitions of $\star$ and $\calH$ is the case that $G$ is a finite group.

% To this end, let $G$ be finite and equipped with the discrete topology (i.e.\ every subset is open).
% In this topology, every subset is compact.
% Then it is easy to see that $G$ is locally compact.
% Furthermore, the Borel $\sigma$-algebra will contain every subset of $G$.

% Define the function $\mu\colon \Sigma\to[0,\infty]$ by $\mu(A) := |A|$, the cardinality of $A$.
% It is easy to verify that this is a measure and it is known as the \emph{counting measure}.
% To see that it is a Haar measure, we can see that $\mu$ is left-invariant since $|S|=|gS|$ for any $g\in G$ and $S\subseteq G$.
% Secondly, notice that $\mu(K)<\infty$ for all compact $K$ since every subset of $G$ is finite.
% Lastly, since every subset of $G$ is open and compact, sub-additivity of measures tells us that the condition $\mu(S) = \inf\{\mu(U)\ |\ \text{open}\ U\supseteq S\} = \sup\{\mu(K)\ |\ \text{compact}\ K\subseteq S\}$ is satisfied by picking $U=K=S$.

% Next, notice that in this setting, all functions $G\to\CC$ have compact support.
% Then pick two functions $f,f'\colon G\to\CC$ and any $x\in G$.
% Observe that
% \[
% 	(f\star f')(x) = \int_G f(xg)f'(g^{-1})\ d\mu(g) = \sum_{x\in G} f(xg)f'(g^{-1}) = \sum_{yz = x} f(y)f'(z),
% \]
% since $\mu$ is the counting measure.
% We see that we truly recover the previous results of Chapter \ref{Chapter1} when we choose $G$ to be a finite group.

% %%%%%%%%%%%%%%%%%%%%%%%%%%%%%%%%%%%%%%%%%%%%%%%%%%%%%%%%%%%%%%%%%%%%%%%%%%%%%%%%%%%%%%%%%%%%%%

% \newpage
% \section{Spherical Hecke Algebras and Iwahori--Hecke Algebras}\label{Chapter5}
% In this chapter, we investigate two Hecke algebras: the spherical Hecke algebra and the Iwahori--Hecke algebra.
% In order to define these algebras, we require an understanding of local fields and their structure.
% In Section \ref{Section5.1}, we present the notions of a field extension, finite extension, absolute values and the $p$-adic numbers.
% This allows us to succinctly classify all local fields.
% Next, we describe the structure of local fields in Section \ref{Section5.2} and explicitly compute some of these objects to aid in our understanding of these fields.
% We define the spherical Hecke algebra and give a proof that it is commutative.
% We conclude with the definition of the Iwahori subgroup and an investigation of the Iwahori--Hecke algebra.

% %%%%%%%%%%%%%%%%%%%%%%%%%%%%%%%%%%%%%%%%%%%%%%%%%%%%%%%%%%%%%%%%%%%%%%%%%%%%%%%%%%%%%%%%%%%%%%

% \subsection{Local fields}\label{Section5.1}
% Fix a field $F$.
% We call $K\subset F$ a \emph{subfield} of $F$ if $K$ is a field with respect to the addition and multiplication operations equipped to $F$.
% There are many immediate examples.
% For instance, $\QQ\subset\QQ(\sqrt{2})$, $\QQ\subset\RR$, $\RR\subset\CC$, and so on.

% If $K\subset F$ is a subfield then we call $F$ a \emph{field extension} of $K$ and denote this extension by $F/K$.
% The \emph{degree} of the extension $F/K$ is the dimension of $F$ when considered as a $K$-vector space.
% The degree of $F/K$ is written $[F:K]$ and we say $F/K$ is a \emph{finite extension} when $[F:K]<\infty$.

% For instance, there holds $[\CC:\RR]=2$.
% To see this, we consider $\CC$ as a real vector space.
% We convince ourselves that $\{1,i\}$ is an appropriate basis that demonstrates $\CC/\RR$ is a field extension of degree $2$.

% As another example, consider $[\RR:\QQ]$.
% We can think of infinitely many irrational numbers that are not rational scalar multiples of each other.
% For example, $\sqrt{2}$, $\log 3$, $\pi$, and so on.
% This tells us that we can't find a finite basis for $\RR$ as a $\QQ$-vector space.
% Then $[\RR:\QQ]$ is not finite and $\RR/\QQ$ is not a finite field extension.

% Our fields of interest are $\QQ_p$, the \emph{$p$-adic numbers}, and $\FF_q(\!(t)\!)$, the \emph{formal Laurent series}.
% We remind the reader that the latter field is given by
% \[
% 	\FF_q(\!(t)\!) := \Bigg\{\sum_{i=N}^\infty a_it^i\ \Bigg|\ a_i\in\FF_q,\ N\in\ZZ,\ a_N\neq 0\Bigg\}.
% \]
% To describe $\QQ_p$, we require a definition that equips fields with a notion of norm.

% A \emph{norm} or an \emph{absolute value} on a ring $R$ is a function $|\cdot|\colon R\to[0,\infty)$ such that, for all $r,s\in R$, there holds $|r\cdot s| = |r|\cdot|s|$, $|r+s|\leq |r|+|s|$, and $|r|=0$ if and only if $r=0$.
% If the second condition is replaced with the \emph{strong triangle inequality} $|r+s|\leq \max\{|x|,|y|\}$ then we say that $|\cdot|$ is \emph{non-archimedian}.
% Otherwise, we say that $|\cdot|$ is \emph{archimedian}.

% All fields have a trivial absolute value, given by $|r|_\triv = 1$ for $r\neq 0$ and $|0|_\triv=0$.
% An example of a non-trivial absolute value is the one defined on $\FF_q(\!(t)\!)$.
% Consider an arbitrary element $\sum_{i=N}^\infty a_it^i$ in $\FF_q(\!(t)\!)$ written so that $a_N\neq 0$ if the chosen element of $\FF_q(\!(t)\!)$ is non-zero.
% Then
% \[
% 	\Bigg|\sum_{i=N}^\infty a_it^i\Bigg| := \begin{cases}
% 		0,\       & \text{if}\ a_i = 0\ \text{for all}\ i, \\
% 		q^{-N},\  & \text{else}.
% 	\end{cases}.
% \]
% Another example of an absolute value is found in the construction of $\QQ_p$.

% Fix a prime $p\geq 2$.
% Given any $\frac{a}{b}\in\QQ^\times$, we can uniquely write $\frac{a}{b} = p^n\cdot\frac{x}{y}$ where $\gcd(x,y)=1$ and $p$ does not divide $x$ any further.
% This is a simple consequence of the fundamental theorem of arithmetic.
% Then the \emph{$p$-adic absolute value} $|\cdot|_p\colon \QQ\to[0,\infty)$ is defined by
% \[
% 	\bigg|\frac{a}{b}\bigg|_p := \begin{cases}
% 		0\       & \text{if}\ \frac{a}{b}=0,                                      \\
% 		p^{-n}\  & \text{if}\ \frac{a}{b} = p^n\cdot\frac{x}{y}\ \text{as above}.
% 	\end{cases}
% \]
% The \emph{$p$-adic numbers}, denoted $\QQ_p$, is the completion of $\QQ$ with respect to the $p$-adic absolute value.
% This is analogous to the usual completion of a normed vector space.
% More precisely, we can explicitly write out the elements of $\QQ_p$ as
% \[
% 	\QQ_p = \Bigg\{\sum_{i=N}^\infty a_ip^i\ \Bigg|\ a_i=0,1,2,\ldots,p-1,\ N\in\ZZ,\ a_N\neq0 \Bigg\}.
% \]
% Then the absolute value defined on $\QQ_p$ is given by
% \[
% 	\Bigg|\sum_{i=N}^\infty a_ip^i\Bigg| = p^{-N}.
% \]
% We are almost ready to define local fields.
% We say that a ring $R$ is a \emph{topological ring} if it is equipped with a topology  such that the addition and multiplication maps $+,\cdot\colon R\times R\to R$ are continuous.
% Here the product space $R\times R$ is equipped with the product topology.
% We say that a topological ring is (locally) compact if its associated topological space is (locally) compact.
% A locally compact topological field is a field equipped with a topological space whose underlying ring is a locally compact topological ring.
% Now we are ready to define local fields.

% A \emph{local field} is a locally compact topological field $k$ with respect to a non-trivial absolute value.
% In fact, as remarked in \cite{MilneANT}, all fields satisfying such a property are isomorphic to a small family of fields.
% We summarise the possibilities for $k$:
% \begin{enumerate}[\itshape(i)]
% 	\item If $|\cdot|$ is archimedian, then $k$ is $\RR$ or $\CC$.
% 	\item If $|\cdot|$ is non-archimedian and $\ch k=0$, then $k$ is a finite extension of $\QQ_p$, where $p$ is prime.
% 	\item If $|\cdot|$ is non-archimedian and $\ch k = p$, then $k$ is $\FF_q(\!(t)\!)$, where $q=p^m$ and $m\geq 1$.
% \end{enumerate}
% We present an example of a finite extension of $\QQ_p$.
% Take $\QQ_p(\!(t^{1/m})\!)$ for some prime $p>2$ and $m\in\ZZ$ with $p\nmid m$.
% This field has the elements
% \[
% 	\QQ_p(\!(t^{1/m})\!) = \Bigg\{\sum_{i=N}^\infty a_it^i\ \Bigg|\ a_i\in\QQ_p,\ N\in\ZZ,\ a_N\neq 0 \Bigg\}.
% \]
% Inside $\QQ_p(\!(t^{1/m})\!)$ lies a subfield of constant functions, $K$.
% That is,
% \[
% 	K = \{f \in  \QQ_p(\!(t^{1/m})\!)\ |\ f(t) = c,\ \text{for some}\ c\in\QQ_p\}.
% \]
% Then $K$ is isomorphic to $\QQ_p$ and we can think of $\QQ_p$ as a subfield of $\QQ_p(\!(t^{1/m})\!)$ via this subfield $K$.
% We see that $B=\{t^{1/m},t^{2/m},\ldots,t\}$ is a basis demonstrating $\QQ_p(\!(t^{1/m})\!)$ is a $\QQ_p$-vector space of dimension $m$.
% Thus, $\QQ_p(\!(t^{1/m})\!)$ is a finite extension of $\QQ_p$ with degree $m$.

% %%%%%%%%%%%%%%%%%%%%%%%%%%%%%%%%%%%%%%%%%%%%%%%%%%%%%%%%%%%%%%%%%%%%%%%%%%%%%%%%%%%%%%%%%%%%%%

% \subsection{The structure of non-archimedian local fields}\label{Section5.2}
% Let $k$ be a non-archimedian local field.
% We require some understanding of the structure of these fields.
% To this end, we define the following important objects:
% \begin{enumerate}[\itshape(i)]
% 	\item $\calO := \{x\in k:|x|\leq 1\}$, the \emph{ring of integers} of $k$.
% 	\item $\calO^\times := \{x\in k: |x|=1\}$, the group of units in the ring of integers.
% 	\item $\calP := \calO-\calO^\times = \{x\in k: |x|<1\}$, the unique maximal ideal of $\calO$.
% 	\item $\calK := \calO/\calP$, the \emph{residue field} of $k$.
% 	\item $\varpi :=$ a generator of $\calP$ called a \emph{uniformiser} of $k$.
% 	      Then we may write $\calK = \calO/\varpi\calO$.
% \end{enumerate}
% We compute some of these objects in some cases.

% Suppose that $k=\FF_q(\!(t)\!)$.
% Then $\calO=\{\sum_{i=N}^\infty a_it^i \in k\ |\ q^{-N}\leq 1\}$.
% Note that $q^{-N}\leq 1$ if and only if $N\geq 0$.
% Then $\calO = \FF_q[\![t]\!]$, the ring of formal power series over $\FF_q$.
% Furthermore, $\calO^\times = \{\sum_{i=N}^\infty a_it^i\in k\ |\ q^{-N}=1\}$.
% Note that $q^{-N}=1$ if and only if $N=0$.
% Then $\calO^\times = \{f\in k\ |\ f(0)\neq 0\}$.
% Next, $\calP = \{x\in k\ |\ |x|<1\}$.
% Note that $q^{-N}<1$ if and only if $N\geq 1$.
% Then $\calP = t\FF_q[\![t]\!]$.
% Lastly, $\calK = \FF_q[\![t]\!]/t\FF_q[\![t]\!] \cong \FF_q$.
% This can be seen by the surjective ring homomorphism $\FF_q[\![t]\!]\to\FF_q$ given by $f\mapsto f(0)$.
% It is easily seen that this map has kernel $t\FF_q[\![t]\!]$, so the first isomorphism theorem for rings yields the result.

% Suppose that $k=\QQ_p$.
% Then $\calO=\{\sum_{i=N}^\infty a_ip^i \in k\ |\ p^{-N}\leq 1\}$.
% Note that $p^{-N}\leq 1$ if and only if $N\geq 0$.
% Then $\calO = \ZZ_p$, the ring of $p$-adic integers.
% Furthermore, $\calO^\times = \{\sum_{i=N}^\infty a_ip^i\in k\ |\ p^{-N}=1\}$.
% Note that $p^{-N}=1$ if and only if $N=0$.
% Then $\calO^\times = \{\sum_{i=N}^\infty a_ip^i\in k\ |\ a_0\neq 0\}$.
% Next, $\calP = \{x\in k\ |\ |x|<1\}$.
% Note that $p^{-N}<1$ if and only if $N\geq 1$.
% Then $\calP = \{\sum_{i=N}^\infty a_ip^i\in k\ |\ N\geq 1\}=\{p\sum_{i=N}^\infty a_ip^i\in k\ |\ N\geq 0\} = p\ZZ_p$.
% Lastly, $\calK = \ZZ_p/p\ZZ_p \cong \FF_p$.
% This can be seen by the surjective ring homomorphism $\ZZ_p\to \FF_p$ given by $\sum_{i=N}^\infty a_ip^i \mapsto a_0$.
% It is easily seen that this map has kernel $p\ZZ_p$, so the first isomorphism theorem for rings yields the result.

% We make the observation that $\calK\cong\FF_q$ for all non-archmedian local fields.
% We have not yet shown this for the case when $k$ is a finite extension of $\QQ_p$.
% In this case, $\calO$ will be a finite extension of $\ZZ_p$ and the above procedure can be followed to see that $\calK$ is a finite extension of $\calK_{\QQ_p}\cong\FF_q$, so $\calK$ is also a finite field.

% %%%%%%%%%%%%%%%%%%%%%%%%%%%%%%%%%%%%%%%%%%%%%%%%%%%%%%%%%%%%%%%%%%%%%%%%%%%%%%%%%%%%%%%%%%%%%%

% \subsection{The spherical Hecke algebra $C_c(K^\circ\backslash G/K^\circ)$}\label{Section5.3}
% Choose a non-archimedian local field $k$.
% Then consider $G:=\GL_n(k)$ and $K^\circ := \GL_n(\calO)$.
% To see that $G$ is locally compact, we recall that it is a subspace of $k^{n^2}$, and every open subset of $k^{n^2}$ is locally compact in the subspace topology.
% The ring of integers $\calO$ is open and compact in $k$, so $K^\circ$ is open and compact in $G$.
% The \emph{spherical Hecke algebra} is the Hecke algebra $C_c(K^\circ\backslash G/K^\circ)$, following the notation of Section \ref{Section4.2}.

% We conclude this section with the result that the spherical Hecke algebra is commutative.
% To this end, we present a preparatory lemma of Bump which is stated and proved in \cite{Bump10}.
% \begin{lem}[The $p$-adic Cartan Decomposition]
% 	Every double coset in $K^\circ\backslash G/K^\circ$ has a unique representative of the form
% 	\[
% 		\begin{pmatrix}
% 			\varpi^{\lambda_1} &                    &        &                    \\
% 			                   & \varpi^{\lambda_2} &        &                    \\
% 			                   &                    & \ddots &                    \\
% 			                   &                    &        & \varpi^{\lambda_n}
% 		\end{pmatrix},
% 	\]
% 	where $\varpi$ is the uniformiser of $\calP$ and $\lambda_1\geq \lambda_2\geq\cdots\geq\lambda_n$ are integers.
% \end{lem}
% \begin{thm}\label{sphericalcomm}
% 	The spherical Hecke algebra $C_c(K^\circ\backslash G/K^\circ)$ is commutative.
% \end{thm}
% The following proof follows the idea of Gelfand's Trick from Section \ref{Section1.7}.
% One can follow the proof of Gelfand's Trick in the finite case in order to produce a proof of the Trick in the locally compact case.
% This is because the requirement that functions are compactly supported reduces the convolution product of the Hecke algebra to a finite sum when performing the proof.
% Then the proof in the locally compact case will reduce to the computation we have already performed in the finite case.
% \begin{proof}[Proof of Theorem \ref{sphericalcomm}]
% 	The anti-automorphism $\varphi\colon G\to G$ given by $\varphi(g):=g^t$ is an involution.
% 	It can be pulled back to the map $\varphi^\ast\colon \Fun(G)\to\Fun(G)$ given by $(\varphi^\ast f)(g) = f(\varphi(g)) = f(g^t)$.
% 	This map $\varphi$ preserves $K^\circ$ (i.e.\ the image of $K^\circ$ lies in $K^\circ$).
% 	The previous lemma tells us that each double coset contains a diagonal representative, on which $\varphi$ is clearly the identity.
% 	Then $\varphi^\ast$ is an involutive anti-automorphism acting as the identity on a basis of the Hecke algebra, so it is commutative.
% \end{proof}
% We generalise the notion of a Geland pairs from Chapter \ref{Chapter1}.
% See \cite{Bump10} and \cite{AGS08}.
% A complex representation $G\to\GL(V)$ is said to be \emph{smooth} if the stabiliser $\{k\in G\ |\ k\cdot v = v\}$ is open for every non-zero $v\in V$.
% Furthermore, a smooth representation is \emph{admissible} if, given any open subgroup $K$, the vector subspace of $K$-fixed vectors $V^K$ is finite-dimensional.

% Then the pair $(G,K)$ is said to be a Gelfand pair if, for every admissible irreducible representation $V$ of $G$, the subspace of $K$-fixed vectors $V^K$ is at most one-dimensional.
% This is equivalent to saying that $C_c(K\backslash G/K)$ is commutative.
% In the case that $G$ is a finite group, this is equivalent to saying that $\Ind_K^G \1$ is multiplicity-free (c.f.\ Section \ref{Section1.8}).
% Theorem \ref{sphericalcomm} tells us that $(G,K^\circ)$ is a Gelfand pair.

% %%%%%%%%%%%%%%%%%%%%%%%%%%%%%%%%%%%%%%%%%%%%%%%%%%%%%%%%%%%%%%%%%%%%%%%%%%%%%%%%%%%%%%%%%%%%%%

% \subsection{The Iwahori subgroup $I$}\label{Section5.4}
% Fix a non-archimedian local field $k$.
% Recall its ring of integers $\calO$ and its residue field $\calO/\calP\cong\FF_q$.
% Now suppose that $G$ is a ``nice''  reductive group.
% In particular, suppose that $G$ is a connected, split reductive group over $k$.
% For example, $G(k)=\SL_n(k)$ for $n\geq 2$.

% Consider the group homomorphism $\phi\colon G(\calO)\to G(\calO/\calP)\cong G(\FF_q)$ given by
% \[
% 	\phi((g_{ij})_{i,j=1,\ldots,n})\mapsto (n(g_{ij}))_{i,j=1,\ldots,n},
% \]
% where $n\colon \calO\to\calO/\calP$ is the natural map $x\mapsto x+\calP$.
% The map $\phi$ amounts to applying the natural map to each entry of the matrix.
% Then the \emph{Iwahori subgroup} of $G(\calO)$ is $I:=\phi^{-1}(B(\FF_q))$.

% As an example, if $k=\FF_q(\!(t)\!)$, the natural map $n$ is the evaluation map $t\mapsto 0$, and the map $\phi$ will evaluate the entries at $t=0$.
% For a matrix to lie in the preimage of $B(\FF_q)$, its entries below the diagonal must have a constant term of $0$, the entries on the diagonal must have a non-zero constant term, and the entries above the diagonal may be chosen freely.
% Then
% \[
% 	I = \phi^{-1}(B(\FF_q)) = \begin{pmatrix}
% 		\calO^\times & \calO        & \hdots & \calO        \\
% 		\calP        & \calO^\times & \ddots & \vdots       \\
% 		\vdots       & \ddots       & \ddots & \calO        \\
% 		\calP        & \hdots       & \calP  & \calO^\times
% 	\end{pmatrix}
% \]
% In fact, this characterises the Iwahori subgroup for any non-archimedian local field.
% We require that the elements below the diagonal vanish under $\phi$, so they must lie in $\calP$.
% The diagonal elements must map to a non-zero element under $\phi$, so they must lie in $\calO^\times$.
% We have no restriction on the elements above the diagonal, so they lie in $\calO$.

% %%%%%%%%%%%%%%%%%%%%%%%%%%%%%%%%%%%%%%%%%%%%%%%%%%%%%%%%%%%%%%%%%%%%%%%%%%%%%%%%%%%%%%%%%%%%%%

% \subsection{The Iwahori--Hecke algebra $C_c(I\backslash G/I)$}
% Take $G=G(k)$ and $I$ as defined above.
% Then the \emph{Iwahori--Hecke algebra} is given by $C_c(I\backslash G/I)$ in the notation of Section \ref{Section4.2}.

% We will see that the Iwahori--Hecke algebra contains a subalgebra we are already familiar with.
% Namely, consider the Hecke algebra $C_c(I\backslash G(\calO)/I)$.
% This is clearly a subalgebra of the Iwahori--Hecke algebra; these are the functions in $C_c(I\backslash G/I)$ that are supported on $G(\calO)$.

% Recall the Hecke algebra $\calH(G,B)$ of Chapter \ref{Chapter3}.
% These are the complex-valued functions on $G(\FF_q)=\SL_n(\FF_q)$ that are constant on $B$-double cosets.
% In the notation of Chapter \ref{Chapter4}, this algebra may also be written as $C_c(B(\FF_q)\backslash G(\FF_q)/B(\FF_q))$.

% Recall the map $\phi\colon G(\calO)\to G(\FF_q)$ given in Section \ref{Section5.4}.
% Take $f$ in $C_c(B(\FF_q)\backslash G(\FF_q)/B(\FF_q))$ and consider the pullback of $f$ by $\phi$, given by $\phi^\ast f := f\circ \phi$.
% This is now a function $G(\calO)\to\CC$.
% Then we have the following proposition:
% \begin{prop}
% 	The map
% 	\[
% 		\phi^\ast\colon C_c(B(\FF_q)\backslash G(\FF_q)/B(\FF_q))\to C_c(I\backslash G(\calO)/I), \quad \phi^\ast f := f\circ \phi
% 	\]
% 	is an algebra isomorphism.
% \end{prop}
% \begin{proof}[Proof (Sketch)]
% 	We must check that the image of $\phi^\ast$ lies in $C_c(I\backslash G(\calO)/I)$.
% 	To this end, let $i,i'\in I$ and $g\in G(\calO)$.
% 	Then
% 	\[
% 		(\phi^\ast f)(igi') = f(\phi(igi')) = f(\phi(i)\phi(g)\phi(i')) = f(\phi(g))=(\phi^\ast f)(g),
% 	\]
% 	since $\phi(i)\in B(\FF_q)$ by the definition of $I$, and $f$ is invariant on $B(\FF_q)$-double cosets.
% 	It is an exercise to show that $\phi^\ast f$ has compact support.

% 	Next, the map $\phi^\ast$ is clearly linear.
% 	Furthermore, it is easy to see that $\ker \phi^\ast$ is trivial since $f\circ \phi=0$ is only satisfied by the constant map $f\equiv 0$.
% 	The spaces $C_c(B(\FF_q)\backslash G(\FF_q)/B(\FF_q))$ and $C_c(I\backslash G(\calO)/I)$ have the same dimension because they both have a basis parameterised by the Weyl group.
% 	Then $\phi^\ast$ is an isomorphism of vector spaces.

% 	Lastly, we must check that $\phi^\ast$ preserves the algebra multiplication.
% 	To see this, take two maps $f,f'\in C_c(B(\FF_q)\backslash G(\FF_q)/B(\FF_q))$ and an element $x\in G(\calO)$.
% 	Then
% 	\begin{multline*}
% 		(\phi^\ast f \star \phi^\ast f')(x) = \int_G f(\phi(xg))f'(\phi(g^{-1}))\ d\mu(g) = \sum_{g\in G(\FF_q)} f(\phi(xg))f'(\phi(g^{-1})) \\
% 		= \sum_{yz=\phi(x)} f(y)f'(z) = (f\star f')(\phi(x)) = \phi^\ast(f\star f')(x). \qedhere
% 	\end{multline*}
% \end{proof}

% %%%%%%%%%%%%%%%%%%%%%%%%%%%%%%%%%%%%%%%%%%%%%%%%%%%%%%%%%%%%%%%%%%%%%%%%%%%%%%%%%%%%%%%%%%%%%%

% \subsection{The Iwahori--Matsumoto presentation}
% We conclude this thesis with a presentation of the Iwahori--Hecke algebra called the \emph{Iwahori--Matsumoto presentation} \cite{HP02, Solleveld21, Prasad05, Bump10}.

% Recall that $G(k)$ is a ``nice'' reductive group, e.g.\ $G(k)=\SL_n(k)$ where $n\geq 2$.
% Fix a maximal split torus $A$ in $G$.
% If $G(k)=\SL_n(k)$, then we may choose $A$ to be the subgroup of diagonal matrices.
% Let $X^\ast(A):=\Hom_\alg(A,\GG_m)$ denote the \emph{rational characters} of $A$, and let $X_\ast(A):= \Hom_\alg(\GG_m,A)$ denote the \emph{rational cocharacters} of $A$, where $\Hom_\alg$ denotes the space of \emph{algebraic homomorphisms}.
% Here $\GG_m$ is the \emph{one-dimensional torus} over $k$.
% Specifically, $\GG_m$ is isomorphic to the group scheme $\Spec(k[t,t^{-1}])$.
% If one is not familiar with these notions, one can think of $\GG_m$ as the multiplicative group $k^\times$.

% Given a rational character $\lambda\in X^\ast(A)$ and a rational cocharacter $\mu\in X_\ast(A)$, their composition is an endomorphism of $\GG_m$ of the form $x\mapsto x^k$.
% This yields a bilinear map
% \[
% 	\langle\cdot,\cdot\rangle\colon X^\ast(A)\times X_\ast(A)\to\ZZ
% \]
% defined by $\langle\lambda,\mu\rangle:=k$, where $k$ is given above.

% The \emph{affine Weyl group} of $G$ is the quotient group $\widetilde{W} = N_{G(k)}(A)/A(\calO)$.
% One may show that $\widetilde{W}$ may be written as $\widetilde{W} = \frac{A}{A(\calO)}\rtimes W$, where $W$ is the Weyl group of $G$ \cite{Humphreys90}.
% Recall from Section \ref{Section3.1} that if $G=\SL_{r+1}$ then $W\cong S_{r+1}$.
% The Weyl group is a Coxeter group with simple reflections $S=\{s_i = (i\ i+1)\ |\ i=1,\ldots, r\}$.
% We can consider these simple reflections as elements of $\SL_{r+1}$ by writing
% \[
% 	s_i = \begin{pmatrix}
% 		I_{i-1} &    &   &         \\
% 		        & 0  & 1 &         \\
% 		        & -1 & 0 &         \\
% 		        &    &   & I_{r-i}
% 	\end{pmatrix}.
% \]
% Similarly, the affine Weyl group has the generators $\widetilde{S} := \{s_0,s_1,\ldots, s_r\}$, where $s_1,\ldots, s_r$ are as before, and $s_0$ is given by
% \[
% 	s_0 = \begin{pmatrix}
% 		                 &         & \varpi^{-1} \\
% 		                 & I_{r-1} &             \\
% 		(-1)^{r+1}\varpi &         &
% 	\end{pmatrix}.
% \]
% This generator is understood as an element of $N_{G(k)}(A)/A(\calO)$.
% Then we have
% \[
% 	\widetilde{W} = \left\langle s_0,\ldots, s_r\ \Bigg|\ \begin{array}{c}
% 		s_i^2 = 1,                         \\
% 		s_is_{i+1}s_i = s_{i+1}s_is_{i+1}, \\
% 		s_is_j=s_js_i\end{array}
% 	\right\rangle.
% \]
% We can define the \emph{length function} on the Weyl group $W$ with associated generators $S$.
% This is the map $l\colon W\to\ZZ_{\geq 0}$ given by $l(w)=k$, where $k$ is the smallest integer such that $w=s_1s_2\cdots s_k$, where $s_i\in S$.
% It is seen in \cite{Bump10} that this map is well-defined and unique.
% The definition of $l$ clearly extends to the affine Weyl group $\widetilde{W}$ and its generators $\widetilde{S}$.

% Now we may present the Iwahori--Matsumoto presentation of the Iwahori--Hecke algebra.
% The Iwahori--Hecke algebra has the basis $\{T_w\ |\ w\in\widetilde{W}\}$ subject to the multiplication relations
% \begin{align*}
% 	T_s^2 = (q-1)T_s + q,\quad & \text{if}\ s\in\widetilde{S},     \\
% 	T_wT_{w'} = T_{ww'},\quad  & \text{if}\ l(ww') = l(w) + l(w').
% \end{align*}

%%%%%%%%%%%%%%%%%%%%%%%%%%%%%%%%%%%%%%%%%%%%%%%%%%%%%%%%%%%%%%%%%%%%%%%%%%%%%%%%%%%%%%%%%%%%%%

\newpage
\linespread{1.58}
\begin{bibdiv}
    \begin{biblist}[\normalsize]*{labels={alphabetic}}

        % Gaussian processes for machine learning / Carl Edward Rasmussen, Christopher K.I. Williams
        \bib{RasmussenCarlEdward2006Gpfm}{book}{
            series = {Adaptive computation and machine learning},
            publisher = {MIT Press},
            booktitle = {Gaussian processes for machine learning},
            isbn = {026218253X},
            year = {2006},
            title = {Gaussian processes for machine learning / Carl Edward Rasmussen, Christopher K.I. Williams.},
            language = {eng},
            address = {Cambridge, Mass.},
            author = {Rasmussen, Carl Edward and Williams, Christopher K. I}
        }

        %  Machine learning : a probabilistic perspective / Kevin P. Murphy.
        \bib{MurphyKevinP2012Ml}{book}{
            series = {Adaptive computation and machine learning},
            publisher = {MIT Press},
            booktitle = {Machine learning : a probabilistic perspective},
            isbn = {9780262018029},
            year = {2012},
            title = {Machine learning : a probabilistic perspective / Kevin P. Murphy.},
            language = {eng},
            address = {Cambridge, MA},
            author = {Murphy, Kevin P.}
        }

        %  Functional analysis. Volume 1 : Berezansky, Z.G. Sheftel, G.F. Us
        \bib{BerezanskyMakarovich1996FaV1}{book}{
            series = {Operator Theory: Advances and Applications, 85},
            publisher = {Basel ; Boston ; Berlin : BirkhaIuser Verlag},
            booktitle = {Functional analysis. Volume 1},
            isbn = {3-7643-5344-9},
            year = {1996},
            title = {Functional analysis. Volume 1 / Y.M. Berezansky, Z.G. Sheftel, G.F. Us ; translated from the Russian by Peter V. Malyshev.},
            edition = {1st ed. 1996.},
            language = {eng},
            address = {Basel ; Boston ; Berlin},
            author = {Berezansky, Z.G. Sheftel, G.F}
        }

        %  Numerical linear algebra / Lloyd N. Trefethen, David Bau
        % https://search.library.uq.edu.au/primo-explore/fulldisplay?docid=61UQ_ALMA21109158490003131&context=L&vid=61UQ&lang=en_US&search_scope=61UQ_All&adaptor=Local%20Search%20Engine&tab=61uq_all&query=any,contains,Numerical%20Linear%20Algebra%20Lloyd%20N.%20Trefethen%20and%20David%20Bau&sortby=rank&facet=rtype,exclude,newspaper_articles,lk&facet=rtype,exclude,reviews,lk&offset=0
        \bib{TrefethenLloydN.LloydNicholas1997Nla/}{book}{
            publisher = {SIAM Society for Industrial and Applied Mathematics},
            booktitle = {Numerical linear algebra},
            isbn = {0898713617},
            year = {1997},
            title = {Numerical linear algebra / Lloyd N. Trefethen, David Bau.},
            language = {eng},
            address = {Philadelphia},
            author = {Trefethen, Lloyd N. (Lloyd Nicholas) and Bau, David},
        }

        %   Applied numerical linear algebra / James W. Demmel.
        % https://search.library.uq.edu.au/primo-explore/fulldisplay?docid=61UQ_ALMA2180009300003131&context=L&vid=61UQ&lang=en_US&search_scope=61UQ_All&adaptor=Local%20Search%20Engine&tab=61uq_all&query=any,contains,Applied%20Numerical%20Linear%20Algebra&sortby=rank&facet=rtype,exclude,newspaper_articles,lk&facet=rtype,exclude,reviews,lk&offset=0
        \bib{DemmelJamesW1997Anla}{book}{
            publisher = {Society for Industrial and Applied Mathematics},
            booktitle = {Applied numerical linear algebra},
            isbn = {0898713897},
            year = {1997},
            title = {Applied numerical linear algebra / James W. Demmel.},
            language = {eng},
            address = {Philadelphia, Pa.},
            author = {Demmel, James W},
        }

        %%%%%%%%%%%%%%%%%%%%%%%%%%%%%%%%%%%%%%%%%%%%%%%%%%%%%%%%%%%%%%%%%%%%%%%%%%%%%%%%%%%%%%%%%%%%%%


        \bib{AGS08}{incollection}{
            author = {Aizenbud, A.},
            author = {Gourevitch, D.},
            author = {Sayag, E.},
            title = {$(\GL_{n+1}(F),\GL_n(F))$ is a Gelfand pair for any local field $F$},
            publisher = {Cambridge University Press},
            volume ={144},
            year = {2008},
            bookTitle = {Compositio Mathematica, Volume 144, Issue 6},
            pages = {1504--1524}
        }

        \bib{Brocker85}{book}{
            author = {Brocker, T.},
            author = {Dieck, T.},
            year = {1985},
            publisher = {Springer-Verlag, Berlin},
            title = {Representations of Compact Lie Groups}
        }

        \bib{Bannai84}{book}{
            author = {Bannai, E.},
            author = {Ito, T.},
            title = {Algebraic Combinatorics I : Association Schemes},
            publisher = {Benjamin Cummings Publishing Company, San Francisco},
            year = {1984}
        }

        \bib{Bump10}{webpage}{
            author = {Bump, D.},
            title = {Hecke Algebras},
            year = {2010},
            url = {http://sporadic.stanford.edu/bump/math263/hecke.pdf},
            accessdate = {12/03/2021}
        }

        \bib{Bump13}{book}{
            author = {Bump, D.},
            year = {2013},
            title = {Lie Groups},
            edition = {Second Edition},
            publisher = {Springer-Verlag, New York}
        }

        \bib{Carter85}{book}{
            author = {Carter, R. W.},
            title = {Finite Groups of Lie Type},
            publisher = {John Wiley \& Sons Ltd},
            year = {1985}
        }

        \bib{CSST20}{book}{
            title={Gelfand Triples and Their Hecke Algebras: Harmonic Analysis for Multiplicity-Free Induced Representations of Finite Groups},
            author={Ceccherini-Silberstein, T.},
            author={Scarabotti, F.},
            author={Tolli, F.},
            series={Lecture Notes in Mathematics},
            year={2020},
            publisher={Springer International Publishing}
        }

        \bib{CMHL03}{book}{
            author = {Cherednik, I.},
            author = {Markov, Y.},
            author = {Howe, R.},
            author = {Lusztig, G.},
            year = {2003},
            title = {Iwahori-Hecke Algebras and Their Representation Theory},
            publisher = {Springer, Berlin, Heidelberg}
        }

        \bib{Coxeter35}{article}{
            author = {Coxeter, H. S. M.},
            year = {1935},
            title = {The Complete Enumeration of Finite Groups of the Form {$R_i^2=(R_iR_j)^{k_{ij}}=1$}},
            publisher = {Journal of the London Mathematical Society},
            volume = {s1-10},
            pages = {21--25}
        }

        \bib{Curtis87v1}{book}{
            author = {Curtis, C. W.},
            author = {Reiner, I.},
            year = {1987},
            title = {Methods of Representation Theory - with applications to finite groups and orders},
            volume = {1},
            publisher = {John Wiley \& Sons}
        }

        \bib{Curtis87v2}{book}{
            author = {Curtis, C. W.},
            author = {Reiner, I.},
            year = {1987},
            title = {Methods of Representation Theory - with applications to finite groups and orders},
            volume = {2},
            publisher = {John Wiley \& Sons}
        }

        \bib{Diaconis88}{book}{
            title = {Group representations in probability and statistics},
            author = {Diaconis, P.},
            organization = {Institute of Mathematical Statistics, Hayward, California},
            bookTitle = {Lecture Notes--Monograph Series, vol. 11},
            year = {1988}
        }

        \bib{Etingof11}{book}{
            title={Introduction to Representation Theory},
            author={Etingof, P. I.},
            author = {Golberg, O.},
            author = {Hensel, S.},
            author = {Liu, T.},
            author = {Schwendner, A.},
            author ={Vaintrob, D.},
            author = {Yudovina, E.},
            year={2011},
            publisher={American Mathematical Society}
        }

        \bib{Folland84}{book}{
            author = {Folland, G. B.},
            title = {Real Analysis},
            year = {1984},
            publisher = {John Wiley \& Sons Inc.}
        }

        \bib{Gross91}{article}{
            author = {Gross, B. H.},
            journal = {Bulletin (New Series) of the American Mathematical Society},
            pages = {277--301},
            publisher = {American Mathematical Society},
            title = {Some applications of Gelfand pairs to number theory},
            volume = {24},
            year = {1991}
        }

        \bib{Hendel20}{webpage}{
            author = {Hendel, Y. I.},
            title = {On Twisted Gelfand Pairs Through Commutativity of a Hecke Algebra},
            year = {2020},
            url = {https://arxiv.org/pdf/1807.02843.pdf},
            accessdate = {20/01/21}
        }

        \bib{HKP}{webpage}{
            title = {Iwahori--Hecke Algebras},
            author = {Haines, T. J.},
            author = {Kottwitz, R. E.},
            author = {Prasad, A.},
            year = {2009},
            url = {https://arxiv.org/abs/math/0309168},
            accessdate = {29/04/2021}
        }

        \bib{HP02}{article}{
            title = {Formulae relating the Bernstein and Iwahori--Matsumoto presentations of an affine Hecke algebra},
            author = {Haines, T. J.},
            author = {Pettet, A.},
            year = {2002},
            journal = {Journal of Algebra},
            pages = {127--149},
            volume = {252}
        }

        \bib{Humphreys90}{book}{
            title = {Reflection Groups and Coxeter Groups},
            author = {Humphreys, J. E.},
            publisher = {Cambridge University Press, New York},
            year = {1990}
        }

        \bib{Koranyi80}{incollection}{
            pages = {334--348},
            booktitle = {Harmonic Analysis and Group Representation},
            title = {Some Applications of Gelfand Pairs in Classical Analysis},
            year = {1980},
            author = {Koranyi, A.}
        }

        \bib{Lam06}{incollection}{
            author = {Lam, T. Y.},
            year = {2003},
            booktitle = {Algebras, Rings and Their Representations: Proceedings of the International Conference on Algebras, Modules and Rings},
            title = {Corner Ring Theory: A Generalization of Peirce Decompositions, I},
            journal = {Algebras, Rings and Their Representations},
            pages = {153--182}
        }

        \bib{Lang02}{book}{
            title = {Algebra},
            author = {Lang, S.},
            year = {2002},
            edition = {revised 3rd edition},
            publisher = {Springer-Verlag, New York}
        }

        \bib{Macdonald95}{book}{
            author = {Macdonald, I. G.},
            title = {Symmetric Functions and Hall Polynomials},
            year = {1995},
            edition = {2},
            publisher = {Oxford University Press Inc., New York}
        }

        \bib{MilneANT}{webpage}{
            author={Milne, J. S.},
            title={Algebraic Number Theory (v3.08)},
            year={2020},
            url={www.jmilne.org/},
            accessdate = {15/04/2021}
        }


        \bib{Morel18}{webpage}{
            author = {Morel, S.},
            year = {2018},
            url = {https://web.math.princeton.edu/~smorel/449/notes_449.pdf},
            title = {MAT 449 : Representation theory},
            accessdate = {14/02/2021}
        }

        \bib{Solleveld21}{webpage}{
            url = {https://arxiv.org/pdf/2009.03007.pdf},
            author = {Maarten Solleveld},
            title = {Affine Hecke Algebras and their Representations},
            year = {2021},
            institution = {Institute for Mathematics, Astrophysics and Particle Physics, Radboud University},
            accessdate = {31/05/21}
        }

        \bib{Murnaghan05}{webpage}{
            author = {Murnaghan, F.},
            title = {MAT445/1196F - Representation Theory Course Notes},
            year = {2005},
            accessdate = {15/02/2021},
            url = {http://www.math.toronto.edu/murnaghan/courses/mat445/notes.pdf}
        }

        \bib{Musili93}{incollection}{
            author = {Musili, C.},
            title = {Representations of Finite Groups},
            bookTitle = {Texts and Readings in Mathematics, vol. 8},
            year = {1993},
            pages = {115--140},
            notes = {Hindustan Book Agency, India}
        }

        \bib{Prasad05}{webpage}{
            author = {Prasad, A.},
            title = {On Bernstein's Presentation of Iwahori--Hecke Algebras and Representations of Split Reductive Groups over Non-Archimedian Local Fields},
            year = {2005},
            url = {https://arxiv.org/pdf/math/0504417.pdf},
            accessdate = {27/05/21}
        }

        \bib{Riggs96}{thesis}{
            author = {Riggs, L. J.},
            year = {1996},
            title = {Polynomial equations and solvability: A historical perspective},
            institution = {California State University, San Bernardino}
        }

        \bib{Williamson21}{webpage}{
            author = {Romanov, A.},
            author = {Williamson, G.},
            title = {Langlands correspondence and Bezrukavnikov’s equivalence},
            year = {2021},
            institution = {University of Sydney},
            url = {https://arxiv.org/abs/2103.02329},
            accessdate = {2/05/2021}
        }

        \bib{Terras99}{book}{
            author = {Terras, A.},
            title = {Fourier  Analysis  on  Finite  Groups  and  Applications},
            year = {1999},
            publisher = {Cambridge University Press},
            organization = {London  Mathematical Society}
        }



    \end{biblist}
\end{bibdiv}

\end{document}