\subsection{Gram-Schmidt Process and QR factorisations}\label{Section1.2}

Many areas of linear algebra involving studing the column space of matrices. The $QR$ factorisation provides us with a powerful tool to better understand the column space of a matrix as well as serving as an important factorisation mechanism for many numerical methods. Suppose that a matrix $\bm{A} = \left[ \bm{a}_1 , \bm{a}_2 , \ldots , \bm{a}_n \right] \in \KK^{n \times n}$ has full rank. The idea of a $QR$ factorisation is to find an alternative orthornormal basis for $\left( \bm{a}_i \right)_{i=1}^{n}$, say $\left( \bm{q}_i \right)_{i=1}^{n}$, and to somehow relate the original matrix $\bm{A}$ to a new matrix whose columns are $\left( \bm{q}_i \right)_{i=1}^{n}$. Consider the following procedure that allows us to find an orthornormal basis $\left( \bm{q}_i \right)_{i=1}^{n}$ for which $\operatorname{l.s} \left\{ \left( \bm{a}_i \right)_{i=1}^{n} \right\} = \operatorname{l.s} \left\{ \left( \bm{q}_i \right)_{i=1}^{n} \right\}$. First set $\bm{q}_1 = \frac{\bm{a}_1}{\| \bm{a}_i \|}$, clearly $\operatorname{l.s} \left\{ \bm{a}_1 \right\} = \operatorname{l.s} \left\{ \bm{q}_1 \right\}$. Next, construct a vector $\bm{q}_2' = \bm{a}_2 - r_{1,2} \cdot \bm{q}_1$ so that $\bm{q}_2' \perp \bm{q}_1$. This means
\begin{align*}
    0       & = \langle \bm{q}_2', \bm{q}_1 \rangle                                                   \\
    0       & = \langle \bm{a}_2 - r_{1,2} \cdot \bm{q}_1, \bm{q}_1 \rangle                           \\
    0       & = \langle \bm{a}_2, \bm{q}_1 \rangle - r_{1,2} \cdot \langle \bm{q}_1, \bm{q}_1 \rangle \\
    r_{1,2} & = \langle \bm{a}_2, \bm{q}_1 \rangle
\end{align*}
Since $\bm{q}_2'$ may not be a unit vector we set $\bm{q}_2 = \frac{\bm{q}_2'}{\| \bm{q}_2' \|}$ where $\operatorname{l.s} \left\{ \bm{a}_1, \bm{a}_2 \right\} = \operatorname{l.s} \left\{ \bm{q}_1, \bm{q}_2 \right\}$. Continuing the vector $\bm{q}_3'$ is constructed so that
\[
    \bm{q}_3' = \bm{a}_3 - \bm{r}_{1,3} \bm{q}_1 - \bm{r}_{2,3} \bm{q}_2
\]
are chosen so that $\bm{q}_3'$ is orthogonal to both $\bm{q}_2$ and $\bm{q}_1$. This amounts to setting $r_{1,3} = \langle \bm{a}_3, \bm{q}_1 \rangle$ and $r_{2,3} = \langle \bm{a}_{3}, \bm{q}_2 \langle$. Similarly, $\bm{q}_3'$ is normalized so that $\bm{q}_3 = \frac{\bm{q}_3'}{\| \bm{q}_3' \|}$ and $\operatorname{l.s} \left\{ \bm{a}_1, \bm{a}_2, \bm{a}_3 \right\} = \operatorname{l.s} \left\{ \bm{q}_1, \bm{q}_2, \bm{q}_3 \right\}$. Continuing in this fashion the $k^{th}$ vector in our orthornormal basis is computed as
\[
    \bm{q}_k = \frac{\bm{a}_k - \sum_{i=1}^{k-1} r_{i,k} \cdot \bm{q}_i}{r_{k,k}}
\]
where $r_{i,k} = \langle \bm{a}_k , \bm{q}_i \rangle$, $r_{k,k} = \| \bm{a}_k - \sum_{i=1}^{k-1} r_{i,k} \cdot \bm{q}_i \|$ and $\operatorname{l.s} \left( \left\{ \bm{a}_1, \bm{a}_2, \ldots , \bm{a}_k \right\} \right) = \operatorname{l.s} \left( \left\{ \bm{q}_1, \bm{q}_2, \ldots , \bm{q}_k \right\} \right)$. This procedure is famiously known as the Gram-Schmidt process and is summarized in the following algorithm.

    {\centering
        \begin{minipage}{.85\linewidth}
            \begin{algorithm}[H]
                \caption{relax}
                \label{alg:relax}
                \SetAlgoLined
                \DontPrintSemicolon
                \SetKwInOut{Input}{input}\SetKwInOut{Output}{output}

                \Input{A basis $\left( \bm{a}_i \right)_{i=1}^{n}$.}
                \Output{An orthornormal basis $\left( \bm{q}_i \right)_{i=1}^{n}$ such that $\operatorname{l.s} \left\{ \left( \bm{a}_i \right)_{i=1}^{n} \right\} = \operatorname{l.s} \left\{ \left( \bm{q}_i \right)_{i=1}^{n} \right\}$}
                \BlankLine
                \For{$k = 1$ \KwTo $n$}{
                    $\bm{q}_k' = \bm{a}_k$\;
                    \For{$i = 1$ \KwTo $k-1$}{
                        $r_{i,k} = \langle \bm{a}_k, \bm{q}_i \rangle$\;
                        $\bm{q}_k' = \bm{q}_k' - r_{i,k} \bm{q}_i$\;
                    }
                    $r_{k,k} = \| \bm{q}_k' \|$\;
                    $\bm{q}_k = \bm{q}_k' / r_{k,k}$\;
                }
                \Return{$\left( \bm{q}_i \right)_{i=1}^{n}$}
                \BlankLine
            \end{algorithm}
        \end{minipage}
        \par
    }