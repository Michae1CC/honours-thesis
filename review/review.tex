\section{A Review of Gaussian Processes and Related Topics}\label{Chapter1}
The aim of this chapter is to review some essential mathematical machinery required for us to understand the core concepts of Gaussian Processes.

\subsection{Krylov Subspace Methods}\label{Section1.1}

In this section we will focus on how iterative methods, in particular a class of iterative methods called Krylov Subspace methods, may be used to solve a linear system $\bm{A} \bm{x} = \bm{b}$. While non-iterative methods exist to solve such systems virtually all of them carry an unwieldy runtime of $\mathcal{O} \left( n^3 \right)$ for a system of $n$ parameters. Even for current computer systems, this renders many common matrix problems untractable. Consequently the focus of solving linear systems has shifted towards iterative methods. While iterative methods typically demand certain structural properties of the matrices, such as symmetry and positive definiteness, this generally is not a problem since the majority of large matrix problems that, by mature, endow these systems with the desired properties. For example, in the context of this paper the Gram matrices used to solve linear systems in Gaussian Processes possess both symmetry and positive definiteness. There are also a number of other properties of iterative methods which make them rather attractive to users. To start, iterative Krylov subspace methods are guranteed to converge to an exact solution within a finite number of iterations and even if the method is prematurely stopped before reaching an exact solution, the approximation obtained on the final iteration will in some sense be a good enough estimate of our exact solution. Furthermore, unlike most non-iterative methods, Krylov subspace methods do not require an explicit form of the matrix $\bm{A}$ and instead only requires some routine or process for computing $\bm{A} \bm{x}$.

\subsubsection{Krylov Subspaces}\label{Section1.1.1}

We will motivate the Krylov subspaces by observing their usefullness in solving linear systems. To this end, consider the problem of solving the linear system
\begin{equation}\label{eq: lin_sys_1}
    \bm{A} \bm{x^{\star}} = \bm{b}
\end{equation}
where no explicit form of $\bm{A}$ is available and instead one must draw information from $\bm{A}$ solely through a routine that can evaluate $\bm{A} \bm{v}$ for any $\bm{v}$. How could this routine be utilized in such a manner to provide with a solution to equation \ref{eq: lin_sys_1}? Before answering this, consider the following theorem

\begin{thm} \label{theorem: invert_mat_norm}
    For $\bm{A} \in \KK^{n \times n}$ if $\| \bm{A} \| = q < 1$ then $\Id - \bm{A}$ is invertible and its inverse admits the following representation
    \[
        \left( \Id - \bm{A} \right)^{-1} = \sum_{k=0}^{\infty} \bm{A}^k.
    \]
    \cite{BerezanskyMakarovich1996FaV1}
\end{thm}
Consider a matrix for which $\| \bm{A} \| < 2$, it follows that $\| \Id - \bm{A} \| < 1$ meaning $\Id - \left( \Id - \bm{A} \right)$ is invertible and $\bm{A}^{-1} = \left( \Id - \left( \Id - \bm{A} \right) \right)^{-1} = \sum_{k=0}^{\infty} \left( \Id - \bm{A} \right)^{k}$. Thinking back to equation \ref{eq: lin_sys_1} for any $x_0 \in \KK^{n}$ we have
\begin{align*}
    \bm{x^{\star}} & = \bm{A}^{-1} \bm{b} = \bm{A}^{-1} \left( \bm{A} \bm{x^{\star}} - \bm{A} \bm{x_0} + \bm{A} \bm{x_0} \right) \\
                   & = \bm{x_0} + \bm{A}^{-1} \bm{r_0}                                                                           \\
                   & = \bm{x_0} + \sum_{k=0}^{\infty} \left( \Id - \bm{A} \right)^k
\end{align*}
where $\bm{r_0} = \bm{A} \bm{x^{\star}} - \bm{A} \bm{x_0}$. A natural question that arises is that can we find a closed form solution of the above equation? To answer this question we need to enlist the help of the Cayley-Hamilton theorem.
\begin{thm}[Cayley-Hamilton] \label{theorem: cayley_amilton}
    Let $p_n \left( \lambda \right) = \sum_{i=0}^{n} c_i \lambda^{i}$ be the characteristic polynomial of the matrix $\bm{A} \in \KK^{n \times n}$, then $p_n \left( \bm{A} \right) = \bm{0}$. {\color{red} \textbf{THIS NEEDS A CITATION}}
\end{thm}
The Cayley-Hamilton theorem implies that
\begin{align*}
    0           & = c_0 + c_1 \bm{A} + \ldots + c_{n-1} \bm{A}^{n-1} + c_{n} \bm{A}^{n}           \\
    0           & = \bm{A}^{-1} c_0 + c_1 + \ldots + c_{n-1} \bm{A}^{n-2} + c_{n} \bm{A}^{n-1}    \\
    \bm{A}^{-1} & = \alpha_0 + c_1 + \ldots + \alpha_{n-1} \bm{A}^{n-2} + \alpha_{n} \bm{A}^{n-1}
\end{align*}
where $\alpha_i = -c_i / c_0$. This demonstrates that $\bm{A}^{-1}$ can be represented as a matrix polynomial of degree $n-1$. This means that $\sum_{k=0}^{\infty} \left( \Id - \bm{A} \right)^k$ indeed possess a closed form solution namely
\[
    \bm{x^{\star}} = \bm{x_0} + \bm{A}^{-1} \bm{r_0} = \alpha_0 + c_1 + \ldots + \alpha_{n-1} \bm{A}^{n-2} + \alpha_{n} \bm{A}^{n-1}.
\]
This also shows that $\bm{x^{\star}} \in \operatorname{l.s} \left\{ \bm{r_0}, \bm{A} \bm{r_0}, \bm{A}^2 \bm{r_0}, \ldots , \bm{A}^{n-1} \bm{r_0} \right\}$. One idea for finding a solution to equation \ref{eq: lin_sys_1} is to use our routine for evaluting $\bm{A} \bm{v}$ to iteratively compute new basis elements for the space generated by $\left\{ \bm{r_0}, \bm{A} \bm{r_0}, \bm{A}^2 \bm{r_0}, \ldots , \bm{A}^{n-1} \bm{r_0} \right\}$ and at each step carefully choosing a $\bm{x_k}$ such that $\bm{x_k}$ approaches $\bm{x^{\star}}$, in some form. The subspace constructed using this technique is so important that is has its own name.
\begin{defe}[Krylov Subspace] \label{defe: krylov_subspace}
    The Krylov Subspace of order $k$ generated by the matrix $\bm{A} \in \KK^{n \times n}$ and the vector $\bm{v} \in \KK$ is defined as
    \[
        \calK_{k} \left( \bm{A},\bm{v} \right) = \operatorname{l.s} \left\{ \bm{r_0}, \bm{A} \bm{r_0}, \bm{A}^2 \bm{r_0}, \ldots , \bm{A}^{n-1} \bm{r_0} \right\}
    \]
    for $k \geq 1$ and $\calK_{k} \left( \bm{A},\bm{v} \right) = \left\{ \bm{0} \right\}$.
\end{defe}
For the purposes of solving equation \ref{eq: lin_sys_1} it is of much interest to understand how $\calK_{k} \left( \bm{A},\bm{v} \right)$ grows for larger and larger $k$ since a solution for equation \ref{eq: lin_sys_1} will be present in a Krylov Subspace that cannot be grown any larger. In other words, an exact solution can be constructed once we have extracted all the information from $\bm{A}$ through multiplication of $\bm{r_0}$. The following theorem provides information on how exactly the Krylov Subspace grows as $k$ increases.
\begin{thm} \label{theorem: grade_of_v}
    There is a positive called the grade of $\bm{v}$ with respect to $\bm{A}$, denoted $t_{\bm{v}, \bm{A}}$, where
    \[
        \operatorname{dim} \left( \calK_{k} \left( \bm{A} , \bm{v} \right) \right) = \left\{
        \begin{matrix}
            k, & k \leq t \\
            t, & k \geq t
        \end{matrix}
        \right.
    \]
\end{thm}
Theorem \ref{theorem: grade_of_v} essentially tells us that for $k \leq t_{\bm{v}, \bm{A}}$ that $\bm{A}^k \bm{v}$ is linearly independent to $\bm{A}^i \bm{v}$ for $0 \leq i \leq k-1$ meaning $\left\{ \bm{v}, \bm{A} \bm{v}, \bm{A}^2 \bm{v}, \ldots , \bm{A}^{n-1} \bm{v} \right\}$ serves as a basis for $\calK_{k} \left( \bm{A},\bm{v} \right)$ and that $\calK_{k-1} \left( \bm{A},\bm{v} \right) \subsetneq \calK_{k} \left( \bm{A},\bm{v} \right)$. Conversely, any new vectors formed beyond $t_{\bm{v}, \bm{A}}$ will be linearly independent meaning $\calK_{k} \left( \bm{A},\bm{v} \right) \subsetneq \calK_{k+1} \left( \bm{A},\bm{v} \right)$ for $k \geq t_{\bm{v}, \bm{A}}$. While $t_{\bm{v}, \bm{A}}$ clearly plays a role in determining a suitable basis for which $\bm{A}^{-1} \bm{b}$ lies in its importance is made abundantly clear in the following corollary.
\begin{cor} \label{theorem: grade_as_min}
    \[
        t_{\bm{v}, \bm{A}} = \min \left\{k \mid \bm{A}^{-1} \bm{v} \in \calK_{k} \left( \bm{A},\bm{v} \right) \right\}
    \]
\end{cor}
\begin{proof}
    Recall from Cayley-Hamilton (theorem \ref{theorem: cayley_amilton}) that
    \[
        \bm{A}^{-1} \bm{v} = \sum_{i=0}^{n-1} \alpha_{i} \bm{A}^{i} \bm{v}
    \]
    But since $\calK_{k} \left( \bm{A},\bm{v} \right) = \calK_{k+1} \left( \bm{A},\bm{v} \right)$ for $k \geq t_{\bm{v}, \bm{A}}$
    \[
        \bm{A}^{-1} \bm{v} = \sum_{i=0}^{t-1} \beta_{i} \bm{A}^{i} \bm{v}
    \]
    meaing $\bm{A}^{-1} \bm{v} \in \calK_{k} \left( \bm{A},\bm{v} \right)$ for $k \geq t_{\bm{v}, \bm{A}}$. Suppose for the sake of contradiction that this also holds for $k = t_{\bm{v}, \bm{A}} - 1$, that is, $\bm{A}^{-1} \bm{v} = \sum_{i=0}^{t-2} \gamma_{i} \bm{A}^{i} \bm{v}$. However, this gives
    \[
        \bm{v} = \sum_{i=0}^{t-2} \gamma_{i} \bm{A}^{i+1} \bm{v} = \sum_{i=0}^{t-1} \gamma_{i-1} \bm{A}^{i} \bm{v}
    \]
    implying $\left\{ \bm{v}, \bm{A} \bm{v}, \bm{A}^2 \bm{v}, \ldots , \bm{A}^{t-1} \bm{v} \right\}$ are linearly dependent which means that $\operatorname{dim} \left( \calK_{k} \left( \bm{A} , \bm{v} \right) \right) < t$, which provides us with our contrdiction.
\end{proof}
This machinery allows us to make a much stronger statement on the where abouts of $\bm{x^{\star}}$ in relation to the Krylov Subspaces.
\begin{cor} \label{theorem: sol_in_krylov}
    For any $\bm{x_0}$, we have
    \[
        \bm{x^{\star}} \in \bm{x_0} + \calK_{t_{\bm{r_0}, \bm{A}}} \left( \bm{A},\bm{r_0} \right)
    \]
    where $\bm{r_0} = \bm{b} - \bm{A} \bm{x_0}$.
\end{cor}

\subsubsection{Gram-Schmidt Process and QR factorisations}\label{Section1.1.2}

Many areas of linear algebra involving studing the column space of matrices. The $QR$ factorisation provides us with a powerful tool to better understand the column space of a matrix as well as serving as an important factorisation mechanism for many numerical methods. Suppose that a matrix $\bm{A} = \left[ \bm{a}_1 , \bm{a}_2 , \ldots , \bm{a}_n \right] \in \KK^{n \times n}$ has full rank. The idea of a $QR$ factorisation is to find an alternative orthornormal basis for $\left( \bm{a}_i \right)_{i=1}^{n}$, say $\left( \bm{q}_i \right)_{i=1}^{n}$, and to somehow relate the original matrix $\bm{A}$ to a new matrix whose columns are $\left( \bm{q}_i \right)_{i=1}^{n}$. Consider the following procedure that allows us to find an orthornormal basis $\left( \bm{q}_i \right)_{i=1}^{n}$ for which $\operatorname{l.s} \left\{ \left( \bm{a}_i \right)_{i=1}^{n} \right\} = \operatorname{l.s} \left\{ \left( \bm{q}_i \right)_{i=1}^{n} \right\}$. First set $\bm{q}_1 = \frac{\bm{a}_1}{\| \bm{a}_i \|}$, clearly $\operatorname{l.s} \left\{ \bm{a}_1 \right\} = \operatorname{l.s} \left\{ \bm{q}_1 \right\}$. Next, construct a vector $\bm{q}_2' = \bm{a}_2 - r_{1,2} \cdot \bm{q}_1$ so that $\bm{q}_2' \perp \bm{q}_1$. This means
\begin{align*}
    0       & = \langle \bm{q}_1, \bm{q}_2' \rangle                                                   \\
    0       & = \langle \bm{q}_1, \bm{a}_2 - r_{1,2} \cdot \bm{q}_1 \rangle                           \\
    0       & = \langle \bm{q}_1, \bm{a}_2 \rangle - r_{1,2} \cdot \langle \bm{q}_1, \bm{q}_1 \rangle \\
    r_{1,2} & = \langle \bm{q}_1, \bm{a}_2 \rangle
\end{align*}
Since $\bm{q}_2'$ may not be a unit vector we set $\bm{q}_2 = \frac{\bm{q}_2'}{\| \bm{q}_2' \|}$ where $\operatorname{l.s} \left( \left\{ \bm{a}_1, \bm{a}_2 \right\} \right) = \operatorname{l.s} \left( \left\{ \bm{q}_1, \bm{q}_2 \right\} \right)$. Continuing the vector $\bm{q}_3'$ is constructed so that
\[
    \bm{q}_3' = \bm{a}_3 - \bm{r}_{1,3} \bm{q}_1 - \bm{r}_{2,3} \bm{q}_2
\]
are chosen so that $\bm{q}_3'$ is orthogonal to both $\bm{q}_2$ and $\bm{q}_1$. This amounts to setting $r_{1,3} = \langle \bm{q}_1, \bm{a}_3 \rangle$ and $r_{2,3} = \langle \bm{q}_2, \bm{a}_{3} \rangle$. Similarly, $\bm{q}_3'$ is normalized so that $\bm{q}_3 = \frac{\bm{q}_3'}{\| \bm{q}_3' \|}$ and $\operatorname{l.s} \left( \left\{ \bm{a}_1, \bm{a}_2, \bm{a}_3 \right\} \right) = \operatorname{l.s} \left( \left\{ \bm{q}_1, \bm{q}_2, \bm{q}_3 \right\} \right)$. Continuing in this fashion the $k^{th}$ vector in our orthornormal basis is computed as
\begin{equation}\label{eq: comp_orth_basis}
    \bm{q}_k = \frac{\bm{a}_k - \sum_{i=1}^{k-1} r_{i,k} \cdot \bm{q}_i}{r_{k,k}}
\end{equation}
where $r_{i,k} = \langle \bm{q}_i, \bm{a}_k \rangle$, $r_{k,k} = \| \bm{a}_k - \sum_{i=1}^{k-1} r_{i,k} \cdot \bm{q}_i \|$ and $\operatorname{l.s} \left( \left\{ \bm{a}_1, \bm{a}_2, \ldots , \bm{a}_k \right\} \right) = \operatorname{l.s} \left( \left\{ \bm{q}_1, \bm{q}_2, \ldots , \bm{q}_k \right\} \right)$. This procedure is famiously known as the Gram-Schmidt process \cite{BerezanskyMakarovich1996FaV1,TrefethenLloydN.LloydNicholas1997Nla/,DemmelJamesW1997Anla} and is summarized in the following algorithm.

    % https://tex.stackexchange.com/questions/463359/algorithm-inside-a-tcolorbox-how-to-put-a-label-to-the-algorithm-but-the-captio
    % http://cfrgtkky.blogspot.com/2018/12/algorithm-inside-tcolorbox-how-to-put.html
    % {\centering
    %     \begin{minipage}{.85\linewidth}
    %         \begin{tcolorbox}[colback=white!100,colframe=black!100]
    %             \begin{algorithm}[H]
    %                 \caption{Classical Gram-Schmidt}
    % \label{alg: Classical_Gram-Schmidt}
    % \SetAlgoLined
    % \DontPrintSemicolon
    % \SetKwInOut{Input}{input}\SetKwInOut{Output}{output}

    % \Input{A basis $\left( \bm{a}_i \right)_{i=1}^{n}$.}
    % \Output{An orthornormal basis $\left( \bm{q}_i \right)_{i=1}^{n}$ such that $\operatorname{l.s} \left\{ \left( \bm{a}_i \right)_{i=1}^{n} \right\} = \operatorname{l.s} \left\{ \left( \bm{q}_i \right)_{i=1}^{n} \right\}$}
    % \BlankLine
    % \For{$k = 1$ \KwTo $n$}{
    %     $\bm{q}_k' = \bm{a}_k$\;
    %     \For{$i = 1$ \KwTo $k-1$}{
    %         $r_{i,k} = \langle \bm{q}_i, \bm{a}_k \rangle$\;
    %         $\bm{q}_k' = \bm{q}_k' - r_{i,k} \bm{q}_i$\;
    %     }
    %     $r_{k,k} = \| \bm{q}_k' \|$\;
    %     $\bm{q}_k = \bm{q}_k' / r_{k,k}$\;
    % }
    % \Return{$\left( \bm{q}_i \right)_{i=1}^{n}$}
    % \BlankLine
    %             \end{algorithm}
    %         \end{tcolorbox}
    %     \end{minipage}
    %     \par
    % }


    {\centering
        \begin{minipage}{.85\linewidth}
            \begin{algorithm}[H]
                \caption{Classical Gram-Schmidt}
                \label{alg: Classical_Gram-Schmidt}
                \SetAlgoLined
                \DontPrintSemicolon
                \SetKwInOut{Input}{input}\SetKwInOut{Output}{output}

                \Input{A basis $\left( \bm{a}_i \right)_{i=1}^{n}$.}
                \Output{An orthornormal basis $\left( \bm{q}_i \right)_{i=1}^{n}$ such that $\operatorname{l.s} \left\{ \left( \bm{a}_i \right)_{i=1}^{n} \right\} = \operatorname{l.s} \left\{ \left( \bm{q}_i \right)_{i=1}^{n} \right\}$}
                \BlankLine
                \For{$k = 1$ \KwTo $n$}{
                    $\bm{q}_k' = \bm{a}_k$\;
                    \For{$i = 1$ \KwTo $k-1$}{
                        $r_{i,k} = \langle \bm{q}_i, \bm{a}_k \rangle$\;
                        $\bm{q}_k' = \bm{q}_k' - r_{i,k} \bm{q}_i$\;
                    }
                    $r_{k,k} = \| \bm{q}_k' \|$\;
                    $\bm{q}_k = \bm{q}_k' / r_{k,k}$\;
                }
                \Return{$\left( \bm{q}_i \right)_{i=1}^{n}$}
                \BlankLine
            \end{algorithm}
        \end{minipage}
        \par
    }

Relating the column space of $\bm{A}$ to the orthornormal basis $\left( \bm{q}_{i} \right)_{i=1}^{n}$ in a matrix form
\[
    \left[ \bm{a}_1 , \bm{a}_2 , \ldots \bm{a}_n \right] =
    \left[ \bm{q}_1 , \bm{q}_2 , \ldots \bm{q}_n \right]
    \begin{bmatrix}
        r_{1,1} & r_{1,2} & \cdots & r_{1,n} \\
                & r_{2,2} &        & \vdots  \\
                &         & \ddots & \vdots  \\
                &         &        & r_{n,n}
    \end{bmatrix}
\]
or more succinctly
\begin{equation}\label{eq: QR_factorisation}
    \bm{A} = \bm{Q} \bm{R}
\end{equation}
where $\bm{Q} = \left[ \bm{q}_1 , \bm{q}_2 , \ldots \bm{q}_n \right]$ and $\left( \bm{R} \right)_{i,j} = r_{i,j}$ for $i \leq j$ and $\left( \bm{R} \right)_{i,j} = 0$ for $i > j$. This is exactly the $QR$ factorisation for a full rank matrix. Note that $\operatorname{Range} \left( \bm{A} \right) = \operatorname{Range} \left( \bm{Q} \right)$. In general, any square matrix  $\bm{A} \in \KK^{m \times n}$ may be decomposed as $\bm{A} = \bm{Q} \bm{R}$ where $\bm{Q} \in \KK^{m \times m}$ is an orthogonal matrix and $\bm{R} \in \KK^{m \times n}$ is an upper triangular matrix. This is known as a full $QR$ factorisation. Since bottom $(m-n)$ rows of this $\bm{R}$ consists entirely of zeros, it is often useful to partition the full $QR$ factorisation in the following manner to shed vacuous entries
\[
    \bm{A} = \bm{Q} \bm{R} = \bm{Q}
    \begin{bmatrix}
        \hat{\bm{R}} \\
        \bm{0}_{(m-n) \times n}
    \end{bmatrix}
    =
    \begin{bmatrix}
        \hat{\bm{Q}} & \bm{Q}'
    \end{bmatrix}
    \begin{bmatrix}
        \hat{\bm{R}} \\
        \bm{0}_{(m-n) \times n}
    \end{bmatrix}
    = \hat{\bm{Q}} \hat{\bm{R}}.
\]
This alternate decomposition is called the reduced (or somtimes the thin) QR-factorization. We shall state the following two theorems on the QR-factorization are stated without proof.

\begin{thm} \label{theorem: QR_general_existence}
    Every $\bm{A} \in \KK^{m \times n}, \; (m \geq n)$ has a full $QR$ factorisation, hence also a reduced $QR$ factorisation.
    \cite{TrefethenLloydN.LloydNicholas1997Nla/}
\end{thm}

\begin{thm} \label{theorem: QR_full_rank_unique}
    Each $\bm{A} \in \KK^{m \times n}, \; (m \geq n)$ of full rank has a unique reduced $QR$ factorisation $\bm{A} = \hat{\bm{Q}} \hat{\bm{R}}$ with $r_{k,k} > 0$.
    \cite{TrefethenLloydN.LloydNicholas1997Nla/}
\end{thm}

In practice the classical Gram-Schmidt process described in algorithm \ref{alg: Classical_Gram-Schmidt} is rarely used as the procedure becomes numerically unstable if $\left( \bm{a}_i \right)_{i=1}^{n}$ are almost linearly dependent. Before looking for ways to resolve these numerical instabilities a quick recap of projectors has been devised. A square matrix $\bm{P}_{G}$ acting on a Hilbert space $H$ that sends $\bm{x} \in H$ to its projection onto a subspace $G$ is called the projector onto $G$. If $\left( \bm{q}_k \right)_{k=1}^{m}$ is an orthornormal basis in $G$ then
\[
    \bm{P}_{G} = \bm{Q} \bm{Q}^{\ast}
\]
where $\bm{Q} = \left[ \bm{q}_1 , \bm{q}_2 , \ldots \bm{q}_m, 0 , \ldots , 0 \right] \in \KK^{n \times n}$. A special class of projectors which isolates the components of a given vector onto a one dimensional subspace spanned by a single unit vector $\bm{q}$ called a rank one orthogonal projector, denoted as $\bm{P}_{q}$. Each $k$ in the classical Gram-Schmidt process $\bm{q}_k'$ using the following orthogonal projection
\begin{equation}\label{eq: classical_GS_proj}
    \bm{q}_k' = \bm{P}_{A_{k}^{\perp}} \bm{a}_k
\end{equation}
where $A_k = \operatorname{l.s} \left\{ \bm{a}_i \right\}_{i=1}^{k}$ and $\bm{P}_{A_{1}^{\perp}} = \Id$ for convenience. A modified version of the Gram-Schmidt process performs the same orthogonal projection broken up as $k-1$ orthogonal projections of rank $n-1$ as so
\begin{align*}
    \bm{q}_k' & = \bm{P}_{A_{k}^{\perp}} \bm{a}_k                                                                                                                                   \\
              & = \left( \Id - \bm{Q}_{k} \bm{Q}_{k}^{\ast} \right) \bm{a}_k                                                                                                        \\
              & = \left( \prod_{i=1}^{k-1} \left( \Id - \bm{q}_i \bm{q}_i^{\ast} \right) \right)\bm{a}_k                                                                            \\
              & = \left( \Id - \bm{q}_1 \bm{q}_1^{\ast} \right) \left( \Id - \bm{q}_1 \bm{q}_1^{\ast} \right) \cdots \left( \Id - \bm{q}_{k-1} \bm{q}_{k-1}^{\ast} \right) \bm{a}_k \\
              & = \bm{P}_{\bm{q}_{k}^{\perp}} \cdots \bm{P}_{\bm{q}_{1}^{\perp}} \bm{a}_k
\end{align*}

While its clear that $\bm{P}_{A_{k}^{\perp}} \bm{a} $ and $\bm{P}_{\bm{q}_{k}^{\perp}} \cdots \bm{P}_{\bm{q}_{1}^{\perp}} \bm{a}_k$ used for computing $\bm{q}_k'$ are algebraically, they differ arithmetically as the latter expression evaluates $\bm{q}_k'$ using the follow procedure

\begin{align*}
    \bm{q}_k^{(1)}             & = \bm{a}_k                                       \\
    \bm{q}_k^{(2)}             & = \bm{P}_{\bm{q}_{1}^{\perp}} \bm{q}_k^{(1)}     \\
    \bm{q}_k^{(3)}             & = \bm{P}_{\bm{q}_{2}^{\perp}} \bm{q}_k^{(2)}     \\
                               & \vdots                                           \\
    \bm{q}_k' = \bm{q}_k^{(k)} & = \bm{P}_{\bm{q}_{k-1}^{\perp}} \bm{q}_k^{(k-1)}
\end{align*}

Applying projections sequentially in this manner produces smaller numerical errors. The modified Gram-Schmidt process \cite{TrefethenLloydN.LloydNicholas1997Nla/,DemmelJamesW1997Anla} is summarized in the following algorithm.

    {\centering
        \begin{minipage}{.85\linewidth}
            \begin{algorithm}[H]
                \caption{Modified Gram-Schmidt}
                \label{alg: Modified_Gram-Schmidt}
                \SetAlgoLined
                \DontPrintSemicolon
                \SetKwInOut{Input}{input}\SetKwInOut{Output}{output}

                \Input{A basis $\left\{ \bm{a}_i \right\}_{i=1}^{n}$.}
                \Output{An orthornormal basis $\left\{ \bm{q}_i \right\}_{i=1}^{n}$ such that $\operatorname{l.s} \left\{ \bm{a}_i \right\}_{i=1}^{n} = \operatorname{l.s} \left\{  \bm{q}_i \right\}_{i=1}^{n}$}
                \BlankLine
                \For{$k = 1$ \KwTo $n$}{
                    $\bm{q}_k' = \bm{a}_k$\;
                }
                \For{$k = 1$ \KwTo $n$}{
                    $r_{k,k} = \| \bm{q}_k' \|$\;
                    $\bm{q}_k = \bm{q}_k' / r_{k,k}$\;
                    \For{$i = k+1$ \KwTo $n$}{
                        $r_{i,k} = \langle \bm{q}_k, \bm{q}_i' \rangle$\;
                        $\bm{q}_i = \bm{q}_i - r_{i,k} \bm{q}_i$\;
                    }
                }
                \Return{$\left\{ \bm{q}_i \right\}_{i=1}^{n}$}
                \BlankLine
            \end{algorithm}
        \end{minipage}
        \par
    }

\subsubsection{Arnoldi and Lanczos Algorithm}\label{Section1.1.3}

As a quick reminder, we are in search of an iterative process to solve the linear system $\bm{A} \bm{x}^{\star} = \bm{b}$ where no explicit form of $\bm{A}$ is available and we may only rely on a routine that computes $\bm{A} \bm{v}$ for any $\bm{v}$ to extract information on $\bm{A}$. In section \ref{Section1.1.1} we discovered that $\bm{x}^{\star} \in \calK_{t_{\bm{r}_0}, \bm{A}} \left( \bm{A}, \bm{r}_0 \right)$. With many iterative methods, computing an exact value for $\bm{x}^{\star}$ is out the question with the view that $t_{\bm{r}_0, \bm{A}}$ is impractically large. We must instead resort to approximating $\bm{x}^{\star}$ by $\bm{x}_k$ for which $\bm{x}^{k} \in \calK_{k} \left( \bm{A}, \bm{r}_0 \right)$ where $k \ll t_{\bm{r}_0}$. To find an appropriate value for $\bm{x}_k$, a good start would be to find a basis $\calK_{k} \left( \bm{A}, \bm{r}_0 \right)$. Definition \ref{defe: krylov_subspace} showed us that $\left\{ \bm{A}^{i-1} \bm{r}_0 \right\}_{i=1}^{k}$ serves as a basis for $\calK_{k} \left( \bm{A}, \bm{r}_0 \right)$. However, for numericcal reasons this is a poor choice of basis since this each consecutive term becomes closer and closer to being linearly dependent. To search for a more appriporate basis, set $n = t_{\bm{r}_0, \bm{A}}$ so that $\bm{x}^{\star} \in \calK_{n} \left( \bm{A}, \bm{r}_0 \right)$. Let $\bm{K} \in \KK^{n \times n}$ be the invertible matrix
\[
    \bm{K} = \left[ \bm{r}_0 , \bm{A} \bm{r}_0, \ldots , \bm{A}^{n-1} \bm{r}_0 \right].
\]
Since $\bm{K}$ is invertible we can compute $\bm{c} = - \bm{K}^{-1} \bm{A}^{n} \bm{r}_0$ so that
\begin{align*}
    \bm{A} \bm{K} & = \left[ \bm{A} \bm{r}_0, \bm{A}^{2} \bm{r}_0, \ldots , \bm{A}^{n} \bm{r}_0 \right]                     \\
    \bm{A} \bm{K} & = \bm{K} \cdot \left[ \bm{e}_2, \bm{e}_3, \ldots , \bm{e}_n, - \bm{c}  \right] \triangleq \bm{K} \bm{C}
\end{align*}
or, in a more verbose form
\[
    \bm{K}^{-1} \bm{A} \bm{K} = \bm{C} =
    \begin{bmatrix}
        0      & 0      & \cdots & 0      & -c_1   \\
        1      & 0      & \cdots & 0      & -c_2   \\
        0      & 1      & \cdots & 0      & \vdots \\
        \vdots & \vdots & \cdots & \vdots & \vdots \\
        0      & 0      & \cdots & 1      & -c_n
    \end{bmatrix}.
\]
Note here that $\bm{C}$ is upper Hessenberg. While this form is simple, it is of little practical use since the matrix $\bm{K}$ is very likely to be ill-conditioned. To remedy this we can replace $\bm{K}$ with an orthogonal matrix which spans the same space. These are exactly the properties that the $\bm{Q}$ matrix offers in the $QR$-factorisation of $\bm{K}$. With this in mind let $\bm{K} = \bm{Q} \bm{R}$ be the full $QR$-factorisation of $\bm{K}$. Then
\begin{align*}
    \bm{A} \bm{Q} \bm{R} & = \bm{A} \bm{K}                    \\
    \bm{A} \bm{Q}        & = \bm{A} \bm{K} \bm{R}^{-1}        \\
    \bm{A} \bm{Q}        & = \bm{K} \bm{C} \bm{R}^{-1}        \\
    \bm{A} \bm{Q}        & = \bm{Q} \bm{R} \bm{C} \bm{R}^{-1} \\
    \bm{A} \bm{Q}        & \triangleq \bm{Q} \bm{H}.
\end{align*}
Since $\bm{R}$ and $\bm{R}^{-1}$ and both upper triangular and $\bm{C}$ is upper Hessenberg, $\bm{H}$ is also upper Hessenberg. This form provides us with a $\bm{Q}$ such that the range of $\bm{Q}$ is $\calK_{n} \left( \bm{A}, \bm{r}_0 \right)$ and
\begin{equation}\label{eq: QTAQ_eq_H}
    \bm{Q}^{\intercal} \bm{A} \bm{Q} = \bm{H}.
\end{equation}
Again, in practice, it may be very difficult to compute this entire expression forcing us to search for approximative alternatives. Consider equation \ref{eq: QTAQ_eq_H} for which the only first $k$ columns of $\bm{Q}$ have been computed. Let $\bm{Q}_k = \left[ \bm{q}_1 , \bm{q}_2 , \ldots , \bm{q}_k \right]$ and $\bm{Q}_u = \left[ \bm{q}_{k+1} , \bm{q}_{k+2} , \ldots , \bm{q}_{n} \right]$. Then
\begin{align*}
    \bm{Q}^{\intercal} \bm{A} \bm{Q}                                                         & = \bm{H} \\
    \left[ \bm{Q}_k , \bm{Q}_u \right]^{\intercal} \bm{A} \left[ \bm{Q}_k , \bm{Q}_u \right] & =
    \begin{bmatrix}
        \bm{H}_k     & \bm{H}_{u,k} \\
        \bm{H}_{k,u} & \bm{H}_{u}
    \end{bmatrix}                                                                           \\
    \begin{bmatrix}
        \bm{Q}_{k}^{\intercal} \bm{A} \bm{Q}_{k} & \bm{Q}_{k}^{\intercal} \bm{A} \bm{Q}_{u} \\
        \bm{Q}_{u}^{\intercal} \bm{A} \bm{Q}_{k} & \bm{Q}_{u}^{\intercal} \bm{A} \bm{Q}_{u}
    \end{bmatrix}
                                                                                             & =
    \begin{bmatrix}
        \bm{H}_k     & \bm{H}_{u,k} \\
        \bm{H}_{k,u} & \bm{H}_{u}
    \end{bmatrix}
\end{align*}
where $\bm{H}_k , \bm{H}_{u,k}, \bm{H}_{k,u}$ and $\bm{H}_u$ are the relevant sub matrices. This provides us with the equality
\begin{equation}\label{eq: QTkAQk_eq_Hk}
    \bm{Q}_{k}^{\intercal} \bm{A} \bm{Q}_{k} = \bm{H}_k
\end{equation}
noting that $\bm{H}_{k}$ is upper Hessenberg for the same reason that $\bm{H}$ is. We know that when $n = t_{\bm{r}_0, \bm{A}}$ we can find a $\bm{Q} \in \KK^{n \times n}$ and $\bm{H} \in \KK^{n \times n}$ that satisfies $\bm{A} \bm{Q} = \bm{Q} \bm{H}$. However, in general, we may not be so fortunate in finding a $\bm{Q}_{k} \in \KK^{n \times k}$ and $\bm{H}_{k} \in \KK^{n \times k}$ so satisfy $\bm{A} \bm{Q}_{k} = \bm{Q}_{k} \bm{H}_k$ for any $k < n$. Instead we can adjust this equality by adding an error $\bm{E}_k \in \KK^{n \times k}$ so that we do get equality. Our expression now becomes
\begin{equation}\label{eq: QTkAQk_eq_HkEk}
    \bm{Q}_{k}^{\intercal} \bm{A} \bm{Q}_{k} = \bm{H}_k + \bm{E}_k.
\end{equation}
A careful choice of $\bm{E}_k$ must be made to also retain equality in equation \ref{eq: QTkAQk_eq_Hk}, meaning $\bm{Q}_{k}^{\intercal} \bm{E}_k = \bm{0}$. Since $\left\{ \bm{q}_i \right\}_{i=1}^{k}$ forms an orthornormal basis for $\calK_{n} \left( \bm{A}, \bm{r}_0 \right)$, consider the following choice of $\bm{E}_k$,
\[
    \bm{E}_k = \bm{q}_{k+1} \bm{h}_{k}^{\intercal}
\]
where $\bm{h}_k$ is any vector in $\KK^{k}$. Notice that
\[
    \bm{Q}_{k}^{\intercal} \bm{E} = \bm{Q}^{\intercal} \left( \bm{q}_{k+1} \bm{h}_k \right) = \left( \bm{Q}^{\intercal} \bm{q}_{k+1} \right) \bm{h}_{k}^{\intercal} = \bm{0}.
\]
Since this holds for any $\bm{h}_k \in \KK^{k}$, to preserve sparsity and to keep this form as simple as possible we can set $\bm{h}_k = \left[ 0,0, \ldots , h_{k+1,k} \right]^{\intercal}$. This means $\bm{A} \bm{Q}_k$ can be written as
\begin{equation}\label{eq: QTkAQk_eq_Hk_p_qkhk}
    \bm{A} \bm{Q}_k =  \bm{Q}_k \bm{H}_k + \bm{q}_{k+1} \bm{h}_{k}^{\intercal}
\end{equation}
where
\[
    \bm{Q}_k \bm{H}_k =
    \left[ \bm{q}_1 , \bm{q}_2 , \ldots , \bm{q}_k \right]
    \begin{bmatrix}
        h_{1,1} & \cdots & \cdots & \cdots    & h_{1,k}   \\
        h_{2,1} & \cdots & \cdots & \cdots    & \vdots    \\
        0       & \ddots & \ddots & \ddots    & \vdots    \\
        \vdots  & \ddots & \ddots & \ddots    & \vdots    \\
        0       & \cdots & 0      & h_{k,k-1} & h_{k,k}   \\
        0       & \cdots & 0      & 0         & h_{k+1,k}
    \end{bmatrix}.
\]
Equating the $j^{th}$ columns of equation \ref{eq: QTkAQk_eq_Hk_p_qkhk} yields
\[
    \bm{A} \bm{q}_j = \sum_{i=1}^{j+1} h_{i,j} \bm{q}_{i}.
\]
Again since $\left\{ \bm{q}_i \right\}_{i=1}^{n}$ form an orthornormal basis, multiplying both sides by $\bm{q}_m$ for $1 \leq m \leq j$ gives
\[
    \bm{q}_m^{\intercal} \bm{A} \bm{q}_j = \sum_{i=1}^{j+1} h_{i,j} \bm{q}_m^{\intercal} \bm{q}_{i} = h_{m,j}
\]
and so
\begin{equation}\label{eq: arn_eq_1}
    h_{j+1,j} \bm{q}_{j+1} = \bm{A} \bm{q}_j - \sum_{i=1}^{j} h_{i,j} \bm{q}_{i}.
\end{equation}
From equation \ref{eq: arn_eq_1} we find that $\bm{q}_{j+1}$ can be computed using a recurrance involving its previous Krylov factors. Notice this bears a striking resemblance to equation \ref{eq: comp_orth_basis} having a virtually an identical setup to computing an orthornormal basis using the modified Gram-Schmidt process (algorithm \ref{alg: Modified_Gram-Schmidt}). As such, values for $\bm{q}_{j+1}$ and $h_{j+1,j}$ can be evaluted using a procedure very similar to the modified Gram-Schmidt process better known as the Arnoldi algorithm \cite{TrefethenLloydN.LloydNicholas1997Nla/,DemmelJamesW1997Anla}, presented in algorithm \ref{alg: Arnoldi_Algorithm}.

{\centering
\begin{minipage}{.85\linewidth}
    \begin{algorithm}[H]
        \caption{Arnoldi Algorithm}
        \label{alg: Arnoldi_Algorithm}
        \SetAlgoLined
        \DontPrintSemicolon
        \SetKwInOut{Input}{input}\SetKwInOut{Output}{output}

        \Input{$\bm{A}, \bm{r}_0$ and $k$, the number of columns of $\bm{Q}$ to compute.}
        \Output{$\bm{Q}_k , \bm{H}_k$.}
        \BlankLine
        $\bm{q}_1 = \bm{r}_0 / \| \bm{r}_0 \|$\;
        \For{$j = 1$ \KwTo $k$}{
            $\bm{z} = \bm{A} \bm{q}_j$\;
            \For{$i = 1$ \KwTo $j$}{
                $h_{i,j} = \langle \bm{q}_{i} , \bm{z} \rangle$\;
                $\bm{z} = \bm{z} - h_{i,j} \bm{q}_{i}$\;
            }
            $h_{j+1,j} = \| \bm{z} \|$\;
            \If{$h_{j+1,j} = 0$}{
                \Return{$\bm{Q}_k , \bm{H}_k$}
            }
            $\bm{q}_{j+1} = \bm{z} / h_{j+1,j}$\;
        }
        \Return{$\bm{Q}_k , \bm{H}_k$}
        \BlankLine
    \end{algorithm}
\end{minipage}
\par
}

When $\bm{A}$ is symmertic then $\bm{H} = \bm{T}$ becomes a tridiagonal matrix, simplifying a large amount of the Arnoldi algorithm since a considerably large number of matrix elements from $\bm{T}$ can be written as
\[
    \bm{T} =
    \begin{bmatrix}
        \alpha_1 & \beta_1 &        &             &             \\
        \beta_1  & \ddots  & \ddots &             &             \\
                 & \ddots  & \ddots & \ddots      &             \\
                 &         & \ddots & \ddots      & \beta_{n-1} \\
                 &         &        & \beta_{n-1} & \alpha_{n}
    \end{bmatrix}.
\]
As before, equating the $j^{th}$ columns of $\bm{A} \bm{Q} = \bm{Q} \bm{T}$ yields
\begin{equation}\label{eq: lancz_orth_basis}
    \bm{A} \bm{q}_{j} = \beta_{j-1} \bm{q}_{j-1} + \alpha_{j} \bm{q}_j + \beta_j \bm{q}_{j+1}.
\end{equation}
Again since $\left\{ \bm{q}_{i} \right\}_{i=1}^{n}$ form an orthornormal basis, multiplying both sides of equation \ref{eq: lancz_orth_basis} by $\bm{q}_j$ gives $\bm{q}_j = \bm{A} \bm{q}_j = \alpha_j$. A simplified version of the Arnoldi algorithm can be devised can be used to compute $\left\{ \bm{q}_{i} \right\}_{i=1}^{n}$ and $\bm{T}$ for symmetric matrices known as the Lanczos algorithm \cite{DemmelJamesW1997Anla}. The Lanczos algorithm is presented in algorithm \ref{alg: Lanczos_Algorithm}.

{\centering
\begin{minipage}{.85\linewidth}
    \begin{algorithm}[H]
        \caption{Lanczos Algorithm}
        \label{alg: Lanczos_Algorithm}
        \SetAlgoLined
        \DontPrintSemicolon
        \SetKwInOut{Input}{input}\SetKwInOut{Output}{output}

        \Input{$\bm{A}, \bm{r}_0$ and $k$, the number of columns of $\bm{Q}$ to compute.}
        \Output{$\bm{Q}_k , \bm{T}_k$.}
        \BlankLine
        $\bm{q}_1 = \bm{r}_0 / \| \bm{r}_0 \|$, $\beta_0 = 0$, $\bm{q}_0 = 0$\;
        \For{$j = 1$ \KwTo $k$}{
            $\bm{z} = \bm{A} \bm{q}_j$\;
            $\alpha_j = \langle \bm{q}_{j}, \bm{z} \rangle$\;
            $\bm{z} = \bm{z} - \alpha_j \bm{q}_{j} - \beta_{j-1} \bm{q}_{j-1}$\;
            $\beta_j = \| z \|$\;
            \If{$\beta_{j} = 0$}{
                \Return{$\bm{Q}_k , \bm{T}_k$}
            }
            $\bm{q}_{j+1} = \bm{z} / \beta_{j}$\;
        }
        \Return{$\bm{Q}_k , \bm{T}_k$}
        \BlankLine
    \end{algorithm}
\end{minipage}
\par
}

\subsubsection{Optimality Conditions}\label{Section1.1.4}

So far we have shown that $\bm{x}^{\star} \in \calK_{t_{\bm{r}_0}, \bm{A}} \left( \bm{A}, \bm{r}_0 \right)$ where $n = t_{\bm{r}_0}$ is the grade of $\bm{r}_0$ with respect to $\bm{A}$. Moreover from section \ref{Section1.1.3} we found ways to construct a basis for $\calK_{t_{\bm{r}_0}, \bm{A}} \left( \bm{A}, \bm{r}_0 \right)$ allowing us to generate vectors with these affine spaces, namely the Arnoldi algorithm (algorithm \ref{alg: Arnoldi_Algorithm})and Lanczos algorithm (algorithm \ref{alg: Lanczos_Algorithm}) for non-symmertic and symmertic systems respectively. From now on $\calK_{t_{\bm{r}_0}, \bm{A}} \left( \bm{A}, \bm{r}_0 \right)$ will be abbreviated to $\calK_{t_{\bm{r}_0}, \bm{A}}$ when the context is clear. The question still remains however, how should one choose an $\bm{x}_k$ that best approximates $\bm{x}^{\ast}$ satisfying equation \ref{eq: lin_sys_1}? Here are a few of the most well known methods for selecting a suitable $\bm{x}_k$.

\begin{enumerate}

    \item Select an $\bm{x}_k \in \bm{x}_0 + \calK_k$ which minimizes $\| \bm{x}_k - \bm{x}^{\ast} \|_2$. While this method seems like the most intuitive and natural way to select $\bm{x}_k$, it is unfortunately of no practical use since there is not enough information in the Krylov subspace to find an $\bm{x}_k$ which matches this specification.

    \item Select an $\bm{x}_k \in \bm{x}_0 + \calK_k$ which minimizes $\| \bm{r}_k \|_2$ (recall is the residual of $\bm{x}_k$, that is, $\bm{r}_k = \bm{b} - \bm{A} \bm{x}_k$). This method is possible to implement. Two well known algorithms stem from this class of methods, notably MINRES (minimum residual) and GMRES (general minimum residual) which solve linear systems for symmetric and non-symmertic $\bm{A}$ respectively.

    \item When $\bm{A}$ is a positive definite matrix it defines a norm $\| r \|_{\bm{A}} = \left( \bm{r}^{\intercal} \bm{A} \bm{r} \right)^{\frac{1}{2}}$, called the energy norm. Select an $\bm{x}_k \in \bm{x}_0 + \calK_k$ which minimizes $\| r \|_{\bm{A}^{-1}}$ which is equivalent to minimizing $\| \bm{x}_k - \bm{x} \|_{A}$. This technique is known as the CG (conjugate gradient) algorithm.

    \item Select an $\bm{x}_k \in \bm{x}_0 + \calK_k$ for which $\bm{r}_{k} \perp \calW_k$ where $\calW_k$ is some $k$-dimensional subspace. Two well known algorithms that belong to this family of methods are SYMMLQ (Symmetric LQ Method) and a variant of GMRES used for solving symmetric and non-symmetric methods respectively.

\end{enumerate}

Interestingly, when $\bm{A}$ is symmetric positive definite and $\calW_k = \calK_k$ the last two selection methods are equivalent. This is stated more precisely in thoerem

\begin{thm} \label{theorem: 3_4_method_eq}
    In the context of the above selection method, if $\bm{A} \succ \bm{0}$ and $\calW_k = \calK_k$ in method (4) then it produces the same $\bm{x}_k$ in method (3) \cite{DemmelJamesW1997Anla}.
\end{thm}

In fact the very last method can be used to bring together a number of different analytical aspects and unify them in a general framework known as projection methods. Selecting an $\bm{x}_k$ from our Krylov subspace allows $k$ degrees of freedom meaning $k$ constraints must be used to determine a unique $\bm{x}_k$ for selection. As seen in method (4) already, typically orthogonality constraints are imposed on the residual $\bm{r}_k$. Specifically we would like to find a $\bm{x}_k \in \bm{x}_0 + \calK_k$ where $\bm{r}_k \perp \calW_k$. This is sometimes referred to as the Petrov-Galerkin (or just Galerkin) conditions. Projection methods for which $\calW_k = \calK_k$ are known as orthogonal projections while methods for which $\calW_k = \bm{A} \calK_k$ are known as oblique projections. If we set $\bm{x}_k = \bm{x}_0 + \bm{z}_k$ for some $\bm{z}_k \in \calK_k$ then the Petrov-Galerkin conditions imply $\bm{r}_0 - \bm{A} \bm{z}_k \perp \calW_k$, or alternatively $\langle \bm{r}_0 - \bm{A} \bm{z}_k , \bm{w} \rangle = 0$ for every $\bm{w} \in \calW_k$. To impose these conditions it will help to have an appropriate basis for $\calK$ and $\calW$. Suppose we have access to such a basis where $\left\{ \bm{q}_i \right\}_{i=1}^{k}$ and $\left\{ \bm{w}_i \right\}_{i=1}^{k}$ are basis elements for $\calK$ and $\calW$ respectively. Let
\begin{align*}
    \bm{K}_k & \triangleq \left[ \bm{v}_1 , \bm{v}_2 , \ldots , \bm{v}_k \right] \in \KK^{n \times k} \\
    \bm{W}_k & \triangleq \left[ \bm{w}_1 , \bm{w}_2 , \ldots , \bm{w}_k \right] \in \KK^{n \times k}
\end{align*}
then the Petrov-Galerkin conditions can be imposed as follows
\begin{align*}
    \bm{K}_k \bm{y}_k                                                       & = \bm{z}_k , \quad \text{for some} \; \bm{y}_k \in \KK^k \\
    \bm{W}_k^{\intercal} \left( \bm{r}_0 - \bm{A} \bm{K}_k \bm{y}_k \right) & = \bm{0}.
\end{align*}
Moreover if $\bm{W}_k^{\intercal} \bm{A} \bm{K}_k$ is invertible then $\bm{x}_k$ can be expressed as
\begin{equation} \label{eq: expr_x_Petrov_Galerkin_1}
    \bm{x}_k = \bm{x}_0 + \bm{K}_k \left( \bm{W}_k^{\intercal} \bm{A} \bm{K}_k \right)^{-1} \bm{W}_k \bm{r}_0.
\end{equation}
This justifies a general form of the projection method algorithm presented in algorithm \ref{alg: General_Projection}.

{\centering
\begin{minipage}{.85\linewidth}
    \begin{algorithm}[H]
        \caption{General Projection Method}
        \label{alg: General_Projection}
        \SetAlgoLined
        \DontPrintSemicolon
        \SetKwInOut{Input}{input}\SetKwInOut{Output}{output}

        \Output{An approximation of $\bm{x}^{\ast}$, $\bm{x}_k$.}
        \BlankLine
        \For{$k = 1 , \ldots $ \Until convergence}{
        Select $\calK_k$ and $\calW_k$\;
        Form $\bm{K}_k$ and $\bm{W}_k$\;
        Solve $\left( \bm{W}_k^{\intercal} \bm{A} \bm{K}_k \right) \bm{y}_k = \bm{W}_k^{\intercal} \bm{r}_0$\;
        $\bm{x}_k = \bm{x}_0 + \bm{K}_k \bm{y}_k$\;
        }
        \Return{$\bm{x}_k$}
        \BlankLine
    \end{algorithm}
\end{minipage}
\par
}

\subsubsection{Conjugate Gradient Algorithm}\label{Section1.1.5}

From \Cref{Section1.1.4} that the Petrov-Galerkin conditions for the CG algorithm used an orthogonal projection and the matrix $\bm{A}$ was assumed to be positive definite. To derive the CG algorithm we can start be using some machinery that the Lanczos algorithm provides us with. Recall, the Lanczos algorithm produces the form $\bm{A}\bm{Q}_k = \bm{Q}_k \bm{T}_k + \bm{q}_{k+1} \bm{t}_{k}^{\intercal}$ where $\bm{t}_{k} \triangleq \left[ 0,0, \ldots , 0, \beta_k \right]^{\intercal} \in \KK^k$ and the columns of $\bm{Q}_k$ span $\calK_k$. Recall that $\bm{x}_k$ can be expressed as $\bm{x}_k = \bm{x}_0 + \bm{K}_k \left( \bm{W}_k^{\intercal} \bm{A} \bm{K}_k \right)^{-1} \bm{W}_k \bm{r}_0$ (\Cref{eq: expr_x_Petrov_Galerkin_1}) when $\bm{W}_k^{\intercal} \bm{A} \bm{K}_k$ is invertible. For the CG algorithm $\calK = \calW$ and $\bm{A} \succ \bm{0}$. Under these conditions we can easily show that $\bm{W}_k^{\intercal} \bm{A} \bm{K}_k$ is indeed invertible. This means the approximate vector can be expressed as $\bm{x}_k = \bm{x}_0 + \bm{z}_k$ where $\bm{z}_k \in \calK_k$. In terms of the Petrov-Galerkin conditions this means that $\bm{z}_k$ must satisfy $\bm{r}_0 - \bm{A} \bm{z}_k \perp \calW_k$. Furthermore since $\calK_k = \operatorname{Range} \left( \bm{Q}_k \right)$ where $\bm{Q}_k$ has full column rank then $\bm{z}_k$ can be represented as $\bm{z}_k = \bm{Q}_k \bm{y}$ for a unique $\bm{y} \in \KK^k$ so that
\begin{equation} \label{eq: x_eq_Qky}
    \bm{x}_k = \bm{x}_0 + \bm{Q}_k \bm{y}.
\end{equation}
Coupling this with the Petrov-Galerkin conditions means
\begin{align} \label{eq: Tky_eq_normr0e1}
    \bm{Q}_k^{\intercal} \left( \bm{r}_0 - \bm{A} \bm{Q}_k \bm{y} \right) & = \bm{0}                        \nonumber \\
    \bm{Q}_k^{\intercal} \bm{A} \bm{Q}_k \bm{y}                           & = \bm{Q}_k^{\intercal} \bm{r}_0 \nonumber \\
    \bm{T}_k \bm{y}                                                       & = \| \bm{r}_0 \| \bm{e}_1.
\end{align}
In the CG algorithm $\bm{x}_{k+1}$ is computed as the recurrance of the following three sets of vectors
\begin{enumerate}
    \item The approximate solutions $\bm{x}_{k}$
    \item The residual vectors $\bm{r}_{k}$
    \item The conjugate gradient vectors $\bm{p}_k$
\end{enumerate}
The conjugate gradient vectors are given the name gradient since the attempt to find the direction of steepest descent that minimizes $\| \bm{r}_{k} \|_{\bm{A}^{-1}}$. The are also given the name conjugate since $\langle \bm{p}_k, \bm{A} \bm{p}_j \rangle = 0$ for $i \neq j$, that is, vectors $\bm{p}_i$ and $\bm{p}_j$ are mutally $A$-conjugate.

Since $\bm{A}$ is symmetric positive definite then so is $\bm{T}_k  = \bm{Q}_k \bm{A} \bm{Q}_k$. We can take the Cholesky decomposition of $\bm{T}_k$ to get
\begin{equation} \label{eq: Tk_Cholesky}
    \bm{T}_k = \bm{L}_k \bm{D}_k \bm{L}_k^{\intercal}
\end{equation}
where $\bm{L}_k$ is a unit lower bidiagonal matrix and $\bm{D}_k$ is diagonal written as
\[
    \bm{L}_k =
    \begin{bmatrix}
        1   &        &         &   \\
        l_1 & \ddots &         &   \\
            & \ddots & \ddots  &   \\
            &        & l_{k-1} & 1
    \end{bmatrix}, \quad
    \bm{D}_k =
    \begin{bmatrix}
        d_1 &     &        &     \\
            & d_2 &        &     \\
            &     & \ddots &     \\
            &     &        & d_k
    \end{bmatrix}.
\]
Combining equations \Cref{eq: x_eq_Qky}, \Cref{eq: Tky_eq_normr0e1} and \Cref{eq: Tk_Cholesky}
\begin{align*}
    \bm{x}_k & = \bm{x}_0 + \bm{Q}_k \bm{y}                                                                                                  \\
    \bm{x}_k & = \bm{x}_0 + \| \bm{r}_0 \| \bm{Q}_k \bm{T}_k^{-1} \bm{e}_1                                                                   \\
    \bm{x}_k & = \bm{x}_0 + \| \bm{r}_0 \| \bm{Q}_k \left( \bm{L}_k \bm{D}_k \bm{L}_k^{\intercal} \right)^{-1} \bm{e}_1                      \\
    \bm{x}_k & = \bm{x}_0 + \left( \bm{Q}_k \bm{L}_k^{-\intercal} \right) \left( \| \bm{r}_0 \| \bm{D}_k^{-1} \bm{L}_k^{-1} \bm{e}_1 \right) \\
    \bm{x}_k & \triangleq \bm{x}_0 + \tilde{\bm{P}}_k \tilde{\bm{y}}_k
\end{align*}
where $\tilde{\bm{P}}_k = \bm{Q}_k \bm{L}_k^{-\intercal}$ and $\tilde{\bm{y}}_k = \| \bm{r}_0 \| \bm{D}_k^{-1} \bm{L}_k^{-1} \bm{e}_1$. The matrix $\tilde{\bm{P}}_k$ can be written as
$\tilde{\bm{P}}_k = \left[ \tilde{\bm{p}}_1 , \tilde{\bm{p}}_2 , \ldots , \tilde{\bm{p}}_k \right]$. \Cref{lemma: Pk_cols_A_conj} shows that the columns of $\tilde{\bm{P}}_k$ are $A$-conjugate.

\begin{lem} \label{lemma: Pk_cols_A_conj}
    The columns of $\tilde{\bm{P}}_k$ are $A$-conjugate, in otherwise $\tilde{\bm{P}}_k^{\intercal} \bm{A} \tilde{\bm{P}}_k$ is diagonal.
\end{lem}

\begin{proof}
    We compute
    \begin{align*}
        \tilde{\bm{P}}_k^{\intercal} \bm{A} \tilde{\bm{P}}_k & = \left( \bm{Q}_k \bm{L}_k^{-\intercal} \right)^{\intercal} \bm{A} \left( \bm{Q}_k \bm{L}_k^{-\intercal} \right)         \\
                                                             & = \bm{L}_k^{-1} \left( \bm{Q}_k^{\intercal} \bm{A} \bm{Q}_k \right) \bm{L}_k^{-\intercal}                                \\
                                                             & = \bm{L}_k^{-1} \left( \bm{T}_k \right) \bm{L}_k^{-\intercal}                                                            \\
                                                             & = \bm{L}_k^{-1} \left( \bm{L}_k \bm{D}_k \bm{L}_k^{\intercal} \right) \bm{L}_k^{-\intercal} \tag{\Cref{eq: Tk_Cholesky}} \\
                                                             & = \bm{D}_k
    \end{align*}
    as wanted.
\end{proof}

Since $\bm{L}_k$ is a lower bidiagonal, setting $\bm{a} \triangleq l_{k-1} \bm{e}_{k-1}$, it can be written in the form
\[
    \bm{L}_k =
    \begin{bmatrix}
        \bm{L}_{k-1}       & \bm{0} \\
        \bm{a}^{\intercal} & 1
    \end{bmatrix}
\]
meaning
\[
    \bm{L}_k^{-1} =
    \begin{bmatrix}
        \bm{L}_{k-1}^{-1} & \bm{0} \\
        \star             & 1
    \end{bmatrix}.
\]

With this a recurrance for the columns of $\tilde{\bm{P}}_k$ can now be derived in terms of $\bm{y}_k$. To start we can show that the first $k-1$ entries of $\tilde{\bm{y}}_{k}$ shares the first $k-1$ entires with $\tilde{\bm{y}}_{k-1}$ and that $\tilde{\bm{P}}_k$ and $\tilde{\bm{P}}_{k-1}$ share the same first $k-1$ columns. To start we can compute a recurrance for $\tilde{\bm{y}}_{k}$ as follows
\begin{align*}
    \tilde{\bm{y}}_{k} & = \| \bm{r}_0 \| \bm{D}_k^{-1} \bm{L}_k^{-1} \bm{e}_1^k \\
                       & = \| \bm{r}_0 \|
    \begin{bmatrix}
        \bm{D}_{k-1}^{-1} & \bm{0}   \\
        \bm{0}            & d_k^{-1}
    \end{bmatrix}
    \begin{bmatrix}
        \bm{L}_{k-1}^{-1} & \bm{0} \\
        \star             & 1
    \end{bmatrix}
    \bm{e}_1^k                                                                   \\
                       & = \| \bm{r}_0 \|
    \begin{bmatrix}
        \bm{D}_{k-1}^{-1} \bm{L}_{k-1}^{-1} & \bm{0}   \\
        \star                               & d_k^{-1}
    \end{bmatrix}
    \begin{bmatrix}
        \bm{e}_1^k \\
        0
    \end{bmatrix}                                                   \\
                       & =
    \begin{bmatrix}
        \tilde{\bm{y}}_{k-1} \\
        \eta_k
    \end{bmatrix}
\end{align*}
To get a recurrance for the columns of $\tilde{\bm{P}}_{k-1} = \left[ \tilde{\bm{p}}_1 , \tilde{\bm{p}}_2 , \ldots , \tilde{\bm{p}}_k \right]$ since $\bm{L}_{k-1}^{\intercal}$ is upper triangular then so is $\bm{L}_{k-1}^{-\intercal}$, thus forming the leading $(k-1)-\text{by}-(k-1)$ submatrix of $\bm{L}_{k}^{-\intercal}$. This means that $\tilde{\bm{P}}_{k-1}$ is identical to the leading $k-1$ columns of
\[
    \tilde{\bm{P}}_{k} = \bm{Q}_k \bm{L}_k^{-\intercal} = \left[ \bm{Q}_{k-1} , \bm{q}_k \right]
    \begin{bmatrix}
        \bm{L}_{k-1}^{-1} & \bm{0} \\
        \star             & 1
    \end{bmatrix}
    = \left[ \bm{Q}_{k-1} \bm{L}_{k-1}^{-1} , \tilde{\bm{p}}_{k} \right]
    = \left[ \tilde{\bm{P}}_{k-1} , \tilde{\bm{p}}_{k} \right].
\]
Moreover rearranging $\tilde{\bm{P}}_{k} = \bm{Q}_k \bm{L}_k^{-\intercal}$ we get $\tilde{\bm{P}}_{k} \bm{L}_k^{\intercal} = \bm{Q}_k$. Equating the $k^{th}$ column yields
\begin{equation} \label{eq: pk_rec}
    \tilde{\bm{p}}_{k} = \bm{q}_k - l_{k-1} \tilde{\bm{p}}_{k-1}.
\end{equation}
Finally we can use
\begin{equation} \label{eq: xk_rec}
    \bm{x}_k = \bm{x}_0 + \tilde{\bm{P}}_{k} \tilde{\bm{y}}_{k}                                 \\
    = \bm{x}_0 + \left[ \tilde{\bm{P}}_{k-1} , \tilde{\bm{p}}_{k} \right]
    \begin{bmatrix}
        \tilde{\bm{y}}_{k-1} \\
        \eta_k
    \end{bmatrix}                                                                    \\
    = \bm{x}_0 + \tilde{\bm{P}}_{k-1} \tilde{\bm{y}}_{k-1} + \eta_k \tilde{\bm{p}}_{k} \\
    = \bm{x}_{k-1} + \eta_k \tilde{\bm{p}}_{k}
\end{equation}
as a recurrance for $\bm{x}_k$. A recurrance for $\bm{r}_k$ is easily computed as
\begin{equation} \label{eq: rk_rec}
    \bm{r}_{k} = b - \bm{A} \bm{x}_k = b - \bm{A} \left( \bm{x}_{k-1} + \eta_k \tilde{\bm{p}}_{k} \right) = \left( b - \bm{A} \bm{x}_{k-1} \right) - \eta_k \bm{A} \tilde{\bm{p}}_{k} = \bm{r}_{k-1} - \eta_k \bm{A} \tilde{\bm{p}}_{k}
\end{equation}
Altogether we are left with recurrences for $\bm{q}_k$ from Lanczos, $\tilde{\bm{p}}_{k}$ (\Cref{eq: pk_rec}), the residual $\bm{r}_k$ (\Cref{eq: pk_rec}),  and for the approximate solution $\bm{x}_k$ (\Cref{eq: xk_rec}). However, futher simplification can be made for a more efficient algorithm. Recall from \Cref{Section1.1.3} that $\bm{A} \bm{Q}_k =  \bm{Q}_k \bm{T}_k + \bm{q}_{k+1} \bm{t}_{k}^{\intercal}$ where $\bm{t}_k = \left[ 0,0, \ldots , 0, \beta_k \right]^{\intercal} \in \KK^k$ meaning
\[
    \bm{r}_k = \bm{r}_0 - \bm{A} \bm{Q}_k \bm{y}_k = \bm{r}_0 - \bm{Q}_k \bm{T}_k \bm{y}_k - \langle \bm{t}_k , \bm{y} \rangle \bm{q}_{k+1} = - \beta_k y_k \bm{q}_{k+1}.
\]
This tells us that $\bm{r}_k$ is parallel to $\bm{q}_{k+1}$ and orthogonal to all $\bm{q}_{i}, \; 1 \leq i \leq k$. This further implies that $\bm{r}_k$ is orthogonal to all $\bm{r}_i, \; 1 \leq i \leq k-1$ since they are just $\bm{q}_{i}$ scaled by some constant factor. So replacing $\bm{r}_{k-1}$ with $\bm{q}_k / \eta_k$ and defining $\bm{p}_k \triangleq \tilde{\bm{p}}_k / \gamma_k$ gives us a new set of recurrences
\begin{align*}
    \bm{x}_k & = \bm{x}_{k-1} + \alpha_k \bm{p}_k        \\
    \bm{r}_k & = \bm{r}_{k-1} - \alpha_k \bm{A} \bm{p}_k \\
    \bm{p}_k & = \bm{r}_{k-1} + \beta_k \bm{p}_{k-1}
\end{align*}
where $\alpha_k = \eta_k / \gamma_k$. From \Cref{lemma: Pk_cols_A_conj} we have shown that the columns of $\tilde{\bm{P}}_k$ are $A$-conjugate (that is $\langle \tilde{\bm{p}}_i , \bm{A} \tilde{\bm{p}}_j \rangle = 0, \; i \neq j$) and that $\tilde{\bm{P}}_k^{\intercal} \bm{A} \tilde{\bm{P}}_k = \bm{D}_k$. This also means that $\langle \bm{r}_i , \bm{r}_j \rangle = 0, \; i \neq j$. Now note that from our recurrence for $\bm{p}_k = \bm{r}_{k-1} + \beta_k \bm{p}_{k-1}$ that
\[
    \langle \bm{A} \bm{p}_k ,\bm{p}_k \rangle = \langle \bm{A} \bm{p}_k , \bm{r}_{k-1} + \beta_k \bm{p}_{k-1} \rangle = \langle \bm{A} \bm{p}_k , \bm{r}_{k-1} \rangle.
\]
We can now find an expression for $\alpha_k$ as
\begin{align*}
    \langle \bm{r}_{k-1} , \bm{r}_{k} \rangle & = \langle \bm{r}_{k-1} , \bm{r}_{k-1} - \alpha_k \bm{A} \bm{p}_k \rangle                           \\
    \langle \bm{r}_{k-1} -1 \rangle           & = \langle \bm{r}_{k-1} , \bm{r}_{k-1} \rangle - \alpha_k \langle \bm{p}_k, \bm{A} \bm{p}_k \rangle \\
    \alpha_k                                  & = \frac{\langle \bm{r}_{k-1} , \bm{r}_{k-1} \rangle}{\langle \bm{p}_k, \bm{A} \bm{p}_k \rangle}.
\end{align*}
Similarly, using the recurrence for $\bm{p}_k$, an expression for $\beta_k$ can be computed as
\begin{align*}
    \langle \bm{A} \bm{p}_{k-1} , \bm{p}_k \rangle & = \langle \bm{A} \bm{p}_{k-1}, \bm{r}_{k-1} + \beta_k \bm{p}_{k-1} \rangle                                      \\
    \langle \bm{A} \bm{p}_{k-1} , \bm{p}_k \rangle & = \langle \bm{A} \bm{p}_{k-1}, \bm{r}_{k-1} \rangle + \beta_k \langle \bm{A} \bm{p}_{k-1}, \bm{p}_{k-1} \rangle \\
    \beta_k                                        & = - \frac{\langle \bm{A} \bm{p}_{k-1}, \bm{r}_{k-1} \rangle}{\langle \bm{A} \bm{p}_{k-1}, \bm{p}_{k-1} \rangle}
\end{align*}
This formula requires am additional dot product which was not present before. Fortunately, this dot product can be eliminated using our recurrence for $\bm{r}_k$
\begin{align*}
    \langle \bm{r}_k , \bm{r}_k \rangle & = \langle \bm{r}_k , \bm{r}_{k-1} - \alpha_k \bm{A} \bm{p}_k \rangle                            \\
    \langle \bm{r}_k , \bm{r}_k \rangle & = \langle \bm{r}_k , \bm{r}_{k-1} \rangle - \alpha_k \langle \bm{r}_k , \bm{A} \bm{p}_k \rangle \\
    \alpha_k                            & = - \frac{\langle \bm{r}_k , \bm{r}_k \rangle}{\langle \bm{r}_k , \bm{A} \bm{p}_k \rangle}.
\end{align*}
Equating the two expressions for $\bm{a}_k$ yields
\begin{align*}
    - \frac{\langle \bm{r}_k , \bm{r}_k \rangle}{\langle \bm{r}_k , \bm{A} \bm{p}_k \rangle}  & = \frac{\langle \bm{r}_{k-1} , \bm{r}_{k-1} \rangle}{\langle \bm{p}_k, \bm{A} \bm{p}_k \rangle} \\
    - \frac{\langle \bm{r}_k , \bm{r}_k \rangle}{\langle \bm{r}_{k-1} , \bm{r}_{k-1} \rangle} & = \frac{\langle \bm{r}_k , \bm{A} \bm{p}_k \rangle}{\langle \bm{p}_k, \bm{A} \bm{p}_k \rangle}.
\end{align*}
This means that
\[
    \beta_k = \frac{\langle \bm{r}_{k-1} , \bm{r}_{k-1} \rangle}{\langle \bm{r}_{k-2} , \bm{r}_{k-2} \rangle}.
\]
These recurrences are computed iteratively to form the basis of the CG algorithm, seen in \Cref{alg: CG}.

{\centering
\begin{minipage}{.85\linewidth}
    \begin{algorithm}[H]
        \caption{CG Algorithm}
        \label{alg: CG}
        \SetAlgoLined
        \DontPrintSemicolon
        \SetKwInOut{Input}{input}\SetKwInOut{Output}{output}

        \Input{$\bm{A} \succ \bm{0}$, $\bm{b}$ and an initial guess $\bm{x}_0$.}
        \Output{An approximation of $\bm{x}^{\ast}$, $\bm{x}_k$.}
        \BlankLine
        $\bm{r}_0 = \bm{b} - \bm{A} \bm{x}_0$, $\bm{p}_1 = \bm{r}_0$\;
        \For{$k = 1 , \ldots $ \Until $\| r_{k-1} \| \leq \tau$}{
            $\alpha_k = \frac{\langle \bm{r}_{k-1} , \bm{r}_{k-1} \rangle}{\langle \bm{p}_k, \bm{A} \bm{p}_k \rangle}$ \;
            $\bm{x}_k = \bm{x}_{k-1} + \alpha_k \bm{p}_k$ \;
            $\bm{r}_k = \bm{r}_{k-1} - \alpha_k \bm{A} \bm{p}_k$ \;
            $\beta_{k+1} = \frac{\langle \bm{r}_{k} , \bm{r}_{k} \rangle}{\langle \bm{r}_{k-1} , \bm{r}_{k-1} \rangle}$ \;
            $\bm{p}_{k+1} = \bm{r}_{k} + \beta_{k+1} \bm{p}_k$ \;
        }
        \Return{$\bm{x}_k$}
        \BlankLine
    \end{algorithm}
\end{minipage}
\par
}


\subsection{Gaussian Processes}\label{Section1.4}
A {\it Gaussian Process} (GP) is a collection of random variables with index set $I$, such that every finite subset of random variables has a joint Gaussian distribution \cite{RasmussenCarlEdward2006Gpfm,MurphyKevinP2012Ml}.

A GP is completely characterised by a mean function $m(\bm{x})$ and a covariance function $k (\bm{x}, \bm{x'})$ on a real process as
\begin{align*}
    m(\bm{x})           & = \EE \left[ f(\bm{x}) \right]                                         \\
    k (\bm{x}, \bm{x'}) & = \EE \left[ (f(\bm{x}) - m(\bm{x})) (f(\bm{x'}) - m(\bm{x'})) \right]
\end{align*}
A function $f(\bm{x})$ sampled from a GP with mean $m(\bm{x})$ and covariance $k (\bm{x}, \bm{x'})$ is written as
\[
    f(\bm{x}) \sim \calG \calP \left( m(\bm{x}), k (\bm{x}, \bm{x'}) \right)
\]
Since a GP is a collection of random variables it must satisfy the consistency requirement, that is, an observation of a set of variables should not the distribution of any small sub set of the observed values. More specifically if
\[
    (\bm{y_1}, \bm{y_2}) \sim \calN (\bm{\mu}, \bm{\Sigma})
\]
then
\begin{align*}
    \bm{y_1} & \sim \calN (\bm{\mu_1}, \bm{\Sigma_{1,1}}) \\
    \bm{y_2} & \sim \calN (\bm{\mu_2}, \bm{\Sigma_{2,2}})
\end{align*}

where $\bm{\Sigma_{1,1}}$ and $\bm{\Sigma_{2,2}}$ are the relevant sub matrices.

\subsubsection{Noise-free observations}\label{Section1.4.1}
Typically when using GP we would like to incorporate data from observations, or training data, into our predictions on unobserved values.
Let us suppose there is some obsevered data $D = \left\{ (\bm{x}_i, \bm{f}_i) \mid i \in \left\{ 1,2, \ldots , n \right\} \right\}$ which is (unrealistically) noise-free that we would like to model as a GP. In other words, for any sample in our dataset we can be certain that the observed value is the true value of the underlying function we wish to model. Then for the observed data
\[
    \bm{f} \sim \calN \left( \bm{0}, \bm{K_{XX}} \right).
\]
where $\bm{K_{XX}} = k(\bm{X}, \bm{X}) \in \RR^{n \times n}$. We would then like to make a prediction for unobserved values say $X^{\ast} = \left[ \bm{x}_1^{\ast}, \bm{x}_2^{\ast}, \ldots , \bm{x}_{n_\ast}^{\ast} \right]$ with value $f_{\ast}$ as has a distribution of
\[
    \bm{f}_{\ast} \sim \calN \left( \bm{0}, \bm{K_{X^{\ast}X^{\ast}}} \right).
\]
where $\bm{K_{X^{\ast}X^{\ast}}} = k(\bm{X^{\ast}}, \bm{X^{\ast}}) \in \RR^{n_\ast \times n_\ast}$. Here $\bm{f}$ and $\bm{f}_{\ast}$ are independent but we would like to give them some sort of correlation. We can do this by having them originate from the same joint distribution. According to the prior, we can write the joint distribution of the training points $\bm{f}$ and the test points $\bm{f}_{\ast}$ as
\[
    \begin{bmatrix}
        \bm{f} \\
        \bm{f}_{\ast}
    \end{bmatrix}
    \sim \calN
    \begin{bmatrix}
        \bm{0}, &
        {
                \begin{bmatrix}
                    \bm{K_{XX}}                    & \bm{K_{XX^{\ast}}}        \\
                    \bm{K_{XX^{\ast}}}^{\intercal} & \bm{K_{X^{\ast}X^{\ast}}}
                \end{bmatrix}
            }
    \end{bmatrix}
\]
where $\bm{K_{XX^{\ast}}} = k(\bm{X}, \bm{X^{\ast}}) \in \RR^{n \times n_\ast}$.

While the above does give us some information on $\bm{f}_{\ast}$ is related to the observed data and the test inputs, it does not provide any method to evalute $\bm{f}_{\ast}$. To do this we shall need the assistance of the following lemma
\begin{thm}\label{theorem: cond_of_MVN}
    (Marginals and conditionals of an MVN \cite{MurphyKevinP2012Ml}) Suppose $\bm{x} = (\bm{x}_1, \bm{x}_2)$ is jointly Gaussian with parameters
    \[
        \bm{\mu} =
        \begin{bmatrix}
            \bm{\mu}_1 \\
            \bm{\mu}_2
        \end{bmatrix}, \quad
        \bm{\Sigma} =
        \begin{bmatrix}
            \bm{\Sigma}_{1,1} & \bm{\Sigma}_{1,2} \\
            \bm{\Sigma}_{2,1} & \bm{\Sigma}_{2,2}
        \end{bmatrix}
    \]
    then the posterior conditional is given by
    \begin{align*}
        \bm{x}_2 \mid \bm{x}_1 & \sim \calN \left( \bm{x}_2 \mid \bm{\mu}_{2 \mid 1}, \bm{\Sigma}_{2 \mid 1} \right)          \\
        \bm{\mu}_{2 \mid 1}    & = \bm{\mu}_2 + \bm{\Sigma}_{2,1} \bm{\Sigma}_{1,1}^{-1} \left( \bm{x}_1 - \bm{\mu}_1 \right) \\
        \bm{\Sigma}_{2 \mid 1} & = \bm{\Sigma}_{2,2} - \bm{\Sigma}_{2,1} \bm{\Sigma}_{1,1}^{-1} \bm{\Sigma}_{1,2}
    \end{align*}
\end{thm}

Thus finding a mean an covariance for $\bm{f}_{\ast}$ requires a direct application of Theorem \ref{theorem: cond_of_MVN} which gives
\begin{align*}
    \bm{f}_{\ast} \mid \bm{K_{XX^{\ast}}} , \bm{K_{XX}}, \bm{f} \sim \calN \left( \bm{\mu}^{\ast}, \bm{\Sigma}^{\ast} \right)
\end{align*}
where
\begin{align*}
    \bm{\mu}^{\ast} & = \bm{0} + \bm{K_{XX^{\ast}}}^{\intercal} \bm{K_{XX}}^{-1} \left( \bm{f} - \bm{0} \right) \\
                    & = \bm{K_{XX^{\ast}}}^{\intercal} \bm{K_{XX}}^{-1} \bm{f}
\end{align*}
and
\begin{align*}
    \bm{\Sigma}^{\ast} & = \bm{K_{X^{\ast}X^{\ast}}} - \bm{K_{XX^{\ast}}}^{\intercal} \bm{K_{XX}}^{-1} \bm{K_{XX^{\ast}}}
\end{align*}
meaning we can write a distribution for $\bm{f}_{\ast}$ as
\begin{equation}\label{prop:GP_train_distr1}
    \bm{f}_{\ast} \mid \bm{K_{XX^{\ast}}} , \bm{K_{XX}}, \bm{f} \sim \calN \left( \bm{K_{XX^{\ast}}}^{\intercal} \bm{K_{XX}}^{-1} \bm{f},  \bm{K_{X^{\ast}X^{\ast}}} - \bm{K_{XX^{\ast}}}^{\intercal} \bm{K_{XX}}^{-1} \bm{K_{XX^{\ast}}}  \right)
\end{equation}
Function values from the unobserved inputs $\bm{X^{\ast}}$ can be estimated using the mean of $\bm{f}_{\ast}$ evaluted in \ref{prop:GP_train_distr1}.

\subsubsection{Prediction with Noisy observations}\label{Section1.1.2}
When attempting to model our value function we usually do not have access to the value function itself but a noisy version thereof, $y = f(\bm{x}) + \varepsilon$ where $\varepsilon \calN (0, \sigma_n^2)$ meaning the prior on the noisy values becomes
\[
    \operatorname{cov} (\bm{y}) = \bm{K_{XX}} + \sigma_n^2 \bm{I}
\]
The reason why noise is only added along the diagonal follows from the assumption of independence in our data.
We can write out the new distribution of the observed noisy values along the points at which we wish to test the underlying function as
\[
    \begin{bmatrix}
        \bm{f} \\
        \bm{f}_{\ast}
    \end{bmatrix}
    \sim \calN
    \begin{bmatrix}
        \bm{0}, &
        {
                \begin{bmatrix}
                    \bm{K_{XX}} + \sigma_n^2 \bm{I} & \bm{K_{XX^{\ast}}}        \\
                    \bm{K_{XX^{\ast}}}^{\intercal}  & \bm{K_{X^{\ast}X^{\ast}}}
                \end{bmatrix}
            }
    \end{bmatrix}
\]
Using a similar we arrive at a similar condition distribution of $\bm{f}_{\ast} \mid \bm{K_{XX^{\ast}}} , \bm{K_{XX}}, \bm{f}$ we arrive at one of the most fundamental equations for GP regression tasks
\begin{align*}\label{prop:GP_train_distr2}
    \bm{f}_{\ast} \mid \bm{K_{XX^{\ast}}} , \bm{K_{XX}}, \bm{y} \sim & \calN \left( \overline{\bm{f}_{\ast}}, \operatorname{cov} (\bm{f}_{\ast}) \right)                                                   \\
    \overline{\bm{f}_{\ast}}                                         & := \bm{K_{XX^{\ast}}}^{\intercal} \left[ \bm{K_{XX}} + \sigma_n^2 \bm{I} \right]^{-1} \bm{y}                                        \\
    \operatorname{cov} (\bm{f}_{\ast})                               & = \bm{K_{X^{\ast}X^{\ast}}} - \bm{K_{XX^{\ast}}}^{\intercal} \left[ \bm{K_{XX}} + \sigma_n^2 \bm{I} \right]^{-1} \bm{K_{XX^{\ast}}}
\end{align*}

%%%%%%%%%%%%%%%%%%%%%%%%%%%%%%%%%%%%%%%%%%%%%%%%%%%%%%%%%%%%%%%%%%%%%%%%%%%%%%%%%%%%%%%%%%%%%%

\subsection{The induced representation $\Ind_K^G \1$}\label{Section1.2}
Consider the space of functions in $\Fun(G)$ that are invariant under right-multiplication by elements of $K$.
Explicitly, this space is defined by
\[
    W: = \{f\colon G\to \CC\ |\ f(gk) = f(g),\ \forall g\in G, \forall k \in K\} \subseteq \Fun(G).
\]
Note that the action of $G$ on $\Fun(G)$ leaves $W$ invariant.
The resulting action of $G$ on $W$ is called the \emph{induced representation} and denoted $\Ind_K^G \1$.
When $K=\{1\}$, the representation $\Ind_{\{1\}}^G\1=\Fun(G)$ is the \emph{left regular} representation of $G$.
For future use, we prove the following lemma.
\begin{lem}\label{lemma: W_left_ideal}
    The space $W$ is a left ideal of $(\Fun(G),\star)$.
\end{lem}
\begin{proof}
    We verify that $f\star w\in W$ whenever $w\in W$ and $f\in\Fun(G)$.
    Let $g\in G$ and $k\in K$.
    Then
    \begin{multline*}
        (f\star w)(gk) = \sum_{xy=gk} f(x)w(y) = \sum_{x\in G} f(x)w(x^{-1}gk) \\
        = \sum_{x\in G} f(x)w(x^{-1}g) = \sum_{xy=g} f(x)w(y) = (f\star w)(g).\qedhere
    \end{multline*}
\end{proof}

%%%%%%%%%%%%%%%%%%%%%%%%%%%%%%%%%%%%%%%%%%%%%%%%%%%%%%%%%%%%%%%%%%%%%%%%%%%%%%%%%%%%%%%%%%%%%%

\subsection{The Hecke algebra of a finite group $\calH(G,K)$}\label{Section1.3}
Consider the space of functions in $\Fun(G)$ that are invariant under right- and left-multiplication by elements of $K$.
Explicitly, this space is defined by
\[
    \calH(G,K) := \{f\colon G\to \CC\ |\ f(k_1gk_2) = f(g),\ \forall g\in G,\ \forall k_1,k_2\in K\} \subseteq \Fun(G).
\]
This is the \emph{Hecke algebra} associated to $G$ and $K$ and we will write $\calH$ to mean $\calH(G,K)$ when there is no ambiguity.
The proof of Lemma \ref{lemma: W_left_ideal} can be adapted to show that $\calH$ is a two-sided ideal in $(\Fun(G),\star)$.
Notice that the identity of $(\Fun(G),\star)$ does not lie in $\calH$.
Nevertheless, $\calH$ does have an identity of its own.
It is easy to verify that the identity is $\iota_K$, which we define below.
\[
    \iota_K:G\to\CC,\quad \iota_K(g) := \begin{cases}
        \frac{1}{|K|},\  & \text{if}\ g\in K, \\
        0,\              & \text{else}.
    \end{cases}
\]
We see that $\iota_K$ is an idempotent element, since $(\iota_K\star\iota_K)(g)=0$ for $g\notin K$, and
\[
    (\iota_K\star\iota_K)(k) = \sum_{x\in G} \iota_K(kx)\iota_K(x^{-1}) = \sum_{x\in K} \frac{1}{|K|^2} = \frac{1}{|K|}
\]
for $k\in K$.

This is a special case of a more general situation: if $R$ is a ring and $e$ is an idempotent, then $eRe$ will be a ring in which $e$ serves as a unit.
This is clear since $eree=ere=eere$ for all $ere\in eRe$.
The ring $eRe$ is sometimes called an \emph{idempotented ring} or a \emph{Pierce corner} \cite{Bump10,Lam06}.

We present a basis for $\calH$.
For $KxK\in K\backslash G/K$, the $K$-double cosets in $G$, we define
\[
    \chi_{KxK}(y) := \begin{cases}
        1,\  & \text{if}\ y\in KxK, \\
        0,\  & \text{else}.
    \end{cases}
\]
Recall that double cosets partition $G$ so there is no ambiguity in this definition.
We call $\chi_{KxK}$ the \emph{characteristic function} of the $K$-double coset $KxK$.
As an abuse of notation for the sake of brevity, we will denote this family by $\{\chi_x\}_{x\in G}$, where $x$ ranges over the $K$-double coset representatives as written above.

It is not hard to see that the characteristic functions form a basis of $\calH$.
By the definition of $\calH$, we see characteristic functions span the space.
To see that they're linearly independent, assume that
\[
    \alpha_1 \chi_{x_1} + \cdots + \alpha_n \chi_{x_n}  = 0,
\]
for some complete collection of $K$-double coset representatives $x_i\in G$ and scalars $\alpha_i\in\CC$.
Here $0$ denotes the zero function $g\mapsto 0$ for all $g\in G$.
Evaluating both sides at $x_i$ tells us that $\alpha_i=0$, so the only solution is the trivial solution and we have linear independence.

%%%%%%%%%%%%%%%%%%%%%%%%%%%%%%%%%%%%%%%%%%%%%%%%%%%%%%%%%%%%%%%%%%%%%%%%%%%%%%%%%%%%%%%%%%%%%%

\subsection{The group algebra $\CC[G]$}\label{Section1.4}
We can associate to $G$ another algebra, $\CC[G]$, called the \emph{group algebra} of $G$ over $\CC$.
This algebra is defined by
\[
    \CC[G] := \Bigg\{ \sum_{g\in G} a_g e_g\ \Bigg|\ a_g\in \CC \Bigg\}.
\]
Clearly, the set $\{e_g\}_{g\in G}$ serves as a basis of this space.
We endow the space with a multiplication defined on basis elements by $e_ge_h := e_{gh}$.
The following lemma illustrates the relevance of the group algebra.
\begin{lem}
    The map $\Phi\colon \Fun(G)\to\CC[G]$ defined on basis elements by $\delta_g\mapsto e_g$ and extended linearly is an algebra isomorphism.
\end{lem}
\begin{proof}
    By construction, $\Phi$ is a linear map of vector spaces.
    It is also clear that this map is bijective since it is a bijection on basis elements.
    Thus $\Phi$ is a vector space isomorphism.

    We need to check that $\Phi$ respects the algebra multiplication.
    This amounts to verifying that $\delta_g \star \delta_h = \delta_{gh}$.
    Notice that $(\delta_g\star\delta_h)(x) = \sum_{ab=x} \delta_g(a)\delta_h(b)$ is equal to $1$ when $g=a$ and $h=b$, and $0$ otherwise.
    This is exactly $\delta_{gh}(x)$.
\end{proof}
We may ask ourselves: what is the image of the induced representation and the Hecke algebra inside of the group algebra? To answer this, we define the group algebra element
\[
    e := \frac{1}{|K|} \sum_{k\in K} e_k.
\]
Note that $e$ is an idempotent element.
Then the following proposition answers our question.
\begin{prop}
    \begin{enumerate}[\itshape(i)]
        \item $\Phi(W) = \CC[G]e$.
        \item $\Phi(\calH) = e \CC[G] e$.
    \end{enumerate}
\end{prop}
\begin{proof}
    \begin{enumerate}[\itshape(i)]
        \item We begin by showing that $\CC[G]e\subseteq \Phi(W)$.
              To see this, take an arbitrary element $(\sum_{g\in G} a_ge_g)e$ in $\CC[G]e$.
              Then notice
              \[
                  \bigg(\sum_{g\in G} a_ge_g\bigg)e = \frac{1}{|K|}\bigg(\sum_{g\in G} a_ge_g\bigg)\bigg(\sum_{k\in K}e_k\bigg) = \frac{1}{|K|}\sum_{\substack{g\in G \\ k\in K}} a_ge_ge_k = \frac{1}{|K|}\sum_{\substack{g\in G \\ k\in K}} a_ge_{gk}.
              \]
              Then we apply $\Phi^{-1}$ to see that
              \[
                  \frac{1}{|K|}\sum_{\substack{g\in G \\ k\in K}} a_ge_{gk} \mapsto \frac{1}{|K|}\sum_{\substack{g\in G \\ k\in K}} a_g\delta_{gk}.
              \]
              We wish to show that this lies in $W$, so we wish to check that this map is invariant under right-multiplication by an element of $K$.
              To this end, let $g'\in G$, $k'\in K$ and apply $\frac{1}{|K|}\sum_{\substack{g\in G \\ k\in K}} a_g\delta_{gk}$ to $g'k'$.
              Note that $\delta_{gk}(g'k')=1$ if and only if $gk=g'k'$ (and $0$ otherwise).
              This is equivalent to $g=g'k'k^{-1}$.
              Thus
              \[
                  \frac{1}{|K|}\sum_{\substack{g\in G \\ k\in K}} a_g\delta_{gk}(g'k') = \frac{1}{|K|}\sum_{k\in K} a_{g'k'k^{-1}}\delta_{g'k'}(g'k') = \frac{1}{|K|}\sum_{k\in K} a_{g'k'k^{-1}}.
              \]
              Similarly, we apply the map $\frac{1}{|K|}\sum_{\substack{g\in G \\ k\in K}} a_g\delta_{gk}$ to $g'$.
              This yields
              \[
                  \frac{1}{|K|}\sum_{\substack{g\in G \\ k\in K}} a_g\delta_{gk}(g') = \frac{1}{|K|}\sum_{k\in K} a_{g'k^{-1}}
              \]
              Since right-multiplication by any element of $K$ is an automorphism of $G$, we see that
              \[
                  \frac{1}{|K|}\sum_{k\in K} a_{g'k'k^{-1}}=\frac{1}{|K|}\sum_{k\in K} a_{g'k^{-1}},
              \]
              which shows that $\CC[G]e\subseteq \Phi(W)$.
              Conversely, take $f=\sum_{g\in G} a_g\delta_g\in W$.
              Let $g'\in G$, $k'\in K$ and notice that
              \[
                  a_{g'k'} = \sum_{g\in G} a_g\delta_g(g'k') = f(g'k') = f(g') = \sum_{g\in G} a_g \delta_g(g') = a_{g'}.
              \]
              Then $a_{g'k'}=a_{g'}$ for any $g'\in G$ and $k'\in K$.
              Then observe
              \begin{multline*}
                  \Phi(f)e = \bigg(\sum_{g\in G} a_g\delta_g\bigg)\bigg(\frac{1}{|K|}\sum_{k\in K} e_k\bigg) = \frac{1}{|K|} \sum_{\substack{g\in G \\ k\in K}} a_g e_{gk} = \frac{1}{|K|} \sum_{\substack{g\in G \\ k\in K}} a_{gk^{-1}} e_g \\
                  = \frac{1}{|K|} \sum_{\substack{g\in G \\ k\in K}} a_g e_g = \frac{1}{|K|}\sum_{k\in K}\sum_{g\in G} a_ge_g = \frac{1}{|K|}\sum_{k\in K} \Phi(f) = \Phi(f).
              \end{multline*}
              Then $\varphi(f)=\varphi(f)e\in\CC[G]e$, so $\Phi(W)\subseteq \CC[G]e$ as required.

        \item The proof is similar to that of {\itshape(i)}.\qedhere
    \end{enumerate}
\end{proof}

%%%%%%%%%%%%%%%%%%%%%%%%%%%%%%%%%%%%%%%%%%%%%%%%%%%%%%%%%%%%%%%%%%%%%%%%%%%%%%%%%%%%%%%%%%%%%%

\subsection{Identifying $\calH(G,K)$ with the endomorphism algebra $\End_G(W)$}\label{Section1.5}
For any representation $V$ of $G$, define the space of \emph{$G$-intertwining endomorphisms on $V$} by
\[
    \End_G(V) := \{ f\in\End(V)\ |\ g\cdot f(v) = f(g\cdot v),\ \forall\ v\in V,\ g\in G \}\subseteq \End(V).
\]
These are the endomorphisms of $V$ that respect the action of $G$ on $V$.
It is easy to see that this is a vector space.
It has the additional structure of a unital associative algebra when endowed with the product of endomorphism composition.

Now set $V$ to be $W$, the induced representation of the trivial character from $K$ to $G$, and define the linear map
\[
    \Psi\colon \calH\to\End(W),\quad \alpha\mapsto(w\mapsto w\star\alpha).
\]
Lemma \ref{lemma: W_left_ideal} tells us that $w \star\alpha$ is indeed an element of $W$ so the image of $\Psi$ is indeed $\End(W)$.
The following proposition highlights the significance of this map.
\begin{prop}\label{prop: H_iso_End_G(W)}
    The map $\Psi$ defines an algebra isomorphism $\calH \cong \End_G(W)$.
\end{prop}
\begin{proof}
    First we observe that $\Psi(\alpha)$ is indeed a $G$-intertwiner.
    Given $g,h\in G$ and $w\in W$, we have
    \begin{multline*}
        (\Psi(\alpha)(g\cdot w))(h) = ((g\cdot w)\star\alpha)(h) = \sum_{xy=h} w(g^{-1}x)\alpha(y) = \sum_{x\in G} w(g^{-1}x)\alpha(x^{-1}h) \\
        = \sum_{ab=g^{-1}h} w(a)\alpha(b) = (g\cdot(w\star\alpha))(h) = (g\cdot \Psi(\alpha)(w))(h).
    \end{multline*}
    Thus, the image of $\Psi$ lies in $\End_G(W)$.
    Next, we check that $\Psi$ is an algebra isomorphism.
    Let $\alpha_1,\alpha_2\in\calH$ and observe
    \[
        \Psi(\alpha_1\star\alpha_2)(w) = w\star(\alpha_1\star\alpha_2) = (w\star\alpha_1)\star\alpha_2 = \Psi(\alpha_1)(w)\star\alpha_2 = (\Psi(\alpha_1)\circ\Psi(\alpha_2))(w).
    \]
    Thus $\Psi$ is an algebra homomorphism.
    To see that $\Psi$ is injective, we compute
    \[
        \ker \Psi = \{\alpha\in\calH\ |\ \Psi(\alpha)(w) = w\} = \{\alpha\in\calH\ |\ w\star\alpha = w\} = \{\delta_{1_G}\}.
    \]
    We see that $\Psi$ has trivial kernel so it is injective.
    It is easy to see that surjectivity is a consequence of Theorem 13 in \cite{Murnaghan05} which also contains its proof.
\end{proof}

%%%%%%%%%%%%%%%%%%%%%%%%%%%%%%%%%%%%%%%%%%%%%%%%%%%%%%%%%%%%%%%%%%%%%%%%%%%%%%%%%%%%%%%%%%%%%%

\subsection{Consequences for representation theory}\label{Section1.6}
We prove a general property of representations.
Namely, the decomposition of a representation is linked to its corresponding algebra of $G$-intertwining endomorphisms.
We apply this to the induced representation $W$ and Proposition \ref{prop: H_iso_End_G(W)} lets us conclude that $W$ is multiplicity-free if and only if $\calH$ is commutative.

First, suppose that $V$ is a complex representation of $G$.
Write $V = \bigoplus_{i=1}^n V_i$ as the decomposition of $V$ into irreducible constituents, using Maschke's theorem.
Notice that some of these $V_i$ may be isomorphic to each other as $G$-representations.
We group these mutually isomorphic irreducible representations together by writing
\[
    V = \bigoplus_{i=1}^n V_i = \bigoplus_{i=1}^n U_i^{\oplus m_i},
\]
where $m_i$ is the number of times $U_i$ appears in the decomposition of $V$, henceforth referred to as the \emph{multiplicity} of $U_i$ in $V$.
We say $V$ is \emph{multiplicity-free} if $m_i=1$ for all $i$.
The $U_i^{\oplus m_i}$ are called the \emph{isotypical components} of $V$.
We now prove the main proposition of this section.
\begin{prop}\label{prop: V_mult-free_iff_End_G(V)_commutative}
    \begin{enumerate}[\itshape(i)]
        \item If $V$ is a representation of $G$ with the decomposition into isotypical components as above, then $\End_G(V)\cong \bigoplus_{i=1}^n \Mat_{m_i}(\CC)$.
        \item $V$ is multiplicity-free if and only if $\End_G(V)$ is commutative.
    \end{enumerate}
\end{prop}
\begin{proof}
    \begin{enumerate}[\itshape(i)]
        \item Observe that
              \[
                  \End_G(V) = \Hom_G(V_1\oplus \cdots \oplus V_n, V_1\oplus \cdots \oplus V_n) \cong \bigoplus_{i,j=1,\ldots,n} \Hom_G(V_i, V_j).
              \]
              Then we compute
              \[
                  \Hom_G(V_i,V_j) = \Hom_G(U_i^{\oplus m_i}, U_j^{\oplus m_j}) \cong \Hom_G(U_i,U_j)^{\oplus m_im_j}.
              \]
              Schur's lemma tells us that
              \[
                  \Hom_G(U_i,U_j) \cong \begin{cases}
                      \CC,\    & \text{if}\ U_i\cong U_j,     \\
                      \{0\},\  & \text{if}\ U_i\not\cong U_j.
                  \end{cases}
              \]
              Then $\Hom_G(U_i,U_j)^{\oplus m_im_j} =\{0\}$ if $i\neq j$ and
              \[
                  \Hom_G(U_i,U_i)^{\oplus m_i^2} \cong \CC^{m_i^2} \cong \Mat_{m_i}(\CC).
              \]
              Thus $\End_G(V) \cong \bigoplus_{i=1}^n \Mat_{n_i}(\CC)$.

        \item We know from $(i)$ that we can identify $\End_G(V)$ with an algebra of block-diagonal matrices over $\CC$.
              The sizes of the blocks correspond to $m_i$, the multiplicity of $U_i$ in $V$.
              Composing two $f,g\in\End_G(V)$ corresponds to multiplying their associated matrices.
              Then $\End_G(V)$ is commutative if and only if the block sizes are all $1$.
              That is, if $m_i = 1$ for all $i$. \qedhere
    \end{enumerate}
\end{proof}
\begin{cor}\label{cor: H_commutative}
    \begin{enumerate}[\itshape(i)]
        \item The induced representation $W$ is multiplicity-free if and only if its associated Hecke algebra $\calH$ is commutative.
        \item $W$ is irreducible if and only if $\calH \cong \CC$.
    \end{enumerate}
\end{cor}
\begin{proof}
    \begin{enumerate}[\itshape(i)]
        \item Apply Proposition \ref{prop: V_mult-free_iff_End_G(V)_commutative} with $V=W$.
              Then $W$ is multiplicity-free if and only if $\End_G(W)$ is commutative.
              Proposition \ref{prop: H_iso_End_G(W)} tells us that $\End_G(W)\cong\calH$.
              Thus $W$ is multiplicity-free if and only if $\calH$ is commutative.

        \item Suppose that $W$ is irreducible.
              Schur's Lemma tells us that $\End_G(W)\cong\CC$, so $\calH\cong\CC$.
              Conversely, suppose that $\calH\cong\CC$.
              Write the decomposition of $W$ into irreducible constituents
              \[
                  W = \bigoplus_{i=1}^n W_i.
              \]
              Schur's lemma tells us that $\End_G(W_i)\cong \CC$ for each $i$.
              Then
              \[
                  \End_G(W) = \End_G\bigg(\bigoplus_{i=1}^n W_i\bigg) \cong \bigoplus_{i=1}^n \End_G(W_i) \cong \bigoplus_{i=1}^n \CC = \CC^n.
              \]
              However $\CC\cong\calH\cong\End_G(W)\cong\CC^n$.
              Thus $n=1$ and $W$ is irreducible. \qedhere
    \end{enumerate}
\end{proof}

%%%%%%%%%%%%%%%%%%%%%%%%%%%%%%%%%%%%%%%%%%%%%%%%%%%%%%%%%%%%%%%%%%%%%%%%%%%%%%%%%%%%%%%%%%%%%%

\subsection{Gelfand's Trick}\label{Section1.7}
Our goal in this section is to prove the following theorem.
\begin{thm}[Gelfand's Trick] \label{theorem: Gelfand's_Trick}
    Suppose that $G$ is a finite group and $K\leq G$ is a subgroup.
    Let $\varphi\colon G\to G$ be an anti-automorphism with
    \begin{enumerate}[(i)]
        \item $\varphi^2=1$, and
        \item $K\varphi(x)K=KxK$ for all $x\in G$.
    \end{enumerate}
    Then $\calH(G,K)$ is commutative.
\end{thm}
The key idea of this theorem is the following lemma.
\begin{lem}\label{lemma: subalgebra_commutative}
    Let $A$ be an algebra and $B\subseteq A$ be a subalgebra with basis $\{b_i\}_{i\in I}$.
    Suppose $F\colon A\to A$ is an anti-homomorphism (i.e.\ $F(a_1a_2)=F(a_2)F(a_1)$) and $F(b_i) = b_i$.
    Then $B$ is commutative.
\end{lem}
\begin{proof}
    Since $F$ is the identity on basis elements of $B$, there holds $F|_B = \Id_B$.
    Let $b_i,b_j\in B$ be basis elements and notice
    \[
        b_ib_j = F(b_ib_j) = F(b_j)F(b_i) = b_jb_i.
    \]
    Then basis elements of $B$ commute as desired.
\end{proof}
We employ Lemma \ref{lemma: subalgebra_commutative} by applying it to the case where $A=\Fun(G)$ and $B=\calH(G,K)$. Recall from Section \ref{Section1.3} that the characteristic functions $\{\chi_x\}_{x\in G}$ form a basis of $\calH(G,K)$.
\begin{cor}\label{cor: comm}
    Suppose $F\colon \Fun(G) \to \Fun(G)$ is an anti-homomorphism such that $F(\chi_x) = \chi_x$ for all $x\in X$.
    Then $\calH(G,K)$ is commutative.
\end{cor}
This gives us a clear direction going forward: we want to find such a map $F$.

Given an anti-homomorphism of groups $\varphi\colon G\to G$, we can consider the map $\varphi^\ast\colon\Fun(G)\to\Fun(G)$ defined by $\varphi^\ast f := f\circ \varphi$.
This is the \emph{pullback} of $f$ by $\varphi$.
In general, $\varphi^\ast$ is not an anti-homomorphism of convolution algebras.
For instance, consider $G=\ZZ/2\ZZ=\{0,1\}$ and the map $\varphi(x)=x+x=0$.
Clearly $\varphi$ is an anti-homomorphism.
However, consider the maps $f,g\in\Fun(G)$ given by $f(x)=g(x)=0$ if $x=0$ and $f(x)=g(x)=1$ if $x=1$.
Then
\[
    (\varphi^\ast(f\star g))(0) = \sum_{x+y = \varphi(0)} f(x)g(y) = \sum_{x+y = 0} f(x)g(y) = f(0)g(0) + f(1)g(1) = 1,
\]
\[
    ((\varphi^\ast g)\star(\varphi^\ast f))(0) = \sum_{x+y = 0} g(\varphi(x))f(\varphi(y)) = \sum_{x+y = 0} g(0)f(0) = 2g(0)f(0) = 0.
\]
Thus $\varphi^\ast$ is not an anti-homomorphism.
However, when $\varphi$ has the stronger anti-automorphism property, we can say the same for $\varphi^\ast$.
More precisely, we have the following lemma.
\begin{lem}
    Suppose $\varphi\colon G\to G$ is a group anti-automorphism.
    Then $\varphi^\ast\colon\Fun(G)\to\Fun(G)$ is an algebra anti-automorphism.
\end{lem}
\begin{proof}
    Let $\varphi$ be a group anti-automorphism.
    Thus $\varphi$ is a bijection and an anti-homomorphism.
    This lets us write $yz=x \iff \varphi(yz)=\varphi(x)$ since $\varphi$ is a bijection.
    We can also write $\varphi(yz)=\varphi(x) \iff \varphi(z)\varphi(y)=\varphi(x)$ since $\varphi$ is an anti-homomorphism.
    Then we compute
    \begin{multline*}
        ((\varphi^\ast f)\star(\varphi^\ast g))(x) = \sum_{yz=x} (\varphi^\ast f)(y)(\varphi^\ast g)(z) = \sum_{yz=x} f(\varphi(y)) g(\varphi(z)) = \\
        \sum_{\varphi(z)\varphi(y)=\varphi(x)} g(\varphi(z))f(\varphi(y)) = \sum_{z'y'=\varphi(x)} g(z')f(y') = (\varphi^\ast(g\star f))(x).
    \end{multline*}
    Thus $\varphi^\ast(g\star f) = (\varphi^\ast f)\star(\varphi^\ast g)$.
    We also need to check that $\varphi^\ast$ is a bijection.
    We check this on the basis elements $\{\delta_g\}_{g\in G}$ of $\Fun(G)$.
    Let $g,h\in G$ and we compute
    \[
        (\varphi^\ast\delta_g)(h) = \begin{cases}
            1,\  & \text{if}\ g=\varphi(h), \\
            0,\  & \text{else}.
        \end{cases} = \begin{cases}
            1,\  & \text{if}\ h=\varphi^{-1}(g), \\
            0,\  & \text{else}.
        \end{cases} = \delta_{\varphi^{-1}(g)}(h).
    \]
    We see that $\varphi^\ast$ sends $\delta_g$ to $\delta_{\varphi^{-1}(g)}$.
    We know $\varphi$ and $\varphi^{-1}$ are bijections on $G$, so $\varphi^\ast$ acts bijectively on the basis of $\Fun(G)$.
\end{proof}
Now we know that an anti-automorphism $\varphi$ of $G$ induces an anti-automorphism $\varphi^\ast$ of $\Fun(G)$.
We ask ourselves: when does this anti-automorphism restrict to an anti-automorphism of $\calH(G,K)$?
That is, when is $\varphi^\ast$ also an anti-automorphism of $\calH(G,K)$? The following lemma provides an answer.
\begin{lem}
    Suppose that $\varphi\colon G\to G$ is an anti-automorphism.
    If $\varphi(K)=K$ then $\varphi^\ast$ restricts to an anti-automorphism of $\calH(G,K)$.
\end{lem}
\begin{proof}
    Suppose $f\in\calH$.
    Then notice
    \[
        (\varphi^\ast f)(k_1gk_2) = f(\varphi(k_1gk_2)) = f(\varphi(k_2)\varphi(g)\varphi(k_1)) = f(k_2'\varphi(g)k_1') = f(\varphi(g)) = (\varphi^\ast f)(g).
    \]
    Thus $\varphi^\ast f \in\calH$ since it's constant on $K$-double cosets.
\end{proof}
Now we explore the effect of $\varphi^\ast$ on the basis elements $\{\chi_x\}_{x\in G}$ of $\calH(G,K)$.
\begin{lem}\label{lem: id_on_basis}
    Suppose $\varphi\colon G\to G$ is an anti-automorphism.
    If $\varphi^2=1$ and $K\varphi(x)K=KxK$ for all $x\in G$, then $\varphi^\ast\chi_x=\chi_x$.
\end{lem}
Before we present the proof, notice that $\varphi(K)=K$ is a consequence of the assumption that $K\varphi(x)K=KxK$ for all $x\in G$.
This assumption implies that $K\varphi(x)K=KxK$ for all $x\in K$, which in turn implies that $\varphi(K)=K$.
\begin{proof}
    First, if $g\in KxK$, then
    \[
        \varphi(g)\in \varphi(KxK) = \varphi(K)\varphi(x)\varphi(K) = K\varphi(x)K = KxK.
    \]
    On the other hand, if $\varphi(g)\in KxK$, then
    \[
        g = \varphi(\varphi(g)) \in \varphi(KxK) = \varphi(K)\varphi(x)\varphi(K) = K\varphi(x)K = KxK.
    \]
    We see that $g\in KxK$ if and only if $\varphi(g)\in KxK$.
    Then we compute
    \[
        (\varphi^\ast\chi_x)(g) = \chi_x(\varphi(g)) = \begin{cases}
            1,\  & \text{if}\ \varphi(g)\in KxK, \\
            0,\  & \text{else}.
        \end{cases} = \begin{cases}
            1,\  & \text{if}\ g\in KxK, \\
            0,\  & \text{else}.
        \end{cases} = \chi_x(g).\qedhere
    \]
\end{proof}
We are now ready to prove Theorem \ref{theorem: Gelfand's_Trick}.
\begin{proof}[Proof of Theorem \ref{theorem: Gelfand's_Trick}]
    Lemma \ref{lem: id_on_basis} tells us that $\varphi^\ast$ is the identity on the characteristic functions $\chi_x$.
    These are the basis elements of $\calH(G,K)$.
    Since $\varphi$ is an anti-automorphism, $\varphi^\ast$ will be too.
    We apply Corollary \ref{cor: comm} with $F=\varphi^\ast$ to see that the basis elements commute.
    Thus $\calH(G,K)$ is commutative.
\end{proof}
When applying Gelfand's Trick, we will often consider $\varphi(x)=x^{-1}$ or $\varphi(x)=x^t$ (the latter of which is understood as the transpose map when $G$ is a matrix group).
It is easy to see that they are both involutive anti-automorphisms, so the condition $K\varphi(x)K=KxK$ for all $x\in G$ will be the only condition left to verify.

%%%%%%%%%%%%%%%%%%%%%%%%%%%%%%%%%%%%%%%%%%%%%%%%%%%%%%%%%%%%%%%%%%%%%%%%%%%%%%%%%%%%%%%%%%%%%%

\subsection{Gelfand pairs}\label{Section1.8}
We say that a pair of groups $(G,K)$ with $K\leq G$ is a \emph{Gelfand pair} if $\Ind_K^G \1$ is multiplicity-free.
To be a Gelfand pair, it is sufficient to find an anti-automorphism satisfying the conditions of Theorem \ref{theorem: Gelfand's_Trick}.
We present some examples of applications of this technique.

\subsubsection{Example: $(G,K)$ with $G$ abelian}
For any abelian group $G$, the identity map $\varphi(g)=g$ is an anti-automorphism.
This map clearly satisfies $\varphi^2=1$ and $K\varphi(x)K=KxK$ for all $x\in G$.

\subsubsection{Example: $(G,K)$ with $[G\! :\! K]=2$}
The condition $[G\! :\! K]=2$ tells us that $K$ is a normal subgroup of $G$.
Thus, the quotient group $G/K$ is defined and contains two cosets, $K$ and $G- K$.
Consider the involutive anti-automorphism $\varphi(g)=g^{-1}$.
We verify that double cosets are preserved.
If $x\in K$, then $K\varphi(x)K = Kx^{-1}K = K = KxK$.
On the other hand, if $x\in G- K$, then $K\varphi(x)K = Kx^{-1}K = G\setminus K = KxK$.
We see that $K\varphi(x)K=KxK$ in all cases.

\subsubsection{Example: $(G\times G,G)$}
We can embed the group $G$ inside $G\times G$ by the injective map $g\mapsto (g,g)$.
Then it makes sense to consider $G$ as a subgroup of $G\times G$.
We apply Gelfand's Trick with the involutive anti-automorphism $\varphi(g_1,g_2)=(g_1,g_2)^{-1}=(g_1^{-1},g_2^{-1})$.
There holds
\begin{multline*}
    G\varphi(g_1,g_2)G = \{(hg_1^{-1}k,hg_2^{-1}k)\ |\ h,k\in G\} \\
    = \{(k^{-1}g_1h^{-1},k^{-1}g_2h^{-1})^{-1}\ |\ h,k\in G\} = \{(xg_1y,xg_2y)\ |\ x,y\in G\} = G(g_1,g_2)G.
\end{multline*}
We see that $\varphi$ preserves double cosets and we have a Gelfand pair.


\subsubsection{Example: $(S_{n+m},S_n\times S_m)$}
We present an original proof, but one may also see \cite{Bump13} for an alternate proof.
The group $S_n\times S_m$ can be embedded inside $S_{n+m}$ by taking $w=(w_1,w_2)\in S_n\times S_m$ and forming an element of $S_{n+m}$ by having $w_1$ act on the first $n$ elements of $\{1,2,\ldots,n+m\}$ and having $w_2$ act on the last $m$ elements of $\{1,2,\ldots,n+m\}$.

Consider the involutive anti-automorphism $\varphi(w)=w^{-1}$.
We must verify that $K\varphi(w)K = KwK$ for each double coset.
If $w\in K$, then $K\varphi(w)K=Kw^{-1}K=K=KwK$ so all that is left is to verify double cosets are preserved for $w\in G-K$.

We wish to show that $Kw^{-1}K\subseteq KwK$ and $KwK \subseteq Kw^{-1}K$.
Note that it suffices to show only one of these.
We will show that $Kw^{-1}K\subseteq KwK$.
Again, note that it suffices to show that $w^{-1} \in KwK$.
This is equivalent to showing that $w^{-1} = k_1wk_2$ for some $k_1,k_2\in K$.
This equation is equivalent to $k_2^{-1} = wk_1w$.
Then it suffices to show that $wkw\in K$ for some $k\in K$.

We call $i\in\{1,\ldots,n+m\}$ a \emph{crossing point} of $w$ if one of two mutually exclusive conditions hold: $i\in\{1,\ldots,n\}$ and $w(i)\in\{n+1,\ldots,n+m\}$, or $i\in\{n+1,\ldots,n+m\}$ and $w(i)\in\{1,\ldots,n\}$.
Notice that the number of crossing points in $\{1,\ldots,n\}$ must equal the number of crossing points in $\{n+1,\ldots,n+m\}$ since $w$ is a bijection.
Then there is a bijection $f\colon \{\text{crossing points}\leq n\} \to \{\text{crossing points}>n\}$.
This yields two other bijections $g\colon \{1,\ldots,n\}-\{\text{crossing points}\leq n\} \to \{1,\ldots,n\}-w(\{\text{crossing points}>n\})$ and $h\colon \{n+1,\ldots,n+m\}-\{\text{crossing points}>n\} \to \{n+1,\ldots,n+m\} - w(\{\text{crossing points}\leq n\})$. Define $k\in S_{n+m}$ by
\[
    k(w(i)) := \begin{cases}
        f(i),\       & \text{if}\ i\leq n\ \text{is a crossing point},     \\
        f^{-1}(i),\  & \text{if}\ i>n\ \text{is a crossing point},         \\
        g(i),\       & \text{if}\ i\leq n\ \text{is not a crossing point}, \\
        h(i),\       & \text{if}\ i> n\ \text{is not a crossing point}.
    \end{cases}
\]
It is easy to check that $k$ and $wkw$ lie in $K$ as desired.

\subsubsection{Example: ($\mO_{n+1}(\FF_q),\mO_n(\FF_q))$ with $q\neq 2^k$}
We can embed the group $\mO_n(\FF_q)$ inside $\mO_{n+1}(\FF_q)$ by the injection
\[
    \mO_n(\FF_q) \hookrightarrow \mO_{n+1}(\FF_q),\quad A \mapsto \begin{bmatrix} A & 0 \\ 0 & 1 \end{bmatrix}.
\]
Consider the involutive anti-automorphism $\varphi(x)=x^t=x^{-1}$.
We verify that $\varphi$ preserves double cosets.
First note, for any group $G$ and subgroup $H$, the action of $G$ on $G/H$ by left translation gives rise to an action of $G$ on $G/H\times G/H$.
The orbits of this action are the double cosets $H\backslash G/H$.
This yields an identification of $H\backslash G/H$ with $G\backslash (G/H\times G/H)$.
Explicitly, the identification is given by $(g_1H,g_2H)\mapsto Hg_1g_2^{-1}H$.

Notice that $G/H := \mO_{n+1}(\FF_q)/\mO_n(\FF_q)$ is isomorphic to the unit sphere.
Given the previous discussion, it suffices to show that, given two unit vectors $u,v\in\RR^n$, there exists $g\in\mO_n(\FF_q)$ with $g(u) = v$ and $g(v) = u$, since the transpose map sends $(u,v)$ to $(v,u)$.
If $u-v$ is not orthogonal to itself, take $g$ to be the reflection relative to the hyperplane orthogonal to $u-v$.
More specifically, set $g(x) := x - \frac{2\langle u-v,x\rangle}{\langle u-v,u-v\rangle}(u-v)$.
Then
\begin{align*}
    g(u) & = u - \frac{2\langle u-v,u\rangle}{\langle u-v,u-v\rangle}(u-v) = u - \frac{2\|u\|^2-2\langle u,v\rangle}{\|u\|^2+\|v\|^2-2\langle u,v\rangle}(u-v) = u - (u-v) = v, \\
    g(v) & = v - \frac{2\langle u-v,v\rangle}{\langle u-v,u-v\rangle}(u-v) = v - \frac{2\langle u,v\rangle-2\|v\|^2}{\|u\|^2+\|v\|^2-2\langle u,v\rangle}(u-v) = v + (u-v) = u.
\end{align*}
If $u-v$ is orthogonal to itself, this tells us that $0=\langle u-v,u-v\rangle = \|u\|^2+\|v\|^2-\langle u,v\rangle = 2-2\langle u,v\rangle$ so $\langle u,v\rangle = 1$.
Then $\langle u+v,u+v\rangle = 4$ so $u+v$ is not orthgonal to itself, and we take $g$ to be the reflection relative to $u+v$.
That is, $g(x) := \frac{2\langle u+v,x\rangle}{\langle u+v,u+v\rangle}(u+v) - x$.
Then
\begin{align*}
    g(u) & = \frac{2\langle u+v,u\rangle}{\langle u+v,u+v\rangle}(u+v) - u = \frac{2\langle u,v\rangle + 2\|u\|^2}{4}(u+v)-u = (u+v)-u = v, \\
    g(v) & = \frac{2\langle u+v,v\rangle}{\langle u+v,u+v\rangle}(u+v) - v = \frac{2\langle u,v\rangle + 2\|v\|^2}{4}(u+v)-v = (u+v)-v = u.
\end{align*}

%%%%%%%%%%%%%%%%%%%%%%%%%%%%%%%%%%%%%%%%%%%%%%%%%%%%%%%%%%%%%%%%%%%%%%%%%%%%%%%%%%%%%%%%%%%%%%